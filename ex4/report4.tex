
% ---------- Titelblad Masterproef Faculteit Wetenschappen -----------
% Dit document is opgesteld voor compilatie met pdflatex.  Indien je
% wilt compileren met latex naar dvi/ps, dien je de figuren naar
% (e)ps-formaat om te zetten.
%                           -- december 2012
% -------------------------------------------------------------------
\RequirePackage{fix-cm}
\documentclass[12pt,a4paper,oneside]{article}

% --------------------- In te laden pakketten -----------------------
% Deze kan je eventueel toevoegen aan de pakketten die je al inlaadt
% als je dit titelblad integreert met de rest van thesis.
% -------------------------------------------------------------------
\usepackage{graphicx,xcolor,textpos}
\usepackage{helvet}

% -------------------- Pagina-instellingen --------------------------
% Indien je deze wijzigt, zal het titelblad ook wijzigen.  Dit dien je
% dan manueel aan te passen.
% --------------------------------------------------------------------

\topmargin -10mm
\textwidth 160truemm
\textheight 240truemm
\oddsidemargin 0mm
\evensidemargin 0mm

% ------------------- textpos-instellingen ---------------------------
% Enkele andere instellingen voor het voorblad.
% --------------------------------------------------------------------

\definecolor{green}{RGB}{172,196,0}
\definecolor{bluetitle}{RGB}{29,141,176}
\definecolor{blueaff}{RGB}{0,0,128}
\definecolor{blueline}{RGB}{82,189,236}
\setlength{\TPHorizModule}{1mm}
\setlength{\TPVertModule}{1mm}



%----------------------- Custom stuff -------------------------------

\graphicspath{./}
\usepackage{makeidx}
\index{hoofd}
\makeindex
\usepackage{amsmath}
\usepackage[english]{babel}
\usepackage{hyperref}
\usepackage{listings}
\usepackage{eurosym}


%------------------------ Plot packages ----------------------------
\usepackage{tikz}
\usepackage{pgfplots}
\usepackage{pgf}
\usepackage{units}
\usepackage{metalogo}
\usepackage{graphicx}
\usepackage{caption}
\usepackage{subcaption}
\usepackage{standalone}


%----------------------Title etc----------------------------------
\title{Non linear Systems: Laser Model Analysis}
\author{Moritz Wolter}
\date{\today}


% --------------------------------------------------------------------

\begin{document}

\title{Study of a predator prey model.}
\author{Moritz Wolter}

\maketitle

\section{The equation}
\begin{align}
\dot{x} &= x(x-a)(1-x) - bxy \\
\dot{y} &= xy - cy - d.
\label{eq:toBeAn}
\end{align}
With $a = 0.4, \; b = 0.3, \text{ and } c \in [0.65 0.75]$. $x$ represents prey and $y$ predators. The $xy$ products of the system govern the interaction of the two species. 

\section{Analysis of a simplified model $d=0$}
\subsection{One-dimensional approach}
Setting d and y equal to zero turns the system into:
\begin{equation}
\dot{x} = x(x-a)(1-x).
\label{eq:simple}
\end{equation}
For this simplified case the fixed points may be read off easily. $\dot{x} = 0$ yields $x_1 = 0, \; x_2 = a, \; x_3 = 1$. Linear analysis will lead to further insight into the nature of these fixed points reading of $f(x) =  x(x-a)(1-x)$ and computing $f'(x)$ leads to:
\begin{equation}
f'(x) = -3x^2 + 2x + 2xa -a.
\end{equation}
Substituting x with the fixed points yields:
\begin{align}
f'(x_1) &= -a \\
f'(x_2) &= -a^2 +a = -0.4^2 + 0.4 > 0 \\
f'(x_3) &= -3 + 2 + 2a - a = -1 + a = -0.6 < 0 
\end{align}
Thus it may be concluded, that $x_2$ is unstable and $x_3 \wedge x_1$ are stable.  
\begin{figure}
\centering
% This file was created by matlab2tikz.
% Minimal pgfplots version: 1.3
%
%The latest updates can be retrieved from
%  http://www.mathworks.com/matlabcentral/fileexchange/22022-matlab2tikz
%where you can also make suggestions and rate matlab2tikz.
%
\documentclass[tikz]{standalone}
\usepackage{pgfplots}
\usepackage{grffile}
\pgfplotsset{compat=newest}
\usetikzlibrary{plotmarks}
\usepackage{amsmath}

\begin{document}
\definecolor{mycolor1}{rgb}{0.00000,0.44700,0.74100}%
\definecolor{mycolor2}{rgb}{0.85000,0.32500,0.09800}%
\definecolor{mycolor3}{rgb}{0.92900,0.69400,0.12500}%
\definecolor{mycolor4}{rgb}{0.49400,0.18400,0.55600}%
\definecolor{mycolor5}{rgb}{0.46600,0.67400,0.18800}%
\definecolor{mycolor6}{rgb}{0.30100,0.74500,0.93300}%
\definecolor{mycolor7}{rgb}{0.63500,0.07800,0.18400}%
%
\begin{tikzpicture}

\begin{axis}[%
width=3in,
height=2in,
at={(0.744792in,0.478958in)},
scale only axis,
xmin=0,
xmax=10,
xlabel={time [s]},
ymin=0,
ymax=1.5,
ylabel={x}
]
\addplot [color=mycolor1,solid,forget plot]
  table[row sep=crcr]{%
0	0\\
0.25	0\\
0.5	0\\
0.75	0\\
1	0\\
1.25	0\\
1.5	0\\
1.75	0\\
2	0\\
2.25	0\\
2.5	0\\
2.75	0\\
3	0\\
3.25	0\\
3.5	0\\
3.75	0\\
4	0\\
4.25	0\\
4.5	0\\
4.75	0\\
5	0\\
5.25	0\\
5.5	0\\
5.75	0\\
6	0\\
6.25	0\\
6.5	0\\
6.75	0\\
7	0\\
7.25	0\\
7.5	0\\
7.75	0\\
8	0\\
8.25	0\\
8.5	0\\
8.75	0\\
9	0\\
9.25	0\\
9.5	0\\
9.75	0\\
10	0\\
};
\addplot [color=mycolor2,solid,forget plot]
  table[row sep=crcr]{%
0	0.1\\
0.186065661593302	0.0950473631707389\\
0.372131323186604	0.0902408256707844\\
0.558196984779907	0.0855854294778743\\
0.744262646373209	0.0810852846831479\\
0.994262646373209	0.0752887263991574\\
1.24426264637321	0.0697832440054908\\
1.49426264637321	0.0645712588179216\\
1.74426264637321	0.0596525182073924\\
1.99426264637321	0.0550242758801567\\
2.24426264637321	0.0506817395861209\\
2.49426264637321	0.0466183248510188\\
2.74426264637321	0.0428257705978301\\
2.99426264637321	0.0392943813137015\\
3.24426264637321	0.0360136296002224\\
3.49426264637321	0.0329722507814812\\
3.74426264637321	0.0301583157569112\\
3.99426264637321	0.0275594534312763\\
4.24426264637321	0.0251634035969796\\
4.49426264637321	0.0229579211288335\\
4.74426264637321	0.0209307821848653\\
4.99426264637321	0.0190699554052319\\
5.24426264637321	0.0173640208556066\\
5.49426264637321	0.015801952895518\\
5.74426264637321	0.0143730793522497\\
5.99426264637321	0.0130672020742472\\
6.24426264637321	0.0118748827992332\\
6.49426264637321	0.0107871873919343\\
6.74426264637321	0.00979562585402563\\
6.99426264637321	0.00889223375024768\\
7.24426264637321	0.00806975713285761\\
7.49426264637321	0.00732141352200719\\
7.74426264637321	0.00664083201716912\\
7.99426264637321	0.00602210739488229\\
8.24426264637321	0.00545991720135689\\
8.49426264637321	0.00494932414239661\\
8.74426264637321	0.00448572528024961\\
8.99426264637321	0.00406488782003703\\
9.24426264637321	0.00368302305360603\\
9.49426264637321	0.00333663456968276\\
9.74426264637321	0.0030224787957342\\
9.80819698477991	0.00294694818405799\\
9.8721313231866	0.00287328595605968\\
9.9360656615933	0.00280144684577873\\
10	0.00273138661093691\\
};
\addplot [color=mycolor3,solid,forget plot]
  table[row sep=crcr]{%
0	0.2\\
0.25	0.191964175445535\\
0.5	0.183874178360581\\
0.75	0.175757085922189\\
1	0.167641125053278\\
1.25	0.159555404785902\\
1.5	0.151529434576239\\
1.75	0.14359247153584\\
2	0.135773195766996\\
2.25	0.12809939914824\\
2.5	0.120597362968057\\
2.75	0.113291373074574\\
3	0.106203454499729\\
3.25	0.0993531808235956\\
3.5	0.0927572757048673\\
3.75	0.0864295265285032\\
4	0.0803806978091195\\
4.25	0.0746185498495501\\
4.5	0.0691479028726998\\
4.75	0.0639708564222577\\
5	0.0590868597775438\\
5.25	0.0544928956512089\\
5.5	0.0501839362699888\\
5.75	0.0461531944143232\\
6	0.0423922387690108\\
6.25	0.0388912364274936\\
6.5	0.0356395563518158\\
6.75	0.0326258580268452\\
7	0.0298381623626457\\
7.25	0.0272640728344241\\
7.5	0.0248913250310948\\
7.75	0.0227076860024539\\
8	0.0207009586653294\\
8.25	0.0188591513918266\\
8.5	0.0171708927138507\\
8.75	0.0156252119347572\\
9	0.0142114973533061\\
9.25	0.0129196154702571\\
9.5	0.0117401933391636\\
9.75	0.0106643626710023\\
10	0.00968369962538085\\
};
\addplot [color=mycolor4,solid,forget plot]
  table[row sep=crcr]{%
0	0.3\\
0.25	0.29463805678257\\
0.5	0.289050985170633\\
0.75	0.283237889438719\\
1	0.277199163480021\\
1.25	0.270936606044933\\
1.5	0.264453649189452\\
1.75	0.257755538087673\\
2	0.250849430058369\\
2.25	0.243744480172066\\
2.5	0.236452042479212\\
2.75	0.228985704479947\\
3	0.221361317145054\\
3.25	0.213596980867398\\
3.5	0.20571308541489\\
3.75	0.19773205252314\\
4	0.189678222726795\\
4.25	0.181577685150032\\
4.5	0.173458013820053\\
4.75	0.165347703335334\\
5	0.157275898441052\\
5.25	0.149272095720937\\
5.5	0.141365583593175\\
5.75	0.133584804246744\\
6	0.125957029313425\\
6.25	0.118508070403879\\
6.5	0.111261687987539\\
6.75	0.104239223712173\\
7	0.0974593854198682\\
7.25	0.0909381206332745\\
7.5	0.0846883548456782\\
7.75	0.07872002778164\\
8	0.0730400662067604\\
8.25	0.0676524655095758\\
8.5	0.0625585005645293\\
8.75	0.0577569855594424\\
9	0.053244371898911\\
9.25	0.0490149623462337\\
9.5	0.0450614423770349\\
9.75	0.0413750919238877\\
10	0.0379458929819096\\
};
\addplot [color=mycolor5,solid,forget plot]
  table[row sep=crcr]{%
0	0.4\\
0.25	0.4\\
0.5	0.4\\
0.75	0.4\\
1	0.4\\
1.25	0.4\\
1.5	0.4\\
1.75	0.4\\
2	0.4\\
2.25	0.4\\
2.5	0.4\\
2.75	0.4\\
3	0.4\\
3.25	0.4\\
3.5	0.4\\
3.75	0.4\\
4	0.4\\
4.25	0.4\\
4.5	0.4\\
4.75	0.4\\
5	0.4\\
5.25	0.4\\
5.5	0.4\\
5.75	0.4\\
6	0.4\\
6.25	0.4\\
6.5	0.4\\
6.75	0.4\\
7	0.4\\
7.25	0.4\\
7.5	0.4\\
7.75	0.4\\
8	0.4\\
8.25	0.4\\
8.5	0.4\\
8.75	0.4\\
9	0.4\\
9.25	0.4\\
9.5	0.4\\
9.75	0.4\\
10	0.4\\
};
\addplot [color=mycolor6,solid,forget plot]
  table[row sep=crcr]{%
0	0.5\\
0.25	0.506449082396747\\
0.5	0.513311637571175\\
0.75	0.520610847942769\\
1	0.528370189855291\\
1.25	0.536613163515774\\
1.5	0.545362854493291\\
1.75	0.554641201770392\\
2	0.564468629866309\\
2.25	0.57486347191569\\
2.5	0.585840994653731\\
2.75	0.597411922479223\\
3	0.609581693559258\\
3.25	0.622349550466762\\
3.5	0.635706773794713\\
3.75	0.649634902059092\\
4	0.664104817765994\\
4.25	0.679075994315484\\
4.5	0.694494479242512\\
4.75	0.710292988642329\\
5	0.72639085780252\\
5.25	0.742694661428322\\
5.5	0.759098530049686\\
5.75	0.775488610312741\\
6	0.791745066499918\\
6.25	0.807744611954209\\
6.5	0.823365454255885\\
6.75	0.838493151677356\\
7	0.853023448366809\\
7.25	0.866864586483304\\
7.5	0.879942291574251\\
7.75	0.892200581891367\\
8	0.903602688505274\\
8.25	0.914130759939149\\
8.5	0.923784358923342\\
8.75	0.932577624572531\\
9	0.940538532160206\\
9.25	0.947706794393916\\
9.5	0.954127779303864\\
9.75	0.959851591162071\\
10	0.964932692767807\\
};
\addplot [color=mycolor7,solid,forget plot]
  table[row sep=crcr]{%
0	0.6\\
0.25	0.612299812901764\\
0.5	0.625196695467512\\
0.75	0.638679610930008\\
1	0.652727919111458\\
1.25	0.667310514344741\\
1.5	0.682383727609995\\
1.75	0.697890801995765\\
2	0.713761555061741\\
2.25	0.729912628751586\\
2.5	0.746247065238195\\
2.75	0.762657897054947\\
3	0.779029735103866\\
3.25	0.79524100538844\\
3.5	0.811168078474461\\
3.75	0.826691422744679\\
4	0.841698515729568\\
4.25	0.856086454842726\\
4.5	0.86976760082298\\
4.75	0.88267159442879\\
5	0.894746758257742\\
5.25	0.905960348791703\\
5.5	0.916298454563344\\
5.75	0.925763380416267\\
6	0.934373061581365\\
6.25	0.942159154140788\\
6.5	0.949161406211357\\
6.75	0.955426113107914\\
7	0.961005569133934\\
7.25	0.965955861682459\\
7.5	0.970330832673213\\
7.75	0.974183664326144\\
8	0.97756721470784\\
8.25	0.980532344504553\\
8.5	0.98312383749042\\
8.75	0.985383501972512\\
9	0.987350891222817\\
9.25	0.989062184669164\\
9.5	0.990547871472421\\
9.75	0.991835698434454\\
10	0.992951305965978\\
};
\addplot [color=mycolor1,solid,forget plot]
  table[row sep=crcr]{%
0	0.7\\
0.25	0.715913673140071\\
0.5	0.732094887809705\\
0.75	0.748446006945702\\
1	0.764858820235082\\
1.25	0.781216379806755\\
1.5	0.797396095095385\\
1.75	0.813275854423579\\
2	0.828736869599653\\
2.25	0.843666472784918\\
2.5	0.857964098193271\\
2.75	0.871544610713617\\
3	0.884340218418093\\
3.25	0.896301346184009\\
3.5	0.907398110532371\\
3.75	0.917618254210786\\
4	0.926966839730199\\
4.25	0.935464644440715\\
4.5	0.943143295749738\\
4.75	0.950043120471941\\
5	0.956212444615705\\
5.25	0.961705335462597\\
5.5	0.966575316441525\\
5.75	0.9708763537411\\
6	0.974663022301857\\
6.25	0.977988701917189\\
6.5	0.980901042255131\\
6.75	0.983444910204352\\
7	0.985663080083172\\
7.25	0.987595005631169\\
7.5	0.989274182630066\\
7.75	0.990731217473631\\
8	0.991994501142284\\
8.25	0.993089428018109\\
8.5	0.99403696888487\\
8.75	0.994856011836646\\
9	0.995563837062381\\
9.25	0.996175643964692\\
9.5	0.99670378950895\\
9.75	0.99715932395932\\
10	0.997552283793207\\
};
\addplot [color=mycolor2,solid,forget plot]
  table[row sep=crcr]{%
0	0.8\\
0.25	0.815820226034934\\
0.5	0.831201695826699\\
0.75	0.846034679088846\\
1	0.860221555112395\\
1.25	0.87367886943698\\
1.5	0.88634172210534\\
1.75	0.898163820708214\\
2	0.909118091985625\\
2.25	0.919196049260238\\
2.5	0.92840542241709\\
2.75	0.936767310818937\\
3	0.944315391718171\\
3.25	0.951093738829737\\
3.5	0.957150558742577\\
3.75	0.962537699889425\\
4	0.967310395305861\\
4.25	0.971525197125516\\
4.5	0.975234490313337\\
4.75	0.978488801275115\\
5	0.981337330644785\\
5.25	0.9838264721932\\
5.5	0.985996378830404\\
5.75	0.987884146816719\\
6	0.989524540741121\\
6.25	0.990949025316372\\
6.5	0.992183865785363\\
6.75	0.99325284609905\\
7	0.994177839968772\\
7.25	0.99497821210104\\
7.5	0.995669801973442\\
7.75	0.996266812387941\\
8	0.996782178964794\\
8.25	0.997227215523025\\
8.5	0.997611070738371\\
8.75	0.997941907195444\\
9	0.998227120899609\\
9.25	0.998473137563475\\
9.5	0.998685119939727\\
9.75	0.998867662376326\\
10	0.999024916177123\\
};
\addplot [color=mycolor3,solid,forget plot]
  table[row sep=crcr]{%
0	0.9\\
0.25	0.910812461623111\\
0.5	0.920748818039132\\
0.75	0.929818478200031\\
1	0.93804545323369\\
1.25	0.945466326473351\\
1.5	0.952124332375219\\
1.75	0.95806817248844\\
2	0.963351564502009\\
2.25	0.968031079178228\\
2.5	0.972160277358748\\
2.75	0.975791576827072\\
3	0.978976665860834\\
3.25	0.981764898848435\\
3.5	0.984199451564601\\
3.75	0.986320470675039\\
4	0.98816580076278\\
4.25	0.989769920593186\\
4.5	0.991161782715511\\
4.75	0.992367685577523\\
5	0.993411887294857\\
5.25	0.994315940833207\\
5.5	0.995097534810212\\
5.75	0.995772554506192\\
6	0.996355490449292\\
6.25	0.996859041505354\\
6.5	0.997293495736756\\
6.75	0.997668039557587\\
7	0.997991003588974\\
7.25	0.998269633425031\\
7.5	0.998509756686829\\
7.75	0.998716561545868\\
8	0.998894737979855\\
8.25	0.999048348183151\\
8.5	0.99918064620545\\
8.75	0.999294524487324\\
9	0.999392593236276\\
9.25	0.99947710810268\\
9.5	0.999549871957848\\
9.75	0.999612485974338\\
10	0.999666393735973\\
};
\addplot [color=mycolor4,solid,forget plot]
  table[row sep=crcr]{%
0	1\\
0.25	1\\
0.5	1\\
0.75	1\\
1	1\\
1.25	1\\
1.5	1\\
1.75	1\\
2	1\\
2.25	1\\
2.5	1\\
2.75	1\\
3	1\\
3.25	1\\
3.5	1\\
3.75	1\\
4	1\\
4.25	1\\
4.5	1\\
4.75	1\\
5	1\\
5.25	1\\
5.5	1\\
5.75	1\\
6	1\\
6.25	1\\
6.5	1\\
6.75	1\\
7	1\\
7.25	1\\
7.5	1\\
7.75	1\\
8	1\\
8.25	1\\
8.5	1\\
8.75	1\\
9	1\\
9.25	1\\
9.5	1\\
9.75	1\\
10	1\\
};
\addplot [color=mycolor5,solid,forget plot]
  table[row sep=crcr]{%
0	1.1\\
0.25	1.08277735898578\\
0.5	1.06902047865195\\
0.75	1.05793216863778\\
1	1.04883023268385\\
1.25	1.04125225411422\\
1.5	1.03495559760308\\
1.75	1.02970131839883\\
2	1.02528691695878\\
2.25	1.02155711379417\\
2.5	1.01840424345887\\
2.75	1.01573401888912\\
3	1.01346476303214\\
3.25	1.01153028051212\\
3.5	1.00988116565486\\
3.75	1.00847425559899\\
4	1.007271550324\\
4.25	1.00624137026516\\
4.5	1.00535930852791\\
4.75	1.00460392921658\\
5	1.00395617375417\\
5.25	1.00339991214974\\
5.5	1.00292251900585\\
5.75	1.0025128595799\\
6	1.002160977736\\
6.25	1.00185837801204\\
6.5	1.00159835515083\\
6.75	1.0013749802588\\
7	1.00118293508212\\
7.25	1.00101766130067\\
7.5	1.00087554499794\\
7.75	1.0007533858918\\
8	1.00064830821532\\
8.25	1.00055784109076\\
8.5	1.00048002079272\\
8.75	1.00041310688219\\
9	1.00035553387081\\
9.25	1.000305954877\\
9.5	1.00026329804737\\
9.75	1.00022661295594\\
10	1.00019504425276\\
};
\addplot [color=mycolor6,solid,forget plot]
  table[row sep=crcr]{%
0	1.2\\
0.25	1.15844880064465\\
0.5	1.12824716188127\\
0.75	1.10639734713305\\
1	1.08900071911549\\
1.25	1.07400201240788\\
1.5	1.06191594736442\\
1.75	1.05209716892024\\
2	1.04399896079454\\
2.25	1.0372405030559\\
2.5	1.03160471774849\\
2.75	1.02688740577607\\
3	1.02291493488655\\
3.25	1.01955264331821\\
3.5	1.0167055918683\\
3.75	1.01429081831055\\
4	1.01223624182447\\
4.25	1.01048311479921\\
4.5	1.00898730132524\\
4.75	1.00771021307348\\
5	1.00661781356162\\
5.25	1.00568164305034\\
5.5	1.00487970555911\\
5.75	1.0041926680756\\
6	1.00360332284981\\
6.25	1.00309708290159\\
6.5	1.00266251174424\\
6.75	1.00228951822489\\
7	1.00196907373759\\
7.25	1.00169346752228\\
7.5	1.00145660829875\\
7.75	1.00125310850189\\
8	1.00107813378105\\
8.25	1.00092753858812\\
8.5	1.00079803468945\\
8.75	1.00068670973012\\
9	1.00059094617275\\
9.25	1.00050849442171\\
9.5	1.00043756614969\\
9.75	1.00037657622622\\
10	1.00032409870913\\
};
\addplot [color=mycolor7,solid,forget plot]
  table[row sep=crcr]{%
0	1.3\\
0.186065661593302	1.24338903910851\\
0.372131323186604	1.20206939477602\\
0.558196984779907	1.17192950031586\\
0.744262646373209	1.14758541231878\\
0.946296840123378	1.12461140032858\\
1.14833103387355	1.10607275497641\\
1.35036522762372	1.09093331004956\\
1.55239942137389	1.07833176949678\\
1.80239942137389	1.06541059277248\\
2.05239942137389	1.05491139166382\\
2.30239942137389	1.04631864246718\\
2.55239942137389	1.03919787914037\\
2.80239942137389	1.03323850223038\\
3.05239942137389	1.02825151299888\\
3.30239942137389	1.02406455723173\\
3.55239942137389	1.02053050406126\\
3.80239942137389	1.01753395280251\\
4.05239942137389	1.01499225959552\\
4.30239942137389	1.01283329459491\\
4.55239942137389	1.01099419232176\\
4.80239942137389	1.00942342439084\\
5.05239942137389	1.00808202642738\\
5.30239942137389	1.00693590019258\\
5.55239942137389	1.00595491138029\\
5.80239942137389	1.00511378701502\\
6.05239942137389	1.00439293165469\\
6.30239942137389	1.00377510845092\\
6.55239942137389	1.00324495816611\\
6.80239942137389	1.00278943953333\\
7.05239942137389	1.00239831085878\\
7.30239942137389	1.00206253064813\\
7.55239942137389	1.00177400430754\\
7.80239942137389	1.00152581259924\\
8.05239942137388	1.00131248463015\\
8.30239942137388	1.00112918011358\\
8.55239942137388	1.00097155424909\\
8.80239942137388	1.0008358794713\\
9.05239942137388	1.00071919765384\\
9.30239942137388	1.00061888837388\\
9.55239942137388	1.00053259601446\\
9.66429956603041	1.00049796519051\\
9.77619971068694	1.00046558908512\\
9.88809985534347	1.00043532085192\\
10	1.00040702249989\\
};
\addplot [color=mycolor1,solid,forget plot]
  table[row sep=crcr]{%
0	1.4\\
0.125594321575479	1.33826077376974\\
0.251188643150958	1.29101801216889\\
0.376782964726437	1.25425286179904\\
0.502377286301916	1.22396944421311\\
0.657624747429572	1.19242944322825\\
0.812872208557227	1.16685314634466\\
0.968119669684883	1.14580472586189\\
1.12336713081254	1.12811332558215\\
1.37117733415766	1.10500344868328\\
1.61898753750279	1.08693171350335\\
1.86679774084791	1.07265059620199\\
2.11460794419304	1.06104323507051\\
2.36460794419304	1.05132425337288\\
2.61460794419304	1.04332199962768\\
2.86460794419304	1.03669695799516\\
3.11460794419304	1.03116298964812\\
3.36460794419304	1.02650690390066\\
3.61460794419304	1.02258768737405\\
3.86460794419304	1.01928056319934\\
4.11460794419304	1.01647817215857\\
4.36460794419304	1.01409470335031\\
4.61460794419304	1.01206717882955\\
4.86460794419304	1.01034064308984\\
5.11460794419304	1.00886692867626\\
5.36460794419304	1.00760617006904\\
5.61460794419304	1.00652789454437\\
5.86460794419304	1.0056053839858\\
6.11460794419304	1.00481494575149\\
6.36460794419304	1.0041366060692\\
6.61460794419304	1.00355479395786\\
6.86460794419304	1.00305579274171\\
7.11460794419304	1.0026273559763\\
7.36460794419304	1.00225905610722\\
7.61460794419304	1.00194268059775\\
7.86460794419304	1.00167097282635\\
8.11460794419304	1.00143742900194\\
8.36460794419304	1.00123648114904\\
8.61460794419304	1.00106372000309\\
8.86460794419304	1.00091524251493\\
9.11460794419304	1.00078754303893\\
9.33595595814478	1.00068939999145\\
9.55730397209652	1.0006035119108\\
9.77865198604826	1.0005283633729\\
10	1.00046258593362\\
};
\addplot [color=mycolor2,solid,forget plot]
  table[row sep=crcr]{%
0	1.5\\
0.0913413247821665	1.43322729477461\\
0.182682649564333	1.38078304011729\\
0.2740239743465	1.33874647243676\\
0.365365299128666	1.30364779309686\\
0.488366918930436	1.2640510481543\\
0.611368538732205	1.23186721967296\\
0.734370158533974	1.20527338886581\\
0.857371778335744	1.18279830127887\\
1.05294653628572	1.15331599649886\\
1.2485212942357	1.12994647833736\\
1.44409605218568	1.11116263295264\\
1.63967081013566	1.09565174666543\\
1.88967081013566	1.07931829750749\\
2.13967081013566	1.06622604275208\\
2.38967081013566	1.05564004592219\\
2.63967081013566	1.04693427080318\\
2.88967081013566	1.0396797481757\\
3.13967081013566	1.03364338970062\\
3.38967081013566	1.02860021529076\\
3.63967081013566	1.0243593243394\\
3.88967081013566	1.02077367858892\\
4.13967081013566	1.01774064569229\\
4.38967081013566	1.01517043326074\\
4.63967081013566	1.01298516856112\\
4.88967081013566	1.01112160058504\\
5.13967081013566	1.00953239914644\\
5.38967081013566	1.00817620309136\\
5.63967081013566	1.00701657207103\\
5.88967081013566	1.00602309153518\\
6.13967081013566	1.00517230007631\\
6.38967081013566	1.00444358622341\\
6.63967081013566	1.00381861658518\\
6.88967081013566	1.00328186520356\\
7.13967081013566	1.00282117164912\\
7.38967081013566	1.00242580920965\\
7.63967081013566	1.00208618427488\\
7.88967081013566	1.00179410792783\\
8.13967081013566	1.00154311463693\\
8.38967081013566	1.00132748683429\\
8.63967081013566	1.00114209513573\\
8.88967081013566	1.00098254228212\\
9.13967081013566	1.00084534139987\\
9.38967081013566	1.00072740451117\\
9.63967081013566	1.0006259565585\\
9.72975310760174	1.00059296983052\\
9.81983540506783	1.00056172409875\\
9.90991770253391	1.00053212731968\\
10	1.00050409205657\\
};
\addplot [color=mycolor3,solid,forget plot]
  table[row sep=crcr]{%
0	0\\
0.25	0\\
0.5	0\\
0.75	0\\
1	0\\
1.25	0\\
1.5	0\\
1.75	0\\
2	0\\
2.25	0\\
2.5	0\\
2.75	0\\
3	0\\
3.25	0\\
3.5	0\\
3.75	0\\
4	0\\
4.25	0\\
4.5	0\\
4.75	0\\
5	0\\
5.25	0\\
5.5	0\\
5.75	0\\
6	0\\
6.25	0\\
6.5	0\\
6.75	0\\
7	0\\
7.25	0\\
7.5	0\\
7.75	0\\
8	0\\
8.25	0\\
8.5	0\\
8.75	0\\
9	0\\
9.25	0\\
9.5	0\\
9.75	0\\
10	0\\
};
\addplot [color=mycolor4,solid,forget plot]
  table[row sep=crcr]{%
0	0.1\\
0.186065661593302	0.0950473631707389\\
0.372131323186604	0.0902408256707844\\
0.558196984779907	0.0855854294778743\\
0.744262646373209	0.0810852846831479\\
0.994262646373209	0.0752887263991574\\
1.24426264637321	0.0697832440054908\\
1.49426264637321	0.0645712588179216\\
1.74426264637321	0.0596525182073924\\
1.99426264637321	0.0550242758801567\\
2.24426264637321	0.0506817395861209\\
2.49426264637321	0.0466183248510188\\
2.74426264637321	0.0428257705978301\\
2.99426264637321	0.0392943813137015\\
3.24426264637321	0.0360136296002224\\
3.49426264637321	0.0329722507814812\\
3.74426264637321	0.0301583157569112\\
3.99426264637321	0.0275594534312763\\
4.24426264637321	0.0251634035969796\\
4.49426264637321	0.0229579211288335\\
4.74426264637321	0.0209307821848653\\
4.99426264637321	0.0190699554052319\\
5.24426264637321	0.0173640208556066\\
5.49426264637321	0.015801952895518\\
5.74426264637321	0.0143730793522497\\
5.99426264637321	0.0130672020742472\\
6.24426264637321	0.0118748827992332\\
6.49426264637321	0.0107871873919343\\
6.74426264637321	0.00979562585402563\\
6.99426264637321	0.00889223375024768\\
7.24426264637321	0.00806975713285761\\
7.49426264637321	0.00732141352200719\\
7.74426264637321	0.00664083201716912\\
7.99426264637321	0.00602210739488229\\
8.24426264637321	0.00545991720135689\\
8.49426264637321	0.00494932414239661\\
8.74426264637321	0.00448572528024961\\
8.99426264637321	0.00406488782003703\\
9.24426264637321	0.00368302305360603\\
9.49426264637321	0.00333663456968276\\
9.74426264637321	0.0030224787957342\\
9.80819698477991	0.00294694818405799\\
9.8721313231866	0.00287328595605968\\
9.9360656615933	0.00280144684577873\\
10	0.00273138661093691\\
};
\addplot [color=mycolor5,solid,forget plot]
  table[row sep=crcr]{%
0	0.2\\
0.25	0.191964175445535\\
0.5	0.183874178360581\\
0.75	0.175757085922189\\
1	0.167641125053278\\
1.25	0.159555404785902\\
1.5	0.151529434576239\\
1.75	0.14359247153584\\
2	0.135773195766996\\
2.25	0.12809939914824\\
2.5	0.120597362968057\\
2.75	0.113291373074574\\
3	0.106203454499729\\
3.25	0.0993531808235956\\
3.5	0.0927572757048673\\
3.75	0.0864295265285032\\
4	0.0803806978091195\\
4.25	0.0746185498495501\\
4.5	0.0691479028726998\\
4.75	0.0639708564222577\\
5	0.0590868597775438\\
5.25	0.0544928956512089\\
5.5	0.0501839362699888\\
5.75	0.0461531944143232\\
6	0.0423922387690108\\
6.25	0.0388912364274936\\
6.5	0.0356395563518158\\
6.75	0.0326258580268452\\
7	0.0298381623626457\\
7.25	0.0272640728344241\\
7.5	0.0248913250310948\\
7.75	0.0227076860024539\\
8	0.0207009586653294\\
8.25	0.0188591513918266\\
8.5	0.0171708927138507\\
8.75	0.0156252119347572\\
9	0.0142114973533061\\
9.25	0.0129196154702571\\
9.5	0.0117401933391636\\
9.75	0.0106643626710023\\
10	0.00968369962538085\\
};
\addplot [color=mycolor6,solid,forget plot]
  table[row sep=crcr]{%
0	0.3\\
0.25	0.29463805678257\\
0.5	0.289050985170633\\
0.75	0.283237889438719\\
1	0.277199163480021\\
1.25	0.270936606044933\\
1.5	0.264453649189452\\
1.75	0.257755538087673\\
2	0.250849430058369\\
2.25	0.243744480172066\\
2.5	0.236452042479212\\
2.75	0.228985704479947\\
3	0.221361317145054\\
3.25	0.213596980867398\\
3.5	0.20571308541489\\
3.75	0.19773205252314\\
4	0.189678222726795\\
4.25	0.181577685150032\\
4.5	0.173458013820053\\
4.75	0.165347703335334\\
5	0.157275898441052\\
5.25	0.149272095720937\\
5.5	0.141365583593175\\
5.75	0.133584804246744\\
6	0.125957029313425\\
6.25	0.118508070403879\\
6.5	0.111261687987539\\
6.75	0.104239223712173\\
7	0.0974593854198682\\
7.25	0.0909381206332745\\
7.5	0.0846883548456782\\
7.75	0.07872002778164\\
8	0.0730400662067604\\
8.25	0.0676524655095758\\
8.5	0.0625585005645293\\
8.75	0.0577569855594424\\
9	0.053244371898911\\
9.25	0.0490149623462337\\
9.5	0.0450614423770349\\
9.75	0.0413750919238877\\
10	0.0379458929819096\\
};
\addplot [color=mycolor7,solid,forget plot]
  table[row sep=crcr]{%
0	0.4\\
0.25	0.4\\
0.5	0.4\\
0.75	0.4\\
1	0.4\\
1.25	0.4\\
1.5	0.4\\
1.75	0.4\\
2	0.4\\
2.25	0.4\\
2.5	0.4\\
2.75	0.4\\
3	0.4\\
3.25	0.4\\
3.5	0.4\\
3.75	0.4\\
4	0.4\\
4.25	0.4\\
4.5	0.4\\
4.75	0.4\\
5	0.4\\
5.25	0.4\\
5.5	0.4\\
5.75	0.4\\
6	0.4\\
6.25	0.4\\
6.5	0.4\\
6.75	0.4\\
7	0.4\\
7.25	0.4\\
7.5	0.4\\
7.75	0.4\\
8	0.4\\
8.25	0.4\\
8.5	0.4\\
8.75	0.4\\
9	0.4\\
9.25	0.4\\
9.5	0.4\\
9.75	0.4\\
10	0.4\\
};
\addplot [color=mycolor1,solid,forget plot]
  table[row sep=crcr]{%
0	0.5\\
0.25	0.506449082396747\\
0.5	0.513311637571175\\
0.75	0.520610847942769\\
1	0.528370189855291\\
1.25	0.536613163515774\\
1.5	0.545362854493291\\
1.75	0.554641201770392\\
2	0.564468629866309\\
2.25	0.57486347191569\\
2.5	0.585840994653731\\
2.75	0.597411922479223\\
3	0.609581693559258\\
3.25	0.622349550466762\\
3.5	0.635706773794713\\
3.75	0.649634902059092\\
4	0.664104817765994\\
4.25	0.679075994315484\\
4.5	0.694494479242512\\
4.75	0.710292988642329\\
5	0.72639085780252\\
5.25	0.742694661428322\\
5.5	0.759098530049686\\
5.75	0.775488610312741\\
6	0.791745066499918\\
6.25	0.807744611954209\\
6.5	0.823365454255885\\
6.75	0.838493151677356\\
7	0.853023448366809\\
7.25	0.866864586483304\\
7.5	0.879942291574251\\
7.75	0.892200581891367\\
8	0.903602688505274\\
8.25	0.914130759939149\\
8.5	0.923784358923342\\
8.75	0.932577624572531\\
9	0.940538532160206\\
9.25	0.947706794393916\\
9.5	0.954127779303864\\
9.75	0.959851591162071\\
10	0.964932692767807\\
};
\addplot [color=mycolor2,solid,forget plot]
  table[row sep=crcr]{%
0	0.6\\
0.25	0.612299812901764\\
0.5	0.625196695467512\\
0.75	0.638679610930008\\
1	0.652727919111458\\
1.25	0.667310514344741\\
1.5	0.682383727609995\\
1.75	0.697890801995765\\
2	0.713761555061741\\
2.25	0.729912628751586\\
2.5	0.746247065238195\\
2.75	0.762657897054947\\
3	0.779029735103866\\
3.25	0.79524100538844\\
3.5	0.811168078474461\\
3.75	0.826691422744679\\
4	0.841698515729568\\
4.25	0.856086454842726\\
4.5	0.86976760082298\\
4.75	0.88267159442879\\
5	0.894746758257742\\
5.25	0.905960348791703\\
5.5	0.916298454563344\\
5.75	0.925763380416267\\
6	0.934373061581365\\
6.25	0.942159154140788\\
6.5	0.949161406211357\\
6.75	0.955426113107914\\
7	0.961005569133934\\
7.25	0.965955861682459\\
7.5	0.970330832673213\\
7.75	0.974183664326144\\
8	0.97756721470784\\
8.25	0.980532344504553\\
8.5	0.98312383749042\\
8.75	0.985383501972512\\
9	0.987350891222817\\
9.25	0.989062184669164\\
9.5	0.990547871472421\\
9.75	0.991835698434454\\
10	0.992951305965978\\
};
\addplot [color=mycolor3,solid,forget plot]
  table[row sep=crcr]{%
0	0.7\\
0.25	0.715913673140071\\
0.5	0.732094887809705\\
0.75	0.748446006945702\\
1	0.764858820235082\\
1.25	0.781216379806755\\
1.5	0.797396095095385\\
1.75	0.813275854423579\\
2	0.828736869599653\\
2.25	0.843666472784918\\
2.5	0.857964098193271\\
2.75	0.871544610713617\\
3	0.884340218418093\\
3.25	0.896301346184009\\
3.5	0.907398110532371\\
3.75	0.917618254210786\\
4	0.926966839730199\\
4.25	0.935464644440715\\
4.5	0.943143295749738\\
4.75	0.950043120471941\\
5	0.956212444615705\\
5.25	0.961705335462597\\
5.5	0.966575316441525\\
5.75	0.9708763537411\\
6	0.974663022301857\\
6.25	0.977988701917189\\
6.5	0.980901042255131\\
6.75	0.983444910204352\\
7	0.985663080083172\\
7.25	0.987595005631169\\
7.5	0.989274182630066\\
7.75	0.990731217473631\\
8	0.991994501142284\\
8.25	0.993089428018109\\
8.5	0.99403696888487\\
8.75	0.994856011836646\\
9	0.995563837062381\\
9.25	0.996175643964692\\
9.5	0.99670378950895\\
9.75	0.99715932395932\\
10	0.997552283793207\\
};
\addplot [color=mycolor4,solid,forget plot]
  table[row sep=crcr]{%
0	0.8\\
0.25	0.815820226034934\\
0.5	0.831201695826699\\
0.75	0.846034679088846\\
1	0.860221555112395\\
1.25	0.87367886943698\\
1.5	0.88634172210534\\
1.75	0.898163820708214\\
2	0.909118091985625\\
2.25	0.919196049260238\\
2.5	0.92840542241709\\
2.75	0.936767310818937\\
3	0.944315391718171\\
3.25	0.951093738829737\\
3.5	0.957150558742577\\
3.75	0.962537699889425\\
4	0.967310395305861\\
4.25	0.971525197125516\\
4.5	0.975234490313337\\
4.75	0.978488801275115\\
5	0.981337330644785\\
5.25	0.9838264721932\\
5.5	0.985996378830404\\
5.75	0.987884146816719\\
6	0.989524540741121\\
6.25	0.990949025316372\\
6.5	0.992183865785363\\
6.75	0.99325284609905\\
7	0.994177839968772\\
7.25	0.99497821210104\\
7.5	0.995669801973442\\
7.75	0.996266812387941\\
8	0.996782178964794\\
8.25	0.997227215523025\\
8.5	0.997611070738371\\
8.75	0.997941907195444\\
9	0.998227120899609\\
9.25	0.998473137563475\\
9.5	0.998685119939727\\
9.75	0.998867662376326\\
10	0.999024916177123\\
};
\addplot [color=mycolor5,solid,forget plot]
  table[row sep=crcr]{%
0	0.9\\
0.25	0.910812461623111\\
0.5	0.920748818039132\\
0.75	0.929818478200031\\
1	0.93804545323369\\
1.25	0.945466326473351\\
1.5	0.952124332375219\\
1.75	0.95806817248844\\
2	0.963351564502009\\
2.25	0.968031079178228\\
2.5	0.972160277358748\\
2.75	0.975791576827072\\
3	0.978976665860834\\
3.25	0.981764898848435\\
3.5	0.984199451564601\\
3.75	0.986320470675039\\
4	0.98816580076278\\
4.25	0.989769920593186\\
4.5	0.991161782715511\\
4.75	0.992367685577523\\
5	0.993411887294857\\
5.25	0.994315940833207\\
5.5	0.995097534810212\\
5.75	0.995772554506192\\
6	0.996355490449292\\
6.25	0.996859041505354\\
6.5	0.997293495736756\\
6.75	0.997668039557587\\
7	0.997991003588974\\
7.25	0.998269633425031\\
7.5	0.998509756686829\\
7.75	0.998716561545868\\
8	0.998894737979855\\
8.25	0.999048348183151\\
8.5	0.99918064620545\\
8.75	0.999294524487324\\
9	0.999392593236276\\
9.25	0.99947710810268\\
9.5	0.999549871957848\\
9.75	0.999612485974338\\
10	0.999666393735973\\
};
\addplot [color=mycolor6,solid,forget plot]
  table[row sep=crcr]{%
0	1\\
0.25	1\\
0.5	1\\
0.75	1\\
1	1\\
1.25	1\\
1.5	1\\
1.75	1\\
2	1\\
2.25	1\\
2.5	1\\
2.75	1\\
3	1\\
3.25	1\\
3.5	1\\
3.75	1\\
4	1\\
4.25	1\\
4.5	1\\
4.75	1\\
5	1\\
5.25	1\\
5.5	1\\
5.75	1\\
6	1\\
6.25	1\\
6.5	1\\
6.75	1\\
7	1\\
7.25	1\\
7.5	1\\
7.75	1\\
8	1\\
8.25	1\\
8.5	1\\
8.75	1\\
9	1\\
9.25	1\\
9.5	1\\
9.75	1\\
10	1\\
};
\addplot [color=mycolor7,solid,forget plot]
  table[row sep=crcr]{%
0	1.1\\
0.25	1.08277735898578\\
0.5	1.06902047865195\\
0.75	1.05793216863778\\
1	1.04883023268385\\
1.25	1.04125225411422\\
1.5	1.03495559760308\\
1.75	1.02970131839883\\
2	1.02528691695878\\
2.25	1.02155711379417\\
2.5	1.01840424345887\\
2.75	1.01573401888912\\
3	1.01346476303214\\
3.25	1.01153028051212\\
3.5	1.00988116565486\\
3.75	1.00847425559899\\
4	1.007271550324\\
4.25	1.00624137026516\\
4.5	1.00535930852791\\
4.75	1.00460392921658\\
5	1.00395617375417\\
5.25	1.00339991214974\\
5.5	1.00292251900585\\
5.75	1.0025128595799\\
6	1.002160977736\\
6.25	1.00185837801204\\
6.5	1.00159835515083\\
6.75	1.0013749802588\\
7	1.00118293508212\\
7.25	1.00101766130067\\
7.5	1.00087554499794\\
7.75	1.0007533858918\\
8	1.00064830821532\\
8.25	1.00055784109076\\
8.5	1.00048002079272\\
8.75	1.00041310688219\\
9	1.00035553387081\\
9.25	1.000305954877\\
9.5	1.00026329804737\\
9.75	1.00022661295594\\
10	1.00019504425276\\
};
\addplot [color=mycolor1,solid,forget plot]
  table[row sep=crcr]{%
0	1.2\\
0.25	1.15844880064465\\
0.5	1.12824716188127\\
0.75	1.10639734713305\\
1	1.08900071911549\\
1.25	1.07400201240788\\
1.5	1.06191594736442\\
1.75	1.05209716892024\\
2	1.04399896079454\\
2.25	1.0372405030559\\
2.5	1.03160471774849\\
2.75	1.02688740577607\\
3	1.02291493488655\\
3.25	1.01955264331821\\
3.5	1.0167055918683\\
3.75	1.01429081831055\\
4	1.01223624182447\\
4.25	1.01048311479921\\
4.5	1.00898730132524\\
4.75	1.00771021307348\\
5	1.00661781356162\\
5.25	1.00568164305034\\
5.5	1.00487970555911\\
5.75	1.0041926680756\\
6	1.00360332284981\\
6.25	1.00309708290159\\
6.5	1.00266251174424\\
6.75	1.00228951822489\\
7	1.00196907373759\\
7.25	1.00169346752228\\
7.5	1.00145660829875\\
7.75	1.00125310850189\\
8	1.00107813378105\\
8.25	1.00092753858812\\
8.5	1.00079803468945\\
8.75	1.00068670973012\\
9	1.00059094617275\\
9.25	1.00050849442171\\
9.5	1.00043756614969\\
9.75	1.00037657622622\\
10	1.00032409870913\\
};
\addplot [color=mycolor2,solid,forget plot]
  table[row sep=crcr]{%
0	1.3\\
0.186065661593302	1.24338903910851\\
0.372131323186604	1.20206939477602\\
0.558196984779907	1.17192950031586\\
0.744262646373209	1.14758541231878\\
0.946296840123378	1.12461140032858\\
1.14833103387355	1.10607275497641\\
1.35036522762372	1.09093331004956\\
1.55239942137389	1.07833176949678\\
1.80239942137389	1.06541059277248\\
2.05239942137389	1.05491139166382\\
2.30239942137389	1.04631864246718\\
2.55239942137389	1.03919787914037\\
2.80239942137389	1.03323850223038\\
3.05239942137389	1.02825151299888\\
3.30239942137389	1.02406455723173\\
3.55239942137389	1.02053050406126\\
3.80239942137389	1.01753395280251\\
4.05239942137389	1.01499225959552\\
4.30239942137389	1.01283329459491\\
4.55239942137389	1.01099419232176\\
4.80239942137389	1.00942342439084\\
5.05239942137389	1.00808202642738\\
5.30239942137389	1.00693590019258\\
5.55239942137389	1.00595491138029\\
5.80239942137389	1.00511378701502\\
6.05239942137389	1.00439293165469\\
6.30239942137389	1.00377510845092\\
6.55239942137389	1.00324495816611\\
6.80239942137389	1.00278943953333\\
7.05239942137389	1.00239831085878\\
7.30239942137389	1.00206253064813\\
7.55239942137389	1.00177400430754\\
7.80239942137389	1.00152581259924\\
8.05239942137388	1.00131248463015\\
8.30239942137388	1.00112918011358\\
8.55239942137388	1.00097155424909\\
8.80239942137388	1.0008358794713\\
9.05239942137388	1.00071919765384\\
9.30239942137388	1.00061888837388\\
9.55239942137388	1.00053259601446\\
9.66429956603041	1.00049796519051\\
9.77619971068694	1.00046558908512\\
9.88809985534347	1.00043532085192\\
10	1.00040702249989\\
};
\addplot [color=mycolor3,solid,forget plot]
  table[row sep=crcr]{%
0	1.4\\
0.125594321575479	1.33826077376974\\
0.251188643150958	1.29101801216889\\
0.376782964726437	1.25425286179904\\
0.502377286301916	1.22396944421311\\
0.657624747429572	1.19242944322825\\
0.812872208557227	1.16685314634466\\
0.968119669684883	1.14580472586189\\
1.12336713081254	1.12811332558215\\
1.37117733415766	1.10500344868328\\
1.61898753750279	1.08693171350335\\
1.86679774084791	1.07265059620199\\
2.11460794419304	1.06104323507051\\
2.36460794419304	1.05132425337288\\
2.61460794419304	1.04332199962768\\
2.86460794419304	1.03669695799516\\
3.11460794419304	1.03116298964812\\
3.36460794419304	1.02650690390066\\
3.61460794419304	1.02258768737405\\
3.86460794419304	1.01928056319934\\
4.11460794419304	1.01647817215857\\
4.36460794419304	1.01409470335031\\
4.61460794419304	1.01206717882955\\
4.86460794419304	1.01034064308984\\
5.11460794419304	1.00886692867626\\
5.36460794419304	1.00760617006904\\
5.61460794419304	1.00652789454437\\
5.86460794419304	1.0056053839858\\
6.11460794419304	1.00481494575149\\
6.36460794419304	1.0041366060692\\
6.61460794419304	1.00355479395786\\
6.86460794419304	1.00305579274171\\
7.11460794419304	1.0026273559763\\
7.36460794419304	1.00225905610722\\
7.61460794419304	1.00194268059775\\
7.86460794419304	1.00167097282635\\
8.11460794419304	1.00143742900194\\
8.36460794419304	1.00123648114904\\
8.61460794419304	1.00106372000309\\
8.86460794419304	1.00091524251493\\
9.11460794419304	1.00078754303893\\
9.33595595814478	1.00068939999145\\
9.55730397209652	1.0006035119108\\
9.77865198604826	1.0005283633729\\
10	1.00046258593362\\
};
\addplot [color=mycolor4,solid,forget plot]
  table[row sep=crcr]{%
0	1.5\\
0.0913413247821665	1.43322729477461\\
0.182682649564333	1.38078304011729\\
0.2740239743465	1.33874647243676\\
0.365365299128666	1.30364779309686\\
0.488366918930436	1.2640510481543\\
0.611368538732205	1.23186721967296\\
0.734370158533974	1.20527338886581\\
0.857371778335744	1.18279830127887\\
1.05294653628572	1.15331599649886\\
1.2485212942357	1.12994647833736\\
1.44409605218568	1.11116263295264\\
1.63967081013566	1.09565174666543\\
1.88967081013566	1.07931829750749\\
2.13967081013566	1.06622604275208\\
2.38967081013566	1.05564004592219\\
2.63967081013566	1.04693427080318\\
2.88967081013566	1.0396797481757\\
3.13967081013566	1.03364338970062\\
3.38967081013566	1.02860021529076\\
3.63967081013566	1.0243593243394\\
3.88967081013566	1.02077367858892\\
4.13967081013566	1.01774064569229\\
4.38967081013566	1.01517043326074\\
4.63967081013566	1.01298516856112\\
4.88967081013566	1.01112160058504\\
5.13967081013566	1.00953239914644\\
5.38967081013566	1.00817620309136\\
5.63967081013566	1.00701657207103\\
5.88967081013566	1.00602309153518\\
6.13967081013566	1.00517230007631\\
6.38967081013566	1.00444358622341\\
6.63967081013566	1.00381861658518\\
6.88967081013566	1.00328186520356\\
7.13967081013566	1.00282117164912\\
7.38967081013566	1.00242580920965\\
7.63967081013566	1.00208618427488\\
7.88967081013566	1.00179410792783\\
8.13967081013566	1.00154311463693\\
8.38967081013566	1.00132748683429\\
8.63967081013566	1.00114209513573\\
8.88967081013566	1.00098254228212\\
9.13967081013566	1.00084534139987\\
9.38967081013566	1.00072740451117\\
9.63967081013566	1.0006259565585\\
9.72975310760174	1.00059296983052\\
9.81983540506783	1.00056172409875\\
9.90991770253391	1.00053212731968\\
10	1.00050409205657\\
};
\end{axis}
\end{tikzpicture}%
\end{document}
\caption{Simulation of the simplified system described by equation~\ref{eq:simple}.}
\label{fig:simpleSim}
\end{figure}
Figure~\ref{fig:simpleSim} shows simulation results produced by a Runge-Kutta type numerical integration routine. The fixed point positions that where read off from the simplified system equation are confirmed by the results to be at $x_1 = 0, \; x_2 = a = 0.4, \; x_3 = 1$. Furthermore the fixed points show the predicted characteristics.

\subsection{Two-dimensional approach}
Once more the analysis starts with the computation of the fixed point locations. Setting the system equations to zero leads to:
\begin{align}
0 &= x(x-a)(1-x) - bxy \\
\label{eq:topDzero}
0 &= xy - cy.
\end{align}
Starting from the top equation~\ref{eq:topDzero} first x may be factored out:
\begin{equation}
0 = x[(x-a)(1-x) - by].
\end{equation}
Therefore $x_1 = 0$. In order to obtain the remaining zeros the equation:
\begin{align}
0 = (x - a)(1 - x) - by
\end{align} 
has to be solved. After factoring out the brackets the pq-Formula is applicable thus the following expression is obtained:
\begin{equation}
x_{2,3} = \frac{1+a}{2} \pm \sqrt{\frac{(1+a)^2}{4} - (a + by)}.
\label{eq:pq}
\end{equation}
Which will be simplified further once more is known about y. To finish the quest for the fixed points x values y is factored out in the second equation:
\begin{equation}
0 = y (x - c).
\label{eq:facDzero}
\end{equation}
The equation~\ref{eq:facDzero} is zero when $x_4 = c$. Which is the missing x component. Looking at y, $y_1 = 0$ is quickly read off from ~\ref{eq:facDzero}. Turning back to equation~\ref{eq:topDzero} and solving for $y$ while assuming $x \ne 0$ gives:
\begin{equation}
y_2 = \frac{(x - a)(1 - x)}{b}
\end{equation}
At this point two steady state solutions at $\begin{pmatrix} x_1 & y_1 \end{pmatrix}^T = \begin{pmatrix} 0 & 0 \end{pmatrix}^T$ and $\begin{pmatrix} x_4 & y_2 \end{pmatrix}^T = \begin{pmatrix} c & \frac{(c -a )(1-c)}{b} \end{pmatrix}^T$ are already known. Using $y_1$ again equation \ref{eq:pq} can be simplified further after plugging in and factoring out one obtains:
\begin{align}
x_{2/3} = \frac{1+a}{2} \pm \sqrt{\frac{1 - 2a + a^2}{4} } \\
x_{2/3} = \frac{1+a}{2} \pm \sqrt{(\frac{1 - a}{2})^2 } \\
x_{2/3} = \frac{1+a}{2} \pm \frac{1 - a}{2} \\
\Rightarrow x_2 = 1 \wedge x_3 = a
\end{align}
Now two more fixed points are known $\begin{pmatrix} x_2 & y_1 \end{pmatrix}^T = \begin{pmatrix} 1 & 0 \end{pmatrix}^T$ and 
$\begin{pmatrix} x_3 & y_1 \end{pmatrix}^T = \begin{pmatrix} a & 0 \end{pmatrix}^T$. 

\begin{table}
\centering
\begin{tabular}{|c|}
\hline
$\mathbf{x}_1^* = \begin{pmatrix} x_1 \\ y_1 \end{pmatrix} = \begin{pmatrix} 0 \\ 0 \end{pmatrix}$ \\
\hline
$\mathbf{x}_2^* =\begin{pmatrix} x_4 \\ y_2 \end{pmatrix} = \begin{pmatrix} c \\ \frac{(c -a )(1-c)}{b} \end{pmatrix}$\\
\hline
$\mathbf{x}_3^* =\begin{pmatrix} x_2 \\ y_1 \end{pmatrix} = \begin{pmatrix} 1 \\ 0 \end{pmatrix}$\\
\hline
$\mathbf{x}_4^* =\begin{pmatrix} x_3 \\ y_1 \end{pmatrix} = \begin{pmatrix} a \\ 0 \end{pmatrix}$\\
\hline
\end{tabular}
\caption{Fixed point positions.}
\end{table}

Next the obtained points will be classified according to their properties. Starting from the system equations after factoring out the Jacobian is computed:
\begin{equation}
J = \begin{pmatrix}
-3x^2 + 2x + 2xa - a - yb & - bx \\
y & x-c \\
\end{pmatrix}
\end{equation}
Linear analysis proceeds by plugging the fixed points into the Jacobian and compute the trace $\tau$ as well as the determinant $\triangle$. For the first fixed point $\mathbf{x}_1^*$ this gives:
\begin{equation}
J(\mathbf{x}_1^*) = \begin{pmatrix}
-a & 0 \\
0  & -c \end{pmatrix}.
\end{equation}
Therefore the trace and determinant are $\tau_1 = -a-c \wedge \triangle_1 = ac$. Thus this node is a saddle point if $c < 0$, if $c > 0$ it is stable if $-a < c$ and spirals if $a < c$.
The second fixed point $\mathbf{x}_2^*$ has the Jacobian:
\begin{equation}
J(\mathbf{x_2^*}) = \begin{pmatrix}
c(c+a-2c) & - bc \\
(c-a)(1-c)/b &	0
\end{pmatrix})
\end{equation}
With the determinant and trace $\tau_2 = c(1+a-2c) \wedge \triangle_2 = c(c-a)(1-c)$. From these two expressions it is possible to deduce, that if $c>0$, $\mathbf{x_2^*}$ is a saddle point if additionally, $c > a \wedge c < 1$. If that is not the case then the determinant is positive, now the trace determines stability. $\tau_2 < 0$ is the case if $\frac{1+a}{2} < c$. However it $c < 0$ then the determinant will always be negative, making $\mathbf{x_2^*}$ a saddle point.















\end{document}