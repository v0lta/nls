
% ---------- Titelblad Masterproef Faculteit Wetenschappen -----------
% Dit document is opgesteld voor compilatie met pdflatex.  Indien je
% wilt compileren met latex naar dvi/ps, dien je de figuren naar
% (e)ps-formaat om te zetten.
%                           -- december 2012
% -------------------------------------------------------------------
\RequirePackage{fix-cm}
\documentclass[12pt,a4paper,oneside]{article}

% --------------------- In te laden pakketten -----------------------
% Deze kan je eventueel toevoegen aan de pakketten die je al inlaadt
% als je dit titelblad integreert met de rest van thesis.
% -------------------------------------------------------------------
\usepackage{graphicx,xcolor,textpos}
\usepackage{helvet}

% -------------------- Pagina-instellingen --------------------------
% Indien je deze wijzigt, zal het titelblad ook wijzigen.  Dit dien je
% dan manueel aan te passen.
% --------------------------------------------------------------------

\topmargin -10mm
\textwidth 160truemm
\textheight 240truemm
\oddsidemargin 0mm
\evensidemargin 0mm

% ------------------- textpos-instellingen ---------------------------
% Enkele andere instellingen voor het voorblad.
% --------------------------------------------------------------------

\definecolor{green}{RGB}{172,196,0}
\definecolor{bluetitle}{RGB}{29,141,176}
\definecolor{blueaff}{RGB}{0,0,128}
\definecolor{blueline}{RGB}{82,189,236}
\setlength{\TPHorizModule}{1mm}
\setlength{\TPVertModule}{1mm}



%----------------------- Custom stuff -------------------------------

\graphicspath{./}
\usepackage{makeidx}
\index{hoofd}
\makeindex
\usepackage{amsmath}
\usepackage[english]{babel}
\usepackage{hyperref}
\usepackage{listings}
\usepackage{eurosym}


%------------------------ Plot packages ----------------------------
\usepackage{tikz}
\usepackage{pgfplots}
\usepackage{pgf}
\usepackage{units}
\usepackage{metalogo}
\usepackage{graphicx}
\usepackage{caption}
\usepackage{subcaption}
\usepackage{standalone}


%----------------------Title etc----------------------------------
\title{Non linear Systems: Laser Model Analysis}
\author{Moritz Wolter}
\date{\today}


% --------------------------------------------------------------------

\begin{document}

\title{Exercise 1 Laser Model}
\author{Moritz Wolter}

\maketitle

\section{The equations}
\begin{eqnarray}
\dot{n} = GnN - kn. \\
\dot{N} = -GnN - fN + p.
\end{eqnarray}


\section{First Order Analysis}
Assume that the number of excited atoms remains quasi-static. 
\begin{eqnarray}
\dot{N} \approx 0. \\
\Rightarrow \dot{n} = Gn \cdot \frac{p}{Gn + f} - kn. \label{eq:oneParam}\\
\end{eqnarray}
\subsection{Linear Stability analysis}
In the two dimensional case fixed points are at the intersections of the first derivative with the real axis. Thus the can be found by solving $\dot{n}=0$. Leading to the problem:
\begin{equation}
0 = n ( \frac{Gp}{Gn+f} - k)
\end{equation}
Which has the zeros $0$ and $\frac{p}{k} - \frac{f}{G}$. To learn more about the nature of the fixed points one has to set the derivative of \ref{eq:oneParam} to zero. Using the Quotient rule the expression: 
\begin{equation}
\ddot{n} = \frac{Gfp}{(Gn+f)^2} - k   
\label{eq:secondDiv} 
\end{equation}
is obtained. A fixed point is stable if the second derivative is negative. Likewise it is unstable if the second derivative is positive \footnote{Strogatz, p.25}.
Evaluating the second derivative for the fixed point a zero leads to $\ddot{n}(0) = \frac{Gp}{f} - k$ which leads to a values of $p_c = \frac{fk}{G}$. Thus the fixed points becomes unstable if $p > p_c$.
%TODO: TIMESCALE p 66.

\begin{figure}
% This file was created by matlab2tikz.
% Minimal pgfplots version: 1.3
%
%The latest updates can be retrieved from
%  http://www.mathworks.com/matlabcentral/fileexchange/22022-matlab2tikz
%where you can also make suggestions and rate matlab2tikz.
%
\documentclass[tikz]{standalone}
\usepackage{pgfplots}
\usepackage{grffile}
\pgfplotsset{compat=newest}
\usetikzlibrary{plotmarks}
\usepackage{amsmath}

\begin{document}
\definecolor{mycolor1}{rgb}{0.00000,0.44700,0.74100}%
\definecolor{mycolor2}{rgb}{0.85000,0.32500,0.09800}%
\definecolor{mycolor3}{rgb}{0.92900,0.69400,0.12500}%
\definecolor{mycolor4}{rgb}{0.49400,0.18400,0.55600}%
\definecolor{mycolor5}{rgb}{0.46600,0.67400,0.18800}%
\definecolor{mycolor6}{rgb}{0.30100,0.74500,0.93300}%
\definecolor{mycolor7}{rgb}{0.63500,0.07800,0.18400}%
%
\begin{tikzpicture}

\begin{axis}[%
width=2.001474in,
height=2.582604in,
at={(1.165937in,0.483542in)},
scale only axis,
xmin=0,
xmax=2,
xlabel={p},
ymin=-0.2,
ymax=1,
ylabel={zero}
]
\addplot [color=mycolor1,only marks,mark=asterisk,mark options={solid},forget plot]
  table[row sep=crcr]{%
0	-5.55111512312578e-17\\
0.1	8.74787441744869e-20\\
0.2	-3.25739628826474e-22\\
0.3	-1.2412215694122e-16\\
0.4	-3.80323207149667e-19\\
0.5	-1.42196127852538e-18\\
0.6	-3.00559044891048e-17\\
0.7	1.9283552904372e-18\\
0.8	1.578393658814e-18\\
0.9	2.626844099944e-23\\
1	5.06656246881054e-18\\
1.1	0.1\\
1.2	0.2\\
1.3	0.3\\
1.4	0.4\\
1.5	0.5\\
1.6	0.6\\
1.7	0.7\\
1.8	0.8\\
1.9	0.9\\
2	1\\
};
\end{axis}

\begin{axis}[%
width=2.001474in,
height=2.582604in,
at={(5.115245in,0.483542in)},
scale only axis,
xmin=-0.167093683858749,
xmax=1.33006572351565,
xlabel={n},
ymin=-0.117242428339643,
ymax=0.176632639264467,
ylabel={$\dot{n}$},
ytick={0.1,0.15,0,-0.1}
]
\addplot [color=mycolor1,solid,forget plot]
  table[row sep=crcr]{%
0	0\\
0.01	-0.01\\
0.02	-0.02\\
0.03	-0.03\\
0.04	-0.04\\
0.05	-0.05\\
0.06	-0.06\\
0.07	-0.07\\
0.08	-0.08\\
0.09	-0.09\\
0.1	-0.1\\
0.11	-0.11\\
0.12	-0.12\\
0.13	-0.13\\
0.14	-0.14\\
0.15	-0.15\\
0.16	-0.16\\
0.17	-0.17\\
0.18	-0.18\\
0.19	-0.19\\
0.2	-0.2\\
0.21	-0.21\\
0.22	-0.22\\
0.23	-0.23\\
0.24	-0.24\\
0.25	-0.25\\
0.26	-0.26\\
0.27	-0.27\\
0.28	-0.28\\
0.29	-0.29\\
0.3	-0.3\\
0.31	-0.31\\
0.32	-0.32\\
0.33	-0.33\\
0.34	-0.34\\
0.35	-0.35\\
0.36	-0.36\\
0.37	-0.37\\
0.38	-0.38\\
0.39	-0.39\\
0.4	-0.4\\
0.41	-0.41\\
0.42	-0.42\\
0.43	-0.43\\
0.44	-0.44\\
0.45	-0.45\\
0.46	-0.46\\
0.47	-0.47\\
0.48	-0.48\\
0.49	-0.49\\
0.5	-0.5\\
0.51	-0.51\\
0.52	-0.52\\
0.53	-0.53\\
0.54	-0.54\\
0.55	-0.55\\
0.56	-0.56\\
0.57	-0.57\\
0.58	-0.58\\
0.59	-0.59\\
0.6	-0.6\\
0.61	-0.61\\
0.62	-0.62\\
0.63	-0.63\\
0.64	-0.64\\
0.65	-0.65\\
0.66	-0.66\\
0.67	-0.67\\
0.68	-0.68\\
0.69	-0.69\\
0.7	-0.7\\
0.71	-0.71\\
0.72	-0.72\\
0.73	-0.73\\
0.74	-0.74\\
0.75	-0.75\\
0.76	-0.76\\
0.77	-0.77\\
0.78	-0.78\\
0.79	-0.79\\
0.8	-0.8\\
0.81	-0.81\\
0.82	-0.82\\
0.83	-0.83\\
0.84	-0.84\\
0.85	-0.85\\
0.86	-0.86\\
0.87	-0.87\\
0.88	-0.88\\
0.89	-0.89\\
0.9	-0.9\\
0.91	-0.91\\
0.92	-0.92\\
0.93	-0.93\\
0.94	-0.94\\
0.95	-0.95\\
0.96	-0.96\\
0.97	-0.97\\
0.98	-0.98\\
0.99	-0.99\\
1	-1\\
1.01	-1.01\\
1.02	-1.02\\
1.03	-1.03\\
1.04	-1.04\\
1.05	-1.05\\
1.06	-1.06\\
1.07	-1.07\\
1.08	-1.08\\
1.09	-1.09\\
1.1	-1.1\\
1.11	-1.11\\
1.12	-1.12\\
1.13	-1.13\\
1.14	-1.14\\
1.15	-1.15\\
1.16	-1.16\\
1.17	-1.17\\
1.18	-1.18\\
1.19	-1.19\\
1.2	-1.2\\
1.21	-1.21\\
1.22	-1.22\\
1.23	-1.23\\
1.24	-1.24\\
1.25	-1.25\\
1.26	-1.26\\
1.27	-1.27\\
1.28	-1.28\\
1.29	-1.29\\
1.3	-1.3\\
1.31	-1.31\\
1.32	-1.32\\
1.33	-1.33\\
1.34	-1.34\\
1.35	-1.35\\
1.36	-1.36\\
1.37	-1.37\\
1.38	-1.38\\
1.39	-1.39\\
1.4	-1.4\\
1.41	-1.41\\
1.42	-1.42\\
1.43	-1.43\\
1.44	-1.44\\
1.45	-1.45\\
1.46	-1.46\\
1.47	-1.47\\
1.48	-1.48\\
1.49	-1.49\\
1.5	-1.5\\
1.51	-1.51\\
1.52	-1.52\\
1.53	-1.53\\
1.54	-1.54\\
1.55	-1.55\\
1.56	-1.56\\
1.57	-1.57\\
1.58	-1.58\\
1.59	-1.59\\
1.6	-1.6\\
1.61	-1.61\\
1.62	-1.62\\
1.63	-1.63\\
1.64	-1.64\\
1.65	-1.65\\
1.66	-1.66\\
1.67	-1.67\\
1.68	-1.68\\
1.69	-1.69\\
1.7	-1.7\\
1.71	-1.71\\
1.72	-1.72\\
1.73	-1.73\\
1.74	-1.74\\
1.75	-1.75\\
1.76	-1.76\\
1.77	-1.77\\
1.78	-1.78\\
1.79	-1.79\\
1.8	-1.8\\
1.81	-1.81\\
1.82	-1.82\\
1.83	-1.83\\
1.84	-1.84\\
1.85	-1.85\\
1.86	-1.86\\
1.87	-1.87\\
1.88	-1.88\\
1.89	-1.89\\
1.9	-1.9\\
1.91	-1.91\\
1.92	-1.92\\
1.93	-1.93\\
1.94	-1.94\\
1.95	-1.95\\
1.96	-1.96\\
1.97	-1.97\\
1.98	-1.98\\
1.99	-1.99\\
2	-2\\
2.01	-2.01\\
2.02	-2.02\\
2.03	-2.03\\
2.04	-2.04\\
2.05	-2.05\\
2.06	-2.06\\
2.07	-2.07\\
2.08	-2.08\\
2.09	-2.09\\
2.1	-2.1\\
2.11	-2.11\\
2.12	-2.12\\
2.13	-2.13\\
2.14	-2.14\\
2.15	-2.15\\
2.16	-2.16\\
2.17	-2.17\\
2.18	-2.18\\
2.19	-2.19\\
2.2	-2.2\\
2.21	-2.21\\
2.22	-2.22\\
2.23	-2.23\\
2.24	-2.24\\
2.25	-2.25\\
2.26	-2.26\\
2.27	-2.27\\
2.28	-2.28\\
2.29	-2.29\\
2.3	-2.3\\
2.31	-2.31\\
2.32	-2.32\\
2.33	-2.33\\
2.34	-2.34\\
2.35	-2.35\\
2.36	-2.36\\
2.37	-2.37\\
2.38	-2.38\\
2.39	-2.39\\
2.4	-2.4\\
2.41	-2.41\\
2.42	-2.42\\
2.43	-2.43\\
2.44	-2.44\\
2.45	-2.45\\
2.46	-2.46\\
2.47	-2.47\\
2.48	-2.48\\
2.49	-2.49\\
2.5	-2.5\\
2.51	-2.51\\
2.52	-2.52\\
2.53	-2.53\\
2.54	-2.54\\
2.55	-2.55\\
2.56	-2.56\\
2.57	-2.57\\
2.58	-2.58\\
2.59	-2.59\\
2.6	-2.6\\
2.61	-2.61\\
2.62	-2.62\\
2.63	-2.63\\
2.64	-2.64\\
2.65	-2.65\\
2.66	-2.66\\
2.67	-2.67\\
2.68	-2.68\\
2.69	-2.69\\
2.7	-2.7\\
2.71	-2.71\\
2.72	-2.72\\
2.73	-2.73\\
2.74	-2.74\\
2.75	-2.75\\
2.76	-2.76\\
2.77	-2.77\\
2.78	-2.78\\
2.79	-2.79\\
2.8	-2.8\\
2.81	-2.81\\
2.82	-2.82\\
2.83	-2.83\\
2.84	-2.84\\
2.85	-2.85\\
2.86	-2.86\\
2.87	-2.87\\
2.88	-2.88\\
2.89	-2.89\\
2.9	-2.9\\
2.91	-2.91\\
2.92	-2.92\\
2.93	-2.93\\
2.94	-2.94\\
2.95	-2.95\\
2.96	-2.96\\
2.97	-2.97\\
2.98	-2.98\\
2.99	-2.99\\
3	-3\\
3.01	-3.01\\
3.02	-3.02\\
3.03	-3.03\\
3.04	-3.04\\
3.05	-3.05\\
3.06	-3.06\\
3.07	-3.07\\
3.08	-3.08\\
3.09	-3.09\\
3.1	-3.1\\
3.11	-3.11\\
3.12	-3.12\\
3.13	-3.13\\
3.14	-3.14\\
3.15	-3.15\\
3.16	-3.16\\
3.17	-3.17\\
3.18	-3.18\\
3.19	-3.19\\
3.2	-3.2\\
3.21	-3.21\\
3.22	-3.22\\
3.23	-3.23\\
3.24	-3.24\\
3.25	-3.25\\
3.26	-3.26\\
3.27	-3.27\\
3.28	-3.28\\
3.29	-3.29\\
3.3	-3.3\\
3.31	-3.31\\
3.32	-3.32\\
3.33	-3.33\\
3.34	-3.34\\
3.35	-3.35\\
3.36	-3.36\\
3.37	-3.37\\
3.38	-3.38\\
3.39	-3.39\\
3.4	-3.4\\
3.41	-3.41\\
3.42	-3.42\\
3.43	-3.43\\
3.44	-3.44\\
3.45	-3.45\\
3.46	-3.46\\
3.47	-3.47\\
3.48	-3.48\\
3.49	-3.49\\
3.5	-3.5\\
3.51	-3.51\\
3.52	-3.52\\
3.53	-3.53\\
3.54	-3.54\\
3.55	-3.55\\
3.56	-3.56\\
3.57	-3.57\\
3.58	-3.58\\
3.59	-3.59\\
3.6	-3.6\\
3.61	-3.61\\
3.62	-3.62\\
3.63	-3.63\\
3.64	-3.64\\
3.65	-3.65\\
3.66	-3.66\\
3.67	-3.67\\
3.68	-3.68\\
3.69	-3.69\\
3.7	-3.7\\
3.71	-3.71\\
3.72	-3.72\\
3.73	-3.73\\
3.74	-3.74\\
3.75	-3.75\\
3.76	-3.76\\
3.77	-3.77\\
3.78	-3.78\\
3.79	-3.79\\
3.8	-3.8\\
3.81	-3.81\\
3.82	-3.82\\
3.83	-3.83\\
3.84	-3.84\\
3.85	-3.85\\
3.86	-3.86\\
3.87	-3.87\\
3.88	-3.88\\
3.89	-3.89\\
3.9	-3.9\\
3.91	-3.91\\
3.92	-3.92\\
3.93	-3.93\\
3.94	-3.94\\
3.95	-3.95\\
3.96	-3.96\\
3.97	-3.97\\
3.98	-3.98\\
3.99	-3.99\\
4	-4\\
4.01	-4.01\\
4.02	-4.02\\
4.03	-4.03\\
4.04	-4.04\\
4.05	-4.05\\
4.06	-4.06\\
4.07	-4.07\\
4.08	-4.08\\
4.09	-4.09\\
4.1	-4.1\\
4.11	-4.11\\
4.12	-4.12\\
4.13	-4.13\\
4.14	-4.14\\
4.15	-4.15\\
4.16	-4.16\\
4.17	-4.17\\
4.18	-4.18\\
4.19	-4.19\\
4.2	-4.2\\
4.21	-4.21\\
4.22	-4.22\\
4.23	-4.23\\
4.24	-4.24\\
4.25	-4.25\\
4.26	-4.26\\
4.27	-4.27\\
4.28	-4.28\\
4.29	-4.29\\
4.3	-4.3\\
4.31	-4.31\\
4.32	-4.32\\
4.33	-4.33\\
4.34	-4.34\\
4.35	-4.35\\
4.36	-4.36\\
4.37	-4.37\\
4.38	-4.38\\
4.39	-4.39\\
4.4	-4.4\\
4.41	-4.41\\
4.42	-4.42\\
4.43	-4.43\\
4.44	-4.44\\
4.45	-4.45\\
4.46	-4.46\\
4.47	-4.47\\
4.48	-4.48\\
4.49	-4.49\\
4.5	-4.5\\
4.51	-4.51\\
4.52	-4.52\\
4.53	-4.53\\
4.54	-4.54\\
4.55	-4.55\\
4.56	-4.56\\
4.57	-4.57\\
4.58	-4.58\\
4.59	-4.59\\
4.6	-4.6\\
4.61	-4.61\\
4.62	-4.62\\
4.63	-4.63\\
4.64	-4.64\\
4.65	-4.65\\
4.66	-4.66\\
4.67	-4.67\\
4.68	-4.68\\
4.69	-4.69\\
4.7	-4.7\\
4.71	-4.71\\
4.72	-4.72\\
4.73	-4.73\\
4.74	-4.74\\
4.75	-4.75\\
4.76	-4.76\\
4.77	-4.77\\
4.78	-4.78\\
4.79	-4.79\\
4.8	-4.8\\
4.81	-4.81\\
4.82	-4.82\\
4.83	-4.83\\
4.84	-4.84\\
4.85	-4.85\\
4.86	-4.86\\
4.87	-4.87\\
4.88	-4.88\\
4.89	-4.89\\
4.9	-4.9\\
4.91	-4.91\\
4.92	-4.92\\
4.93	-4.93\\
4.94	-4.94\\
4.95	-4.95\\
4.96	-4.96\\
4.97	-4.97\\
4.98	-4.98\\
4.99	-4.99\\
5	-5\\
};
\addplot [color=mycolor2,solid,forget plot]
  table[row sep=crcr]{%
0	0\\
0.01	-0.00900990099009901\\
0.02	-0.0180392156862745\\
0.03	-0.0270873786407767\\
0.04	-0.0361538461538462\\
0.05	-0.0452380952380952\\
0.06	-0.0543396226415094\\
0.07	-0.0634579439252337\\
0.08	-0.0725925925925926\\
0.09	-0.081743119266055\\
0.1	-0.0909090909090909\\
0.11	-0.10009009009009\\
0.12	-0.109285714285714\\
0.13	-0.118495575221239\\
0.14	-0.127719298245614\\
0.15	-0.13695652173913\\
0.16	-0.146206896551724\\
0.17	-0.155470085470085\\
0.18	-0.164745762711864\\
0.19	-0.174033613445378\\
0.2	-0.183333333333333\\
0.21	-0.192644628099174\\
0.22	-0.201967213114754\\
0.23	-0.21130081300813\\
0.24	-0.220645161290323\\
0.25	-0.23\\
0.26	-0.239365079365079\\
0.27	-0.248740157480315\\
0.28	-0.258125\\
0.29	-0.267519379844961\\
0.3	-0.276923076923077\\
0.31	-0.286335877862595\\
0.32	-0.295757575757576\\
0.33	-0.305187969924812\\
0.34	-0.314626865671642\\
0.35	-0.324074074074074\\
0.36	-0.333529411764706\\
0.37	-0.342992700729927\\
0.38	-0.352463768115942\\
0.39	-0.361942446043165\\
0.4	-0.371428571428571\\
0.41	-0.380921985815603\\
0.42	-0.390422535211268\\
0.43	-0.39993006993007\\
0.44	-0.409444444444444\\
0.45	-0.418965517241379\\
0.46	-0.428493150684932\\
0.47	-0.438027210884354\\
0.48	-0.447567567567568\\
0.49	-0.457114093959732\\
0.5	-0.466666666666667\\
0.51	-0.476225165562914\\
0.52	-0.485789473684211\\
0.53	-0.495359477124183\\
0.54	-0.504935064935065\\
0.55	-0.514516129032258\\
0.56	-0.524102564102564\\
0.57	-0.533694267515924\\
0.58	-0.543291139240506\\
0.59	-0.552893081761006\\
0.6	-0.5625\\
0.61	-0.572111801242236\\
0.62	-0.581728395061728\\
0.63	-0.591349693251534\\
0.64	-0.600975609756098\\
0.65	-0.610606060606061\\
0.66	-0.620240963855422\\
0.67	-0.629880239520958\\
0.68	-0.63952380952381\\
0.69	-0.649171597633136\\
0.7	-0.658823529411765\\
0.71	-0.668479532163743\\
0.72	-0.678139534883721\\
0.73	-0.687803468208092\\
0.74	-0.697471264367816\\
0.75	-0.707142857142857\\
0.76	-0.716818181818182\\
0.77	-0.726497175141243\\
0.78	-0.736179775280899\\
0.79	-0.745865921787709\\
0.8	-0.755555555555556\\
0.81	-0.76524861878453\\
0.82	-0.774945054945055\\
0.83	-0.784644808743169\\
0.84	-0.794347826086957\\
0.85	-0.804054054054054\\
0.86	-0.813763440860215\\
0.87	-0.823475935828877\\
0.88	-0.833191489361702\\
0.89	-0.842910052910053\\
0.9	-0.852631578947368\\
0.91	-0.862356020942408\\
0.92	-0.872083333333333\\
0.93	-0.881813471502591\\
0.94	-0.891546391752577\\
0.95	-0.901282051282051\\
0.96	-0.911020408163265\\
0.97	-0.920761421319797\\
0.98	-0.930505050505051\\
0.99	-0.940251256281407\\
1	-0.95\\
1.01	-0.959751243781095\\
1.02	-0.969504950495049\\
1.03	-0.979261083743842\\
1.04	-0.989019607843137\\
1.05	-0.998780487804878\\
1.06	-1.00854368932039\\
1.07	-1.01830917874396\\
1.08	-1.02807692307692\\
1.09	-1.03784688995215\\
1.1	-1.04761904761905\\
1.11	-1.05739336492891\\
1.12	-1.06716981132075\\
1.13	-1.07694835680751\\
1.14	-1.08672897196262\\
1.15	-1.09651162790698\\
1.16	-1.1062962962963\\
1.17	-1.11608294930876\\
1.18	-1.12587155963303\\
1.19	-1.13566210045662\\
1.2	-1.14545454545455\\
1.21	-1.15524886877828\\
1.22	-1.16504504504504\\
1.23	-1.17484304932735\\
1.24	-1.18464285714286\\
1.25	-1.19444444444444\\
1.26	-1.20424778761062\\
1.27	-1.21405286343612\\
1.28	-1.22385964912281\\
1.29	-1.23366812227074\\
1.3	-1.24347826086957\\
1.31	-1.25329004329004\\
1.32	-1.26310344827586\\
1.33	-1.27291845493562\\
1.34	-1.28273504273504\\
1.35	-1.29255319148936\\
1.36	-1.30237288135593\\
1.37	-1.312194092827\\
1.38	-1.32201680672269\\
1.39	-1.3318410041841\\
1.4	-1.34166666666667\\
1.41	-1.35149377593361\\
1.42	-1.36132231404959\\
1.43	-1.37115226337449\\
1.44	-1.38098360655738\\
1.45	-1.39081632653061\\
1.46	-1.40065040650407\\
1.47	-1.41048582995951\\
1.48	-1.42032258064516\\
1.49	-1.43016064257028\\
1.5	-1.44\\
1.51	-1.4498406374502\\
1.52	-1.45968253968254\\
1.53	-1.4695256916996\\
1.54	-1.47937007874016\\
1.55	-1.48921568627451\\
1.56	-1.4990625\\
1.57	-1.50891050583658\\
1.58	-1.51875968992248\\
1.59	-1.52861003861004\\
1.6	-1.53846153846154\\
1.61	-1.54831417624521\\
1.62	-1.5581679389313\\
1.63	-1.56802281368821\\
1.64	-1.57787878787879\\
1.65	-1.5877358490566\\
1.66	-1.59759398496241\\
1.67	-1.6074531835206\\
1.68	-1.61731343283582\\
1.69	-1.62717472118959\\
1.7	-1.63703703703704\\
1.71	-1.64690036900369\\
1.72	-1.65676470588235\\
1.73	-1.66663003663004\\
1.74	-1.67649635036496\\
1.75	-1.68636363636364\\
1.76	-1.69623188405797\\
1.77	-1.70610108303249\\
1.78	-1.71597122302158\\
1.79	-1.72584229390681\\
1.8	-1.73571428571429\\
1.81	-1.7455871886121\\
1.82	-1.7554609929078\\
1.83	-1.76533568904594\\
1.84	-1.77521126760563\\
1.85	-1.78508771929825\\
1.86	-1.79496503496504\\
1.87	-1.80484320557491\\
1.88	-1.81472222222222\\
1.89	-1.82460207612457\\
1.9	-1.83448275862069\\
1.91	-1.84436426116839\\
1.92	-1.85424657534247\\
1.93	-1.86412969283276\\
1.94	-1.87401360544218\\
1.95	-1.88389830508475\\
1.96	-1.89378378378378\\
1.97	-1.90367003367003\\
1.98	-1.91355704697987\\
1.99	-1.92344481605351\\
2	-1.93333333333333\\
2.01	-1.94322259136213\\
2.02	-1.95311258278146\\
2.03	-1.96300330033003\\
2.04	-1.97289473684211\\
2.05	-1.9827868852459\\
2.06	-1.99267973856209\\
2.07	-2.00257328990228\\
2.08	-2.01246753246753\\
2.09	-2.02236245954693\\
2.1	-2.03225806451613\\
2.11	-2.04215434083601\\
2.12	-2.05205128205128\\
2.13	-2.06194888178914\\
2.14	-2.07184713375796\\
2.15	-2.08174603174603\\
2.16	-2.09164556962025\\
2.17	-2.10154574132492\\
2.18	-2.1114465408805\\
2.19	-2.12134796238244\\
2.2	-2.13125\\
2.21	-2.14115264797508\\
2.22	-2.15105590062112\\
2.23	-2.16095975232198\\
2.24	-2.17086419753086\\
2.25	-2.18076923076923\\
2.26	-2.19067484662577\\
2.27	-2.20058103975535\\
2.28	-2.21048780487805\\
2.29	-2.22039513677812\\
2.3	-2.23030303030303\\
2.31	-2.24021148036254\\
2.32	-2.25012048192771\\
2.33	-2.26003003003003\\
2.34	-2.26994011976048\\
2.35	-2.27985074626866\\
2.36	-2.2897619047619\\
2.37	-2.29967359050445\\
2.38	-2.30958579881657\\
2.39	-2.31949852507375\\
2.4	-2.32941176470588\\
2.41	-2.33932551319648\\
2.42	-2.34923976608187\\
2.43	-2.35915451895044\\
2.44	-2.36906976744186\\
2.45	-2.37898550724638\\
2.46	-2.38890173410405\\
2.47	-2.39881844380403\\
2.48	-2.40873563218391\\
2.49	-2.41865329512894\\
2.5	-2.42857142857143\\
2.51	-2.43849002849003\\
2.52	-2.44840909090909\\
2.53	-2.45832861189802\\
2.54	-2.46824858757062\\
2.55	-2.47816901408451\\
2.56	-2.48808988764045\\
2.57	-2.49801120448179\\
2.58	-2.50793296089385\\
2.59	-2.51785515320334\\
2.6	-2.52777777777778\\
2.61	-2.53770083102493\\
2.62	-2.54762430939227\\
2.63	-2.55754820936639\\
2.64	-2.56747252747253\\
2.65	-2.57739726027397\\
2.66	-2.58732240437159\\
2.67	-2.59724795640327\\
2.68	-2.60717391304348\\
2.69	-2.61710027100271\\
2.7	-2.62702702702703\\
2.71	-2.63695417789757\\
2.72	-2.64688172043011\\
2.73	-2.65680965147453\\
2.74	-2.66673796791444\\
2.75	-2.67666666666667\\
2.76	-2.68659574468085\\
2.77	-2.69652519893899\\
2.78	-2.70645502645503\\
2.79	-2.71638522427441\\
2.8	-2.72631578947368\\
2.81	-2.73624671916011\\
2.82	-2.7461780104712\\
2.83	-2.75610966057441\\
2.84	-2.76604166666667\\
2.85	-2.77597402597403\\
2.86	-2.7859067357513\\
2.87	-2.79583979328165\\
2.88	-2.80577319587629\\
2.89	-2.81570694087404\\
2.9	-2.82564102564103\\
2.91	-2.83557544757033\\
2.92	-2.84551020408163\\
2.93	-2.85544529262087\\
2.94	-2.8653807106599\\
2.95	-2.8753164556962\\
2.96	-2.88525252525253\\
2.97	-2.89518891687657\\
2.98	-2.9051256281407\\
2.99	-2.9150626566416\\
3	-2.925\\
3.01	-2.93493765586035\\
3.02	-2.94487562189055\\
3.03	-2.95481389578164\\
3.04	-2.96475247524752\\
3.05	-2.97469135802469\\
3.06	-2.98463054187192\\
3.07	-2.99457002457002\\
3.08	-3.00450980392157\\
3.09	-3.01444987775061\\
3.1	-3.02439024390244\\
3.11	-3.03433090024331\\
3.12	-3.04427184466019\\
3.13	-3.05421307506053\\
3.14	-3.06415458937198\\
3.15	-3.07409638554217\\
3.16	-3.08403846153846\\
3.17	-3.09398081534772\\
3.18	-3.10392344497608\\
3.19	-3.11386634844869\\
3.2	-3.12380952380952\\
3.21	-3.13375296912114\\
3.22	-3.14369668246445\\
3.23	-3.15364066193853\\
3.24	-3.16358490566038\\
3.25	-3.17352941176471\\
3.26	-3.18347417840376\\
3.27	-3.19341920374707\\
3.28	-3.20336448598131\\
3.29	-3.21331002331002\\
3.3	-3.22325581395349\\
3.31	-3.23320185614849\\
3.32	-3.24314814814815\\
3.33	-3.25309468822171\\
3.34	-3.26304147465438\\
3.35	-3.27298850574713\\
3.36	-3.28293577981651\\
3.37	-3.29288329519451\\
3.38	-3.30283105022831\\
3.39	-3.31277904328018\\
3.4	-3.32272727272727\\
3.41	-3.33267573696145\\
3.42	-3.34262443438914\\
3.43	-3.35257336343115\\
3.44	-3.36252252252252\\
3.45	-3.37247191011236\\
3.46	-3.38242152466368\\
3.47	-3.39237136465324\\
3.48	-3.40232142857143\\
3.49	-3.41227171492205\\
3.5	-3.42222222222222\\
3.51	-3.43217294900222\\
3.52	-3.44212389380531\\
3.53	-3.45207505518764\\
3.54	-3.46202643171806\\
3.55	-3.47197802197802\\
3.56	-3.4819298245614\\
3.57	-3.4918818380744\\
3.58	-3.50183406113537\\
3.59	-3.51178649237473\\
3.6	-3.52173913043478\\
3.61	-3.53169197396963\\
3.62	-3.54164502164502\\
3.63	-3.55159827213823\\
3.64	-3.56155172413793\\
3.65	-3.57150537634409\\
3.66	-3.58145922746781\\
3.67	-3.59141327623126\\
3.68	-3.60136752136752\\
3.69	-3.61132196162047\\
3.7	-3.62127659574468\\
3.71	-3.63123142250531\\
3.72	-3.64118644067797\\
3.73	-3.65114164904863\\
3.74	-3.6610970464135\\
3.75	-3.67105263157895\\
3.76	-3.68100840336134\\
3.77	-3.690964360587\\
3.78	-3.70092050209205\\
3.79	-3.71087682672234\\
3.8	-3.72083333333333\\
3.81	-3.73079002079002\\
3.82	-3.74074688796681\\
3.83	-3.75070393374741\\
3.84	-3.76066115702479\\
3.85	-3.77061855670103\\
3.86	-3.78057613168724\\
3.87	-3.79053388090349\\
3.88	-3.80049180327869\\
3.89	-3.81044989775051\\
3.9	-3.82040816326531\\
3.91	-3.830366598778\\
3.92	-3.84032520325203\\
3.93	-3.85028397565923\\
3.94	-3.86024291497976\\
3.95	-3.87020202020202\\
3.96	-3.88016129032258\\
3.97	-3.89012072434608\\
3.98	-3.90008032128514\\
3.99	-3.91004008016032\\
4	-3.92\\
4.01	-3.92996007984032\\
4.02	-3.9399203187251\\
4.03	-3.94988071570577\\
4.04	-3.95984126984127\\
4.05	-3.96980198019802\\
4.06	-3.9797628458498\\
4.07	-3.98972386587771\\
4.08	-3.99968503937008\\
4.09	-4.0096463654224\\
4.1	-4.01960784313725\\
4.11	-4.02956947162427\\
4.12	-4.03953125\\
4.13	-4.04949317738791\\
4.14	-4.05945525291829\\
4.15	-4.06941747572816\\
4.16	-4.07937984496124\\
4.17	-4.08934235976789\\
4.18	-4.09930501930502\\
4.19	-4.10926782273603\\
4.2	-4.11923076923077\\
4.21	-4.12919385796545\\
4.22	-4.13915708812261\\
4.23	-4.14912045889101\\
4.24	-4.15908396946565\\
4.25	-4.16904761904762\\
4.26	-4.17901140684411\\
4.27	-4.18897533206831\\
4.28	-4.19893939393939\\
4.29	-4.20890359168242\\
4.3	-4.2188679245283\\
4.31	-4.22883239171375\\
4.32	-4.2387969924812\\
4.33	-4.2487617260788\\
4.34	-4.2587265917603\\
4.35	-4.26869158878505\\
4.36	-4.27865671641791\\
4.37	-4.28862197392924\\
4.38	-4.2985873605948\\
4.39	-4.30855287569573\\
4.4	-4.31851851851852\\
4.41	-4.3284842883549\\
4.42	-4.33845018450184\\
4.43	-4.34841620626151\\
4.44	-4.35838235294118\\
4.45	-4.36834862385321\\
4.46	-4.37831501831502\\
4.47	-4.38828153564899\\
4.48	-4.39824817518248\\
4.49	-4.40821493624772\\
4.5	-4.41818181818182\\
4.51	-4.42814882032668\\
4.52	-4.43811594202899\\
4.53	-4.44808318264015\\
4.54	-4.45805054151625\\
4.55	-4.46801801801802\\
4.56	-4.47798561151079\\
4.57	-4.48795332136445\\
4.58	-4.4979211469534\\
4.59	-4.50788908765653\\
4.6	-4.51785714285714\\
4.61	-4.52782531194296\\
4.62	-4.53779359430605\\
4.63	-4.54776198934281\\
4.64	-4.5577304964539\\
4.65	-4.56769911504425\\
4.66	-4.57766784452297\\
4.67	-4.58763668430335\\
4.68	-4.59760563380282\\
4.69	-4.60757469244288\\
4.7	-4.61754385964912\\
4.71	-4.62751313485114\\
4.72	-4.63748251748252\\
4.73	-4.6474520069808\\
4.74	-4.65742160278746\\
4.75	-4.66739130434783\\
4.76	-4.67736111111111\\
4.77	-4.68733102253033\\
4.78	-4.69730103806228\\
4.79	-4.70727115716753\\
4.8	-4.71724137931034\\
4.81	-4.72721170395869\\
4.82	-4.73718213058419\\
4.83	-4.74715265866209\\
4.84	-4.75712328767123\\
4.85	-4.76709401709402\\
4.86	-4.77706484641638\\
4.87	-4.78703577512777\\
4.88	-4.79700680272109\\
4.89	-4.8069779286927\\
4.9	-4.81694915254237\\
4.91	-4.82692047377327\\
4.92	-4.83689189189189\\
4.93	-4.84686340640809\\
4.94	-4.85683501683502\\
4.95	-4.86680672268908\\
4.96	-4.87677852348993\\
4.97	-4.88675041876047\\
4.98	-4.89672240802676\\
4.99	-4.90669449081803\\
5	-4.91666666666667\\
};
\addplot [color=mycolor3,solid,forget plot]
  table[row sep=crcr]{%
0	0\\
0.01	-0.00801980198019802\\
0.02	-0.016078431372549\\
0.03	-0.0241747572815534\\
0.04	-0.0323076923076923\\
0.05	-0.0404761904761905\\
0.06	-0.0486792452830189\\
0.07	-0.0569158878504673\\
0.08	-0.0651851851851852\\
0.09	-0.0734862385321101\\
0.1	-0.0818181818181818\\
0.11	-0.0901801801801802\\
0.12	-0.0985714285714286\\
0.13	-0.106991150442478\\
0.14	-0.115438596491228\\
0.15	-0.123913043478261\\
0.16	-0.132413793103448\\
0.17	-0.140940170940171\\
0.18	-0.149491525423729\\
0.19	-0.158067226890756\\
0.2	-0.166666666666667\\
0.21	-0.175289256198347\\
0.22	-0.183934426229508\\
0.23	-0.19260162601626\\
0.24	-0.201290322580645\\
0.25	-0.21\\
0.26	-0.218730158730159\\
0.27	-0.22748031496063\\
0.28	-0.23625\\
0.29	-0.245038759689922\\
0.3	-0.253846153846154\\
0.31	-0.262671755725191\\
0.32	-0.271515151515152\\
0.33	-0.280375939849624\\
0.34	-0.289253731343284\\
0.35	-0.298148148148148\\
0.36	-0.307058823529412\\
0.37	-0.315985401459854\\
0.38	-0.324927536231884\\
0.39	-0.333884892086331\\
0.4	-0.342857142857143\\
0.41	-0.351843971631206\\
0.42	-0.360845070422535\\
0.43	-0.36986013986014\\
0.44	-0.378888888888889\\
0.45	-0.387931034482759\\
0.46	-0.396986301369863\\
0.47	-0.406054421768707\\
0.48	-0.415135135135135\\
0.49	-0.424228187919463\\
0.5	-0.433333333333333\\
0.51	-0.442450331125828\\
0.52	-0.451578947368421\\
0.53	-0.460718954248366\\
0.54	-0.46987012987013\\
0.55	-0.479032258064516\\
0.56	-0.488205128205128\\
0.57	-0.497388535031847\\
0.58	-0.506582278481013\\
0.59	-0.515786163522013\\
0.6	-0.525\\
0.61	-0.534223602484472\\
0.62	-0.543456790123457\\
0.63	-0.552699386503067\\
0.64	-0.561951219512195\\
0.65	-0.571212121212121\\
0.66	-0.580481927710843\\
0.67	-0.589760479041916\\
0.68	-0.599047619047619\\
0.69	-0.608343195266272\\
0.7	-0.617647058823529\\
0.71	-0.626959064327485\\
0.72	-0.636279069767442\\
0.73	-0.645606936416185\\
0.74	-0.654942528735632\\
0.75	-0.664285714285714\\
0.76	-0.673636363636364\\
0.77	-0.682994350282486\\
0.78	-0.692359550561798\\
0.79	-0.701731843575419\\
0.8	-0.711111111111111\\
0.81	-0.720497237569061\\
0.82	-0.72989010989011\\
0.83	-0.739289617486339\\
0.84	-0.748695652173913\\
0.85	-0.758108108108108\\
0.86	-0.76752688172043\\
0.87	-0.776951871657754\\
0.88	-0.786382978723404\\
0.89	-0.795820105820106\\
0.9	-0.805263157894737\\
0.91	-0.814712041884817\\
0.92	-0.824166666666667\\
0.93	-0.833626943005181\\
0.94	-0.843092783505155\\
0.95	-0.852564102564103\\
0.96	-0.862040816326531\\
0.97	-0.871522842639594\\
0.98	-0.881010101010101\\
0.99	-0.890502512562814\\
1	-0.9\\
1.01	-0.909502487562189\\
1.02	-0.919009900990099\\
1.03	-0.928522167487685\\
1.04	-0.938039215686275\\
1.05	-0.947560975609756\\
1.06	-0.957087378640777\\
1.07	-0.966618357487923\\
1.08	-0.976153846153846\\
1.09	-0.985693779904306\\
1.1	-0.995238095238095\\
1.11	-1.00478672985782\\
1.12	-1.01433962264151\\
1.13	-1.02389671361502\\
1.14	-1.03345794392523\\
1.15	-1.04302325581395\\
1.16	-1.05259259259259\\
1.17	-1.06216589861751\\
1.18	-1.07174311926605\\
1.19	-1.08132420091324\\
1.2	-1.09090909090909\\
1.21	-1.10049773755656\\
1.22	-1.11009009009009\\
1.23	-1.11968609865471\\
1.24	-1.12928571428571\\
1.25	-1.13888888888889\\
1.26	-1.14849557522124\\
1.27	-1.15810572687225\\
1.28	-1.16771929824561\\
1.29	-1.17733624454148\\
1.3	-1.18695652173913\\
1.31	-1.19658008658009\\
1.32	-1.20620689655172\\
1.33	-1.21583690987124\\
1.34	-1.22547008547009\\
1.35	-1.23510638297872\\
1.36	-1.24474576271186\\
1.37	-1.25438818565401\\
1.38	-1.26403361344538\\
1.39	-1.2736820083682\\
1.4	-1.28333333333333\\
1.41	-1.29298755186722\\
1.42	-1.30264462809917\\
1.43	-1.31230452674897\\
1.44	-1.32196721311475\\
1.45	-1.33163265306122\\
1.46	-1.34130081300813\\
1.47	-1.35097165991903\\
1.48	-1.36064516129032\\
1.49	-1.37032128514056\\
1.5	-1.38\\
1.51	-1.3896812749004\\
1.52	-1.39936507936508\\
1.53	-1.40905138339921\\
1.54	-1.41874015748031\\
1.55	-1.42843137254902\\
1.56	-1.438125\\
1.57	-1.44782101167315\\
1.58	-1.45751937984496\\
1.59	-1.46722007722008\\
1.6	-1.47692307692308\\
1.61	-1.48662835249042\\
1.62	-1.4963358778626\\
1.63	-1.50604562737643\\
1.64	-1.51575757575758\\
1.65	-1.52547169811321\\
1.66	-1.53518796992481\\
1.67	-1.5449063670412\\
1.68	-1.55462686567164\\
1.69	-1.56434944237918\\
1.7	-1.57407407407407\\
1.71	-1.58380073800738\\
1.72	-1.59352941176471\\
1.73	-1.60326007326007\\
1.74	-1.61299270072993\\
1.75	-1.62272727272727\\
1.76	-1.63246376811594\\
1.77	-1.64220216606498\\
1.78	-1.65194244604317\\
1.79	-1.66168458781362\\
1.8	-1.67142857142857\\
1.81	-1.6811743772242\\
1.82	-1.6909219858156\\
1.83	-1.70067137809187\\
1.84	-1.71042253521127\\
1.85	-1.72017543859649\\
1.86	-1.72993006993007\\
1.87	-1.73968641114983\\
1.88	-1.74944444444444\\
1.89	-1.75920415224914\\
1.9	-1.76896551724138\\
1.91	-1.77872852233677\\
1.92	-1.78849315068493\\
1.93	-1.79825938566553\\
1.94	-1.80802721088435\\
1.95	-1.81779661016949\\
1.96	-1.82756756756757\\
1.97	-1.83734006734007\\
1.98	-1.84711409395973\\
1.99	-1.85688963210702\\
2	-1.86666666666667\\
2.01	-1.87644518272425\\
2.02	-1.88622516556291\\
2.03	-1.89600660066007\\
2.04	-1.90578947368421\\
2.05	-1.9155737704918\\
2.06	-1.92535947712418\\
2.07	-1.93514657980456\\
2.08	-1.94493506493506\\
2.09	-1.95472491909385\\
2.1	-1.96451612903226\\
2.11	-1.97430868167203\\
2.12	-1.98410256410256\\
2.13	-1.99389776357827\\
2.14	-2.00369426751592\\
2.15	-2.01349206349206\\
2.16	-2.02329113924051\\
2.17	-2.03309148264984\\
2.18	-2.04289308176101\\
2.19	-2.05269592476489\\
2.2	-2.0625\\
2.21	-2.07230529595016\\
2.22	-2.08211180124224\\
2.23	-2.09191950464396\\
2.24	-2.10172839506173\\
2.25	-2.11153846153846\\
2.26	-2.12134969325153\\
2.27	-2.1311620795107\\
2.28	-2.1409756097561\\
2.29	-2.15079027355623\\
2.3	-2.16060606060606\\
2.31	-2.17042296072508\\
2.32	-2.18024096385542\\
2.33	-2.19006006006006\\
2.34	-2.19988023952096\\
2.35	-2.20970149253731\\
2.36	-2.21952380952381\\
2.37	-2.2293471810089\\
2.38	-2.23917159763314\\
2.39	-2.24899705014749\\
2.4	-2.25882352941176\\
2.41	-2.26865102639296\\
2.42	-2.27847953216374\\
2.43	-2.28830903790087\\
2.44	-2.29813953488372\\
2.45	-2.30797101449275\\
2.46	-2.31780346820809\\
2.47	-2.32763688760807\\
2.48	-2.33747126436782\\
2.49	-2.34730659025788\\
2.5	-2.35714285714286\\
2.51	-2.36698005698006\\
2.52	-2.37681818181818\\
2.53	-2.38665722379603\\
2.54	-2.39649717514124\\
2.55	-2.40633802816901\\
2.56	-2.4161797752809\\
2.57	-2.42602240896359\\
2.58	-2.43586592178771\\
2.59	-2.44571030640668\\
2.6	-2.45555555555556\\
2.61	-2.46540166204986\\
2.62	-2.47524861878453\\
2.63	-2.48509641873278\\
2.64	-2.49494505494506\\
2.65	-2.50479452054795\\
2.66	-2.51464480874317\\
2.67	-2.52449591280654\\
2.68	-2.53434782608696\\
2.69	-2.54420054200542\\
2.7	-2.55405405405405\\
2.71	-2.56390835579515\\
2.72	-2.57376344086021\\
2.73	-2.58361930294906\\
2.74	-2.59347593582888\\
2.75	-2.60333333333333\\
2.76	-2.6131914893617\\
2.77	-2.62305039787798\\
2.78	-2.63291005291005\\
2.79	-2.64277044854881\\
2.8	-2.65263157894737\\
2.81	-2.66249343832021\\
2.82	-2.67235602094241\\
2.83	-2.68221932114883\\
2.84	-2.69208333333333\\
2.85	-2.70194805194805\\
2.86	-2.71181347150259\\
2.87	-2.72167958656331\\
2.88	-2.73154639175258\\
2.89	-2.74141388174807\\
2.9	-2.75128205128205\\
2.91	-2.76115089514066\\
2.92	-2.77102040816327\\
2.93	-2.78089058524173\\
2.94	-2.7907614213198\\
2.95	-2.80063291139241\\
2.96	-2.81050505050505\\
2.97	-2.82037783375315\\
2.98	-2.83025125628141\\
2.99	-2.84012531328321\\
3	-2.85\\
3.01	-2.8598753117207\\
3.02	-2.86975124378109\\
3.03	-2.87962779156328\\
3.04	-2.88950495049505\\
3.05	-2.89938271604938\\
3.06	-2.90926108374384\\
3.07	-2.91914004914005\\
3.08	-2.92901960784314\\
3.09	-2.93889975550122\\
3.1	-2.94878048780488\\
3.11	-2.95866180048662\\
3.12	-2.96854368932039\\
3.13	-2.97842615012107\\
3.14	-2.98830917874396\\
3.15	-2.99819277108434\\
3.16	-3.00807692307692\\
3.17	-3.01796163069544\\
3.18	-3.02784688995215\\
3.19	-3.03773269689737\\
3.2	-3.04761904761905\\
3.21	-3.05750593824228\\
3.22	-3.06739336492891\\
3.23	-3.07728132387707\\
3.24	-3.08716981132075\\
3.25	-3.09705882352941\\
3.26	-3.10694835680751\\
3.27	-3.11683840749415\\
3.28	-3.12672897196262\\
3.29	-3.13662004662005\\
3.3	-3.14651162790698\\
3.31	-3.15640371229698\\
3.32	-3.1662962962963\\
3.33	-3.17618937644342\\
3.34	-3.18608294930876\\
3.35	-3.19597701149425\\
3.36	-3.20587155963303\\
3.37	-3.21576659038902\\
3.38	-3.22566210045662\\
3.39	-3.23555808656036\\
3.4	-3.24545454545455\\
3.41	-3.2553514739229\\
3.42	-3.26524886877828\\
3.43	-3.2751467268623\\
3.44	-3.28504504504504\\
3.45	-3.29494382022472\\
3.46	-3.30484304932735\\
3.47	-3.31474272930649\\
3.48	-3.32464285714286\\
3.49	-3.3345434298441\\
3.5	-3.34444444444444\\
3.51	-3.35434589800443\\
3.52	-3.36424778761062\\
3.53	-3.37415011037528\\
3.54	-3.38405286343612\\
3.55	-3.39395604395604\\
3.56	-3.40385964912281\\
3.57	-3.4137636761488\\
3.58	-3.42366812227074\\
3.59	-3.43357298474946\\
3.6	-3.44347826086956\\
3.61	-3.45338394793926\\
3.62	-3.46329004329004\\
3.63	-3.47319654427646\\
3.64	-3.48310344827586\\
3.65	-3.49301075268817\\
3.66	-3.50291845493562\\
3.67	-3.51282655246253\\
3.68	-3.52273504273504\\
3.69	-3.53264392324094\\
3.7	-3.54255319148936\\
3.71	-3.55246284501062\\
3.72	-3.56237288135593\\
3.73	-3.57228329809725\\
3.74	-3.582194092827\\
3.75	-3.59210526315789\\
3.76	-3.60201680672269\\
3.77	-3.611928721174\\
3.78	-3.6218410041841\\
3.79	-3.63175365344468\\
3.8	-3.64166666666667\\
3.81	-3.65158004158004\\
3.82	-3.66149377593361\\
3.83	-3.67140786749482\\
3.84	-3.68132231404959\\
3.85	-3.69123711340206\\
3.86	-3.70115226337449\\
3.87	-3.71106776180698\\
3.88	-3.72098360655738\\
3.89	-3.73089979550102\\
3.9	-3.74081632653061\\
3.91	-3.75073319755601\\
3.92	-3.76065040650407\\
3.93	-3.77056795131846\\
3.94	-3.78048582995951\\
3.95	-3.79040404040404\\
3.96	-3.80032258064516\\
3.97	-3.81024144869215\\
3.98	-3.82016064257028\\
3.99	-3.83008016032064\\
4	-3.84\\
4.01	-3.84992015968064\\
4.02	-3.8598406374502\\
4.03	-3.86976143141153\\
4.04	-3.87968253968254\\
4.05	-3.88960396039604\\
4.06	-3.8995256916996\\
4.07	-3.90944773175542\\
4.08	-3.91937007874016\\
4.09	-3.92929273084479\\
4.1	-3.93921568627451\\
4.11	-3.94913894324853\\
4.12	-3.9590625\\
4.13	-3.96898635477583\\
4.14	-3.97891050583658\\
4.15	-3.98883495145631\\
4.16	-3.99875968992248\\
4.17	-4.00868471953578\\
4.18	-4.01861003861004\\
4.19	-4.02853564547206\\
4.2	-4.03846153846154\\
4.21	-4.0483877159309\\
4.22	-4.05831417624521\\
4.23	-4.06824091778203\\
4.24	-4.0781679389313\\
4.25	-4.08809523809524\\
4.26	-4.09802281368821\\
4.27	-4.10795066413662\\
4.28	-4.11787878787879\\
4.29	-4.12780718336484\\
4.3	-4.1377358490566\\
4.31	-4.1476647834275\\
4.32	-4.15759398496241\\
4.33	-4.1675234521576\\
4.34	-4.1774531835206\\
4.35	-4.18738317757009\\
4.36	-4.19731343283582\\
4.37	-4.20724394785847\\
4.38	-4.21717472118959\\
4.39	-4.22710575139147\\
4.4	-4.23703703703704\\
4.41	-4.2469685767098\\
4.42	-4.25690036900369\\
4.43	-4.26683241252302\\
4.44	-4.27676470588235\\
4.45	-4.28669724770642\\
4.46	-4.29663003663004\\
4.47	-4.30656307129799\\
4.48	-4.31649635036496\\
4.49	-4.32642987249545\\
4.5	-4.33636363636364\\
4.51	-4.34629764065336\\
4.52	-4.35623188405797\\
4.53	-4.36616636528029\\
4.54	-4.37610108303249\\
4.55	-4.38603603603604\\
4.56	-4.39597122302158\\
4.57	-4.40590664272891\\
4.58	-4.41584229390681\\
4.59	-4.42577817531306\\
4.6	-4.43571428571429\\
4.61	-4.44565062388592\\
4.62	-4.4555871886121\\
4.63	-4.46552397868561\\
4.64	-4.4754609929078\\
4.65	-4.4853982300885\\
4.66	-4.49533568904594\\
4.67	-4.5052733686067\\
4.68	-4.51521126760563\\
4.69	-4.52514938488576\\
4.7	-4.53508771929825\\
4.71	-4.54502626970228\\
4.72	-4.55496503496503\\
4.73	-4.56490401396161\\
4.74	-4.57484320557491\\
4.75	-4.58478260869565\\
4.76	-4.59472222222222\\
4.77	-4.60466204506066\\
4.78	-4.61460207612457\\
4.79	-4.62454231433506\\
4.8	-4.63448275862069\\
4.81	-4.64442340791738\\
4.82	-4.65436426116838\\
4.83	-4.66430531732419\\
4.84	-4.67424657534247\\
4.85	-4.68418803418803\\
4.86	-4.69412969283276\\
4.87	-4.70407155025554\\
4.88	-4.71401360544218\\
4.89	-4.7239558573854\\
4.9	-4.73389830508475\\
4.91	-4.74384094754653\\
4.92	-4.75378378378378\\
4.93	-4.76372681281619\\
4.94	-4.77367003367003\\
4.95	-4.78361344537815\\
4.96	-4.79355704697987\\
4.97	-4.80350083752094\\
4.98	-4.81344481605351\\
4.99	-4.82338898163606\\
5	-4.83333333333333\\
};
\addplot [color=mycolor4,solid,forget plot]
  table[row sep=crcr]{%
0	0\\
0.01	-0.00702970297029703\\
0.02	-0.0141176470588235\\
0.03	-0.0212621359223301\\
0.04	-0.0284615384615385\\
0.05	-0.0357142857142857\\
0.06	-0.0430188679245283\\
0.07	-0.0503738317757009\\
0.08	-0.0577777777777778\\
0.09	-0.0652293577981651\\
0.1	-0.0727272727272727\\
0.11	-0.0802702702702703\\
0.12	-0.0878571428571429\\
0.13	-0.0954867256637168\\
0.14	-0.103157894736842\\
0.15	-0.110869565217391\\
0.16	-0.118620689655172\\
0.17	-0.126410256410256\\
0.18	-0.134237288135593\\
0.19	-0.142100840336134\\
0.2	-0.15\\
0.21	-0.157933884297521\\
0.22	-0.165901639344262\\
0.23	-0.17390243902439\\
0.24	-0.181935483870968\\
0.25	-0.19\\
0.26	-0.198095238095238\\
0.27	-0.206220472440945\\
0.28	-0.214375\\
0.29	-0.222558139534884\\
0.3	-0.230769230769231\\
0.31	-0.239007633587786\\
0.32	-0.247272727272727\\
0.33	-0.255563909774436\\
0.34	-0.263880597014925\\
0.35	-0.272222222222222\\
0.36	-0.280588235294118\\
0.37	-0.288978102189781\\
0.38	-0.297391304347826\\
0.39	-0.305827338129496\\
0.4	-0.314285714285714\\
0.41	-0.322765957446809\\
0.42	-0.331267605633803\\
0.43	-0.33979020979021\\
0.44	-0.348333333333333\\
0.45	-0.356896551724138\\
0.46	-0.365479452054794\\
0.47	-0.374081632653061\\
0.48	-0.382702702702703\\
0.49	-0.391342281879195\\
0.5	-0.4\\
0.51	-0.408675496688742\\
0.52	-0.417368421052632\\
0.53	-0.426078431372549\\
0.54	-0.434805194805195\\
0.55	-0.443548387096774\\
0.56	-0.452307692307692\\
0.57	-0.461082802547771\\
0.58	-0.469873417721519\\
0.59	-0.478679245283019\\
0.6	-0.4875\\
0.61	-0.496335403726708\\
0.62	-0.505185185185185\\
0.63	-0.514049079754601\\
0.64	-0.522926829268293\\
0.65	-0.531818181818182\\
0.66	-0.540722891566265\\
0.67	-0.549640718562874\\
0.68	-0.558571428571429\\
0.69	-0.567514792899408\\
0.7	-0.576470588235294\\
0.71	-0.585438596491228\\
0.72	-0.594418604651163\\
0.73	-0.603410404624277\\
0.74	-0.612413793103448\\
0.75	-0.621428571428571\\
0.76	-0.630454545454545\\
0.77	-0.639491525423729\\
0.78	-0.648539325842697\\
0.79	-0.657597765363129\\
0.8	-0.666666666666667\\
0.81	-0.675745856353591\\
0.82	-0.684835164835165\\
0.83	-0.693934426229508\\
0.84	-0.70304347826087\\
0.85	-0.712162162162162\\
0.86	-0.721290322580645\\
0.87	-0.730427807486631\\
0.88	-0.739574468085106\\
0.89	-0.748730158730159\\
0.9	-0.757894736842105\\
0.91	-0.767068062827225\\
0.92	-0.77625\\
0.93	-0.785440414507772\\
0.94	-0.794639175257732\\
0.95	-0.803846153846154\\
0.96	-0.813061224489796\\
0.97	-0.822284263959391\\
0.98	-0.831515151515151\\
0.99	-0.840753768844221\\
1	-0.85\\
1.01	-0.859253731343284\\
1.02	-0.868514851485149\\
1.03	-0.877783251231527\\
1.04	-0.887058823529412\\
1.05	-0.896341463414634\\
1.06	-0.905631067961165\\
1.07	-0.914927536231884\\
1.08	-0.924230769230769\\
1.09	-0.933540669856459\\
1.1	-0.942857142857143\\
1.11	-0.95218009478673\\
1.12	-0.961509433962264\\
1.13	-0.970845070422535\\
1.14	-0.980186915887851\\
1.15	-0.98953488372093\\
1.16	-0.998888888888889\\
1.17	-1.00824884792627\\
1.18	-1.01761467889908\\
1.19	-1.02698630136986\\
1.2	-1.03636363636364\\
1.21	-1.04574660633484\\
1.22	-1.05513513513514\\
1.23	-1.06452914798206\\
1.24	-1.07392857142857\\
1.25	-1.08333333333333\\
1.26	-1.09274336283186\\
1.27	-1.10215859030837\\
1.28	-1.11157894736842\\
1.29	-1.12100436681223\\
1.3	-1.1304347826087\\
1.31	-1.13987012987013\\
1.32	-1.14931034482759\\
1.33	-1.15875536480687\\
1.34	-1.16820512820513\\
1.35	-1.17765957446809\\
1.36	-1.1871186440678\\
1.37	-1.19658227848101\\
1.38	-1.20605042016807\\
1.39	-1.2155230125523\\
1.4	-1.225\\
1.41	-1.23448132780083\\
1.42	-1.24396694214876\\
1.43	-1.25345679012346\\
1.44	-1.26295081967213\\
1.45	-1.27244897959184\\
1.46	-1.2819512195122\\
1.47	-1.29145748987854\\
1.48	-1.30096774193548\\
1.49	-1.31048192771084\\
1.5	-1.32\\
1.51	-1.3295219123506\\
1.52	-1.33904761904762\\
1.53	-1.34857707509881\\
1.54	-1.35811023622047\\
1.55	-1.36764705882353\\
1.56	-1.3771875\\
1.57	-1.38673151750973\\
1.58	-1.39627906976744\\
1.59	-1.40583011583012\\
1.6	-1.41538461538462\\
1.61	-1.42494252873563\\
1.62	-1.43450381679389\\
1.63	-1.44406844106464\\
1.64	-1.45363636363636\\
1.65	-1.46320754716981\\
1.66	-1.47278195488722\\
1.67	-1.4823595505618\\
1.68	-1.49194029850746\\
1.69	-1.50152416356877\\
1.7	-1.51111111111111\\
1.71	-1.52070110701107\\
1.72	-1.53029411764706\\
1.73	-1.53989010989011\\
1.74	-1.54948905109489\\
1.75	-1.55909090909091\\
1.76	-1.56869565217391\\
1.77	-1.57830324909747\\
1.78	-1.58791366906475\\
1.79	-1.59752688172043\\
1.8	-1.60714285714286\\
1.81	-1.6167615658363\\
1.82	-1.6263829787234\\
1.83	-1.63600706713781\\
1.84	-1.6456338028169\\
1.85	-1.65526315789474\\
1.86	-1.66489510489511\\
1.87	-1.67452961672474\\
1.88	-1.68416666666667\\
1.89	-1.6938062283737\\
1.9	-1.70344827586207\\
1.91	-1.71309278350515\\
1.92	-1.7227397260274\\
1.93	-1.73238907849829\\
1.94	-1.74204081632653\\
1.95	-1.75169491525424\\
1.96	-1.76135135135135\\
1.97	-1.7710101010101\\
1.98	-1.7806711409396\\
1.99	-1.79033444816053\\
2	-1.8\\
2.01	-1.80966777408638\\
2.02	-1.81933774834437\\
2.03	-1.8290099009901\\
2.04	-1.83868421052632\\
2.05	-1.8483606557377\\
2.06	-1.85803921568627\\
2.07	-1.86771986970684\\
2.08	-1.8774025974026\\
2.09	-1.88708737864078\\
2.1	-1.89677419354839\\
2.11	-1.90646302250804\\
2.12	-1.91615384615385\\
2.13	-1.92584664536741\\
2.14	-1.93554140127389\\
2.15	-1.94523809523809\\
2.16	-1.95493670886076\\
2.17	-1.96463722397476\\
2.18	-1.97433962264151\\
2.19	-1.98404388714734\\
2.2	-1.99375\\
2.21	-2.00345794392523\\
2.22	-2.01316770186335\\
2.23	-2.02287925696594\\
2.24	-2.03259259259259\\
2.25	-2.04230769230769\\
2.26	-2.0520245398773\\
2.27	-2.06174311926606\\
2.28	-2.07146341463415\\
2.29	-2.08118541033435\\
2.3	-2.09090909090909\\
2.31	-2.10063444108761\\
2.32	-2.11036144578313\\
2.33	-2.12009009009009\\
2.34	-2.12982035928144\\
2.35	-2.13955223880597\\
2.36	-2.14928571428571\\
2.37	-2.15902077151335\\
2.38	-2.1687573964497\\
2.39	-2.17849557522124\\
2.4	-2.18823529411765\\
2.41	-2.19797653958944\\
2.42	-2.20771929824561\\
2.43	-2.21746355685131\\
2.44	-2.22720930232558\\
2.45	-2.23695652173913\\
2.46	-2.24670520231214\\
2.47	-2.2564553314121\\
2.48	-2.26620689655172\\
2.49	-2.27595988538682\\
2.5	-2.28571428571429\\
2.51	-2.29547008547009\\
2.52	-2.30522727272727\\
2.53	-2.31498583569405\\
2.54	-2.32474576271186\\
2.55	-2.33450704225352\\
2.56	-2.34426966292135\\
2.57	-2.35403361344538\\
2.58	-2.36379888268156\\
2.59	-2.37356545961003\\
2.6	-2.38333333333333\\
2.61	-2.39310249307479\\
2.62	-2.4028729281768\\
2.63	-2.41264462809917\\
2.64	-2.42241758241758\\
2.65	-2.43219178082192\\
2.66	-2.44196721311475\\
2.67	-2.45174386920981\\
2.68	-2.46152173913043\\
2.69	-2.47130081300813\\
2.7	-2.48108108108108\\
2.71	-2.49086253369272\\
2.72	-2.50064516129032\\
2.73	-2.51042895442359\\
2.74	-2.52021390374332\\
2.75	-2.53\\
2.76	-2.53978723404255\\
2.77	-2.54957559681698\\
2.78	-2.55936507936508\\
2.79	-2.56915567282322\\
2.8	-2.57894736842105\\
2.81	-2.58874015748032\\
2.82	-2.59853403141361\\
2.83	-2.60832898172324\\
2.84	-2.618125\\
2.85	-2.62792207792208\\
2.86	-2.63772020725389\\
2.87	-2.64751937984496\\
2.88	-2.65731958762887\\
2.89	-2.66712082262211\\
2.9	-2.67692307692308\\
2.91	-2.686726342711\\
2.92	-2.6965306122449\\
2.93	-2.7063358778626\\
2.94	-2.7161421319797\\
2.95	-2.72594936708861\\
2.96	-2.73575757575758\\
2.97	-2.74556675062972\\
2.98	-2.75537688442211\\
2.99	-2.76518796992481\\
3	-2.775\\
3.01	-2.78481296758105\\
3.02	-2.79462686567164\\
3.03	-2.80444168734491\\
3.04	-2.81425742574257\\
3.05	-2.82407407407407\\
3.06	-2.83389162561576\\
3.07	-2.84371007371007\\
3.08	-2.85352941176471\\
3.09	-2.86334963325183\\
3.1	-2.87317073170732\\
3.11	-2.88299270072993\\
3.12	-2.89281553398058\\
3.13	-2.9026392251816\\
3.14	-2.91246376811594\\
3.15	-2.92228915662651\\
3.16	-2.93211538461538\\
3.17	-2.94194244604317\\
3.18	-2.95177033492823\\
3.19	-2.96159904534606\\
3.2	-2.97142857142857\\
3.21	-2.98125890736342\\
3.22	-2.99109004739336\\
3.23	-3.0009219858156\\
3.24	-3.01075471698113\\
3.25	-3.02058823529412\\
3.26	-3.03042253521127\\
3.27	-3.04025761124122\\
3.28	-3.05009345794393\\
3.29	-3.05993006993007\\
3.3	-3.06976744186046\\
3.31	-3.07960556844548\\
3.32	-3.08944444444444\\
3.33	-3.09928406466513\\
3.34	-3.10912442396313\\
3.35	-3.11896551724138\\
3.36	-3.12880733944954\\
3.37	-3.13864988558352\\
3.38	-3.14849315068493\\
3.39	-3.15833712984055\\
3.4	-3.16818181818182\\
3.41	-3.17802721088435\\
3.42	-3.18787330316742\\
3.43	-3.19772009029345\\
3.44	-3.20756756756757\\
3.45	-3.21741573033708\\
3.46	-3.22726457399103\\
3.47	-3.23711409395973\\
3.48	-3.24696428571429\\
3.49	-3.25681514476615\\
3.5	-3.26666666666667\\
3.51	-3.27651884700665\\
3.52	-3.28637168141593\\
3.53	-3.29622516556291\\
3.54	-3.30607929515419\\
3.55	-3.31593406593407\\
3.56	-3.32578947368421\\
3.57	-3.3356455142232\\
3.58	-3.34550218340611\\
3.59	-3.35535947712418\\
3.6	-3.36521739130435\\
3.61	-3.37507592190889\\
3.62	-3.38493506493506\\
3.63	-3.39479481641469\\
3.64	-3.40465517241379\\
3.65	-3.41451612903226\\
3.66	-3.42437768240343\\
3.67	-3.43423982869379\\
3.68	-3.44410256410256\\
3.69	-3.45396588486141\\
3.7	-3.46382978723404\\
3.71	-3.47369426751592\\
3.72	-3.4835593220339\\
3.73	-3.49342494714588\\
3.74	-3.50329113924051\\
3.75	-3.51315789473684\\
3.76	-3.52302521008403\\
3.77	-3.53289308176101\\
3.78	-3.54276150627615\\
3.79	-3.55263048016701\\
3.8	-3.5625\\
3.81	-3.57237006237006\\
3.82	-3.58224066390042\\
3.83	-3.59211180124224\\
3.84	-3.60198347107438\\
3.85	-3.61185567010309\\
3.86	-3.62172839506173\\
3.87	-3.63160164271047\\
3.88	-3.64147540983607\\
3.89	-3.65134969325153\\
3.9	-3.66122448979592\\
3.91	-3.67109979633401\\
3.92	-3.6809756097561\\
3.93	-3.69085192697769\\
3.94	-3.70072874493927\\
3.95	-3.71060606060606\\
3.96	-3.72048387096774\\
3.97	-3.73036217303823\\
3.98	-3.74024096385542\\
3.99	-3.75012024048096\\
4	-3.76\\
4.01	-3.76988023952096\\
4.02	-3.7797609561753\\
4.03	-3.7896421471173\\
4.04	-3.79952380952381\\
4.05	-3.80940594059406\\
4.06	-3.81928853754941\\
4.07	-3.82917159763314\\
4.08	-3.83905511811024\\
4.09	-3.84893909626719\\
4.1	-3.85882352941176\\
4.11	-3.8687084148728\\
4.12	-3.87859375\\
4.13	-3.88847953216374\\
4.14	-3.89836575875486\\
4.15	-3.90825242718447\\
4.16	-3.91813953488372\\
4.17	-3.92802707930367\\
4.18	-3.93791505791506\\
4.19	-3.94780346820809\\
4.2	-3.95769230769231\\
4.21	-3.96758157389635\\
4.22	-3.97747126436782\\
4.23	-3.98736137667304\\
4.24	-3.99725190839695\\
4.25	-4.00714285714286\\
4.26	-4.01703422053232\\
4.27	-4.02692599620493\\
4.28	-4.03681818181818\\
4.29	-4.04671077504726\\
4.3	-4.05660377358491\\
4.31	-4.06649717514124\\
4.32	-4.07639097744361\\
4.33	-4.0862851782364\\
4.34	-4.0961797752809\\
4.35	-4.10607476635514\\
4.36	-4.11597014925373\\
4.37	-4.12586592178771\\
4.38	-4.13576208178439\\
4.39	-4.1456586270872\\
4.4	-4.15555555555556\\
4.41	-4.16545286506469\\
4.42	-4.17535055350553\\
4.43	-4.18524861878453\\
4.44	-4.19514705882353\\
4.45	-4.20504587155963\\
4.46	-4.21494505494505\\
4.47	-4.22484460694698\\
4.48	-4.23474452554745\\
4.49	-4.24464480874317\\
4.5	-4.25454545454545\\
4.51	-4.26444646098004\\
4.52	-4.27434782608696\\
4.53	-4.28424954792043\\
4.54	-4.29415162454874\\
4.55	-4.30405405405405\\
4.56	-4.31395683453237\\
4.57	-4.32385996409336\\
4.58	-4.33376344086022\\
4.59	-4.34366726296959\\
4.6	-4.35357142857143\\
4.61	-4.36347593582888\\
4.62	-4.37338078291815\\
4.63	-4.38328596802842\\
4.64	-4.3931914893617\\
4.65	-4.40309734513274\\
4.66	-4.4130035335689\\
4.67	-4.42291005291005\\
4.68	-4.43281690140845\\
4.69	-4.44272407732865\\
4.7	-4.45263157894737\\
4.71	-4.46253940455342\\
4.72	-4.47244755244755\\
4.73	-4.48235602094241\\
4.74	-4.49226480836237\\
4.75	-4.50217391304348\\
4.76	-4.51208333333333\\
4.77	-4.52199306759099\\
4.78	-4.53190311418685\\
4.79	-4.54181347150259\\
4.8	-4.55172413793103\\
4.81	-4.56163511187608\\
4.82	-4.57154639175258\\
4.83	-4.58145797598628\\
4.84	-4.5913698630137\\
4.85	-4.60128205128205\\
4.86	-4.61119453924915\\
4.87	-4.6211073253833\\
4.88	-4.63102040816327\\
4.89	-4.6409337860781\\
4.9	-4.65084745762712\\
4.91	-4.6607614213198\\
4.92	-4.67067567567568\\
4.93	-4.68059021922428\\
4.94	-4.69050505050505\\
4.95	-4.70042016806723\\
4.96	-4.7103355704698\\
4.97	-4.72025125628141\\
4.98	-4.73016722408027\\
4.99	-4.74008347245409\\
5	-4.75\\
};
\addplot [color=mycolor5,solid,forget plot]
  table[row sep=crcr]{%
0	0\\
0.01	-0.00603960396039604\\
0.02	-0.012156862745098\\
0.03	-0.0183495145631068\\
0.04	-0.0246153846153846\\
0.05	-0.030952380952381\\
0.06	-0.0373584905660377\\
0.07	-0.0438317757009346\\
0.08	-0.0503703703703704\\
0.09	-0.0569724770642202\\
0.1	-0.0636363636363636\\
0.11	-0.0703603603603604\\
0.12	-0.0771428571428571\\
0.13	-0.0839823008849558\\
0.14	-0.0908771929824562\\
0.15	-0.0978260869565217\\
0.16	-0.104827586206897\\
0.17	-0.111880341880342\\
0.18	-0.118983050847458\\
0.19	-0.126134453781513\\
0.2	-0.133333333333333\\
0.21	-0.140578512396694\\
0.22	-0.147868852459016\\
0.23	-0.15520325203252\\
0.24	-0.16258064516129\\
0.25	-0.17\\
0.26	-0.177460317460317\\
0.27	-0.18496062992126\\
0.28	-0.1925\\
0.29	-0.200077519379845\\
0.3	-0.207692307692308\\
0.31	-0.215343511450382\\
0.32	-0.223030303030303\\
0.33	-0.230751879699248\\
0.34	-0.238507462686567\\
0.35	-0.246296296296296\\
0.36	-0.254117647058824\\
0.37	-0.261970802919708\\
0.38	-0.269855072463768\\
0.39	-0.277769784172662\\
0.4	-0.285714285714286\\
0.41	-0.293687943262411\\
0.42	-0.30169014084507\\
0.43	-0.30972027972028\\
0.44	-0.317777777777778\\
0.45	-0.325862068965517\\
0.46	-0.333972602739726\\
0.47	-0.342108843537415\\
0.48	-0.35027027027027\\
0.49	-0.358456375838926\\
0.5	-0.366666666666667\\
0.51	-0.374900662251656\\
0.52	-0.383157894736842\\
0.53	-0.391437908496732\\
0.54	-0.39974025974026\\
0.55	-0.408064516129032\\
0.56	-0.416410256410256\\
0.57	-0.424777070063694\\
0.58	-0.433164556962025\\
0.59	-0.441572327044025\\
0.6	-0.45\\
0.61	-0.458447204968944\\
0.62	-0.466913580246914\\
0.63	-0.475398773006135\\
0.64	-0.48390243902439\\
0.65	-0.492424242424242\\
0.66	-0.500963855421687\\
0.67	-0.509520958083832\\
0.68	-0.518095238095238\\
0.69	-0.526686390532544\\
0.7	-0.535294117647059\\
0.71	-0.543918128654971\\
0.72	-0.552558139534884\\
0.73	-0.56121387283237\\
0.74	-0.569885057471264\\
0.75	-0.578571428571429\\
0.76	-0.587272727272727\\
0.77	-0.595988700564972\\
0.78	-0.604719101123596\\
0.79	-0.613463687150838\\
0.8	-0.622222222222222\\
0.81	-0.630994475138122\\
0.82	-0.63978021978022\\
0.83	-0.648579234972678\\
0.84	-0.657391304347826\\
0.85	-0.666216216216216\\
0.86	-0.67505376344086\\
0.87	-0.683903743315508\\
0.88	-0.692765957446808\\
0.89	-0.701640211640212\\
0.9	-0.710526315789474\\
0.91	-0.719424083769634\\
0.92	-0.728333333333333\\
0.93	-0.737253886010363\\
0.94	-0.746185567010309\\
0.95	-0.755128205128205\\
0.96	-0.764081632653061\\
0.97	-0.773045685279188\\
0.98	-0.782020202020202\\
0.99	-0.791005025125628\\
1	-0.8\\
1.01	-0.809004975124378\\
1.02	-0.818019801980198\\
1.03	-0.82704433497537\\
1.04	-0.836078431372549\\
1.05	-0.845121951219512\\
1.06	-0.854174757281553\\
1.07	-0.863236714975846\\
1.08	-0.872307692307692\\
1.09	-0.881387559808612\\
1.1	-0.890476190476191\\
1.11	-0.89957345971564\\
1.12	-0.908679245283019\\
1.13	-0.917793427230047\\
1.14	-0.926915887850467\\
1.15	-0.936046511627907\\
1.16	-0.945185185185185\\
1.17	-0.954331797235023\\
1.18	-0.96348623853211\\
1.19	-0.972648401826484\\
1.2	-0.981818181818182\\
1.21	-0.990995475113122\\
1.22	-1.00018018018018\\
1.23	-1.00937219730942\\
1.24	-1.01857142857143\\
1.25	-1.02777777777778\\
1.26	-1.03699115044248\\
1.27	-1.04621145374449\\
1.28	-1.05543859649123\\
1.29	-1.06467248908297\\
1.3	-1.07391304347826\\
1.31	-1.08316017316017\\
1.32	-1.09241379310345\\
1.33	-1.10167381974249\\
1.34	-1.11094017094017\\
1.35	-1.12021276595745\\
1.36	-1.12949152542373\\
1.37	-1.13877637130802\\
1.38	-1.14806722689076\\
1.39	-1.1573640167364\\
1.4	-1.16666666666667\\
1.41	-1.17597510373444\\
1.42	-1.18528925619835\\
1.43	-1.19460905349794\\
1.44	-1.20393442622951\\
1.45	-1.21326530612245\\
1.46	-1.22260162601626\\
1.47	-1.23194331983806\\
1.48	-1.24129032258065\\
1.49	-1.25064257028112\\
1.5	-1.26\\
1.51	-1.2693625498008\\
1.52	-1.27873015873016\\
1.53	-1.28810276679842\\
1.54	-1.29748031496063\\
1.55	-1.30686274509804\\
1.56	-1.31625\\
1.57	-1.3256420233463\\
1.58	-1.33503875968992\\
1.59	-1.34444015444015\\
1.6	-1.35384615384615\\
1.61	-1.36325670498084\\
1.62	-1.37267175572519\\
1.63	-1.38209125475285\\
1.64	-1.39151515151515\\
1.65	-1.40094339622642\\
1.66	-1.41037593984962\\
1.67	-1.4198127340824\\
1.68	-1.42925373134328\\
1.69	-1.43869888475836\\
1.7	-1.44814814814815\\
1.71	-1.45760147601476\\
1.72	-1.46705882352941\\
1.73	-1.47652014652015\\
1.74	-1.48598540145985\\
1.75	-1.49545454545455\\
1.76	-1.50492753623188\\
1.77	-1.51440433212996\\
1.78	-1.52388489208633\\
1.79	-1.53336917562724\\
1.8	-1.54285714285714\\
1.81	-1.5523487544484\\
1.82	-1.56184397163121\\
1.83	-1.57134275618375\\
1.84	-1.58084507042254\\
1.85	-1.59035087719298\\
1.86	-1.59986013986014\\
1.87	-1.60937282229965\\
1.88	-1.61888888888889\\
1.89	-1.62840830449827\\
1.9	-1.63793103448276\\
1.91	-1.64745704467354\\
1.92	-1.65698630136986\\
1.93	-1.66651877133106\\
1.94	-1.67605442176871\\
1.95	-1.68559322033898\\
1.96	-1.69513513513514\\
1.97	-1.70468013468013\\
1.98	-1.71422818791946\\
1.99	-1.72377926421405\\
2	-1.73333333333333\\
2.01	-1.74289036544851\\
2.02	-1.75245033112583\\
2.03	-1.76201320132013\\
2.04	-1.77157894736842\\
2.05	-1.78114754098361\\
2.06	-1.79071895424837\\
2.07	-1.80029315960912\\
2.08	-1.80987012987013\\
2.09	-1.8194498381877\\
2.1	-1.82903225806452\\
2.11	-1.83861736334405\\
2.12	-1.84820512820513\\
2.13	-1.85779552715655\\
2.14	-1.86738853503185\\
2.15	-1.87698412698413\\
2.16	-1.88658227848101\\
2.17	-1.89618296529968\\
2.18	-1.90578616352201\\
2.19	-1.91539184952978\\
2.2	-1.925\\
2.21	-1.93461059190031\\
2.22	-1.94422360248447\\
2.23	-1.95383900928793\\
2.24	-1.96345679012346\\
2.25	-1.97307692307692\\
2.26	-1.98269938650307\\
2.27	-1.99232415902141\\
2.28	-2.0019512195122\\
2.29	-2.01158054711246\\
2.3	-2.02121212121212\\
2.31	-2.03084592145015\\
2.32	-2.04048192771084\\
2.33	-2.05012012012012\\
2.34	-2.05976047904192\\
2.35	-2.06940298507463\\
2.36	-2.07904761904762\\
2.37	-2.0886943620178\\
2.38	-2.09834319526627\\
2.39	-2.10799410029499\\
2.4	-2.11764705882353\\
2.41	-2.12730205278592\\
2.42	-2.13695906432749\\
2.43	-2.14661807580175\\
2.44	-2.15627906976744\\
2.45	-2.16594202898551\\
2.46	-2.17560693641618\\
2.47	-2.18527377521614\\
2.48	-2.19494252873563\\
2.49	-2.20461318051576\\
2.5	-2.21428571428571\\
2.51	-2.22396011396011\\
2.52	-2.23363636363636\\
2.53	-2.24331444759207\\
2.54	-2.25299435028249\\
2.55	-2.26267605633803\\
2.56	-2.2723595505618\\
2.57	-2.28204481792717\\
2.58	-2.29173184357542\\
2.59	-2.30142061281337\\
2.6	-2.31111111111111\\
2.61	-2.32080332409972\\
2.62	-2.33049723756906\\
2.63	-2.34019283746556\\
2.64	-2.34989010989011\\
2.65	-2.35958904109589\\
2.66	-2.36928961748634\\
2.67	-2.37899182561308\\
2.68	-2.38869565217391\\
2.69	-2.39840108401084\\
2.7	-2.40810810810811\\
2.71	-2.4178167115903\\
2.72	-2.42752688172043\\
2.73	-2.43723860589812\\
2.74	-2.44695187165775\\
2.75	-2.45666666666667\\
2.76	-2.4663829787234\\
2.77	-2.47610079575597\\
2.78	-2.48582010582011\\
2.79	-2.49554089709763\\
2.8	-2.50526315789474\\
2.81	-2.51498687664042\\
2.82	-2.52471204188482\\
2.83	-2.53443864229765\\
2.84	-2.54416666666667\\
2.85	-2.5538961038961\\
2.86	-2.56362694300518\\
2.87	-2.57335917312662\\
2.88	-2.58309278350515\\
2.89	-2.59282776349614\\
2.9	-2.6025641025641\\
2.91	-2.61230179028133\\
2.92	-2.62204081632653\\
2.93	-2.63178117048346\\
2.94	-2.64152284263959\\
2.95	-2.65126582278481\\
2.96	-2.6610101010101\\
2.97	-2.6707556675063\\
2.98	-2.68050251256281\\
2.99	-2.69025062656642\\
3	-2.7\\
3.01	-2.7097506234414\\
3.02	-2.71950248756219\\
3.03	-2.72925558312655\\
3.04	-2.7390099009901\\
3.05	-2.74876543209877\\
3.06	-2.75852216748768\\
3.07	-2.7682800982801\\
3.08	-2.77803921568627\\
3.09	-2.78779951100244\\
3.1	-2.79756097560976\\
3.11	-2.80732360097324\\
3.12	-2.81708737864078\\
3.13	-2.82685230024213\\
3.14	-2.83661835748792\\
3.15	-2.84638554216867\\
3.16	-2.85615384615385\\
3.17	-2.86592326139089\\
3.18	-2.87569377990431\\
3.19	-2.88546539379475\\
3.2	-2.8952380952381\\
3.21	-2.90501187648456\\
3.22	-2.91478672985782\\
3.23	-2.92456264775414\\
3.24	-2.93433962264151\\
3.25	-2.94411764705882\\
3.26	-2.95389671361502\\
3.27	-2.96367681498829\\
3.28	-2.97345794392523\\
3.29	-2.98324009324009\\
3.3	-2.99302325581395\\
3.31	-3.00280742459397\\
3.32	-3.01259259259259\\
3.33	-3.02237875288684\\
3.34	-3.03216589861751\\
3.35	-3.04195402298851\\
3.36	-3.05174311926605\\
3.37	-3.06153318077803\\
3.38	-3.07132420091324\\
3.39	-3.08111617312073\\
3.4	-3.09090909090909\\
3.41	-3.10070294784581\\
3.42	-3.11049773755656\\
3.43	-3.1202934537246\\
3.44	-3.13009009009009\\
3.45	-3.13988764044944\\
3.46	-3.14968609865471\\
3.47	-3.15948545861298\\
3.48	-3.16928571428571\\
3.49	-3.1790868596882\\
3.5	-3.18888888888889\\
3.51	-3.19869179600887\\
3.52	-3.20849557522124\\
3.53	-3.21830022075055\\
3.54	-3.22810572687225\\
3.55	-3.23791208791209\\
3.56	-3.24771929824561\\
3.57	-3.25752735229759\\
3.58	-3.26733624454148\\
3.59	-3.27714596949891\\
3.6	-3.28695652173913\\
3.61	-3.29676789587852\\
3.62	-3.30658008658009\\
3.63	-3.31639308855292\\
3.64	-3.32620689655172\\
3.65	-3.33602150537634\\
3.66	-3.34583690987124\\
3.67	-3.35565310492505\\
3.68	-3.36547008547008\\
3.69	-3.37528784648188\\
3.7	-3.38510638297872\\
3.71	-3.39492569002123\\
3.72	-3.40474576271186\\
3.73	-3.4145665961945\\
3.74	-3.42438818565401\\
3.75	-3.43421052631579\\
3.76	-3.44403361344538\\
3.77	-3.45385744234801\\
3.78	-3.4636820083682\\
3.79	-3.47350730688935\\
3.8	-3.48333333333333\\
3.81	-3.49316008316008\\
3.82	-3.50298755186722\\
3.83	-3.51281573498965\\
3.84	-3.52264462809917\\
3.85	-3.53247422680412\\
3.86	-3.54230452674897\\
3.87	-3.55213552361396\\
3.88	-3.56196721311475\\
3.89	-3.57179959100204\\
3.9	-3.58163265306122\\
3.91	-3.59146639511202\\
3.92	-3.60130081300813\\
3.93	-3.61113590263692\\
3.94	-3.62097165991903\\
3.95	-3.63080808080808\\
3.96	-3.64064516129032\\
3.97	-3.65048289738431\\
3.98	-3.66032128514056\\
3.99	-3.67016032064128\\
4	-3.68\\
4.01	-3.68984031936128\\
4.02	-3.6996812749004\\
4.03	-3.70952286282306\\
4.04	-3.71936507936508\\
4.05	-3.72920792079208\\
4.06	-3.73905138339921\\
4.07	-3.74889546351085\\
4.08	-3.75874015748031\\
4.09	-3.76858546168959\\
4.1	-3.77843137254902\\
4.11	-3.78827788649706\\
4.12	-3.798125\\
4.13	-3.80797270955166\\
4.14	-3.81782101167315\\
4.15	-3.82766990291262\\
4.16	-3.83751937984496\\
4.17	-3.84736943907157\\
4.18	-3.85722007722008\\
4.19	-3.86707129094412\\
4.2	-3.87692307692308\\
4.21	-3.8867754318618\\
4.22	-3.89662835249042\\
4.23	-3.90648183556405\\
4.24	-3.9163358778626\\
4.25	-3.92619047619048\\
4.26	-3.93604562737643\\
4.27	-3.94590132827324\\
4.28	-3.95575757575758\\
4.29	-3.96561436672968\\
4.3	-3.97547169811321\\
4.31	-3.98532956685499\\
4.32	-3.99518796992481\\
4.33	-4.0050469043152\\
4.34	-4.0149063670412\\
4.35	-4.02476635514019\\
4.36	-4.03462686567164\\
4.37	-4.04448789571695\\
4.38	-4.05434944237918\\
4.39	-4.06421150278293\\
4.4	-4.07407407407407\\
4.41	-4.08393715341959\\
4.42	-4.09380073800738\\
4.43	-4.10366482504604\\
4.44	-4.11352941176471\\
4.45	-4.12339449541284\\
4.46	-4.13326007326007\\
4.47	-4.14312614259598\\
4.48	-4.15299270072993\\
4.49	-4.16285974499089\\
4.5	-4.17272727272727\\
4.51	-4.18259528130671\\
4.52	-4.19246376811594\\
4.53	-4.20233273056058\\
4.54	-4.21220216606498\\
4.55	-4.22207207207207\\
4.56	-4.23194244604317\\
4.57	-4.24181328545781\\
4.58	-4.25168458781362\\
4.59	-4.26155635062612\\
4.6	-4.27142857142857\\
4.61	-4.28130124777184\\
4.62	-4.2911743772242\\
4.63	-4.30104795737123\\
4.64	-4.3109219858156\\
4.65	-4.32079646017699\\
4.66	-4.33067137809187\\
4.67	-4.3405467372134\\
4.68	-4.35042253521127\\
4.69	-4.36029876977153\\
4.7	-4.37017543859649\\
4.71	-4.38005253940455\\
4.72	-4.38993006993007\\
4.73	-4.39980802792321\\
4.74	-4.40968641114983\\
4.75	-4.4195652173913\\
4.76	-4.42944444444444\\
4.77	-4.43932409012132\\
4.78	-4.44920415224914\\
4.79	-4.45908462867012\\
4.8	-4.46896551724138\\
4.81	-4.47884681583477\\
4.82	-4.48872852233677\\
4.83	-4.49861063464837\\
4.84	-4.50849315068493\\
4.85	-4.51837606837607\\
4.86	-4.52825938566553\\
4.87	-4.53814310051107\\
4.88	-4.54802721088435\\
4.89	-4.5579117147708\\
4.9	-4.56779661016949\\
4.91	-4.57768189509306\\
4.92	-4.58756756756757\\
4.93	-4.59745362563238\\
4.94	-4.60734006734007\\
4.95	-4.6172268907563\\
4.96	-4.62711409395973\\
4.97	-4.63700167504188\\
4.98	-4.64688963210702\\
4.99	-4.65677796327212\\
5	-4.66666666666667\\
};
\addplot [color=mycolor6,solid,forget plot]
  table[row sep=crcr]{%
0	0\\
0.01	-0.00504950495049505\\
0.02	-0.0101960784313725\\
0.03	-0.0154368932038835\\
0.04	-0.0207692307692308\\
0.05	-0.0261904761904762\\
0.06	-0.0316981132075472\\
0.07	-0.0372897196261682\\
0.08	-0.042962962962963\\
0.09	-0.0487155963302752\\
0.1	-0.0545454545454545\\
0.11	-0.0604504504504505\\
0.12	-0.0664285714285714\\
0.13	-0.0724778761061947\\
0.14	-0.0785964912280702\\
0.15	-0.0847826086956522\\
0.16	-0.0910344827586207\\
0.17	-0.0973504273504274\\
0.18	-0.103728813559322\\
0.19	-0.110168067226891\\
0.2	-0.116666666666667\\
0.21	-0.123223140495868\\
0.22	-0.12983606557377\\
0.23	-0.13650406504065\\
0.24	-0.143225806451613\\
0.25	-0.15\\
0.26	-0.156825396825397\\
0.27	-0.163700787401575\\
0.28	-0.170625\\
0.29	-0.177596899224806\\
0.3	-0.184615384615385\\
0.31	-0.191679389312977\\
0.32	-0.198787878787879\\
0.33	-0.20593984962406\\
0.34	-0.213134328358209\\
0.35	-0.22037037037037\\
0.36	-0.227647058823529\\
0.37	-0.234963503649635\\
0.38	-0.24231884057971\\
0.39	-0.249712230215827\\
0.4	-0.257142857142857\\
0.41	-0.264609929078014\\
0.42	-0.272112676056338\\
0.43	-0.27965034965035\\
0.44	-0.287222222222222\\
0.45	-0.294827586206897\\
0.46	-0.302465753424658\\
0.47	-0.310136054421769\\
0.48	-0.317837837837838\\
0.49	-0.325570469798658\\
0.5	-0.333333333333333\\
0.51	-0.34112582781457\\
0.52	-0.348947368421053\\
0.53	-0.356797385620915\\
0.54	-0.364675324675325\\
0.55	-0.37258064516129\\
0.56	-0.380512820512821\\
0.57	-0.388471337579618\\
0.58	-0.396455696202532\\
0.59	-0.404465408805031\\
0.6	-0.4125\\
0.61	-0.42055900621118\\
0.62	-0.428641975308642\\
0.63	-0.436748466257669\\
0.64	-0.444878048780488\\
0.65	-0.453030303030303\\
0.66	-0.461204819277108\\
0.67	-0.46940119760479\\
0.68	-0.477619047619048\\
0.69	-0.48585798816568\\
0.7	-0.494117647058824\\
0.71	-0.502397660818713\\
0.72	-0.510697674418605\\
0.73	-0.519017341040462\\
0.74	-0.52735632183908\\
0.75	-0.535714285714286\\
0.76	-0.544090909090909\\
0.77	-0.552485875706215\\
0.78	-0.560898876404494\\
0.79	-0.569329608938548\\
0.8	-0.577777777777778\\
0.81	-0.586243093922652\\
0.82	-0.594725274725275\\
0.83	-0.603224043715847\\
0.84	-0.611739130434783\\
0.85	-0.62027027027027\\
0.86	-0.628817204301075\\
0.87	-0.637379679144385\\
0.88	-0.645957446808511\\
0.89	-0.654550264550265\\
0.9	-0.663157894736842\\
0.91	-0.671780104712042\\
0.92	-0.680416666666667\\
0.93	-0.689067357512953\\
0.94	-0.697731958762887\\
0.95	-0.706410256410256\\
0.96	-0.715102040816326\\
0.97	-0.723807106598985\\
0.98	-0.732525252525252\\
0.99	-0.741256281407035\\
1	-0.75\\
1.01	-0.758756218905473\\
1.02	-0.767524752475248\\
1.03	-0.776305418719212\\
1.04	-0.785098039215686\\
1.05	-0.79390243902439\\
1.06	-0.802718446601942\\
1.07	-0.811545893719807\\
1.08	-0.820384615384615\\
1.09	-0.829234449760766\\
1.1	-0.838095238095238\\
1.11	-0.84696682464455\\
1.12	-0.855849056603774\\
1.13	-0.864741784037559\\
1.14	-0.873644859813084\\
1.15	-0.882558139534884\\
1.16	-0.891481481481481\\
1.17	-0.900414746543779\\
1.18	-0.909357798165138\\
1.19	-0.918310502283105\\
1.2	-0.927272727272727\\
1.21	-0.936244343891403\\
1.22	-0.945225225225225\\
1.23	-0.954215246636771\\
1.24	-0.963214285714286\\
1.25	-0.972222222222222\\
1.26	-0.981238938053097\\
1.27	-0.990264317180617\\
1.28	-0.999298245614035\\
1.29	-1.00834061135371\\
1.3	-1.01739130434783\\
1.31	-1.02645021645022\\
1.32	-1.03551724137931\\
1.33	-1.04459227467811\\
1.34	-1.05367521367521\\
1.35	-1.06276595744681\\
1.36	-1.07186440677966\\
1.37	-1.08097046413502\\
1.38	-1.09008403361345\\
1.39	-1.0992050209205\\
1.4	-1.10833333333333\\
1.41	-1.11746887966805\\
1.42	-1.12661157024793\\
1.43	-1.13576131687243\\
1.44	-1.14491803278689\\
1.45	-1.15408163265306\\
1.46	-1.16325203252033\\
1.47	-1.17242914979757\\
1.48	-1.18161290322581\\
1.49	-1.19080321285141\\
1.5	-1.2\\
1.51	-1.209203187251\\
1.52	-1.2184126984127\\
1.53	-1.22762845849802\\
1.54	-1.23685039370079\\
1.55	-1.24607843137255\\
1.56	-1.2553125\\
1.57	-1.26455252918288\\
1.58	-1.2737984496124\\
1.59	-1.28305019305019\\
1.6	-1.29230769230769\\
1.61	-1.30157088122605\\
1.62	-1.31083969465649\\
1.63	-1.32011406844106\\
1.64	-1.32939393939394\\
1.65	-1.33867924528302\\
1.66	-1.34796992481203\\
1.67	-1.357265917603\\
1.68	-1.3665671641791\\
1.69	-1.37587360594796\\
1.7	-1.38518518518519\\
1.71	-1.39450184501845\\
1.72	-1.40382352941176\\
1.73	-1.41315018315018\\
1.74	-1.42248175182482\\
1.75	-1.43181818181818\\
1.76	-1.44115942028985\\
1.77	-1.45050541516245\\
1.78	-1.45985611510791\\
1.79	-1.46921146953405\\
1.8	-1.47857142857143\\
1.81	-1.4879359430605\\
1.82	-1.49730496453901\\
1.83	-1.50667844522968\\
1.84	-1.51605633802817\\
1.85	-1.52543859649123\\
1.86	-1.53482517482517\\
1.87	-1.54421602787456\\
1.88	-1.55361111111111\\
1.89	-1.56301038062284\\
1.9	-1.57241379310345\\
1.91	-1.58182130584192\\
1.92	-1.59123287671233\\
1.93	-1.60064846416382\\
1.94	-1.61006802721088\\
1.95	-1.61949152542373\\
1.96	-1.62891891891892\\
1.97	-1.63835016835017\\
1.98	-1.64778523489933\\
1.99	-1.65722408026756\\
2	-1.66666666666667\\
2.01	-1.67611295681063\\
2.02	-1.68556291390728\\
2.03	-1.69501650165017\\
2.04	-1.70447368421053\\
2.05	-1.71393442622951\\
2.06	-1.72339869281046\\
2.07	-1.7328664495114\\
2.08	-1.74233766233766\\
2.09	-1.75181229773463\\
2.1	-1.76129032258065\\
2.11	-1.77077170418006\\
2.12	-1.78025641025641\\
2.13	-1.78974440894569\\
2.14	-1.79923566878981\\
2.15	-1.80873015873016\\
2.16	-1.81822784810127\\
2.17	-1.82772870662461\\
2.18	-1.83723270440252\\
2.19	-1.84673981191223\\
2.2	-1.85625\\
2.21	-1.86576323987539\\
2.22	-1.87527950310559\\
2.23	-1.88479876160991\\
2.24	-1.89432098765432\\
2.25	-1.90384615384615\\
2.26	-1.91337423312883\\
2.27	-1.92290519877676\\
2.28	-1.93243902439024\\
2.29	-1.94197568389058\\
2.3	-1.95151515151515\\
2.31	-1.96105740181269\\
2.32	-1.97060240963855\\
2.33	-1.98015015015015\\
2.34	-1.9897005988024\\
2.35	-1.99925373134328\\
2.36	-2.00880952380952\\
2.37	-2.01836795252226\\
2.38	-2.02792899408284\\
2.39	-2.03749262536873\\
2.4	-2.04705882352941\\
2.41	-2.0566275659824\\
2.42	-2.06619883040936\\
2.43	-2.07577259475219\\
2.44	-2.0853488372093\\
2.45	-2.09492753623188\\
2.46	-2.10450867052023\\
2.47	-2.11409221902017\\
2.48	-2.12367816091954\\
2.49	-2.1332664756447\\
2.5	-2.14285714285714\\
2.51	-2.15245014245014\\
2.52	-2.16204545454545\\
2.53	-2.17164305949008\\
2.54	-2.18124293785311\\
2.55	-2.19084507042253\\
2.56	-2.20044943820225\\
2.57	-2.21005602240896\\
2.58	-2.21966480446927\\
2.59	-2.22927576601671\\
2.6	-2.23888888888889\\
2.61	-2.24850415512465\\
2.62	-2.25812154696133\\
2.63	-2.26774104683196\\
2.64	-2.27736263736264\\
2.65	-2.28698630136986\\
2.66	-2.29661202185792\\
2.67	-2.30623978201635\\
2.68	-2.31586956521739\\
2.69	-2.32550135501355\\
2.7	-2.33513513513513\\
2.71	-2.34477088948787\\
2.72	-2.35440860215054\\
2.73	-2.36404825737265\\
2.74	-2.37368983957219\\
2.75	-2.38333333333333\\
2.76	-2.39297872340425\\
2.77	-2.40262599469496\\
2.78	-2.41227513227513\\
2.79	-2.42192612137203\\
2.8	-2.43157894736842\\
2.81	-2.44123359580052\\
2.82	-2.45089005235602\\
2.83	-2.46054830287206\\
2.84	-2.47020833333333\\
2.85	-2.47987012987013\\
2.86	-2.48953367875648\\
2.87	-2.49919896640827\\
2.88	-2.50886597938144\\
2.89	-2.51853470437018\\
2.9	-2.52820512820513\\
2.91	-2.53787723785166\\
2.92	-2.54755102040816\\
2.93	-2.55722646310433\\
2.94	-2.56690355329949\\
2.95	-2.57658227848101\\
2.96	-2.58626262626263\\
2.97	-2.59594458438287\\
2.98	-2.60562814070352\\
2.99	-2.61531328320802\\
3	-2.625\\
3.01	-2.63468827930175\\
3.02	-2.64437810945274\\
3.03	-2.65406947890819\\
3.04	-2.66376237623762\\
3.05	-2.67345679012346\\
3.06	-2.68315270935961\\
3.07	-2.69285012285012\\
3.08	-2.70254901960784\\
3.09	-2.71224938875306\\
3.1	-2.72195121951219\\
3.11	-2.73165450121654\\
3.12	-2.74135922330097\\
3.13	-2.75106537530266\\
3.14	-2.7607729468599\\
3.15	-2.77048192771084\\
3.16	-2.78019230769231\\
3.17	-2.78990407673861\\
3.18	-2.79961722488038\\
3.19	-2.80933174224344\\
3.2	-2.81904761904762\\
3.21	-2.8287648456057\\
3.22	-2.83848341232227\\
3.23	-2.84820330969267\\
3.24	-2.85792452830189\\
3.25	-2.86764705882353\\
3.26	-2.87737089201878\\
3.27	-2.88709601873536\\
3.28	-2.89682242990654\\
3.29	-2.90655011655012\\
3.3	-2.91627906976744\\
3.31	-2.92600928074246\\
3.32	-2.93574074074074\\
3.33	-2.94547344110855\\
3.34	-2.95520737327189\\
3.35	-2.96494252873563\\
3.36	-2.97467889908257\\
3.37	-2.98441647597254\\
3.38	-2.99415525114155\\
3.39	-3.00389521640091\\
3.4	-3.01363636363636\\
3.41	-3.02337868480726\\
3.42	-3.0331221719457\\
3.43	-3.04286681715576\\
3.44	-3.05261261261261\\
3.45	-3.0623595505618\\
3.46	-3.07210762331839\\
3.47	-3.08185682326622\\
3.48	-3.09160714285714\\
3.49	-3.10135857461025\\
3.5	-3.11111111111111\\
3.51	-3.12086474501109\\
3.52	-3.13061946902655\\
3.53	-3.14037527593819\\
3.54	-3.15013215859031\\
3.55	-3.15989010989011\\
3.56	-3.16964912280702\\
3.57	-3.17940919037199\\
3.58	-3.18917030567686\\
3.59	-3.19893246187364\\
3.6	-3.20869565217391\\
3.61	-3.21845986984816\\
3.62	-3.22822510822511\\
3.63	-3.23799136069114\\
3.64	-3.24775862068965\\
3.65	-3.25752688172043\\
3.66	-3.26729613733906\\
3.67	-3.27706638115632\\
3.68	-3.28683760683761\\
3.69	-3.29660980810235\\
3.7	-3.3063829787234\\
3.71	-3.31615711252654\\
3.72	-3.32593220338983\\
3.73	-3.33570824524313\\
3.74	-3.34548523206751\\
3.75	-3.35526315789474\\
3.76	-3.36504201680672\\
3.77	-3.37482180293501\\
3.78	-3.38460251046025\\
3.79	-3.39438413361169\\
3.8	-3.40416666666667\\
3.81	-3.4139501039501\\
3.82	-3.42373443983403\\
3.83	-3.43351966873706\\
3.84	-3.44330578512397\\
3.85	-3.45309278350515\\
3.86	-3.46288065843621\\
3.87	-3.47266940451745\\
3.88	-3.48245901639344\\
3.89	-3.49224948875256\\
3.9	-3.50204081632653\\
3.91	-3.51183299389002\\
3.92	-3.52162601626016\\
3.93	-3.53141987829615\\
3.94	-3.54121457489879\\
3.95	-3.5510101010101\\
3.96	-3.5608064516129\\
3.97	-3.57060362173038\\
3.98	-3.5804016064257\\
3.99	-3.5902004008016\\
4	-3.6\\
4.01	-3.6098003992016\\
4.02	-3.6196015936255\\
4.03	-3.62940357852883\\
4.04	-3.63920634920635\\
4.05	-3.6490099009901\\
4.06	-3.65881422924901\\
4.07	-3.66861932938856\\
4.08	-3.67842519685039\\
4.09	-3.68823182711198\\
4.1	-3.69803921568627\\
4.11	-3.70784735812133\\
4.12	-3.71765625\\
4.13	-3.72746588693957\\
4.14	-3.73727626459144\\
4.15	-3.74708737864078\\
4.16	-3.7568992248062\\
4.17	-3.76671179883946\\
4.18	-3.7765250965251\\
4.19	-3.78633911368015\\
4.2	-3.79615384615385\\
4.21	-3.80596928982726\\
4.22	-3.81578544061303\\
4.23	-3.82560229445507\\
4.24	-3.83541984732824\\
4.25	-3.8452380952381\\
4.26	-3.85505703422053\\
4.27	-3.86487666034156\\
4.28	-3.87469696969697\\
4.29	-3.8845179584121\\
4.3	-3.89433962264151\\
4.31	-3.90416195856874\\
4.32	-3.91398496240602\\
4.33	-3.923808630394\\
4.34	-3.9336329588015\\
4.35	-3.94345794392523\\
4.36	-3.95328358208955\\
4.37	-3.96310986964618\\
4.38	-3.97293680297398\\
4.39	-3.98276437847866\\
4.4	-3.99259259259259\\
4.41	-4.00242144177449\\
4.42	-4.01225092250923\\
4.43	-4.02208103130755\\
4.44	-4.03191176470588\\
4.45	-4.04174311926606\\
4.46	-4.05157509157509\\
4.47	-4.06140767824497\\
4.48	-4.07124087591241\\
4.49	-4.08107468123862\\
4.5	-4.09090909090909\\
4.51	-4.10074410163339\\
4.52	-4.11057971014493\\
4.53	-4.12041591320072\\
4.54	-4.13025270758123\\
4.55	-4.14009009009009\\
4.56	-4.14992805755396\\
4.57	-4.15976660682226\\
4.58	-4.16960573476702\\
4.59	-4.17944543828265\\
4.6	-4.18928571428571\\
4.61	-4.1991265597148\\
4.62	-4.20896797153025\\
4.63	-4.21880994671403\\
4.64	-4.2286524822695\\
4.65	-4.23849557522124\\
4.66	-4.24833922261484\\
4.67	-4.25818342151675\\
4.68	-4.26802816901408\\
4.69	-4.27787346221441\\
4.7	-4.28771929824561\\
4.71	-4.29756567425569\\
4.72	-4.30741258741259\\
4.73	-4.31726003490401\\
4.74	-4.32710801393728\\
4.75	-4.33695652173913\\
4.76	-4.34680555555556\\
4.77	-4.35665511265165\\
4.78	-4.36650519031142\\
4.79	-4.37635578583765\\
4.8	-4.38620689655172\\
4.81	-4.39605851979346\\
4.82	-4.40591065292096\\
4.83	-4.41576329331046\\
4.84	-4.42561643835616\\
4.85	-4.43547008547008\\
4.86	-4.44532423208191\\
4.87	-4.45517887563884\\
4.88	-4.46503401360544\\
4.89	-4.4748896434635\\
4.9	-4.48474576271186\\
4.91	-4.49460236886633\\
4.92	-4.50445945945946\\
4.93	-4.51431703204047\\
4.94	-4.52417508417508\\
4.95	-4.53403361344538\\
4.96	-4.54389261744966\\
4.97	-4.55375209380234\\
4.98	-4.56361204013378\\
4.99	-4.57347245409015\\
5	-4.58333333333333\\
};
\addplot [color=mycolor7,solid,forget plot]
  table[row sep=crcr]{%
0	0\\
0.01	-0.00405940594059406\\
0.02	-0.00823529411764706\\
0.03	-0.0125242718446602\\
0.04	-0.0169230769230769\\
0.05	-0.0214285714285714\\
0.06	-0.0260377358490566\\
0.07	-0.0307476635514019\\
0.08	-0.0355555555555556\\
0.09	-0.0404587155963303\\
0.1	-0.0454545454545454\\
0.11	-0.0505405405405405\\
0.12	-0.0557142857142857\\
0.13	-0.0609734513274336\\
0.14	-0.0663157894736842\\
0.15	-0.0717391304347826\\
0.16	-0.0772413793103448\\
0.17	-0.0828205128205128\\
0.18	-0.0884745762711864\\
0.19	-0.0942016806722689\\
0.2	-0.1\\
0.21	-0.105867768595041\\
0.22	-0.111803278688525\\
0.23	-0.11780487804878\\
0.24	-0.123870967741935\\
0.25	-0.13\\
0.26	-0.136190476190476\\
0.27	-0.14244094488189\\
0.28	-0.14875\\
0.29	-0.155116279069767\\
0.3	-0.161538461538462\\
0.31	-0.168015267175572\\
0.32	-0.174545454545455\\
0.33	-0.181127819548872\\
0.34	-0.187761194029851\\
0.35	-0.194444444444444\\
0.36	-0.201176470588235\\
0.37	-0.207956204379562\\
0.38	-0.214782608695652\\
0.39	-0.221654676258993\\
0.4	-0.228571428571429\\
0.41	-0.235531914893617\\
0.42	-0.242535211267606\\
0.43	-0.24958041958042\\
0.44	-0.256666666666667\\
0.45	-0.263793103448276\\
0.46	-0.270958904109589\\
0.47	-0.278163265306122\\
0.48	-0.285405405405405\\
0.49	-0.292684563758389\\
0.5	-0.3\\
0.51	-0.307350993377483\\
0.52	-0.314736842105263\\
0.53	-0.322156862745098\\
0.54	-0.32961038961039\\
0.55	-0.337096774193548\\
0.56	-0.344615384615385\\
0.57	-0.352165605095541\\
0.58	-0.359746835443038\\
0.59	-0.367358490566038\\
0.6	-0.375\\
0.61	-0.382670807453416\\
0.62	-0.39037037037037\\
0.63	-0.398098159509202\\
0.64	-0.405853658536585\\
0.65	-0.413636363636364\\
0.66	-0.42144578313253\\
0.67	-0.429281437125749\\
0.68	-0.437142857142857\\
0.69	-0.445029585798817\\
0.7	-0.452941176470588\\
0.71	-0.460877192982456\\
0.72	-0.468837209302326\\
0.73	-0.476820809248555\\
0.74	-0.484827586206896\\
0.75	-0.492857142857143\\
0.76	-0.500909090909091\\
0.77	-0.508983050847458\\
0.78	-0.517078651685393\\
0.79	-0.525195530726257\\
0.8	-0.533333333333333\\
0.81	-0.541491712707182\\
0.82	-0.54967032967033\\
0.83	-0.557868852459016\\
0.84	-0.566086956521739\\
0.85	-0.574324324324324\\
0.86	-0.58258064516129\\
0.87	-0.590855614973262\\
0.88	-0.599148936170213\\
0.89	-0.607460317460317\\
0.9	-0.61578947368421\\
0.91	-0.62413612565445\\
0.92	-0.6325\\
0.93	-0.640880829015544\\
0.94	-0.649278350515464\\
0.95	-0.657692307692308\\
0.96	-0.666122448979592\\
0.97	-0.674568527918782\\
0.98	-0.683030303030303\\
0.99	-0.691507537688442\\
1	-0.7\\
1.01	-0.708507462686567\\
1.02	-0.717029702970297\\
1.03	-0.725566502463054\\
1.04	-0.734117647058824\\
1.05	-0.742682926829268\\
1.06	-0.75126213592233\\
1.07	-0.759855072463768\\
1.08	-0.768461538461539\\
1.09	-0.777081339712919\\
1.1	-0.785714285714286\\
1.11	-0.79436018957346\\
1.12	-0.803018867924528\\
1.13	-0.81169014084507\\
1.14	-0.820373831775701\\
1.15	-0.829069767441861\\
1.16	-0.837777777777778\\
1.17	-0.846497695852535\\
1.18	-0.855229357798165\\
1.19	-0.863972602739726\\
1.2	-0.872727272727273\\
1.21	-0.881493212669683\\
1.22	-0.89027027027027\\
1.23	-0.899058295964125\\
1.24	-0.907857142857143\\
1.25	-0.916666666666667\\
1.26	-0.925486725663717\\
1.27	-0.93431718061674\\
1.28	-0.943157894736842\\
1.29	-0.952008733624454\\
1.3	-0.960869565217391\\
1.31	-0.96974025974026\\
1.32	-0.978620689655172\\
1.33	-0.987510729613734\\
1.34	-0.996410256410256\\
1.35	-1.00531914893617\\
1.36	-1.01423728813559\\
1.37	-1.02316455696203\\
1.38	-1.03210084033613\\
1.39	-1.0410460251046\\
1.4	-1.05\\
1.41	-1.05896265560166\\
1.42	-1.06793388429752\\
1.43	-1.07691358024691\\
1.44	-1.08590163934426\\
1.45	-1.09489795918367\\
1.46	-1.10390243902439\\
1.47	-1.11291497975708\\
1.48	-1.12193548387097\\
1.49	-1.13096385542169\\
1.5	-1.14\\
1.51	-1.1490438247012\\
1.52	-1.15809523809524\\
1.53	-1.16715415019763\\
1.54	-1.17622047244094\\
1.55	-1.18529411764706\\
1.56	-1.194375\\
1.57	-1.20346303501946\\
1.58	-1.21255813953488\\
1.59	-1.22166023166023\\
1.6	-1.23076923076923\\
1.61	-1.23988505747126\\
1.62	-1.24900763358779\\
1.63	-1.25813688212928\\
1.64	-1.26727272727273\\
1.65	-1.27641509433962\\
1.66	-1.28556390977444\\
1.67	-1.2947191011236\\
1.68	-1.30388059701493\\
1.69	-1.31304832713755\\
1.7	-1.32222222222222\\
1.71	-1.33140221402214\\
1.72	-1.34058823529412\\
1.73	-1.34978021978022\\
1.74	-1.35897810218978\\
1.75	-1.36818181818182\\
1.76	-1.37739130434783\\
1.77	-1.38660649819495\\
1.78	-1.3958273381295\\
1.79	-1.40505376344086\\
1.8	-1.41428571428571\\
1.81	-1.4235231316726\\
1.82	-1.43276595744681\\
1.83	-1.44201413427562\\
1.84	-1.4512676056338\\
1.85	-1.46052631578947\\
1.86	-1.46979020979021\\
1.87	-1.47905923344948\\
1.88	-1.48833333333333\\
1.89	-1.4976124567474\\
1.9	-1.50689655172414\\
1.91	-1.51618556701031\\
1.92	-1.52547945205479\\
1.93	-1.53477815699659\\
1.94	-1.54408163265306\\
1.95	-1.55338983050847\\
1.96	-1.5627027027027\\
1.97	-1.5720202020202\\
1.98	-1.58134228187919\\
1.99	-1.59066889632107\\
2	-1.6\\
2.01	-1.60933554817276\\
2.02	-1.61867549668874\\
2.03	-1.6280198019802\\
2.04	-1.63736842105263\\
2.05	-1.64672131147541\\
2.06	-1.65607843137255\\
2.07	-1.66543973941368\\
2.08	-1.67480519480519\\
2.09	-1.68417475728155\\
2.1	-1.69354838709677\\
2.11	-1.70292604501608\\
2.12	-1.71230769230769\\
2.13	-1.72169329073482\\
2.14	-1.73108280254777\\
2.15	-1.74047619047619\\
2.16	-1.74987341772152\\
2.17	-1.75927444794953\\
2.18	-1.76867924528302\\
2.19	-1.77808777429467\\
2.2	-1.7875\\
2.21	-1.79691588785047\\
2.22	-1.80633540372671\\
2.23	-1.81575851393189\\
2.24	-1.82518518518519\\
2.25	-1.83461538461538\\
2.26	-1.8440490797546\\
2.27	-1.85348623853211\\
2.28	-1.86292682926829\\
2.29	-1.87237082066869\\
2.3	-1.88181818181818\\
2.31	-1.89126888217523\\
2.32	-1.90072289156626\\
2.33	-1.91018018018018\\
2.34	-1.91964071856287\\
2.35	-1.92910447761194\\
2.36	-1.93857142857143\\
2.37	-1.94804154302671\\
2.38	-1.95751479289941\\
2.39	-1.96699115044248\\
2.4	-1.97647058823529\\
2.41	-1.98595307917889\\
2.42	-1.99543859649123\\
2.43	-2.00492711370262\\
2.44	-2.01441860465116\\
2.45	-2.02391304347826\\
2.46	-2.03341040462428\\
2.47	-2.04291066282421\\
2.48	-2.05241379310345\\
2.49	-2.06191977077364\\
2.5	-2.07142857142857\\
2.51	-2.08094017094017\\
2.52	-2.09045454545455\\
2.53	-2.0999716713881\\
2.54	-2.10949152542373\\
2.55	-2.11901408450704\\
2.56	-2.1285393258427\\
2.57	-2.13806722689076\\
2.58	-2.14759776536313\\
2.59	-2.15713091922006\\
2.6	-2.16666666666667\\
2.61	-2.17620498614958\\
2.62	-2.18574585635359\\
2.63	-2.19528925619835\\
2.64	-2.20483516483516\\
2.65	-2.21438356164384\\
2.66	-2.22393442622951\\
2.67	-2.23348773841962\\
2.68	-2.24304347826087\\
2.69	-2.25260162601626\\
2.7	-2.26216216216216\\
2.71	-2.27172506738544\\
2.72	-2.28129032258065\\
2.73	-2.29085790884718\\
2.74	-2.30042780748663\\
2.75	-2.31\\
2.76	-2.31957446808511\\
2.77	-2.32915119363395\\
2.78	-2.33873015873016\\
2.79	-2.34831134564644\\
2.8	-2.35789473684211\\
2.81	-2.36748031496063\\
2.82	-2.37706806282722\\
2.83	-2.38665796344648\\
2.84	-2.39625\\
2.85	-2.40584415584416\\
2.86	-2.41544041450777\\
2.87	-2.42503875968992\\
2.88	-2.43463917525773\\
2.89	-2.44424164524422\\
2.9	-2.45384615384615\\
2.91	-2.46345268542199\\
2.92	-2.4730612244898\\
2.93	-2.48267175572519\\
2.94	-2.49228426395939\\
2.95	-2.50189873417722\\
2.96	-2.51151515151515\\
2.97	-2.52113350125945\\
2.98	-2.53075376884422\\
2.99	-2.54037593984962\\
3	-2.55\\
3.01	-2.55962593516209\\
3.02	-2.56925373134328\\
3.03	-2.57888337468983\\
3.04	-2.58851485148515\\
3.05	-2.59814814814815\\
3.06	-2.60778325123153\\
3.07	-2.61742014742015\\
3.08	-2.62705882352941\\
3.09	-2.63669926650367\\
3.1	-2.64634146341463\\
3.11	-2.65598540145985\\
3.12	-2.66563106796117\\
3.13	-2.6752784503632\\
3.14	-2.68492753623188\\
3.15	-2.69457831325301\\
3.16	-2.70423076923077\\
3.17	-2.71388489208633\\
3.18	-2.72354066985646\\
3.19	-2.73319809069212\\
3.2	-2.74285714285714\\
3.21	-2.75251781472684\\
3.22	-2.76218009478673\\
3.23	-2.77184397163121\\
3.24	-2.78150943396226\\
3.25	-2.79117647058824\\
3.26	-2.80084507042254\\
3.27	-2.81051522248244\\
3.28	-2.82018691588785\\
3.29	-2.82986013986014\\
3.3	-2.83953488372093\\
3.31	-2.84921113689095\\
3.32	-2.85888888888889\\
3.33	-2.86856812933025\\
3.34	-2.87824884792627\\
3.35	-2.88793103448276\\
3.36	-2.89761467889908\\
3.37	-2.90729977116705\\
3.38	-2.91698630136986\\
3.39	-2.92667425968109\\
3.4	-2.93636363636364\\
3.41	-2.94605442176871\\
3.42	-2.95574660633484\\
3.43	-2.96544018058691\\
3.44	-2.97513513513513\\
3.45	-2.98483146067416\\
3.46	-2.99452914798206\\
3.47	-3.00422818791946\\
3.48	-3.01392857142857\\
3.49	-3.02363028953229\\
3.5	-3.03333333333333\\
3.51	-3.0430376940133\\
3.52	-3.05274336283186\\
3.53	-3.06245033112583\\
3.54	-3.07215859030837\\
3.55	-3.08186813186813\\
3.56	-3.09157894736842\\
3.57	-3.10129102844639\\
3.58	-3.11100436681223\\
3.59	-3.12071895424837\\
3.6	-3.1304347826087\\
3.61	-3.14015184381779\\
3.62	-3.14987012987013\\
3.63	-3.15958963282937\\
3.64	-3.16931034482759\\
3.65	-3.17903225806452\\
3.66	-3.18875536480687\\
3.67	-3.19847965738758\\
3.68	-3.20820512820513\\
3.69	-3.21793176972281\\
3.7	-3.22765957446809\\
3.71	-3.23738853503185\\
3.72	-3.2471186440678\\
3.73	-3.25684989429175\\
3.74	-3.26658227848101\\
3.75	-3.27631578947368\\
3.76	-3.28605042016807\\
3.77	-3.29578616352201\\
3.78	-3.3055230125523\\
3.79	-3.31526096033403\\
3.8	-3.325\\
3.81	-3.33474012474012\\
3.82	-3.34448132780083\\
3.83	-3.35422360248447\\
3.84	-3.36396694214876\\
3.85	-3.37371134020619\\
3.86	-3.38345679012346\\
3.87	-3.39320328542094\\
3.88	-3.40295081967213\\
3.89	-3.41269938650307\\
3.9	-3.42244897959184\\
3.91	-3.43219959266802\\
3.92	-3.4419512195122\\
3.93	-3.45170385395538\\
3.94	-3.46145748987854\\
3.95	-3.47121212121212\\
3.96	-3.48096774193548\\
3.97	-3.49072434607646\\
3.98	-3.50048192771084\\
3.99	-3.51024048096192\\
4	-3.52\\
4.01	-3.52976047904192\\
4.02	-3.5395219123506\\
4.03	-3.54928429423459\\
4.04	-3.55904761904762\\
4.05	-3.56881188118812\\
4.06	-3.57857707509881\\
4.07	-3.58834319526627\\
4.08	-3.59811023622047\\
4.09	-3.60787819253438\\
4.1	-3.61764705882353\\
4.11	-3.6274168297456\\
4.12	-3.6371875\\
4.13	-3.64695906432749\\
4.14	-3.65673151750973\\
4.15	-3.66650485436893\\
4.16	-3.67627906976744\\
4.17	-3.68605415860735\\
4.18	-3.69583011583012\\
4.19	-3.70560693641618\\
4.2	-3.71538461538462\\
4.21	-3.72516314779271\\
4.22	-3.73494252873563\\
4.23	-3.74472275334608\\
4.24	-3.75450381679389\\
4.25	-3.76428571428571\\
4.26	-3.77406844106464\\
4.27	-3.78385199240987\\
4.28	-3.79363636363636\\
4.29	-3.80342155009452\\
4.3	-3.81320754716981\\
4.31	-3.82299435028249\\
4.32	-3.83278195488722\\
4.33	-3.8425703564728\\
4.34	-3.8523595505618\\
4.35	-3.86214953271028\\
4.36	-3.87194029850746\\
4.37	-3.88173184357542\\
4.38	-3.89152416356877\\
4.39	-3.9013172541744\\
4.4	-3.91111111111111\\
4.41	-3.92090573012939\\
4.42	-3.93070110701107\\
4.43	-3.94049723756906\\
4.44	-3.95029411764706\\
4.45	-3.96009174311927\\
4.46	-3.96989010989011\\
4.47	-3.97968921389397\\
4.48	-3.98948905109489\\
4.49	-3.99928961748634\\
4.5	-4.00909090909091\\
4.51	-4.01889292196007\\
4.52	-4.02869565217391\\
4.53	-4.03849909584087\\
4.54	-4.04830324909747\\
4.55	-4.05810810810811\\
4.56	-4.06791366906475\\
4.57	-4.07771992818672\\
4.58	-4.08752688172043\\
4.59	-4.09733452593918\\
4.6	-4.10714285714286\\
4.61	-4.11695187165775\\
4.62	-4.1267615658363\\
4.63	-4.13657193605684\\
4.64	-4.1463829787234\\
4.65	-4.15619469026549\\
4.66	-4.16600706713781\\
4.67	-4.17582010582011\\
4.68	-4.1856338028169\\
4.69	-4.19544815465729\\
4.7	-4.20526315789474\\
4.71	-4.21507880910683\\
4.72	-4.2248951048951\\
4.73	-4.23471204188482\\
4.74	-4.24452961672474\\
4.75	-4.25434782608696\\
4.76	-4.26416666666667\\
4.77	-4.27398613518197\\
4.78	-4.2838062283737\\
4.79	-4.29362694300518\\
4.8	-4.30344827586207\\
4.81	-4.31327022375215\\
4.82	-4.32309278350515\\
4.83	-4.33291595197256\\
4.84	-4.3427397260274\\
4.85	-4.3525641025641\\
4.86	-4.36238907849829\\
4.87	-4.37221465076661\\
4.88	-4.38204081632653\\
4.89	-4.3918675721562\\
4.9	-4.40169491525424\\
4.91	-4.41152284263959\\
4.92	-4.42135135135135\\
4.93	-4.43118043844857\\
4.94	-4.4410101010101\\
4.95	-4.45084033613445\\
4.96	-4.4606711409396\\
4.97	-4.47050251256281\\
4.98	-4.48033444816054\\
4.99	-4.49016694490818\\
5	-4.5\\
};
\addplot [color=mycolor1,solid,forget plot]
  table[row sep=crcr]{%
0	0\\
0.01	-0.00306930693069307\\
0.02	-0.00627450980392157\\
0.03	-0.00961165048543689\\
0.04	-0.0130769230769231\\
0.05	-0.0166666666666667\\
0.06	-0.020377358490566\\
0.07	-0.0242056074766355\\
0.08	-0.0281481481481482\\
0.09	-0.0322018348623853\\
0.1	-0.0363636363636364\\
0.11	-0.0406306306306306\\
0.12	-0.045\\
0.13	-0.0494690265486726\\
0.14	-0.0540350877192982\\
0.15	-0.058695652173913\\
0.16	-0.063448275862069\\
0.17	-0.0682905982905983\\
0.18	-0.0732203389830508\\
0.19	-0.078235294117647\\
0.2	-0.0833333333333333\\
0.21	-0.0885123966942149\\
0.22	-0.0937704918032787\\
0.23	-0.0991056910569106\\
0.24	-0.104516129032258\\
0.25	-0.11\\
0.26	-0.115555555555556\\
0.27	-0.121181102362205\\
0.28	-0.126875\\
0.29	-0.132635658914729\\
0.3	-0.138461538461538\\
0.31	-0.144351145038168\\
0.32	-0.15030303030303\\
0.33	-0.156315789473684\\
0.34	-0.162388059701493\\
0.35	-0.168518518518519\\
0.36	-0.174705882352941\\
0.37	-0.180948905109489\\
0.38	-0.187246376811594\\
0.39	-0.193597122302158\\
0.4	-0.2\\
0.41	-0.20645390070922\\
0.42	-0.212957746478873\\
0.43	-0.219510489510489\\
0.44	-0.226111111111111\\
0.45	-0.232758620689655\\
0.46	-0.239452054794521\\
0.47	-0.246190476190476\\
0.48	-0.252972972972973\\
0.49	-0.259798657718121\\
0.5	-0.266666666666667\\
0.51	-0.273576158940397\\
0.52	-0.280526315789474\\
0.53	-0.287516339869281\\
0.54	-0.294545454545455\\
0.55	-0.301612903225806\\
0.56	-0.308717948717949\\
0.57	-0.315859872611465\\
0.58	-0.323037974683544\\
0.59	-0.330251572327044\\
0.6	-0.3375\\
0.61	-0.344782608695652\\
0.62	-0.352098765432099\\
0.63	-0.359447852760736\\
0.64	-0.366829268292683\\
0.65	-0.374242424242424\\
0.66	-0.381686746987952\\
0.67	-0.389161676646707\\
0.68	-0.396666666666667\\
0.69	-0.404201183431953\\
0.7	-0.411764705882353\\
0.71	-0.419356725146199\\
0.72	-0.426976744186046\\
0.73	-0.434624277456647\\
0.74	-0.442298850574713\\
0.75	-0.45\\
0.76	-0.457727272727273\\
0.77	-0.465480225988701\\
0.78	-0.473258426966292\\
0.79	-0.481061452513966\\
0.8	-0.488888888888889\\
0.81	-0.496740331491713\\
0.82	-0.504615384615385\\
0.83	-0.512513661202186\\
0.84	-0.520434782608696\\
0.85	-0.528378378378378\\
0.86	-0.536344086021505\\
0.87	-0.544331550802139\\
0.88	-0.552340425531915\\
0.89	-0.56037037037037\\
0.9	-0.568421052631579\\
0.91	-0.576492146596859\\
0.92	-0.584583333333333\\
0.93	-0.592694300518135\\
0.94	-0.600824742268041\\
0.95	-0.608974358974359\\
0.96	-0.617142857142857\\
0.97	-0.625329949238579\\
0.98	-0.633535353535353\\
0.99	-0.641758793969849\\
1	-0.65\\
1.01	-0.658258706467662\\
1.02	-0.666534653465346\\
1.03	-0.674827586206897\\
1.04	-0.683137254901961\\
1.05	-0.691463414634146\\
1.06	-0.699805825242718\\
1.07	-0.70816425120773\\
1.08	-0.716538461538462\\
1.09	-0.724928229665072\\
1.1	-0.733333333333333\\
1.11	-0.74175355450237\\
1.12	-0.750188679245283\\
1.13	-0.758638497652582\\
1.14	-0.767102803738318\\
1.15	-0.775581395348837\\
1.16	-0.784074074074074\\
1.17	-0.79258064516129\\
1.18	-0.801100917431192\\
1.19	-0.809634703196347\\
1.2	-0.818181818181818\\
1.21	-0.826742081447964\\
1.22	-0.835315315315315\\
1.23	-0.84390134529148\\
1.24	-0.8525\\
1.25	-0.861111111111111\\
1.26	-0.869734513274336\\
1.27	-0.878370044052863\\
1.28	-0.887017543859649\\
1.29	-0.895676855895196\\
1.3	-0.904347826086956\\
1.31	-0.913030303030303\\
1.32	-0.921724137931035\\
1.33	-0.930429184549356\\
1.34	-0.939145299145299\\
1.35	-0.947872340425532\\
1.36	-0.956610169491525\\
1.37	-0.96535864978903\\
1.38	-0.974117647058824\\
1.39	-0.982887029288703\\
1.4	-0.991666666666667\\
1.41	-1.00045643153527\\
1.42	-1.00925619834711\\
1.43	-1.0180658436214\\
1.44	-1.02688524590164\\
1.45	-1.03571428571429\\
1.46	-1.04455284552846\\
1.47	-1.0534008097166\\
1.48	-1.06225806451613\\
1.49	-1.07112449799197\\
1.5	-1.08\\
1.51	-1.08888446215139\\
1.52	-1.09777777777778\\
1.53	-1.10667984189723\\
1.54	-1.1155905511811\\
1.55	-1.12450980392157\\
1.56	-1.1334375\\
1.57	-1.14237354085603\\
1.58	-1.15131782945736\\
1.59	-1.16027027027027\\
1.6	-1.16923076923077\\
1.61	-1.17819923371648\\
1.62	-1.18717557251908\\
1.63	-1.19615969581749\\
1.64	-1.20515151515152\\
1.65	-1.21415094339623\\
1.66	-1.22315789473684\\
1.67	-1.23217228464419\\
1.68	-1.24119402985075\\
1.69	-1.25022304832714\\
1.7	-1.25925925925926\\
1.71	-1.26830258302583\\
1.72	-1.27735294117647\\
1.73	-1.28641025641026\\
1.74	-1.29547445255474\\
1.75	-1.30454545454545\\
1.76	-1.3136231884058\\
1.77	-1.32270758122744\\
1.78	-1.33179856115108\\
1.79	-1.34089605734767\\
1.8	-1.35\\
1.81	-1.3591103202847\\
1.82	-1.36822695035461\\
1.83	-1.37734982332155\\
1.84	-1.38647887323944\\
1.85	-1.39561403508772\\
1.86	-1.40475524475524\\
1.87	-1.41390243902439\\
1.88	-1.42305555555556\\
1.89	-1.43221453287197\\
1.9	-1.44137931034483\\
1.91	-1.45054982817869\\
1.92	-1.45972602739726\\
1.93	-1.46890784982935\\
1.94	-1.47809523809524\\
1.95	-1.48728813559322\\
1.96	-1.49648648648649\\
1.97	-1.50569023569024\\
1.98	-1.51489932885906\\
1.99	-1.52411371237458\\
2	-1.53333333333333\\
2.01	-1.54255813953488\\
2.02	-1.5517880794702\\
2.03	-1.56102310231023\\
2.04	-1.57026315789474\\
2.05	-1.57950819672131\\
2.06	-1.58875816993464\\
2.07	-1.59801302931596\\
2.08	-1.60727272727273\\
2.09	-1.61653721682848\\
2.1	-1.6258064516129\\
2.11	-1.63508038585209\\
2.12	-1.64435897435897\\
2.13	-1.65364217252396\\
2.14	-1.66292993630573\\
2.15	-1.67222222222222\\
2.16	-1.68151898734177\\
2.17	-1.69082018927445\\
2.18	-1.70012578616352\\
2.19	-1.70943573667712\\
2.2	-1.71875\\
2.21	-1.72806853582555\\
2.22	-1.73739130434783\\
2.23	-1.74671826625387\\
2.24	-1.75604938271605\\
2.25	-1.76538461538462\\
2.26	-1.77472392638037\\
2.27	-1.78406727828746\\
2.28	-1.79341463414634\\
2.29	-1.80276595744681\\
2.3	-1.81212121212121\\
2.31	-1.82148036253776\\
2.32	-1.83084337349398\\
2.33	-1.84021021021021\\
2.34	-1.84958083832335\\
2.35	-1.8589552238806\\
2.36	-1.86833333333333\\
2.37	-1.87771513353116\\
2.38	-1.88710059171598\\
2.39	-1.89648967551622\\
2.4	-1.90588235294118\\
2.41	-1.91527859237537\\
2.42	-1.9246783625731\\
2.43	-1.93408163265306\\
2.44	-1.94348837209302\\
2.45	-1.95289855072464\\
2.46	-1.96231213872832\\
2.47	-1.97172910662824\\
2.48	-1.98114942528736\\
2.49	-1.99057306590258\\
2.5	-2\\
2.51	-2.0094301994302\\
2.52	-2.01886363636364\\
2.53	-2.02830028328612\\
2.54	-2.03774011299435\\
2.55	-2.04718309859155\\
2.56	-2.05662921348315\\
2.57	-2.06607843137255\\
2.58	-2.07553072625698\\
2.59	-2.0849860724234\\
2.6	-2.09444444444444\\
2.61	-2.10390581717451\\
2.62	-2.11337016574586\\
2.63	-2.12283746556474\\
2.64	-2.13230769230769\\
2.65	-2.14178082191781\\
2.66	-2.15125683060109\\
2.67	-2.16073569482289\\
2.68	-2.17021739130435\\
2.69	-2.17970189701897\\
2.7	-2.18918918918919\\
2.71	-2.19867924528302\\
2.72	-2.20817204301075\\
2.73	-2.21766756032172\\
2.74	-2.22716577540107\\
2.75	-2.23666666666667\\
2.76	-2.24617021276596\\
2.77	-2.25567639257294\\
2.78	-2.26518518518519\\
2.79	-2.27469656992084\\
2.8	-2.28421052631579\\
2.81	-2.29372703412073\\
2.82	-2.30324607329843\\
2.83	-2.31276762402089\\
2.84	-2.32229166666667\\
2.85	-2.33181818181818\\
2.86	-2.34134715025907\\
2.87	-2.35087855297158\\
2.88	-2.36041237113402\\
2.89	-2.36994858611825\\
2.9	-2.37948717948718\\
2.91	-2.38902813299233\\
2.92	-2.39857142857143\\
2.93	-2.40811704834606\\
2.94	-2.41766497461929\\
2.95	-2.42721518987342\\
2.96	-2.43676767676768\\
2.97	-2.44632241813602\\
2.98	-2.45587939698492\\
2.99	-2.46543859649123\\
3	-2.475\\
3.01	-2.48456359102244\\
3.02	-2.49412935323383\\
3.03	-2.50369727047146\\
3.04	-2.51326732673267\\
3.05	-2.52283950617284\\
3.06	-2.53241379310345\\
3.07	-2.54199017199017\\
3.08	-2.55156862745098\\
3.09	-2.56114914425428\\
3.1	-2.57073170731707\\
3.11	-2.58031630170316\\
3.12	-2.58990291262136\\
3.13	-2.59949152542373\\
3.14	-2.60908212560386\\
3.15	-2.61867469879518\\
3.16	-2.62826923076923\\
3.17	-2.63786570743405\\
3.18	-2.64746411483254\\
3.19	-2.65706443914081\\
3.2	-2.66666666666667\\
3.21	-2.67627078384798\\
3.22	-2.68587677725118\\
3.23	-2.69548463356974\\
3.24	-2.70509433962264\\
3.25	-2.71470588235294\\
3.26	-2.72431924882629\\
3.27	-2.73393442622951\\
3.28	-2.74355140186916\\
3.29	-2.75317016317016\\
3.3	-2.76279069767442\\
3.31	-2.77241299303944\\
3.32	-2.78203703703704\\
3.33	-2.79166281755196\\
3.34	-2.80129032258064\\
3.35	-2.81091954022988\\
3.36	-2.8205504587156\\
3.37	-2.83018306636156\\
3.38	-2.83981735159817\\
3.39	-2.84945330296128\\
3.4	-2.85909090909091\\
3.41	-2.86873015873016\\
3.42	-2.87837104072398\\
3.43	-2.88801354401806\\
3.44	-2.89765765765766\\
3.45	-2.90730337078652\\
3.46	-2.91695067264574\\
3.47	-2.92659955257271\\
3.48	-2.93625\\
3.49	-2.94590200445434\\
3.5	-2.95555555555556\\
3.51	-2.96521064301552\\
3.52	-2.97486725663717\\
3.53	-2.98452538631347\\
3.54	-2.99418502202643\\
3.55	-3.00384615384615\\
3.56	-3.01350877192982\\
3.57	-3.02317286652079\\
3.58	-3.0328384279476\\
3.59	-3.04250544662309\\
3.6	-3.05217391304348\\
3.61	-3.06184381778742\\
3.62	-3.07151515151515\\
3.63	-3.0811879049676\\
3.64	-3.09086206896552\\
3.65	-3.1005376344086\\
3.66	-3.11021459227468\\
3.67	-3.11989293361884\\
3.68	-3.12957264957265\\
3.69	-3.13925373134328\\
3.7	-3.14893617021277\\
3.71	-3.15861995753715\\
3.72	-3.16830508474576\\
3.73	-3.17799154334038\\
3.74	-3.18767932489451\\
3.75	-3.19736842105263\\
3.76	-3.20705882352941\\
3.77	-3.21675052410901\\
3.78	-3.22644351464435\\
3.79	-3.23613778705637\\
3.8	-3.24583333333333\\
3.81	-3.25553014553015\\
3.82	-3.26522821576763\\
3.83	-3.27492753623188\\
3.84	-3.28462809917355\\
3.85	-3.29432989690722\\
3.86	-3.3040329218107\\
3.87	-3.31373716632444\\
3.88	-3.32344262295082\\
3.89	-3.33314928425358\\
3.9	-3.34285714285714\\
3.91	-3.35256619144603\\
3.92	-3.36227642276423\\
3.93	-3.3719878296146\\
3.94	-3.3817004048583\\
3.95	-3.39141414141414\\
3.96	-3.40112903225806\\
3.97	-3.41084507042254\\
3.98	-3.42056224899598\\
3.99	-3.43028056112224\\
4	-3.44\\
4.01	-3.44972055888224\\
4.02	-3.4594422310757\\
4.03	-3.46916500994036\\
4.04	-3.47888888888889\\
4.05	-3.48861386138614\\
4.06	-3.49833992094862\\
4.07	-3.50806706114398\\
4.08	-3.51779527559055\\
4.09	-3.52752455795678\\
4.1	-3.53725490196078\\
4.11	-3.54698630136986\\
4.12	-3.55671875\\
4.13	-3.5664522417154\\
4.14	-3.57618677042802\\
4.15	-3.58592233009709\\
4.16	-3.59565891472868\\
4.17	-3.60539651837524\\
4.18	-3.61513513513513\\
4.19	-3.62487475915222\\
4.2	-3.63461538461538\\
4.21	-3.64435700575816\\
4.22	-3.65409961685824\\
4.23	-3.66384321223709\\
4.24	-3.67358778625954\\
4.25	-3.68333333333333\\
4.26	-3.69307984790874\\
4.27	-3.70282732447818\\
4.28	-3.71257575757576\\
4.29	-3.72232514177694\\
4.3	-3.73207547169811\\
4.31	-3.74182674199623\\
4.32	-3.75157894736842\\
4.33	-3.76133208255159\\
4.34	-3.7710861423221\\
4.35	-3.78084112149533\\
4.36	-3.79059701492537\\
4.37	-3.80035381750466\\
4.38	-3.81011152416357\\
4.39	-3.81987012987013\\
4.4	-3.82962962962963\\
4.41	-3.83939001848429\\
4.42	-3.84915129151292\\
4.43	-3.85891344383057\\
4.44	-3.86867647058823\\
4.45	-3.87844036697248\\
4.46	-3.88820512820513\\
4.47	-3.89797074954296\\
4.48	-3.90773722627737\\
4.49	-3.91750455373406\\
4.5	-3.92727272727273\\
4.51	-3.93704174228675\\
4.52	-3.9468115942029\\
4.53	-3.95658227848101\\
4.54	-3.96635379061372\\
4.55	-3.97612612612613\\
4.56	-3.98589928057554\\
4.57	-3.99567324955117\\
4.58	-4.00544802867384\\
4.59	-4.01522361359571\\
4.6	-4.025\\
4.61	-4.03477718360071\\
4.62	-4.04455516014235\\
4.63	-4.05433392539964\\
4.64	-4.0641134751773\\
4.65	-4.07389380530974\\
4.66	-4.08367491166078\\
4.67	-4.09345679012346\\
4.68	-4.10323943661972\\
4.69	-4.11302284710018\\
4.7	-4.12280701754386\\
4.71	-4.13259194395797\\
4.72	-4.14237762237762\\
4.73	-4.15216404886562\\
4.74	-4.1619512195122\\
4.75	-4.17173913043478\\
4.76	-4.18152777777778\\
4.77	-4.1913171577123\\
4.78	-4.20110726643599\\
4.79	-4.21089810017271\\
4.8	-4.22068965517241\\
4.81	-4.23048192771084\\
4.82	-4.24027491408935\\
4.83	-4.25006861063465\\
4.84	-4.25986301369863\\
4.85	-4.26965811965812\\
4.86	-4.27945392491468\\
4.87	-4.28925042589438\\
4.88	-4.29904761904762\\
4.89	-4.3088455008489\\
4.9	-4.31864406779661\\
4.91	-4.32844331641286\\
4.92	-4.33824324324324\\
4.93	-4.34804384485666\\
4.94	-4.35784511784512\\
4.95	-4.36764705882353\\
4.96	-4.37744966442953\\
4.97	-4.38725293132328\\
4.98	-4.39705685618729\\
4.99	-4.40686143572621\\
5	-4.41666666666667\\
};
\addplot [color=mycolor2,solid,forget plot]
  table[row sep=crcr]{%
0	0\\
0.01	-0.00207920792079208\\
0.02	-0.00431372549019608\\
0.03	-0.00669902912621359\\
0.04	-0.00923076923076923\\
0.05	-0.0119047619047619\\
0.06	-0.0147169811320755\\
0.07	-0.0176635514018692\\
0.08	-0.0207407407407407\\
0.09	-0.0239449541284404\\
0.1	-0.0272727272727273\\
0.11	-0.0307207207207207\\
0.12	-0.0342857142857143\\
0.13	-0.0379646017699115\\
0.14	-0.0417543859649123\\
0.15	-0.0456521739130435\\
0.16	-0.0496551724137931\\
0.17	-0.0537606837606837\\
0.18	-0.0579661016949152\\
0.19	-0.0622689075630252\\
0.2	-0.0666666666666667\\
0.21	-0.0711570247933884\\
0.22	-0.0757377049180328\\
0.23	-0.0804065040650406\\
0.24	-0.0851612903225806\\
0.25	-0.09\\
0.26	-0.0949206349206349\\
0.27	-0.0999212598425197\\
0.28	-0.105\\
0.29	-0.11015503875969\\
0.3	-0.115384615384615\\
0.31	-0.120687022900763\\
0.32	-0.126060606060606\\
0.33	-0.131503759398496\\
0.34	-0.137014925373134\\
0.35	-0.142592592592593\\
0.36	-0.148235294117647\\
0.37	-0.153941605839416\\
0.38	-0.159710144927536\\
0.39	-0.165539568345324\\
0.4	-0.171428571428571\\
0.41	-0.177375886524823\\
0.42	-0.183380281690141\\
0.43	-0.189440559440559\\
0.44	-0.195555555555556\\
0.45	-0.201724137931034\\
0.46	-0.207945205479452\\
0.47	-0.21421768707483\\
0.48	-0.220540540540541\\
0.49	-0.226912751677852\\
0.5	-0.233333333333333\\
0.51	-0.239801324503311\\
0.52	-0.246315789473684\\
0.53	-0.252875816993464\\
0.54	-0.259480519480519\\
0.55	-0.266129032258065\\
0.56	-0.272820512820513\\
0.57	-0.279554140127389\\
0.58	-0.286329113924051\\
0.59	-0.29314465408805\\
0.6	-0.3\\
0.61	-0.306894409937888\\
0.62	-0.313827160493827\\
0.63	-0.32079754601227\\
0.64	-0.327804878048781\\
0.65	-0.334848484848485\\
0.66	-0.341927710843374\\
0.67	-0.349041916167665\\
0.68	-0.356190476190476\\
0.69	-0.363372781065089\\
0.7	-0.370588235294118\\
0.71	-0.377836257309941\\
0.72	-0.385116279069767\\
0.73	-0.39242774566474\\
0.74	-0.399770114942529\\
0.75	-0.407142857142857\\
0.76	-0.414545454545454\\
0.77	-0.421977401129944\\
0.78	-0.429438202247191\\
0.79	-0.436927374301676\\
0.8	-0.444444444444444\\
0.81	-0.451988950276243\\
0.82	-0.45956043956044\\
0.83	-0.467158469945355\\
0.84	-0.474782608695652\\
0.85	-0.482432432432432\\
0.86	-0.49010752688172\\
0.87	-0.497807486631016\\
0.88	-0.505531914893617\\
0.89	-0.513280423280423\\
0.9	-0.521052631578947\\
0.91	-0.528848167539267\\
0.92	-0.536666666666667\\
0.93	-0.544507772020725\\
0.94	-0.552371134020619\\
0.95	-0.56025641025641\\
0.96	-0.568163265306122\\
0.97	-0.576091370558376\\
0.98	-0.584040404040404\\
0.99	-0.592010050251256\\
1	-0.6\\
1.01	-0.608009950248756\\
1.02	-0.616039603960396\\
1.03	-0.624088669950739\\
1.04	-0.632156862745098\\
1.05	-0.640243902439024\\
1.06	-0.648349514563107\\
1.07	-0.656473429951691\\
1.08	-0.664615384615385\\
1.09	-0.672775119617225\\
1.1	-0.680952380952381\\
1.11	-0.68914691943128\\
1.12	-0.697358490566038\\
1.13	-0.705586854460094\\
1.14	-0.713831775700935\\
1.15	-0.722093023255814\\
1.16	-0.73037037037037\\
1.17	-0.738663594470046\\
1.18	-0.74697247706422\\
1.19	-0.755296803652968\\
1.2	-0.763636363636364\\
1.21	-0.771990950226244\\
1.22	-0.78036036036036\\
1.23	-0.788744394618834\\
1.24	-0.797142857142857\\
1.25	-0.805555555555556\\
1.26	-0.813982300884956\\
1.27	-0.822422907488987\\
1.28	-0.830877192982456\\
1.29	-0.839344978165939\\
1.3	-0.847826086956522\\
1.31	-0.856320346320346\\
1.32	-0.864827586206897\\
1.33	-0.873347639484979\\
1.34	-0.881880341880342\\
1.35	-0.890425531914894\\
1.36	-0.898983050847458\\
1.37	-0.907552742616034\\
1.38	-0.916134453781513\\
1.39	-0.924728033472803\\
1.4	-0.933333333333333\\
1.41	-0.94195020746888\\
1.42	-0.950578512396694\\
1.43	-0.959218106995885\\
1.44	-0.967868852459016\\
1.45	-0.976530612244898\\
1.46	-0.98520325203252\\
1.47	-0.993886639676113\\
1.48	-1.00258064516129\\
1.49	-1.01128514056225\\
1.5	-1.02\\
1.51	-1.02872509960159\\
1.52	-1.03746031746032\\
1.53	-1.04620553359684\\
1.54	-1.05496062992126\\
1.55	-1.06372549019608\\
1.56	-1.0725\\
1.57	-1.08128404669261\\
1.58	-1.09007751937985\\
1.59	-1.09888030888031\\
1.6	-1.10769230769231\\
1.61	-1.11651340996169\\
1.62	-1.12534351145038\\
1.63	-1.1341825095057\\
1.64	-1.1430303030303\\
1.65	-1.15188679245283\\
1.66	-1.16075187969925\\
1.67	-1.16962546816479\\
1.68	-1.17850746268657\\
1.69	-1.18739776951673\\
1.7	-1.1962962962963\\
1.71	-1.20520295202952\\
1.72	-1.21411764705882\\
1.73	-1.22304029304029\\
1.74	-1.23197080291971\\
1.75	-1.24090909090909\\
1.76	-1.24985507246377\\
1.77	-1.25880866425993\\
1.78	-1.26776978417266\\
1.79	-1.27673835125448\\
1.8	-1.28571428571429\\
1.81	-1.2946975088968\\
1.82	-1.30368794326241\\
1.83	-1.31268551236749\\
1.84	-1.32169014084507\\
1.85	-1.33070175438597\\
1.86	-1.33972027972028\\
1.87	-1.3487456445993\\
1.88	-1.35777777777778\\
1.89	-1.36681660899654\\
1.9	-1.37586206896552\\
1.91	-1.38491408934708\\
1.92	-1.39397260273973\\
1.93	-1.40303754266212\\
1.94	-1.41210884353741\\
1.95	-1.42118644067797\\
1.96	-1.43027027027027\\
1.97	-1.43936026936027\\
1.98	-1.44845637583893\\
1.99	-1.45755852842809\\
2	-1.46666666666667\\
2.01	-1.47578073089701\\
2.02	-1.48490066225166\\
2.03	-1.49402640264026\\
2.04	-1.50315789473684\\
2.05	-1.51229508196721\\
2.06	-1.52143790849673\\
2.07	-1.53058631921824\\
2.08	-1.53974025974026\\
2.09	-1.5488996763754\\
2.1	-1.55806451612903\\
2.11	-1.5672347266881\\
2.12	-1.57641025641026\\
2.13	-1.5855910543131\\
2.14	-1.59477707006369\\
2.15	-1.60396825396825\\
2.16	-1.61316455696203\\
2.17	-1.62236593059937\\
2.18	-1.63157232704403\\
2.19	-1.64078369905956\\
2.2	-1.65\\
2.21	-1.65922118380062\\
2.22	-1.66844720496894\\
2.23	-1.67767801857585\\
2.24	-1.68691358024691\\
2.25	-1.69615384615385\\
2.26	-1.70539877300614\\
2.27	-1.71464831804281\\
2.28	-1.72390243902439\\
2.29	-1.73316109422492\\
2.3	-1.74242424242424\\
2.31	-1.7516918429003\\
2.32	-1.76096385542169\\
2.33	-1.77024024024024\\
2.34	-1.77952095808383\\
2.35	-1.78880597014925\\
2.36	-1.79809523809524\\
2.37	-1.80738872403561\\
2.38	-1.81668639053254\\
2.39	-1.82598820058997\\
2.4	-1.83529411764706\\
2.41	-1.84460410557185\\
2.42	-1.85391812865497\\
2.43	-1.8632361516035\\
2.44	-1.87255813953488\\
2.45	-1.88188405797101\\
2.46	-1.89121387283237\\
2.47	-1.90054755043228\\
2.48	-1.90988505747126\\
2.49	-1.91922636103152\\
2.5	-1.92857142857143\\
2.51	-1.93792022792023\\
2.52	-1.94727272727273\\
2.53	-1.95662889518414\\
2.54	-1.96598870056497\\
2.55	-1.97535211267606\\
2.56	-1.9847191011236\\
2.57	-1.99408963585434\\
2.58	-2.00346368715084\\
2.59	-2.01284122562674\\
2.6	-2.02222222222222\\
2.61	-2.03160664819945\\
2.62	-2.04099447513812\\
2.63	-2.05038567493113\\
2.64	-2.05978021978022\\
2.65	-2.06917808219178\\
2.66	-2.07857923497268\\
2.67	-2.08798365122616\\
2.68	-2.09739130434783\\
2.69	-2.10680216802168\\
2.7	-2.11621621621622\\
2.71	-2.12563342318059\\
2.72	-2.13505376344086\\
2.73	-2.14447721179625\\
2.74	-2.15390374331551\\
2.75	-2.16333333333333\\
2.76	-2.17276595744681\\
2.77	-2.18220159151194\\
2.78	-2.19164021164021\\
2.79	-2.20108179419525\\
2.8	-2.21052631578947\\
2.81	-2.21997375328084\\
2.82	-2.22942408376963\\
2.83	-2.2388772845953\\
2.84	-2.24833333333333\\
2.85	-2.25779220779221\\
2.86	-2.26725388601036\\
2.87	-2.27671834625323\\
2.88	-2.28618556701031\\
2.89	-2.29565552699229\\
2.9	-2.3051282051282\\
2.91	-2.31460358056266\\
2.92	-2.32408163265306\\
2.93	-2.33356234096692\\
2.94	-2.34304568527919\\
2.95	-2.35253164556962\\
2.96	-2.3620202020202\\
2.97	-2.37151133501259\\
2.98	-2.38100502512563\\
2.99	-2.39050125313283\\
3	-2.4\\
3.01	-2.40950124688279\\
3.02	-2.41900497512438\\
3.03	-2.4285111662531\\
3.04	-2.4380198019802\\
3.05	-2.44753086419753\\
3.06	-2.45704433497537\\
3.07	-2.4665601965602\\
3.08	-2.47607843137255\\
3.09	-2.48559902200489\\
3.1	-2.49512195121951\\
3.11	-2.50464720194647\\
3.12	-2.51417475728155\\
3.13	-2.52370460048426\\
3.14	-2.53323671497585\\
3.15	-2.54277108433735\\
3.16	-2.55230769230769\\
3.17	-2.56184652278177\\
3.18	-2.57138755980861\\
3.19	-2.5809307875895\\
3.2	-2.59047619047619\\
3.21	-2.60002375296912\\
3.22	-2.60957345971564\\
3.23	-2.61912529550827\\
3.24	-2.62867924528302\\
3.25	-2.63823529411765\\
3.26	-2.64779342723005\\
3.27	-2.65735362997658\\
3.28	-2.66691588785047\\
3.29	-2.67648018648019\\
3.3	-2.68604651162791\\
3.31	-2.69561484918794\\
3.32	-2.70518518518519\\
3.33	-2.71475750577367\\
3.34	-2.72433179723502\\
3.35	-2.73390804597701\\
3.36	-2.74348623853211\\
3.37	-2.75306636155606\\
3.38	-2.76264840182648\\
3.39	-2.77223234624146\\
3.4	-2.78181818181818\\
3.41	-2.79140589569161\\
3.42	-2.80099547511312\\
3.43	-2.81058690744921\\
3.44	-2.82018018018018\\
3.45	-2.82977528089888\\
3.46	-2.83937219730942\\
3.47	-2.84897091722595\\
3.48	-2.85857142857143\\
3.49	-2.86817371937639\\
3.5	-2.87777777777778\\
3.51	-2.88738359201774\\
3.52	-2.89699115044248\\
3.53	-2.9066004415011\\
3.54	-2.91621145374449\\
3.55	-2.92582417582418\\
3.56	-2.93543859649123\\
3.57	-2.94505470459519\\
3.58	-2.95467248908297\\
3.59	-2.96429193899782\\
3.6	-2.97391304347826\\
3.61	-2.98353579175705\\
3.62	-2.99316017316017\\
3.63	-3.00278617710583\\
3.64	-3.01241379310345\\
3.65	-3.02204301075269\\
3.66	-3.03167381974249\\
3.67	-3.04130620985011\\
3.68	-3.05094017094017\\
3.69	-3.06057569296375\\
3.7	-3.07021276595745\\
3.71	-3.07985138004246\\
3.72	-3.08949152542373\\
3.73	-3.09913319238901\\
3.74	-3.10877637130802\\
3.75	-3.11842105263158\\
3.76	-3.12806722689076\\
3.77	-3.13771488469602\\
3.78	-3.1473640167364\\
3.79	-3.15701461377871\\
3.8	-3.16666666666667\\
3.81	-3.17632016632017\\
3.82	-3.18597510373444\\
3.83	-3.1956314699793\\
3.84	-3.20528925619835\\
3.85	-3.21494845360825\\
3.86	-3.22460905349794\\
3.87	-3.23427104722793\\
3.88	-3.24393442622951\\
3.89	-3.25359918200409\\
3.9	-3.26326530612245\\
3.91	-3.27293279022403\\
3.92	-3.28260162601626\\
3.93	-3.29227180527383\\
3.94	-3.30194331983806\\
3.95	-3.31161616161616\\
3.96	-3.32129032258065\\
3.97	-3.33096579476861\\
3.98	-3.34064257028112\\
3.99	-3.35032064128257\\
4	-3.36\\
4.01	-3.36968063872255\\
4.02	-3.3793625498008\\
4.03	-3.38904572564612\\
4.04	-3.39873015873016\\
4.05	-3.40841584158416\\
4.06	-3.41810276679842\\
4.07	-3.4277909270217\\
4.08	-3.43748031496063\\
4.09	-3.44717092337917\\
4.1	-3.45686274509804\\
4.11	-3.46655577299413\\
4.12	-3.47625\\
4.13	-3.48594541910331\\
4.14	-3.4956420233463\\
4.15	-3.50533980582524\\
4.16	-3.51503875968992\\
4.17	-3.52473887814313\\
4.18	-3.53444015444015\\
4.19	-3.54414258188825\\
4.2	-3.55384615384615\\
4.21	-3.56355086372361\\
4.22	-3.57325670498084\\
4.23	-3.58296367112811\\
4.24	-3.59267175572519\\
4.25	-3.60238095238095\\
4.26	-3.61209125475285\\
4.27	-3.62180265654649\\
4.28	-3.63151515151515\\
4.29	-3.64122873345936\\
4.3	-3.65094339622642\\
4.31	-3.66065913370998\\
4.32	-3.67037593984962\\
4.33	-3.68009380863039\\
4.34	-3.6898127340824\\
4.35	-3.69953271028037\\
4.36	-3.70925373134328\\
4.37	-3.71897579143389\\
4.38	-3.72869888475836\\
4.39	-3.73842300556586\\
4.4	-3.74814814814815\\
4.41	-3.75787430683919\\
4.42	-3.76760147601476\\
4.43	-3.77732965009208\\
4.44	-3.78705882352941\\
4.45	-3.79678899082569\\
4.46	-3.80652014652015\\
4.47	-3.81625228519196\\
4.48	-3.82598540145985\\
4.49	-3.83571948998179\\
4.5	-3.84545454545455\\
4.51	-3.85519056261343\\
4.52	-3.86492753623188\\
4.53	-3.87466546112116\\
4.54	-3.88440433212996\\
4.55	-3.89414414414414\\
4.56	-3.90388489208633\\
4.57	-3.91362657091562\\
4.58	-3.92336917562724\\
4.59	-3.93311270125224\\
4.6	-3.94285714285714\\
4.61	-3.95260249554367\\
4.62	-3.9623487544484\\
4.63	-3.97209591474245\\
4.64	-3.98184397163121\\
4.65	-3.99159292035398\\
4.66	-4.00134275618375\\
4.67	-4.01109347442681\\
4.68	-4.02084507042253\\
4.69	-4.03059753954306\\
4.7	-4.04035087719298\\
4.71	-4.05010507880911\\
4.72	-4.05986013986014\\
4.73	-4.06961605584642\\
4.74	-4.07937282229965\\
4.75	-4.08913043478261\\
4.76	-4.09888888888889\\
4.77	-4.10864818024263\\
4.78	-4.11840830449827\\
4.79	-4.12816925734024\\
4.8	-4.13793103448276\\
4.81	-4.14769363166953\\
4.82	-4.15745704467354\\
4.83	-4.16722126929674\\
4.84	-4.17698630136986\\
4.85	-4.18675213675214\\
4.86	-4.19651877133106\\
4.87	-4.20628620102215\\
4.88	-4.21605442176871\\
4.89	-4.2258234295416\\
4.9	-4.23559322033898\\
4.91	-4.24536379018613\\
4.92	-4.25513513513514\\
4.93	-4.26490725126475\\
4.94	-4.27468013468014\\
4.95	-4.2844537815126\\
4.96	-4.29422818791946\\
4.97	-4.30400335008375\\
4.98	-4.31377926421405\\
4.99	-4.32355592654424\\
5	-4.33333333333333\\
};
\addplot [color=mycolor3,solid,forget plot]
  table[row sep=crcr]{%
0	0\\
0.01	-0.00108910891089109\\
0.02	-0.00235294117647059\\
0.03	-0.00378640776699029\\
0.04	-0.00538461538461538\\
0.05	-0.00714285714285715\\
0.06	-0.0090566037735849\\
0.07	-0.0111214953271028\\
0.08	-0.0133333333333333\\
0.09	-0.0156880733944954\\
0.1	-0.0181818181818182\\
0.11	-0.0208108108108108\\
0.12	-0.0235714285714286\\
0.13	-0.0264601769911504\\
0.14	-0.0294736842105263\\
0.15	-0.0326086956521739\\
0.16	-0.0358620689655172\\
0.17	-0.0392307692307692\\
0.18	-0.0427118644067797\\
0.19	-0.0463025210084033\\
0.2	-0.05\\
0.21	-0.053801652892562\\
0.22	-0.0577049180327869\\
0.23	-0.0617073170731707\\
0.24	-0.0658064516129032\\
0.25	-0.07\\
0.26	-0.0742857142857143\\
0.27	-0.0786614173228347\\
0.28	-0.083125\\
0.29	-0.0876744186046511\\
0.3	-0.0923076923076923\\
0.31	-0.0970229007633588\\
0.32	-0.101818181818182\\
0.33	-0.106691729323308\\
0.34	-0.111641791044776\\
0.35	-0.116666666666667\\
0.36	-0.121764705882353\\
0.37	-0.126934306569343\\
0.38	-0.132173913043478\\
0.39	-0.137482014388489\\
0.4	-0.142857142857143\\
0.41	-0.148297872340426\\
0.42	-0.153802816901408\\
0.43	-0.159370629370629\\
0.44	-0.165\\
0.45	-0.170689655172414\\
0.46	-0.176438356164384\\
0.47	-0.182244897959184\\
0.48	-0.188108108108108\\
0.49	-0.194026845637584\\
0.5	-0.2\\
0.51	-0.206026490066225\\
0.52	-0.212105263157895\\
0.53	-0.218235294117647\\
0.54	-0.224415584415584\\
0.55	-0.230645161290323\\
0.56	-0.236923076923077\\
0.57	-0.243248407643312\\
0.58	-0.249620253164557\\
0.59	-0.256037735849057\\
0.6	-0.2625\\
0.61	-0.269006211180124\\
0.62	-0.275555555555556\\
0.63	-0.282147239263804\\
0.64	-0.288780487804878\\
0.65	-0.295454545454545\\
0.66	-0.302168674698795\\
0.67	-0.308922155688623\\
0.68	-0.315714285714286\\
0.69	-0.322544378698225\\
0.7	-0.329411764705882\\
0.71	-0.336315789473684\\
0.72	-0.343255813953488\\
0.73	-0.350231213872832\\
0.74	-0.357241379310345\\
0.75	-0.364285714285714\\
0.76	-0.371363636363636\\
0.77	-0.378474576271186\\
0.78	-0.38561797752809\\
0.79	-0.392793296089385\\
0.8	-0.4\\
0.81	-0.407237569060774\\
0.82	-0.414505494505495\\
0.83	-0.421803278688525\\
0.84	-0.429130434782609\\
0.85	-0.436486486486487\\
0.86	-0.443870967741935\\
0.87	-0.451283422459893\\
0.88	-0.458723404255319\\
0.89	-0.466190476190476\\
0.9	-0.473684210526316\\
0.91	-0.481204188481675\\
0.92	-0.48875\\
0.93	-0.496321243523316\\
0.94	-0.503917525773196\\
0.95	-0.511538461538462\\
0.96	-0.519183673469388\\
0.97	-0.526852791878172\\
0.98	-0.534545454545454\\
0.99	-0.542261306532663\\
1	-0.55\\
1.01	-0.557761194029851\\
1.02	-0.565544554455446\\
1.03	-0.573349753694581\\
1.04	-0.581176470588235\\
1.05	-0.589024390243902\\
1.06	-0.596893203883495\\
1.07	-0.604782608695652\\
1.08	-0.612692307692308\\
1.09	-0.620622009569378\\
1.1	-0.628571428571429\\
1.11	-0.63654028436019\\
1.12	-0.644528301886792\\
1.13	-0.652535211267606\\
1.14	-0.660560747663552\\
1.15	-0.668604651162791\\
1.16	-0.676666666666667\\
1.17	-0.684746543778802\\
1.18	-0.692844036697248\\
1.19	-0.700958904109589\\
1.2	-0.709090909090909\\
1.21	-0.717239819004525\\
1.22	-0.725405405405405\\
1.23	-0.733587443946188\\
1.24	-0.741785714285714\\
1.25	-0.75\\
1.26	-0.758230088495575\\
1.27	-0.76647577092511\\
1.28	-0.774736842105263\\
1.29	-0.783013100436681\\
1.3	-0.791304347826087\\
1.31	-0.79961038961039\\
1.32	-0.807931034482759\\
1.33	-0.816266094420601\\
1.34	-0.824615384615385\\
1.35	-0.832978723404255\\
1.36	-0.84135593220339\\
1.37	-0.849746835443038\\
1.38	-0.858151260504202\\
1.39	-0.866569037656904\\
1.4	-0.875\\
1.41	-0.88344398340249\\
1.42	-0.891900826446281\\
1.43	-0.90037037037037\\
1.44	-0.908852459016393\\
1.45	-0.91734693877551\\
1.46	-0.925853658536585\\
1.47	-0.934372469635627\\
1.48	-0.942903225806452\\
1.49	-0.95144578313253\\
1.5	-0.96\\
1.51	-0.968565737051793\\
1.52	-0.977142857142857\\
1.53	-0.985731225296443\\
1.54	-0.994330708661417\\
1.55	-1.00294117647059\\
1.56	-1.0115625\\
1.57	-1.02019455252918\\
1.58	-1.02883720930233\\
1.59	-1.03749034749035\\
1.6	-1.04615384615385\\
1.61	-1.0548275862069\\
1.62	-1.06351145038168\\
1.63	-1.07220532319392\\
1.64	-1.08090909090909\\
1.65	-1.08962264150943\\
1.66	-1.09834586466165\\
1.67	-1.10707865168539\\
1.68	-1.11582089552239\\
1.69	-1.12457249070632\\
1.7	-1.13333333333333\\
1.71	-1.14210332103321\\
1.72	-1.15088235294118\\
1.73	-1.15967032967033\\
1.74	-1.16846715328467\\
1.75	-1.17727272727273\\
1.76	-1.18608695652174\\
1.77	-1.19490974729242\\
1.78	-1.20374100719424\\
1.79	-1.21258064516129\\
1.8	-1.22142857142857\\
1.81	-1.2302846975089\\
1.82	-1.23914893617021\\
1.83	-1.24802120141343\\
1.84	-1.2569014084507\\
1.85	-1.26578947368421\\
1.86	-1.27468531468531\\
1.87	-1.28358885017422\\
1.88	-1.2925\\
1.89	-1.30141868512111\\
1.9	-1.31034482758621\\
1.91	-1.31927835051546\\
1.92	-1.32821917808219\\
1.93	-1.33716723549488\\
1.94	-1.34612244897959\\
1.95	-1.35508474576271\\
1.96	-1.36405405405405\\
1.97	-1.3730303030303\\
1.98	-1.38201342281879\\
1.99	-1.39100334448161\\
2	-1.4\\
2.01	-1.40900332225914\\
2.02	-1.41801324503311\\
2.03	-1.4270297029703\\
2.04	-1.43605263157895\\
2.05	-1.44508196721311\\
2.06	-1.45411764705882\\
2.07	-1.46315960912052\\
2.08	-1.47220779220779\\
2.09	-1.48126213592233\\
2.1	-1.49032258064516\\
2.11	-1.49938906752412\\
2.12	-1.50846153846154\\
2.13	-1.51753993610224\\
2.14	-1.52662420382166\\
2.15	-1.53571428571429\\
2.16	-1.54481012658228\\
2.17	-1.55391167192429\\
2.18	-1.56301886792453\\
2.19	-1.57213166144201\\
2.2	-1.58125\\
2.21	-1.5903738317757\\
2.22	-1.59950310559006\\
2.23	-1.60863777089783\\
2.24	-1.61777777777778\\
2.25	-1.62692307692308\\
2.26	-1.6360736196319\\
2.27	-1.64522935779817\\
2.28	-1.65439024390244\\
2.29	-1.66355623100304\\
2.3	-1.67272727272727\\
2.31	-1.68190332326284\\
2.32	-1.6910843373494\\
2.33	-1.70027027027027\\
2.34	-1.70946107784431\\
2.35	-1.71865671641791\\
2.36	-1.72785714285714\\
2.37	-1.73706231454006\\
2.38	-1.74627218934911\\
2.39	-1.75548672566372\\
2.4	-1.76470588235294\\
2.41	-1.77392961876833\\
2.42	-1.78315789473684\\
2.43	-1.79239067055394\\
2.44	-1.80162790697674\\
2.45	-1.81086956521739\\
2.46	-1.82011560693642\\
2.47	-1.82936599423631\\
2.48	-1.83862068965517\\
2.49	-1.84787965616046\\
2.5	-1.85714285714286\\
2.51	-1.86641025641026\\
2.52	-1.87568181818182\\
2.53	-1.88495750708215\\
2.54	-1.89423728813559\\
2.55	-1.90352112676056\\
2.56	-1.91280898876404\\
2.57	-1.92210084033613\\
2.58	-1.93139664804469\\
2.59	-1.94069637883008\\
2.6	-1.95\\
2.61	-1.95930747922438\\
2.62	-1.96861878453039\\
2.63	-1.97793388429752\\
2.64	-1.98725274725275\\
2.65	-1.99657534246575\\
2.66	-2.00590163934426\\
2.67	-2.01523160762943\\
2.68	-2.0245652173913\\
2.69	-2.03390243902439\\
2.7	-2.04324324324324\\
2.71	-2.05258760107817\\
2.72	-2.06193548387097\\
2.73	-2.07128686327078\\
2.74	-2.08064171122995\\
2.75	-2.09\\
2.76	-2.09936170212766\\
2.77	-2.10872679045093\\
2.78	-2.11809523809524\\
2.79	-2.12746701846966\\
2.8	-2.13684210526316\\
2.81	-2.14622047244095\\
2.82	-2.15560209424084\\
2.83	-2.16498694516971\\
2.84	-2.174375\\
2.85	-2.18376623376623\\
2.86	-2.19316062176166\\
2.87	-2.20255813953488\\
2.88	-2.2119587628866\\
2.89	-2.22136246786632\\
2.9	-2.23076923076923\\
2.91	-2.24017902813299\\
2.92	-2.24959183673469\\
2.93	-2.25900763358779\\
2.94	-2.26842639593909\\
2.95	-2.27784810126582\\
2.96	-2.28727272727273\\
2.97	-2.29670025188917\\
2.98	-2.30613065326633\\
2.99	-2.31556390977444\\
3	-2.325\\
3.01	-2.33443890274314\\
3.02	-2.34388059701493\\
3.03	-2.35332506203474\\
3.04	-2.36277227722772\\
3.05	-2.37222222222222\\
3.06	-2.38167487684729\\
3.07	-2.39113022113022\\
3.08	-2.40058823529412\\
3.09	-2.4100488997555\\
3.1	-2.41951219512195\\
3.11	-2.42897810218978\\
3.12	-2.43844660194175\\
3.13	-2.44791767554479\\
3.14	-2.45739130434783\\
3.15	-2.46686746987952\\
3.16	-2.47634615384615\\
3.17	-2.4858273381295\\
3.18	-2.49531100478469\\
3.19	-2.50479713603819\\
3.2	-2.51428571428571\\
3.21	-2.52377672209026\\
3.22	-2.53327014218009\\
3.23	-2.54276595744681\\
3.24	-2.5522641509434\\
3.25	-2.56176470588235\\
3.26	-2.5712676056338\\
3.27	-2.58077283372365\\
3.28	-2.59028037383178\\
3.29	-2.59979020979021\\
3.3	-2.6093023255814\\
3.31	-2.61881670533643\\
3.32	-2.62833333333333\\
3.33	-2.63785219399538\\
3.34	-2.6473732718894\\
3.35	-2.65689655172414\\
3.36	-2.66642201834862\\
3.37	-2.67594965675057\\
3.38	-2.68547945205479\\
3.39	-2.69501138952164\\
3.4	-2.70454545454545\\
3.41	-2.71408163265306\\
3.42	-2.72361990950226\\
3.43	-2.73316027088036\\
3.44	-2.7427027027027\\
3.45	-2.75224719101124\\
3.46	-2.76179372197309\\
3.47	-2.77134228187919\\
3.48	-2.78089285714286\\
3.49	-2.79044543429844\\
3.5	-2.8\\
3.51	-2.80955654101996\\
3.52	-2.81911504424779\\
3.53	-2.82867549668874\\
3.54	-2.83823788546256\\
3.55	-2.8478021978022\\
3.56	-2.85736842105263\\
3.57	-2.86693654266958\\
3.58	-2.87650655021834\\
3.59	-2.88607843137255\\
3.6	-2.89565217391304\\
3.61	-2.90522776572668\\
3.62	-2.91480519480519\\
3.63	-2.92438444924406\\
3.64	-2.93396551724138\\
3.65	-2.94354838709677\\
3.66	-2.9531330472103\\
3.67	-2.96271948608137\\
3.68	-2.97230769230769\\
3.69	-2.98189765458422\\
3.7	-2.99148936170213\\
3.71	-3.00108280254777\\
3.72	-3.01067796610169\\
3.73	-3.02027484143763\\
3.74	-3.02987341772152\\
3.75	-3.03947368421053\\
3.76	-3.0490756302521\\
3.77	-3.05867924528302\\
3.78	-3.06828451882845\\
3.79	-3.07789144050104\\
3.8	-3.0875\\
3.81	-3.09711018711019\\
3.82	-3.10672199170124\\
3.83	-3.11633540372671\\
3.84	-3.12595041322314\\
3.85	-3.13556701030928\\
3.86	-3.14518518518519\\
3.87	-3.15480492813142\\
3.88	-3.1644262295082\\
3.89	-3.1740490797546\\
3.9	-3.18367346938776\\
3.91	-3.19329938900204\\
3.92	-3.20292682926829\\
3.93	-3.21255578093306\\
3.94	-3.22218623481781\\
3.95	-3.23181818181818\\
3.96	-3.24145161290323\\
3.97	-3.25108651911469\\
3.98	-3.26072289156627\\
3.99	-3.27036072144289\\
4	-3.28\\
4.01	-3.28964071856287\\
4.02	-3.2992828685259\\
4.03	-3.30892644135189\\
4.04	-3.31857142857143\\
4.05	-3.32821782178218\\
4.06	-3.33786561264822\\
4.07	-3.34751479289941\\
4.08	-3.35716535433071\\
4.09	-3.36681728880157\\
4.1	-3.37647058823529\\
4.11	-3.3861252446184\\
4.12	-3.39578125\\
4.13	-3.40543859649123\\
4.14	-3.41509727626459\\
4.15	-3.4247572815534\\
4.16	-3.43441860465116\\
4.17	-3.44408123791102\\
4.18	-3.45374517374517\\
4.19	-3.46341040462428\\
4.2	-3.47307692307692\\
4.21	-3.48274472168906\\
4.22	-3.49241379310345\\
4.23	-3.50208413001912\\
4.24	-3.51175572519084\\
4.25	-3.52142857142857\\
4.26	-3.53110266159696\\
4.27	-3.5407779886148\\
4.28	-3.55045454545455\\
4.29	-3.56013232514178\\
4.3	-3.56981132075472\\
4.31	-3.57949152542373\\
4.32	-3.58917293233083\\
4.33	-3.59885553470919\\
4.34	-3.6085393258427\\
4.35	-3.61822429906542\\
4.36	-3.62791044776119\\
4.37	-3.63759776536313\\
4.38	-3.64728624535316\\
4.39	-3.6569758812616\\
4.4	-3.66666666666667\\
4.41	-3.67635859519409\\
4.42	-3.6860516605166\\
4.43	-3.69574585635359\\
4.44	-3.70544117647059\\
4.45	-3.7151376146789\\
4.46	-3.72483516483516\\
4.47	-3.73453382084095\\
4.48	-3.74423357664234\\
4.49	-3.75393442622951\\
4.5	-3.76363636363636\\
4.51	-3.77333938294011\\
4.52	-3.78304347826087\\
4.53	-3.7927486437613\\
4.54	-3.80245487364621\\
4.55	-3.81216216216216\\
4.56	-3.82187050359712\\
4.57	-3.83157989228007\\
4.58	-3.84129032258065\\
4.59	-3.85100178890877\\
4.6	-3.86071428571429\\
4.61	-3.87042780748663\\
4.62	-3.88014234875445\\
4.63	-3.88985790408526\\
4.64	-3.89957446808511\\
4.65	-3.90929203539823\\
4.66	-3.91901060070671\\
4.67	-3.92873015873016\\
4.68	-3.93845070422535\\
4.69	-3.94817223198594\\
4.7	-3.95789473684211\\
4.71	-3.96761821366024\\
4.72	-3.97734265734266\\
4.73	-3.98706806282723\\
4.74	-3.99679442508711\\
4.75	-4.00652173913043\\
4.76	-4.01625\\
4.77	-4.02597920277296\\
4.78	-4.03570934256055\\
4.79	-4.04544041450777\\
4.8	-4.0551724137931\\
4.81	-4.06490533562823\\
4.82	-4.07463917525773\\
4.83	-4.08437392795883\\
4.84	-4.0941095890411\\
4.85	-4.10384615384615\\
4.86	-4.11358361774744\\
4.87	-4.12332197614992\\
4.88	-4.1330612244898\\
4.89	-4.1428013582343\\
4.9	-4.15254237288136\\
4.91	-4.16228426395939\\
4.92	-4.17202702702703\\
4.93	-4.18177065767285\\
4.94	-4.19151515151515\\
4.95	-4.20126050420168\\
4.96	-4.2110067114094\\
4.97	-4.22075376884422\\
4.98	-4.2305016722408\\
4.99	-4.24025041736227\\
5	-4.25\\
};
\addplot [color=mycolor4,solid,forget plot]
  table[row sep=crcr]{%
0	0\\
0.01	-9.90099009900991e-05\\
0.02	-0.000392156862745099\\
0.03	-0.000873786407766988\\
0.04	-0.00153846153846154\\
0.05	-0.00238095238095239\\
0.06	-0.00339622641509434\\
0.07	-0.00457943925233645\\
0.08	-0.00592592592592593\\
0.09	-0.00743119266055046\\
0.1	-0.00909090909090909\\
0.11	-0.0109009009009009\\
0.12	-0.0128571428571429\\
0.13	-0.0149557522123894\\
0.14	-0.0171929824561404\\
0.15	-0.0195652173913043\\
0.16	-0.0220689655172414\\
0.17	-0.0247008547008547\\
0.18	-0.0274576271186441\\
0.19	-0.0303361344537815\\
0.2	-0.0333333333333333\\
0.21	-0.0364462809917355\\
0.22	-0.039672131147541\\
0.23	-0.0430081300813008\\
0.24	-0.0464516129032258\\
0.25	-0.05\\
0.26	-0.0536507936507937\\
0.27	-0.0574015748031496\\
0.28	-0.06125\\
0.29	-0.0651937984496124\\
0.3	-0.0692307692307693\\
0.31	-0.0733587786259542\\
0.32	-0.0775757575757576\\
0.33	-0.0818796992481203\\
0.34	-0.0862686567164179\\
0.35	-0.0907407407407408\\
0.36	-0.0952941176470588\\
0.37	-0.0999270072992701\\
0.38	-0.10463768115942\\
0.39	-0.109424460431655\\
0.4	-0.114285714285714\\
0.41	-0.119219858156028\\
0.42	-0.124225352112676\\
0.43	-0.129300699300699\\
0.44	-0.134444444444444\\
0.45	-0.139655172413793\\
0.46	-0.144931506849315\\
0.47	-0.150272108843537\\
0.48	-0.155675675675676\\
0.49	-0.161140939597315\\
0.5	-0.166666666666667\\
0.51	-0.172251655629139\\
0.52	-0.177894736842105\\
0.53	-0.18359477124183\\
0.54	-0.189350649350649\\
0.55	-0.195161290322581\\
0.56	-0.201025641025641\\
0.57	-0.206942675159236\\
0.58	-0.212911392405063\\
0.59	-0.218930817610063\\
0.6	-0.225\\
0.61	-0.23111801242236\\
0.62	-0.237283950617284\\
0.63	-0.243496932515337\\
0.64	-0.249756097560976\\
0.65	-0.256060606060606\\
0.66	-0.262409638554217\\
0.67	-0.268802395209581\\
0.68	-0.275238095238095\\
0.69	-0.281715976331361\\
0.7	-0.288235294117647\\
0.71	-0.294795321637427\\
0.72	-0.301395348837209\\
0.73	-0.308034682080925\\
0.74	-0.314712643678161\\
0.75	-0.321428571428571\\
0.76	-0.328181818181818\\
0.77	-0.334971751412429\\
0.78	-0.341797752808989\\
0.79	-0.348659217877095\\
0.8	-0.355555555555556\\
0.81	-0.362486187845304\\
0.82	-0.36945054945055\\
0.83	-0.376448087431694\\
0.84	-0.383478260869565\\
0.85	-0.390540540540541\\
0.86	-0.39763440860215\\
0.87	-0.40475935828877\\
0.88	-0.411914893617021\\
0.89	-0.419100529100529\\
0.9	-0.426315789473684\\
0.91	-0.433560209424084\\
0.92	-0.440833333333333\\
0.93	-0.448134715025907\\
0.94	-0.455463917525773\\
0.95	-0.462820512820513\\
0.96	-0.470204081632653\\
0.97	-0.47761421319797\\
0.98	-0.485050505050505\\
0.99	-0.49251256281407\\
1	-0.5\\
1.01	-0.507512437810945\\
1.02	-0.515049504950495\\
1.03	-0.522610837438424\\
1.04	-0.530196078431373\\
1.05	-0.53780487804878\\
1.06	-0.545436893203884\\
1.07	-0.553091787439614\\
1.08	-0.560769230769231\\
1.09	-0.568468899521531\\
1.1	-0.576190476190476\\
1.11	-0.5839336492891\\
1.12	-0.591698113207547\\
1.13	-0.599483568075117\\
1.14	-0.607289719626168\\
1.15	-0.615116279069768\\
1.16	-0.622962962962963\\
1.17	-0.630829493087558\\
1.18	-0.638715596330275\\
1.19	-0.64662100456621\\
1.2	-0.654545454545455\\
1.21	-0.662488687782805\\
1.22	-0.67045045045045\\
1.23	-0.678430493273543\\
1.24	-0.686428571428571\\
1.25	-0.694444444444444\\
1.26	-0.702477876106195\\
1.27	-0.710528634361233\\
1.28	-0.71859649122807\\
1.29	-0.726681222707424\\
1.3	-0.734782608695652\\
1.31	-0.742900432900433\\
1.32	-0.751034482758621\\
1.33	-0.759184549356223\\
1.34	-0.767350427350427\\
1.35	-0.775531914893617\\
1.36	-0.783728813559322\\
1.37	-0.791940928270042\\
1.38	-0.800168067226891\\
1.39	-0.808410041841004\\
1.4	-0.816666666666667\\
1.41	-0.8249377593361\\
1.42	-0.833223140495868\\
1.43	-0.841522633744856\\
1.44	-0.84983606557377\\
1.45	-0.858163265306123\\
1.46	-0.86650406504065\\
1.47	-0.874858299595142\\
1.48	-0.883225806451613\\
1.49	-0.891606425702811\\
1.5	-0.9\\
1.51	-0.908406374501992\\
1.52	-0.916825396825397\\
1.53	-0.925256916996048\\
1.54	-0.933700787401575\\
1.55	-0.942156862745098\\
1.56	-0.950625\\
1.57	-0.959105058365759\\
1.58	-0.967596899224806\\
1.59	-0.976100386100386\\
1.6	-0.984615384615385\\
1.61	-0.993141762452107\\
1.62	-1.00167938931298\\
1.63	-1.01022813688213\\
1.64	-1.01878787878788\\
1.65	-1.02735849056604\\
1.66	-1.03593984962406\\
1.67	-1.04453183520599\\
1.68	-1.05313432835821\\
1.69	-1.06174721189591\\
1.7	-1.07037037037037\\
1.71	-1.0790036900369\\
1.72	-1.08764705882353\\
1.73	-1.09630036630037\\
1.74	-1.10496350364964\\
1.75	-1.11363636363636\\
1.76	-1.12231884057971\\
1.77	-1.13101083032491\\
1.78	-1.13971223021583\\
1.79	-1.1484229390681\\
1.8	-1.15714285714286\\
1.81	-1.165871886121\\
1.82	-1.17460992907801\\
1.83	-1.18335689045936\\
1.84	-1.19211267605634\\
1.85	-1.20087719298246\\
1.86	-1.20965034965035\\
1.87	-1.21843205574913\\
1.88	-1.22722222222222\\
1.89	-1.23602076124567\\
1.9	-1.2448275862069\\
1.91	-1.25364261168385\\
1.92	-1.26246575342466\\
1.93	-1.27129692832764\\
1.94	-1.28013605442177\\
1.95	-1.28898305084746\\
1.96	-1.29783783783784\\
1.97	-1.30670033670034\\
1.98	-1.31557046979866\\
1.99	-1.32444816053512\\
2	-1.33333333333333\\
2.01	-1.34222591362126\\
2.02	-1.35112582781457\\
2.03	-1.36003300330033\\
2.04	-1.36894736842105\\
2.05	-1.37786885245902\\
2.06	-1.38679738562092\\
2.07	-1.3957328990228\\
2.08	-1.40467532467532\\
2.09	-1.41362459546926\\
2.1	-1.42258064516129\\
2.11	-1.43154340836013\\
2.12	-1.44051282051282\\
2.13	-1.44948881789137\\
2.14	-1.45847133757962\\
2.15	-1.46746031746032\\
2.16	-1.47645569620253\\
2.17	-1.48545741324921\\
2.18	-1.49446540880503\\
2.19	-1.50347962382445\\
2.2	-1.5125\\
2.21	-1.52152647975078\\
2.22	-1.53055900621118\\
2.23	-1.53959752321981\\
2.24	-1.54864197530864\\
2.25	-1.55769230769231\\
2.26	-1.56674846625767\\
2.27	-1.57581039755352\\
2.28	-1.58487804878049\\
2.29	-1.59395136778115\\
2.3	-1.6030303030303\\
2.31	-1.61211480362538\\
2.32	-1.62120481927711\\
2.33	-1.6303003003003\\
2.34	-1.63940119760479\\
2.35	-1.64850746268657\\
2.36	-1.65761904761905\\
2.37	-1.66673590504451\\
2.38	-1.67585798816568\\
2.39	-1.68498525073746\\
2.4	-1.69411764705882\\
2.41	-1.70325513196481\\
2.42	-1.71239766081871\\
2.43	-1.72154518950437\\
2.44	-1.7306976744186\\
2.45	-1.73985507246377\\
2.46	-1.74901734104046\\
2.47	-1.75818443804035\\
2.48	-1.76735632183908\\
2.49	-1.7765329512894\\
2.5	-1.78571428571429\\
2.51	-1.79490028490028\\
2.52	-1.80409090909091\\
2.53	-1.81328611898017\\
2.54	-1.82248587570621\\
2.55	-1.83169014084507\\
2.56	-1.84089887640449\\
2.57	-1.85011204481793\\
2.58	-1.85932960893855\\
2.59	-1.86855153203343\\
2.6	-1.87777777777778\\
2.61	-1.88700831024931\\
2.62	-1.89624309392265\\
2.63	-1.90548209366391\\
2.64	-1.91472527472527\\
2.65	-1.92397260273973\\
2.66	-1.93322404371585\\
2.67	-1.9424795640327\\
2.68	-1.95173913043478\\
2.69	-1.9610027100271\\
2.7	-1.97027027027027\\
2.71	-1.97954177897574\\
2.72	-1.98881720430108\\
2.73	-1.99809651474531\\
2.74	-2.00737967914438\\
2.75	-2.01666666666667\\
2.76	-2.02595744680851\\
2.77	-2.03525198938992\\
2.78	-2.04455026455026\\
2.79	-2.05385224274406\\
2.8	-2.06315789473684\\
2.81	-2.07246719160105\\
2.82	-2.08178010471204\\
2.83	-2.09109660574413\\
2.84	-2.10041666666667\\
2.85	-2.10974025974026\\
2.86	-2.11906735751295\\
2.87	-2.12839793281654\\
2.88	-2.13773195876289\\
2.89	-2.14706940874036\\
2.9	-2.15641025641026\\
2.91	-2.16575447570332\\
2.92	-2.17510204081633\\
2.93	-2.18445292620865\\
2.94	-2.19380710659898\\
2.95	-2.20316455696203\\
2.96	-2.21252525252525\\
2.97	-2.22188916876574\\
2.98	-2.23125628140704\\
2.99	-2.24062656641604\\
3	-2.25\\
3.01	-2.25937655860349\\
3.02	-2.26875621890547\\
3.03	-2.27813895781638\\
3.04	-2.28752475247525\\
3.05	-2.29691358024691\\
3.06	-2.30630541871921\\
3.07	-2.31570024570025\\
3.08	-2.32509803921569\\
3.09	-2.33449877750611\\
3.1	-2.34390243902439\\
3.11	-2.35330900243309\\
3.12	-2.36271844660194\\
3.13	-2.37213075060533\\
3.14	-2.38154589371981\\
3.15	-2.39096385542169\\
3.16	-2.40038461538462\\
3.17	-2.40980815347722\\
3.18	-2.41923444976077\\
3.19	-2.42866348448687\\
3.2	-2.43809523809524\\
3.21	-2.4475296912114\\
3.22	-2.45696682464455\\
3.23	-2.46640661938534\\
3.24	-2.47584905660377\\
3.25	-2.48529411764706\\
3.26	-2.49474178403756\\
3.27	-2.50419203747073\\
3.28	-2.51364485981308\\
3.29	-2.52310023310023\\
3.3	-2.53255813953488\\
3.31	-2.54201856148492\\
3.32	-2.55148148148148\\
3.33	-2.56094688221709\\
3.34	-2.57041474654378\\
3.35	-2.57988505747126\\
3.36	-2.58935779816514\\
3.37	-2.59883295194508\\
3.38	-2.6083105022831\\
3.39	-2.61779043280182\\
3.4	-2.62727272727273\\
3.41	-2.63675736961451\\
3.42	-2.6462443438914\\
3.43	-2.65573363431151\\
3.44	-2.66522522522523\\
3.45	-2.6747191011236\\
3.46	-2.68421524663677\\
3.47	-2.69371364653244\\
3.48	-2.70321428571429\\
3.49	-2.71271714922049\\
3.5	-2.72222222222222\\
3.51	-2.73172949002217\\
3.52	-2.7412389380531\\
3.53	-2.75075055187638\\
3.54	-2.76026431718062\\
3.55	-2.76978021978022\\
3.56	-2.77929824561404\\
3.57	-2.78881838074398\\
3.58	-2.79834061135371\\
3.59	-2.80786492374728\\
3.6	-2.81739130434783\\
3.61	-2.82691973969631\\
3.62	-2.83645021645022\\
3.63	-2.84598272138229\\
3.64	-2.85551724137931\\
3.65	-2.86505376344086\\
3.66	-2.87459227467811\\
3.67	-2.88413276231263\\
3.68	-2.89367521367521\\
3.69	-2.90321961620469\\
3.7	-2.91276595744681\\
3.71	-2.92231422505308\\
3.72	-2.93186440677966\\
3.73	-2.94141649048626\\
3.74	-2.95097046413502\\
3.75	-2.96052631578947\\
3.76	-2.97008403361345\\
3.77	-2.97964360587002\\
3.78	-2.9892050209205\\
3.79	-2.99876826722338\\
3.8	-3.00833333333333\\
3.81	-3.01790020790021\\
3.82	-3.02746887966805\\
3.83	-3.03703933747412\\
3.84	-3.04661157024793\\
3.85	-3.05618556701031\\
3.86	-3.06576131687243\\
3.87	-3.07533880903491\\
3.88	-3.08491803278689\\
3.89	-3.09449897750511\\
3.9	-3.10408163265306\\
3.91	-3.11366598778004\\
3.92	-3.12325203252033\\
3.93	-3.13283975659229\\
3.94	-3.14242914979757\\
3.95	-3.1520202020202\\
3.96	-3.16161290322581\\
3.97	-3.17120724346076\\
3.98	-3.18080321285141\\
3.99	-3.19040080160321\\
4	-3.2\\
4.01	-3.20960079840319\\
4.02	-3.219203187251\\
4.03	-3.22880715705765\\
4.04	-3.2384126984127\\
4.05	-3.2480198019802\\
4.06	-3.25762845849802\\
4.07	-3.26723865877712\\
4.08	-3.27685039370079\\
4.09	-3.28646365422397\\
4.1	-3.29607843137255\\
4.11	-3.30569471624266\\
4.12	-3.3153125\\
4.13	-3.32493177387914\\
4.14	-3.33455252918288\\
4.15	-3.34417475728155\\
4.16	-3.3537984496124\\
4.17	-3.36342359767892\\
4.18	-3.37305019305019\\
4.19	-3.38267822736031\\
4.2	-3.39230769230769\\
4.21	-3.40193857965451\\
4.22	-3.41157088122605\\
4.23	-3.42120458891013\\
4.24	-3.43083969465649\\
4.25	-3.44047619047619\\
4.26	-3.45011406844106\\
4.27	-3.45975332068311\\
4.28	-3.46939393939394\\
4.29	-3.4790359168242\\
4.3	-3.48867924528302\\
4.31	-3.49832391713748\\
4.32	-3.50796992481203\\
4.33	-3.51761726078799\\
4.34	-3.527265917603\\
4.35	-3.53691588785047\\
4.36	-3.5465671641791\\
4.37	-3.55621973929237\\
4.38	-3.56587360594796\\
4.39	-3.57552875695733\\
4.4	-3.58518518518519\\
4.41	-3.59484288354898\\
4.42	-3.60450184501845\\
4.43	-3.6141620626151\\
4.44	-3.62382352941176\\
4.45	-3.63348623853211\\
4.46	-3.64315018315018\\
4.47	-3.65281535648995\\
4.48	-3.66248175182482\\
4.49	-3.67214936247723\\
4.5	-3.68181818181818\\
4.51	-3.69148820326679\\
4.52	-3.70115942028985\\
4.53	-3.71083182640145\\
4.54	-3.72050541516245\\
4.55	-3.73018018018018\\
4.56	-3.73985611510791\\
4.57	-3.74953321364452\\
4.58	-3.75921146953405\\
4.59	-3.76889087656529\\
4.6	-3.77857142857143\\
4.61	-3.78825311942959\\
4.62	-3.7979359430605\\
4.63	-3.80761989342806\\
4.64	-3.81730496453901\\
4.65	-3.82699115044248\\
4.66	-3.83667844522968\\
4.67	-3.84636684303351\\
4.68	-3.85605633802817\\
4.69	-3.86574692442882\\
4.7	-3.87543859649123\\
4.71	-3.88513134851138\\
4.72	-3.89482517482517\\
4.73	-3.90452006980803\\
4.74	-3.91421602787456\\
4.75	-3.92391304347826\\
4.76	-3.93361111111111\\
4.77	-3.94331022530329\\
4.78	-3.95301038062284\\
4.79	-3.9627115716753\\
4.8	-3.97241379310345\\
4.81	-3.98211703958692\\
4.82	-3.99182130584192\\
4.83	-4.00152658662093\\
4.84	-4.01123287671233\\
4.85	-4.02094017094017\\
4.86	-4.03064846416382\\
4.87	-4.04035775127768\\
4.88	-4.05006802721088\\
4.89	-4.05977928692699\\
4.9	-4.06949152542373\\
4.91	-4.07920473773266\\
4.92	-4.08891891891892\\
4.93	-4.09863406408094\\
4.94	-4.10835016835017\\
4.95	-4.11806722689076\\
4.96	-4.12778523489933\\
4.97	-4.13750418760469\\
4.98	-4.14722408026756\\
4.99	-4.1569449081803\\
5	-4.16666666666667\\
};
\addplot [color=mycolor5,solid,forget plot]
  table[row sep=crcr]{%
0	0\\
0.01	0.000891089108910892\\
0.02	0.00156862745098039\\
0.03	0.00203883495145631\\
0.04	0.00230769230769231\\
0.05	0.00238095238095239\\
0.06	0.00226415094339623\\
0.07	0.00196261682242992\\
0.08	0.00148148148148149\\
0.09	0.00082568807339449\\
0.1	0\\
0.11	-0.000990990990990995\\
0.12	-0.00214285714285715\\
0.13	-0.0034513274336283\\
0.14	-0.0049122807017544\\
0.15	-0.00652173913043477\\
0.16	-0.00827586206896549\\
0.17	-0.0101709401709402\\
0.18	-0.0122033898305084\\
0.19	-0.0143697478991596\\
0.2	-0.0166666666666667\\
0.21	-0.0190909090909091\\
0.22	-0.0216393442622951\\
0.23	-0.0243089430894309\\
0.24	-0.0270967741935484\\
0.25	-0.03\\
0.26	-0.033015873015873\\
0.27	-0.0361417322834646\\
0.28	-0.039375\\
0.29	-0.0427131782945736\\
0.3	-0.0461538461538462\\
0.31	-0.0496946564885496\\
0.32	-0.0533333333333333\\
0.33	-0.0570676691729323\\
0.34	-0.0608955223880597\\
0.35	-0.0648148148148148\\
0.36	-0.0688235294117647\\
0.37	-0.0729197080291971\\
0.38	-0.0771014492753623\\
0.39	-0.0813669064748201\\
0.4	-0.0857142857142857\\
0.41	-0.0901418439716312\\
0.42	-0.0946478873239436\\
0.43	-0.0992307692307692\\
0.44	-0.103888888888889\\
0.45	-0.108620689655172\\
0.46	-0.113424657534247\\
0.47	-0.118299319727891\\
0.48	-0.123243243243243\\
0.49	-0.128255033557047\\
0.5	-0.133333333333333\\
0.51	-0.138476821192053\\
0.52	-0.143684210526316\\
0.53	-0.148954248366013\\
0.54	-0.154285714285714\\
0.55	-0.159677419354839\\
0.56	-0.165128205128205\\
0.57	-0.170636942675159\\
0.58	-0.17620253164557\\
0.59	-0.181823899371069\\
0.6	-0.1875\\
0.61	-0.193229813664596\\
0.62	-0.199012345679012\\
0.63	-0.204846625766871\\
0.64	-0.210731707317073\\
0.65	-0.216666666666667\\
0.66	-0.222650602409639\\
0.67	-0.228682634730539\\
0.68	-0.234761904761905\\
0.69	-0.240887573964497\\
0.7	-0.247058823529412\\
0.71	-0.25327485380117\\
0.72	-0.25953488372093\\
0.73	-0.265838150289017\\
0.74	-0.272183908045977\\
0.75	-0.278571428571428\\
0.76	-0.285\\
0.77	-0.291468926553672\\
0.78	-0.297977528089888\\
0.79	-0.304525139664804\\
0.8	-0.311111111111111\\
0.81	-0.317734806629834\\
0.82	-0.324395604395604\\
0.83	-0.331092896174863\\
0.84	-0.337826086956522\\
0.85	-0.344594594594595\\
0.86	-0.351397849462366\\
0.87	-0.358235294117647\\
0.88	-0.365106382978723\\
0.89	-0.372010582010582\\
0.9	-0.378947368421053\\
0.91	-0.385916230366492\\
0.92	-0.392916666666667\\
0.93	-0.399948186528497\\
0.94	-0.40701030927835\\
0.95	-0.414102564102564\\
0.96	-0.421224489795918\\
0.97	-0.428375634517767\\
0.98	-0.435555555555555\\
0.99	-0.442763819095477\\
1	-0.45\\
1.01	-0.45726368159204\\
1.02	-0.464554455445545\\
1.03	-0.471871921182266\\
1.04	-0.47921568627451\\
1.05	-0.486585365853658\\
1.06	-0.493980582524272\\
1.07	-0.501400966183575\\
1.08	-0.508846153846154\\
1.09	-0.516315789473684\\
1.1	-0.523809523809524\\
1.11	-0.53132701421801\\
1.12	-0.538867924528302\\
1.13	-0.546431924882629\\
1.14	-0.554018691588785\\
1.15	-0.561627906976744\\
1.16	-0.569259259259259\\
1.17	-0.576912442396313\\
1.18	-0.584587155963303\\
1.19	-0.592283105022831\\
1.2	-0.6\\
1.21	-0.607737556561086\\
1.22	-0.615495495495495\\
1.23	-0.623273542600897\\
1.24	-0.631071428571429\\
1.25	-0.638888888888889\\
1.26	-0.646725663716814\\
1.27	-0.654581497797357\\
1.28	-0.662456140350877\\
1.29	-0.670349344978166\\
1.3	-0.678260869565217\\
1.31	-0.686190476190476\\
1.32	-0.694137931034483\\
1.33	-0.702103004291845\\
1.34	-0.71008547008547\\
1.35	-0.718085106382979\\
1.36	-0.726101694915254\\
1.37	-0.734135021097046\\
1.38	-0.74218487394958\\
1.39	-0.750251046025105\\
1.4	-0.758333333333333\\
1.41	-0.76643153526971\\
1.42	-0.774545454545454\\
1.43	-0.782674897119341\\
1.44	-0.790819672131147\\
1.45	-0.798979591836735\\
1.46	-0.807154471544715\\
1.47	-0.815344129554656\\
1.48	-0.823548387096774\\
1.49	-0.831767068273092\\
1.5	-0.84\\
1.51	-0.848247011952191\\
1.52	-0.856507936507936\\
1.53	-0.864782608695652\\
1.54	-0.873070866141732\\
1.55	-0.881372549019608\\
1.56	-0.8896875\\
1.57	-0.898015564202335\\
1.58	-0.906356589147287\\
1.59	-0.914710424710425\\
1.6	-0.923076923076923\\
1.61	-0.931455938697318\\
1.62	-0.939847328244275\\
1.63	-0.948250950570342\\
1.64	-0.956666666666667\\
1.65	-0.965094339622642\\
1.66	-0.973533834586466\\
1.67	-0.981985018726592\\
1.68	-0.99044776119403\\
1.69	-0.998921933085502\\
1.7	-1.00740740740741\\
1.71	-1.01590405904059\\
1.72	-1.02441176470588\\
1.73	-1.0329304029304\\
1.74	-1.0414598540146\\
1.75	-1.05\\
1.76	-1.05855072463768\\
1.77	-1.0671119133574\\
1.78	-1.07568345323741\\
1.79	-1.08426523297491\\
1.8	-1.09285714285714\\
1.81	-1.1014590747331\\
1.82	-1.11007092198582\\
1.83	-1.1186925795053\\
1.84	-1.12732394366197\\
1.85	-1.1359649122807\\
1.86	-1.14461538461538\\
1.87	-1.15327526132404\\
1.88	-1.16194444444444\\
1.89	-1.17062283737024\\
1.9	-1.17931034482759\\
1.91	-1.18800687285223\\
1.92	-1.19671232876712\\
1.93	-1.20542662116041\\
1.94	-1.21414965986395\\
1.95	-1.2228813559322\\
1.96	-1.23162162162162\\
1.97	-1.24037037037037\\
1.98	-1.24912751677852\\
1.99	-1.25789297658863\\
2	-1.26666666666667\\
2.01	-1.27544850498339\\
2.02	-1.28423841059603\\
2.03	-1.29303630363036\\
2.04	-1.30184210526316\\
2.05	-1.31065573770492\\
2.06	-1.31947712418301\\
2.07	-1.32830618892508\\
2.08	-1.33714285714286\\
2.09	-1.34598705501618\\
2.1	-1.35483870967742\\
2.11	-1.36369774919614\\
2.12	-1.3725641025641\\
2.13	-1.38143769968051\\
2.14	-1.39031847133758\\
2.15	-1.39920634920635\\
2.16	-1.40810126582278\\
2.17	-1.41700315457413\\
2.18	-1.42591194968553\\
2.19	-1.4348275862069\\
2.2	-1.44375\\
2.21	-1.45267912772586\\
2.22	-1.4616149068323\\
2.23	-1.4705572755418\\
2.24	-1.47950617283951\\
2.25	-1.48846153846154\\
2.26	-1.49742331288344\\
2.27	-1.50639143730887\\
2.28	-1.51536585365854\\
2.29	-1.52434650455927\\
2.3	-1.53333333333333\\
2.31	-1.54232628398792\\
2.32	-1.55132530120482\\
2.33	-1.56033033033033\\
2.34	-1.56934131736527\\
2.35	-1.57835820895522\\
2.36	-1.58738095238095\\
2.37	-1.59640949554896\\
2.38	-1.60544378698225\\
2.39	-1.61448377581121\\
2.4	-1.62352941176471\\
2.41	-1.63258064516129\\
2.42	-1.64163742690058\\
2.43	-1.65069970845481\\
2.44	-1.65976744186047\\
2.45	-1.66884057971015\\
2.46	-1.67791907514451\\
2.47	-1.68700288184438\\
2.48	-1.69609195402299\\
2.49	-1.70518624641834\\
2.5	-1.71428571428571\\
2.51	-1.72339031339031\\
2.52	-1.7325\\
2.53	-1.74161473087819\\
2.54	-1.75073446327684\\
2.55	-1.75985915492958\\
2.56	-1.76898876404494\\
2.57	-1.77812324929972\\
2.58	-1.7872625698324\\
2.59	-1.79640668523677\\
2.6	-1.80555555555556\\
2.61	-1.81470914127424\\
2.62	-1.82386740331492\\
2.63	-1.8330303030303\\
2.64	-1.8421978021978\\
2.65	-1.8513698630137\\
2.66	-1.86054644808743\\
2.67	-1.86972752043597\\
2.68	-1.87891304347826\\
2.69	-1.88810298102981\\
2.7	-1.8972972972973\\
2.71	-1.90649595687332\\
2.72	-1.91569892473118\\
2.73	-1.92490616621984\\
2.74	-1.93411764705882\\
2.75	-1.94333333333333\\
2.76	-1.95255319148936\\
2.77	-1.96177718832891\\
2.78	-1.97100529100529\\
2.79	-1.98023746701847\\
2.8	-1.98947368421053\\
2.81	-1.99871391076115\\
2.82	-2.00795811518325\\
2.83	-2.01720626631854\\
2.84	-2.02645833333333\\
2.85	-2.03571428571429\\
2.86	-2.04497409326425\\
2.87	-2.05423772609819\\
2.88	-2.06350515463917\\
2.89	-2.0727763496144\\
2.9	-2.08205128205128\\
2.91	-2.09132992327366\\
2.92	-2.10061224489796\\
2.93	-2.10989821882952\\
2.94	-2.11918781725888\\
2.95	-2.12848101265823\\
2.96	-2.13777777777778\\
2.97	-2.14707808564232\\
2.98	-2.15638190954774\\
2.99	-2.16568922305764\\
3	-2.175\\
3.01	-2.18431421446384\\
3.02	-2.19363184079602\\
3.03	-2.20295285359802\\
3.04	-2.21227722772277\\
3.05	-2.2216049382716\\
3.06	-2.23093596059113\\
3.07	-2.24027027027027\\
3.08	-2.24960784313725\\
3.09	-2.25894865525672\\
3.1	-2.26829268292683\\
3.11	-2.2776399026764\\
3.12	-2.28699029126214\\
3.13	-2.29634382566586\\
3.14	-2.30570048309179\\
3.15	-2.31506024096386\\
3.16	-2.32442307692308\\
3.17	-2.33378896882494\\
3.18	-2.34315789473684\\
3.19	-2.35252983293556\\
3.2	-2.36190476190476\\
3.21	-2.37128266033254\\
3.22	-2.380663507109\\
3.23	-2.39004728132388\\
3.24	-2.39943396226415\\
3.25	-2.40882352941176\\
3.26	-2.41821596244131\\
3.27	-2.4276112412178\\
3.28	-2.43700934579439\\
3.29	-2.44641025641026\\
3.3	-2.45581395348837\\
3.31	-2.46522041763341\\
3.32	-2.47462962962963\\
3.33	-2.4840415704388\\
3.34	-2.49345622119816\\
3.35	-2.50287356321839\\
3.36	-2.51229357798165\\
3.37	-2.52171624713959\\
3.38	-2.53114155251142\\
3.39	-2.540569476082\\
3.4	-2.55\\
3.41	-2.55943310657596\\
3.42	-2.56886877828054\\
3.43	-2.57830699774266\\
3.44	-2.58774774774775\\
3.45	-2.59719101123595\\
3.46	-2.60663677130045\\
3.47	-2.61608501118568\\
3.48	-2.62553571428571\\
3.49	-2.63498886414254\\
3.5	-2.64444444444444\\
3.51	-2.65390243902439\\
3.52	-2.66336283185841\\
3.53	-2.67282560706402\\
3.54	-2.68229074889868\\
3.55	-2.69175824175824\\
3.56	-2.70122807017544\\
3.57	-2.71070021881838\\
3.58	-2.72017467248908\\
3.59	-2.729651416122\\
3.6	-2.73913043478261\\
3.61	-2.74861171366594\\
3.62	-2.75809523809524\\
3.63	-2.76758099352052\\
3.64	-2.77706896551724\\
3.65	-2.78655913978495\\
3.66	-2.79605150214592\\
3.67	-2.8055460385439\\
3.68	-2.81504273504273\\
3.69	-2.82454157782516\\
3.7	-2.83404255319149\\
3.71	-2.84354564755839\\
3.72	-2.85305084745763\\
3.73	-2.86255813953488\\
3.74	-2.87206751054852\\
3.75	-2.88157894736842\\
3.76	-2.89109243697479\\
3.77	-2.90060796645702\\
3.78	-2.91012552301255\\
3.79	-2.91964509394572\\
3.8	-2.92916666666667\\
3.81	-2.93869022869023\\
3.82	-2.94821576763486\\
3.83	-2.95774327122153\\
3.84	-2.96727272727273\\
3.85	-2.97680412371134\\
3.86	-2.98633744855967\\
3.87	-2.9958726899384\\
3.88	-3.00540983606557\\
3.89	-3.01494887525562\\
3.9	-3.02448979591837\\
3.91	-3.03403258655804\\
3.92	-3.04357723577236\\
3.93	-3.05312373225152\\
3.94	-3.06267206477733\\
3.95	-3.07222222222222\\
3.96	-3.08177419354839\\
3.97	-3.09132796780684\\
3.98	-3.10088353413655\\
3.99	-3.11044088176353\\
4	-3.12\\
4.01	-3.12956087824351\\
4.02	-3.1391235059761\\
4.03	-3.14868787276342\\
4.04	-3.15825396825397\\
4.05	-3.16782178217822\\
4.06	-3.17739130434783\\
4.07	-3.18696252465483\\
4.08	-3.19653543307087\\
4.09	-3.20611001964637\\
4.1	-3.2156862745098\\
4.11	-3.22526418786693\\
4.12	-3.23484375\\
4.13	-3.24442495126706\\
4.14	-3.25400778210117\\
4.15	-3.26359223300971\\
4.16	-3.27317829457364\\
4.17	-3.28276595744681\\
4.18	-3.29235521235521\\
4.19	-3.30194605009634\\
4.2	-3.31153846153846\\
4.21	-3.32113243761996\\
4.22	-3.33072796934866\\
4.23	-3.34032504780115\\
4.24	-3.34992366412214\\
4.25	-3.35952380952381\\
4.26	-3.36912547528517\\
4.27	-3.37872865275142\\
4.28	-3.38833333333333\\
4.29	-3.39793950850662\\
4.3	-3.40754716981132\\
4.31	-3.41715630885122\\
4.32	-3.42676691729323\\
4.33	-3.43637898686679\\
4.34	-3.4459925093633\\
4.35	-3.45560747663551\\
4.36	-3.46522388059702\\
4.37	-3.4748417132216\\
4.38	-3.48446096654275\\
4.39	-3.49408163265306\\
4.4	-3.5037037037037\\
4.41	-3.51332717190388\\
4.42	-3.5229520295203\\
4.43	-3.53257826887661\\
4.44	-3.54220588235294\\
4.45	-3.55183486238532\\
4.46	-3.5614652014652\\
4.47	-3.57109689213894\\
4.48	-3.5807299270073\\
4.49	-3.59036429872495\\
4.5	-3.6\\
4.51	-3.60963702359347\\
4.52	-3.61927536231884\\
4.53	-3.62891500904159\\
4.54	-3.6385559566787\\
4.55	-3.6481981981982\\
4.56	-3.6578417266187\\
4.57	-3.66748653500898\\
4.58	-3.67713261648746\\
4.59	-3.68677996422182\\
4.6	-3.69642857142857\\
4.61	-3.70607843137255\\
4.62	-3.71572953736655\\
4.63	-3.72538188277087\\
4.64	-3.73503546099291\\
4.65	-3.74469026548673\\
4.66	-3.75434628975265\\
4.67	-3.76400352733686\\
4.68	-3.77366197183099\\
4.69	-3.78332161687171\\
4.7	-3.79298245614035\\
4.71	-3.80264448336252\\
4.72	-3.81230769230769\\
4.73	-3.82197207678883\\
4.74	-3.83163763066202\\
4.75	-3.84130434782609\\
4.76	-3.85097222222222\\
4.77	-3.86064124783362\\
4.78	-3.87031141868512\\
4.79	-3.87998272884283\\
4.8	-3.88965517241379\\
4.81	-3.89932874354561\\
4.82	-3.90900343642612\\
4.83	-3.91867924528302\\
4.84	-3.92835616438356\\
4.85	-3.93803418803419\\
4.86	-3.9477133105802\\
4.87	-3.95739352640545\\
4.88	-3.96707482993197\\
4.89	-3.97675721561969\\
4.9	-3.9864406779661\\
4.91	-3.99612521150592\\
4.92	-4.00581081081081\\
4.93	-4.01549747048904\\
4.94	-4.02518518518519\\
4.95	-4.03487394957983\\
4.96	-4.04456375838926\\
4.97	-4.05425460636516\\
4.98	-4.06394648829431\\
4.99	-4.07363939899833\\
5	-4.08333333333333\\
};
\addplot [color=mycolor6,solid,forget plot]
  table[row sep=crcr]{%
0	0\\
0.01	0.00188118811881188\\
0.02	0.00352941176470588\\
0.03	0.00495145631067961\\
0.04	0.00615384615384615\\
0.05	0.00714285714285714\\
0.06	0.00792452830188679\\
0.07	0.00850467289719625\\
0.08	0.00888888888888888\\
0.09	0.00908256880733943\\
0.1	0.00909090909090908\\
0.11	0.0089189189189189\\
0.12	0.00857142857142856\\
0.13	0.00805309734513276\\
0.14	0.00736842105263158\\
0.15	0.00652173913043477\\
0.16	0.00551724137931037\\
0.17	0.00435897435897439\\
0.18	0.0030508474576271\\
0.19	0.00159663865546217\\
0.2	0\\
0.21	-0.00173553719008263\\
0.22	-0.00360655737704918\\
0.23	-0.00560975609756098\\
0.24	-0.00774193548387098\\
0.25	-0.01\\
0.26	-0.0123809523809524\\
0.27	-0.0148818897637795\\
0.28	-0.0175\\
0.29	-0.0202325581395349\\
0.3	-0.0230769230769231\\
0.31	-0.0260305343511451\\
0.32	-0.0290909090909091\\
0.33	-0.0322556390977444\\
0.34	-0.0355223880597015\\
0.35	-0.0388888888888889\\
0.36	-0.0423529411764705\\
0.37	-0.0459124087591241\\
0.38	-0.0495652173913043\\
0.39	-0.0533093525179857\\
0.4	-0.0571428571428571\\
0.41	-0.0610638297872341\\
0.42	-0.0650704225352113\\
0.43	-0.0691608391608391\\
0.44	-0.0733333333333333\\
0.45	-0.0775862068965517\\
0.46	-0.0819178082191781\\
0.47	-0.0863265306122449\\
0.48	-0.0908108108108108\\
0.49	-0.0953691275167786\\
0.5	-0.1\\
0.51	-0.104701986754967\\
0.52	-0.109473684210526\\
0.53	-0.114313725490196\\
0.54	-0.119220779220779\\
0.55	-0.124193548387097\\
0.56	-0.129230769230769\\
0.57	-0.134331210191083\\
0.58	-0.139493670886076\\
0.59	-0.144716981132075\\
0.6	-0.15\\
0.61	-0.155341614906832\\
0.62	-0.160740740740741\\
0.63	-0.166196319018405\\
0.64	-0.171707317073171\\
0.65	-0.177272727272727\\
0.66	-0.18289156626506\\
0.67	-0.188562874251497\\
0.68	-0.194285714285714\\
0.69	-0.200059171597633\\
0.7	-0.205882352941177\\
0.71	-0.211754385964912\\
0.72	-0.217674418604651\\
0.73	-0.22364161849711\\
0.74	-0.229655172413793\\
0.75	-0.235714285714286\\
0.76	-0.241818181818182\\
0.77	-0.247966101694915\\
0.78	-0.254157303370787\\
0.79	-0.260391061452514\\
0.8	-0.266666666666667\\
0.81	-0.272983425414365\\
0.82	-0.279340659340659\\
0.83	-0.285737704918033\\
0.84	-0.292173913043478\\
0.85	-0.298648648648649\\
0.86	-0.305161290322581\\
0.87	-0.311711229946524\\
0.88	-0.318297872340426\\
0.89	-0.324920634920635\\
0.9	-0.331578947368421\\
0.91	-0.338272251308901\\
0.92	-0.345\\
0.93	-0.351761658031088\\
0.94	-0.358556701030928\\
0.95	-0.365384615384616\\
0.96	-0.372244897959184\\
0.97	-0.379137055837564\\
0.98	-0.386060606060606\\
0.99	-0.393015075376885\\
1	-0.4\\
1.01	-0.407014925373134\\
1.02	-0.414059405940594\\
1.03	-0.421133004926108\\
1.04	-0.428235294117647\\
1.05	-0.435365853658537\\
1.06	-0.44252427184466\\
1.07	-0.449710144927536\\
1.08	-0.456923076923077\\
1.09	-0.464162679425837\\
1.1	-0.471428571428572\\
1.11	-0.47872037914692\\
1.12	-0.486037735849057\\
1.13	-0.493380281690141\\
1.14	-0.500747663551402\\
1.15	-0.508139534883721\\
1.16	-0.515555555555556\\
1.17	-0.522995391705069\\
1.18	-0.53045871559633\\
1.19	-0.537945205479452\\
1.2	-0.545454545454546\\
1.21	-0.552986425339367\\
1.22	-0.56054054054054\\
1.23	-0.568116591928251\\
1.24	-0.575714285714286\\
1.25	-0.583333333333333\\
1.26	-0.590973451327434\\
1.27	-0.59863436123348\\
1.28	-0.606315789473684\\
1.29	-0.614017467248908\\
1.3	-0.621739130434783\\
1.31	-0.62948051948052\\
1.32	-0.637241379310345\\
1.33	-0.645021459227468\\
1.34	-0.652820512820513\\
1.35	-0.660638297872341\\
1.36	-0.668474576271187\\
1.37	-0.676329113924051\\
1.38	-0.684201680672269\\
1.39	-0.692092050209205\\
1.4	-0.7\\
1.41	-0.70792531120332\\
1.42	-0.715867768595041\\
1.43	-0.723827160493827\\
1.44	-0.731803278688525\\
1.45	-0.739795918367347\\
1.46	-0.74780487804878\\
1.47	-0.75582995951417\\
1.48	-0.763870967741936\\
1.49	-0.771927710843374\\
1.5	-0.78\\
1.51	-0.78808764940239\\
1.52	-0.796190476190476\\
1.53	-0.804308300395257\\
1.54	-0.81244094488189\\
1.55	-0.820588235294118\\
1.56	-0.82875\\
1.57	-0.836926070038911\\
1.58	-0.845116279069767\\
1.59	-0.853320463320463\\
1.6	-0.861538461538462\\
1.61	-0.869770114942529\\
1.62	-0.878015267175573\\
1.63	-0.886273764258555\\
1.64	-0.894545454545455\\
1.65	-0.902830188679245\\
1.66	-0.911127819548872\\
1.67	-0.919438202247191\\
1.68	-0.927761194029851\\
1.69	-0.936096654275093\\
1.7	-0.944444444444444\\
1.71	-0.95280442804428\\
1.72	-0.961176470588235\\
1.73	-0.96956043956044\\
1.74	-0.977956204379562\\
1.75	-0.986363636363636\\
1.76	-0.994782608695652\\
1.77	-1.00321299638989\\
1.78	-1.01165467625899\\
1.79	-1.02010752688172\\
1.8	-1.02857142857143\\
1.81	-1.0370462633452\\
1.82	-1.04553191489362\\
1.83	-1.05402826855124\\
1.84	-1.06253521126761\\
1.85	-1.07105263157895\\
1.86	-1.07958041958042\\
1.87	-1.08811846689895\\
1.88	-1.09666666666667\\
1.89	-1.10522491349481\\
1.9	-1.11379310344828\\
1.91	-1.12237113402062\\
1.92	-1.13095890410959\\
1.93	-1.13955631399317\\
1.94	-1.14816326530612\\
1.95	-1.15677966101695\\
1.96	-1.16540540540541\\
1.97	-1.1740404040404\\
1.98	-1.18268456375839\\
1.99	-1.19133779264214\\
2	-1.2\\
2.01	-1.20867109634552\\
2.02	-1.21735099337748\\
2.03	-1.2260396039604\\
2.04	-1.23473684210526\\
2.05	-1.24344262295082\\
2.06	-1.2521568627451\\
2.07	-1.26087947882736\\
2.08	-1.26961038961039\\
2.09	-1.27834951456311\\
2.1	-1.28709677419355\\
2.11	-1.29585209003215\\
2.12	-1.30461538461538\\
2.13	-1.31338658146965\\
2.14	-1.32216560509554\\
2.15	-1.33095238095238\\
2.16	-1.33974683544304\\
2.17	-1.34854889589905\\
2.18	-1.35735849056604\\
2.19	-1.36617554858934\\
2.2	-1.375\\
2.21	-1.38383177570093\\
2.22	-1.39267080745342\\
2.23	-1.40151702786378\\
2.24	-1.41037037037037\\
2.25	-1.41923076923077\\
2.26	-1.4280981595092\\
2.27	-1.43697247706422\\
2.28	-1.44585365853659\\
2.29	-1.45474164133739\\
2.3	-1.46363636363636\\
2.31	-1.47253776435045\\
2.32	-1.48144578313253\\
2.33	-1.49036036036036\\
2.34	-1.49928143712575\\
2.35	-1.50820895522388\\
2.36	-1.51714285714286\\
2.37	-1.52608308605341\\
2.38	-1.53502958579882\\
2.39	-1.54398230088496\\
2.4	-1.55294117647059\\
2.41	-1.56190615835777\\
2.42	-1.57087719298246\\
2.43	-1.57985422740525\\
2.44	-1.58883720930233\\
2.45	-1.59782608695652\\
2.46	-1.60682080924855\\
2.47	-1.61582132564842\\
2.48	-1.6248275862069\\
2.49	-1.63383954154728\\
2.5	-1.64285714285714\\
2.51	-1.65188034188034\\
2.52	-1.66090909090909\\
2.53	-1.6699433427762\\
2.54	-1.67898305084746\\
2.55	-1.68802816901408\\
2.56	-1.69707865168539\\
2.57	-1.70613445378151\\
2.58	-1.71519553072626\\
2.59	-1.72426183844011\\
2.6	-1.73333333333333\\
2.61	-1.74240997229917\\
2.62	-1.75149171270718\\
2.63	-1.76057851239669\\
2.64	-1.76967032967033\\
2.65	-1.77876712328767\\
2.66	-1.78786885245902\\
2.67	-1.79697547683924\\
2.68	-1.80608695652174\\
2.69	-1.81520325203252\\
2.7	-1.82432432432432\\
2.71	-1.83345013477089\\
2.72	-1.84258064516129\\
2.73	-1.85171581769437\\
2.74	-1.86085561497326\\
2.75	-1.87\\
2.76	-1.87914893617021\\
2.77	-1.8883023872679\\
2.78	-1.89746031746032\\
2.79	-1.90662269129288\\
2.8	-1.91578947368421\\
2.81	-1.92496062992126\\
2.82	-1.93413612565445\\
2.83	-1.94331592689295\\
2.84	-1.9525\\
2.85	-1.96168831168831\\
2.86	-1.97088082901554\\
2.87	-1.98007751937984\\
2.88	-1.98927835051546\\
2.89	-1.99848329048843\\
2.9	-2.00769230769231\\
2.91	-2.01690537084399\\
2.92	-2.02612244897959\\
2.93	-2.03534351145038\\
2.94	-2.04456852791878\\
2.95	-2.05379746835443\\
2.96	-2.0630303030303\\
2.97	-2.07226700251889\\
2.98	-2.08150753768844\\
2.99	-2.09075187969925\\
3	-2.1\\
3.01	-2.10925187032419\\
3.02	-2.11850746268657\\
3.03	-2.12776674937965\\
3.04	-2.1370297029703\\
3.05	-2.1462962962963\\
3.06	-2.15556650246305\\
3.07	-2.1648402948403\\
3.08	-2.17411764705882\\
3.09	-2.18339853300733\\
3.1	-2.19268292682927\\
3.11	-2.20197080291971\\
3.12	-2.21126213592233\\
3.13	-2.22055690072639\\
3.14	-2.22985507246377\\
3.15	-2.23915662650602\\
3.16	-2.24846153846154\\
3.17	-2.25776978417266\\
3.18	-2.26708133971292\\
3.19	-2.27639618138425\\
3.2	-2.28571428571429\\
3.21	-2.29503562945368\\
3.22	-2.30436018957346\\
3.23	-2.31368794326241\\
3.24	-2.32301886792453\\
3.25	-2.33235294117647\\
3.26	-2.34169014084507\\
3.27	-2.35103044496487\\
3.28	-2.3603738317757\\
3.29	-2.36972027972028\\
3.3	-2.37906976744186\\
3.31	-2.3884222737819\\
3.32	-2.39777777777778\\
3.33	-2.40713625866051\\
3.34	-2.41649769585253\\
3.35	-2.42586206896552\\
3.36	-2.43522935779816\\
3.37	-2.4445995423341\\
3.38	-2.45397260273973\\
3.39	-2.46334851936219\\
3.4	-2.47272727272727\\
3.41	-2.48210884353742\\
3.42	-2.49149321266968\\
3.43	-2.50088036117381\\
3.44	-2.51027027027027\\
3.45	-2.51966292134831\\
3.46	-2.52905829596413\\
3.47	-2.53845637583893\\
3.48	-2.54785714285714\\
3.49	-2.55726057906459\\
3.5	-2.56666666666667\\
3.51	-2.57607538802661\\
3.52	-2.58548672566372\\
3.53	-2.59490066225166\\
3.54	-2.60431718061674\\
3.55	-2.61373626373626\\
3.56	-2.62315789473684\\
3.57	-2.63258205689278\\
3.58	-2.64200873362445\\
3.59	-2.65143790849673\\
3.6	-2.66086956521739\\
3.61	-2.67030368763557\\
3.62	-2.67974025974026\\
3.63	-2.68917926565875\\
3.64	-2.69862068965517\\
3.65	-2.70806451612903\\
3.66	-2.71751072961373\\
3.67	-2.72695931477516\\
3.68	-2.73641025641026\\
3.69	-2.74586353944563\\
3.7	-2.75531914893617\\
3.71	-2.76477707006369\\
3.72	-2.77423728813559\\
3.73	-2.78369978858351\\
3.74	-2.79316455696203\\
3.75	-2.80263157894737\\
3.76	-2.81210084033613\\
3.77	-2.82157232704403\\
3.78	-2.8310460251046\\
3.79	-2.84052192066806\\
3.8	-2.85\\
3.81	-2.85948024948025\\
3.82	-2.86896265560166\\
3.83	-2.87844720496894\\
3.84	-2.88793388429752\\
3.85	-2.89742268041237\\
3.86	-2.90691358024691\\
3.87	-2.91640657084189\\
3.88	-2.92590163934426\\
3.89	-2.93539877300613\\
3.9	-2.94489795918367\\
3.91	-2.95439918533605\\
3.92	-2.96390243902439\\
3.93	-2.97340770791075\\
3.94	-2.98291497975708\\
3.95	-2.99242424242424\\
3.96	-3.00193548387097\\
3.97	-3.01144869215292\\
3.98	-3.02096385542169\\
3.99	-3.03048096192385\\
4	-3.04\\
4.01	-3.04952095808383\\
4.02	-3.05904382470119\\
4.03	-3.06856858846919\\
4.04	-3.07809523809524\\
4.05	-3.08762376237624\\
4.06	-3.09715415019763\\
4.07	-3.10668639053254\\
4.08	-3.11622047244094\\
4.09	-3.12575638506876\\
4.1	-3.13529411764706\\
4.11	-3.14483365949119\\
4.12	-3.154375\\
4.13	-3.16391812865497\\
4.14	-3.17346303501945\\
4.15	-3.18300970873786\\
4.16	-3.19255813953488\\
4.17	-3.2021083172147\\
4.18	-3.21166023166023\\
4.19	-3.22121387283237\\
4.2	-3.23076923076923\\
4.21	-3.24032629558541\\
4.22	-3.24988505747126\\
4.23	-3.25944550669216\\
4.24	-3.26900763358779\\
4.25	-3.27857142857143\\
4.26	-3.28813688212928\\
4.27	-3.29770398481973\\
4.28	-3.30727272727273\\
4.29	-3.31684310018904\\
4.3	-3.32641509433962\\
4.31	-3.33598870056497\\
4.32	-3.34556390977444\\
4.33	-3.35514071294559\\
4.34	-3.3647191011236\\
4.35	-3.37429906542056\\
4.36	-3.38388059701493\\
4.37	-3.39346368715084\\
4.38	-3.40304832713755\\
4.39	-3.41263450834879\\
4.4	-3.42222222222222\\
4.41	-3.43181146025878\\
4.42	-3.44140221402214\\
4.43	-3.45099447513812\\
4.44	-3.46058823529412\\
4.45	-3.47018348623853\\
4.46	-3.47978021978022\\
4.47	-3.48937842778793\\
4.48	-3.49897810218978\\
4.49	-3.50857923497268\\
4.5	-3.51818181818182\\
4.51	-3.52778584392015\\
4.52	-3.53739130434783\\
4.53	-3.54699819168174\\
4.54	-3.55660649819495\\
4.55	-3.56621621621622\\
4.56	-3.5758273381295\\
4.57	-3.58543985637343\\
4.58	-3.59505376344086\\
4.59	-3.60466905187835\\
4.6	-3.61428571428571\\
4.61	-3.62390374331551\\
4.62	-3.6335231316726\\
4.63	-3.64314387211368\\
4.64	-3.65276595744681\\
4.65	-3.66238938053097\\
4.66	-3.67201413427562\\
4.67	-3.68164021164021\\
4.68	-3.6912676056338\\
4.69	-3.70089630931459\\
4.7	-3.71052631578947\\
4.71	-3.72015761821366\\
4.72	-3.72979020979021\\
4.73	-3.73942408376963\\
4.74	-3.74905923344948\\
4.75	-3.75869565217391\\
4.76	-3.76833333333333\\
4.77	-3.77797227036395\\
4.78	-3.7876124567474\\
4.79	-3.79725388601036\\
4.8	-3.80689655172414\\
4.81	-3.8165404475043\\
4.82	-3.82618556701031\\
4.83	-3.83583190394511\\
4.84	-3.84547945205479\\
4.85	-3.8551282051282\\
4.86	-3.86477815699659\\
4.87	-3.87442930153322\\
4.88	-3.88408163265306\\
4.89	-3.89373514431239\\
4.9	-3.90338983050848\\
4.91	-3.91304568527919\\
4.92	-3.9227027027027\\
4.93	-3.93236087689713\\
4.94	-3.9420202020202\\
4.95	-3.95168067226891\\
4.96	-3.96134228187919\\
4.97	-3.97100502512563\\
4.98	-3.98066889632107\\
4.99	-3.99033388981636\\
5	-4\\
};
\addplot [color=mycolor7,solid,forget plot]
  table[row sep=crcr]{%
0	0\\
0.01	0.00287128712871287\\
0.02	0.00549019607843137\\
0.03	0.00786407766990291\\
0.04	0.00999999999999999\\
0.05	0.0119047619047619\\
0.06	0.0135849056603773\\
0.07	0.0150467289719626\\
0.08	0.0162962962962963\\
0.09	0.0173394495412844\\
0.1	0.0181818181818182\\
0.11	0.0188288288288288\\
0.12	0.0192857142857142\\
0.13	0.0195575221238938\\
0.14	0.0196491228070175\\
0.15	0.0195652173913043\\
0.16	0.0193103448275862\\
0.17	0.0188888888888889\\
0.18	0.0183050847457627\\
0.19	0.017563025210084\\
0.2	0.0166666666666667\\
0.21	0.0156198347107438\\
0.22	0.0144262295081967\\
0.23	0.0130894308943089\\
0.24	0.0116129032258064\\
0.25	0.00999999999999995\\
0.26	0.00825396825396824\\
0.27	0.00637795275590541\\
0.28	0.00437499999999996\\
0.29	0.00224806201550382\\
0.3	-5.55111512312578e-17\\
0.31	-0.00236641221374051\\
0.32	-0.00484848484848494\\
0.33	-0.00744360902255642\\
0.34	-0.0101492537313433\\
0.35	-0.012962962962963\\
0.36	-0.0158823529411765\\
0.37	-0.0189051094890512\\
0.38	-0.0220289855072464\\
0.39	-0.0252517985611512\\
0.4	-0.0285714285714286\\
0.41	-0.031985815602837\\
0.42	-0.0354929577464789\\
0.43	-0.0390909090909091\\
0.44	-0.0427777777777778\\
0.45	-0.0465517241379311\\
0.46	-0.0504109589041096\\
0.47	-0.0543537414965987\\
0.48	-0.0583783783783784\\
0.49	-0.0624832214765101\\
0.5	-0.0666666666666667\\
0.51	-0.0709271523178809\\
0.52	-0.0752631578947369\\
0.53	-0.0796732026143792\\
0.54	-0.0841558441558443\\
0.55	-0.0887096774193549\\
0.56	-0.0933333333333334\\
0.57	-0.0980254777070065\\
0.58	-0.102784810126582\\
0.59	-0.107610062893082\\
0.6	-0.1125\\
0.61	-0.117453416149068\\
0.62	-0.122469135802469\\
0.63	-0.127546012269939\\
0.64	-0.132682926829268\\
0.65	-0.137878787878788\\
0.66	-0.143132530120482\\
0.67	-0.148443113772455\\
0.68	-0.153809523809524\\
0.69	-0.159230769230769\\
0.7	-0.164705882352941\\
0.71	-0.170233918128655\\
0.72	-0.175813953488372\\
0.73	-0.181445086705202\\
0.74	-0.187126436781609\\
0.75	-0.192857142857143\\
0.76	-0.198636363636364\\
0.77	-0.204463276836158\\
0.78	-0.210337078651685\\
0.79	-0.216256983240224\\
0.8	-0.222222222222222\\
0.81	-0.228232044198895\\
0.82	-0.234285714285714\\
0.83	-0.240382513661202\\
0.84	-0.246521739130435\\
0.85	-0.252702702702703\\
0.86	-0.258924731182796\\
0.87	-0.265187165775401\\
0.88	-0.271489361702128\\
0.89	-0.277830687830688\\
0.9	-0.28421052631579\\
0.91	-0.290628272251309\\
0.92	-0.297083333333333\\
0.93	-0.303575129533679\\
0.94	-0.310103092783505\\
0.95	-0.316666666666667\\
0.96	-0.323265306122449\\
0.97	-0.329898477157361\\
0.98	-0.336565656565657\\
0.99	-0.343266331658291\\
1	-0.35\\
1.01	-0.356766169154229\\
1.02	-0.363564356435644\\
1.03	-0.370394088669951\\
1.04	-0.377254901960784\\
1.05	-0.384146341463415\\
1.06	-0.391067961165049\\
1.07	-0.398019323671498\\
1.08	-0.405\\
1.09	-0.412009569377991\\
1.1	-0.419047619047619\\
1.11	-0.42611374407583\\
1.12	-0.433207547169811\\
1.13	-0.440328638497653\\
1.14	-0.447476635514019\\
1.15	-0.454651162790698\\
1.16	-0.461851851851852\\
1.17	-0.469078341013825\\
1.18	-0.476330275229358\\
1.19	-0.483607305936073\\
1.2	-0.490909090909091\\
1.21	-0.498235294117647\\
1.22	-0.505585585585586\\
1.23	-0.512959641255605\\
1.24	-0.520357142857143\\
1.25	-0.527777777777778\\
1.26	-0.535221238938053\\
1.27	-0.542687224669604\\
1.28	-0.550175438596491\\
1.29	-0.557685589519651\\
1.3	-0.565217391304348\\
1.31	-0.572770562770563\\
1.32	-0.580344827586207\\
1.33	-0.58793991416309\\
1.34	-0.595555555555556\\
1.35	-0.603191489361702\\
1.36	-0.610847457627119\\
1.37	-0.618523206751055\\
1.38	-0.626218487394958\\
1.39	-0.633933054393306\\
1.4	-0.641666666666667\\
1.41	-0.64941908713693\\
1.42	-0.657190082644628\\
1.43	-0.664979423868313\\
1.44	-0.672786885245902\\
1.45	-0.680612244897959\\
1.46	-0.688455284552846\\
1.47	-0.696315789473684\\
1.48	-0.704193548387097\\
1.49	-0.712088353413655\\
1.5	-0.72\\
1.51	-0.72792828685259\\
1.52	-0.735873015873016\\
1.53	-0.743833992094862\\
1.54	-0.751811023622047\\
1.55	-0.759803921568628\\
1.56	-0.7678125\\
1.57	-0.775836575875487\\
1.58	-0.783875968992248\\
1.59	-0.791930501930502\\
1.6	-0.8\\
1.61	-0.80808429118774\\
1.62	-0.81618320610687\\
1.63	-0.824296577946768\\
1.64	-0.832424242424243\\
1.65	-0.840566037735849\\
1.66	-0.848721804511278\\
1.67	-0.85689138576779\\
1.68	-0.865074626865672\\
1.69	-0.873271375464684\\
1.7	-0.881481481481482\\
1.71	-0.889704797047971\\
1.72	-0.897941176470588\\
1.73	-0.906190476190476\\
1.74	-0.914452554744526\\
1.75	-0.922727272727273\\
1.76	-0.931014492753623\\
1.77	-0.939314079422383\\
1.78	-0.947625899280576\\
1.79	-0.955949820788531\\
1.8	-0.964285714285714\\
1.81	-0.972633451957295\\
1.82	-0.980992907801419\\
1.83	-0.989363957597173\\
1.84	-0.99774647887324\\
1.85	-1.00614035087719\\
1.86	-1.01454545454545\\
1.87	-1.02296167247387\\
1.88	-1.03138888888889\\
1.89	-1.03982698961938\\
1.9	-1.04827586206897\\
1.91	-1.056735395189\\
1.92	-1.06520547945205\\
1.93	-1.07368600682594\\
1.94	-1.0821768707483\\
1.95	-1.09067796610169\\
1.96	-1.09918918918919\\
1.97	-1.10771043771044\\
1.98	-1.11624161073826\\
1.99	-1.12478260869565\\
2	-1.13333333333333\\
2.01	-1.14189368770764\\
2.02	-1.15046357615894\\
2.03	-1.15904290429043\\
2.04	-1.16763157894737\\
2.05	-1.17622950819672\\
2.06	-1.18483660130719\\
2.07	-1.19345276872964\\
2.08	-1.20207792207792\\
2.09	-1.21071197411003\\
2.1	-1.21935483870968\\
2.11	-1.22800643086817\\
2.12	-1.23666666666667\\
2.13	-1.24533546325879\\
2.14	-1.2540127388535\\
2.15	-1.26269841269841\\
2.16	-1.27139240506329\\
2.17	-1.28009463722397\\
2.18	-1.28880503144654\\
2.19	-1.29752351097179\\
2.2	-1.30625\\
2.21	-1.31498442367601\\
2.22	-1.32372670807453\\
2.23	-1.33247678018576\\
2.24	-1.34123456790123\\
2.25	-1.35\\
2.26	-1.35877300613497\\
2.27	-1.36755351681957\\
2.28	-1.37634146341463\\
2.29	-1.3851367781155\\
2.3	-1.39393939393939\\
2.31	-1.40274924471299\\
2.32	-1.41156626506024\\
2.33	-1.42039039039039\\
2.34	-1.42922155688623\\
2.35	-1.43805970149254\\
2.36	-1.44690476190476\\
2.37	-1.45575667655786\\
2.38	-1.46461538461538\\
2.39	-1.4734808259587\\
2.4	-1.48235294117647\\
2.41	-1.49123167155425\\
2.42	-1.50011695906433\\
2.43	-1.50900874635569\\
2.44	-1.51790697674419\\
2.45	-1.5268115942029\\
2.46	-1.5357225433526\\
2.47	-1.54463976945245\\
2.48	-1.5535632183908\\
2.49	-1.56249283667622\\
2.5	-1.57142857142857\\
2.51	-1.58037037037037\\
2.52	-1.58931818181818\\
2.53	-1.59827195467422\\
2.54	-1.60723163841808\\
2.55	-1.61619718309859\\
2.56	-1.62516853932584\\
2.57	-1.63414565826331\\
2.58	-1.64312849162011\\
2.59	-1.65211699164345\\
2.6	-1.66111111111111\\
2.61	-1.6701108033241\\
2.62	-1.67911602209945\\
2.63	-1.68812672176309\\
2.64	-1.69714285714286\\
2.65	-1.70616438356164\\
2.66	-1.7151912568306\\
2.67	-1.72422343324251\\
2.68	-1.73326086956522\\
2.69	-1.74230352303523\\
2.7	-1.75135135135135\\
2.71	-1.76040431266846\\
2.72	-1.7694623655914\\
2.73	-1.7785254691689\\
2.74	-1.7875935828877\\
2.75	-1.79666666666667\\
2.76	-1.80574468085106\\
2.77	-1.8148275862069\\
2.78	-1.82391534391534\\
2.79	-1.83300791556728\\
2.8	-1.84210526315789\\
2.81	-1.85120734908136\\
2.82	-1.86031413612565\\
2.83	-1.86942558746736\\
2.84	-1.87854166666667\\
2.85	-1.88766233766234\\
2.86	-1.89678756476684\\
2.87	-1.9059173126615\\
2.88	-1.91505154639175\\
2.89	-1.92419023136247\\
2.9	-1.93333333333333\\
2.91	-1.94248081841432\\
2.92	-1.95163265306122\\
2.93	-1.96078880407125\\
2.94	-1.96994923857868\\
2.95	-1.97911392405063\\
2.96	-1.98828282828283\\
2.97	-1.99745591939547\\
2.98	-2.00663316582915\\
2.99	-2.01581453634085\\
3	-2.025\\
3.01	-2.03418952618454\\
3.02	-2.04338308457711\\
3.03	-2.05258064516129\\
3.04	-2.06178217821782\\
3.05	-2.07098765432099\\
3.06	-2.08019704433498\\
3.07	-2.08941031941032\\
3.08	-2.09862745098039\\
3.09	-2.10784841075795\\
3.1	-2.11707317073171\\
3.11	-2.12630170316302\\
3.12	-2.13553398058252\\
3.13	-2.14476997578693\\
3.14	-2.15400966183575\\
3.15	-2.16325301204819\\
3.16	-2.1725\\
3.17	-2.18175059952038\\
3.18	-2.191004784689\\
3.19	-2.20026252983294\\
3.2	-2.20952380952381\\
3.21	-2.21878859857482\\
3.22	-2.22805687203791\\
3.23	-2.23732860520095\\
3.24	-2.24660377358491\\
3.25	-2.25588235294118\\
3.26	-2.26516431924883\\
3.27	-2.27444964871194\\
3.28	-2.28373831775701\\
3.29	-2.2930303030303\\
3.3	-2.30232558139535\\
3.31	-2.31162412993039\\
3.32	-2.32092592592593\\
3.33	-2.33023094688222\\
3.34	-2.33953917050691\\
3.35	-2.34885057471264\\
3.36	-2.35816513761468\\
3.37	-2.3674828375286\\
3.38	-2.37680365296804\\
3.39	-2.38612756264237\\
3.4	-2.39545454545455\\
3.41	-2.40478458049887\\
3.42	-2.41411764705882\\
3.43	-2.42345372460497\\
3.44	-2.43279279279279\\
3.45	-2.44213483146067\\
3.46	-2.4514798206278\\
3.47	-2.46082774049217\\
3.48	-2.47017857142857\\
3.49	-2.47953229398664\\
3.5	-2.48888888888889\\
3.51	-2.49824833702882\\
3.52	-2.50761061946903\\
3.53	-2.51697571743929\\
3.54	-2.5263436123348\\
3.55	-2.53571428571429\\
3.56	-2.54508771929825\\
3.57	-2.55446389496718\\
3.58	-2.56384279475983\\
3.59	-2.57322440087146\\
3.6	-2.58260869565217\\
3.61	-2.59199566160521\\
3.62	-2.60138528138528\\
3.63	-2.61077753779698\\
3.64	-2.6201724137931\\
3.65	-2.62956989247312\\
3.66	-2.63896995708155\\
3.67	-2.64837259100642\\
3.68	-2.65777777777778\\
3.69	-2.6671855010661\\
3.7	-2.67659574468085\\
3.71	-2.686008492569\\
3.72	-2.69542372881356\\
3.73	-2.70484143763214\\
3.74	-2.71426160337553\\
3.75	-2.72368421052632\\
3.76	-2.73310924369748\\
3.77	-2.74253668763103\\
3.78	-2.75196652719665\\
3.79	-2.7613987473904\\
3.8	-2.77083333333333\\
3.81	-2.78027027027027\\
3.82	-2.78970954356847\\
3.83	-2.79915113871636\\
3.84	-2.80859504132231\\
3.85	-2.8180412371134\\
3.86	-2.82748971193416\\
3.87	-2.83694045174538\\
3.88	-2.84639344262295\\
3.89	-2.85584867075665\\
3.9	-2.86530612244898\\
3.91	-2.87476578411405\\
3.92	-2.88422764227642\\
3.93	-2.89369168356998\\
3.94	-2.90315789473684\\
3.95	-2.91262626262626\\
3.96	-2.92209677419355\\
3.97	-2.93156941649899\\
3.98	-2.94104417670683\\
3.99	-2.95052104208417\\
4	-2.96\\
4.01	-2.96948103792415\\
4.02	-2.97896414342629\\
4.03	-2.98844930417495\\
4.04	-2.99793650793651\\
4.05	-3.00742574257426\\
4.06	-3.01691699604743\\
4.07	-3.02641025641026\\
4.08	-3.03590551181102\\
4.09	-3.04540275049116\\
4.1	-3.05490196078431\\
4.11	-3.06440313111546\\
4.12	-3.07390625\\
4.13	-3.08341130604289\\
4.14	-3.09291828793774\\
4.15	-3.10242718446602\\
4.16	-3.11193798449612\\
4.17	-3.12145067698259\\
4.18	-3.13096525096525\\
4.19	-3.1404816955684\\
4.2	-3.15\\
4.21	-3.15952015355086\\
4.22	-3.16904214559387\\
4.23	-3.17856596558317\\
4.24	-3.18809160305344\\
4.25	-3.19761904761905\\
4.26	-3.20714828897338\\
4.27	-3.21667931688805\\
4.28	-3.22621212121212\\
4.29	-3.23574669187146\\
4.3	-3.24528301886792\\
4.31	-3.25482109227872\\
4.32	-3.26436090225564\\
4.33	-3.27390243902439\\
4.34	-3.2834456928839\\
4.35	-3.29299065420561\\
4.36	-3.30253731343284\\
4.37	-3.31208566108007\\
4.38	-3.32163568773234\\
4.39	-3.33118738404453\\
4.4	-3.34074074074074\\
4.41	-3.35029574861368\\
4.42	-3.35985239852399\\
4.43	-3.36941068139963\\
4.44	-3.37897058823529\\
4.45	-3.38853211009174\\
4.46	-3.39809523809524\\
4.47	-3.40765996343693\\
4.48	-3.41722627737226\\
4.49	-3.4267941712204\\
4.5	-3.43636363636364\\
4.51	-3.44593466424682\\
4.52	-3.45550724637681\\
4.53	-3.46508137432188\\
4.54	-3.47465703971119\\
4.55	-3.48423423423423\\
4.56	-3.49381294964029\\
4.57	-3.50339317773788\\
4.58	-3.51297491039427\\
4.59	-3.52255813953488\\
4.6	-3.53214285714286\\
4.61	-3.54172905525847\\
4.62	-3.55131672597865\\
4.63	-3.56090586145648\\
4.64	-3.57049645390071\\
4.65	-3.58008849557522\\
4.66	-3.58968197879859\\
4.67	-3.59927689594356\\
4.68	-3.60887323943662\\
4.69	-3.61847100175747\\
4.7	-3.6280701754386\\
4.71	-3.6376707530648\\
4.72	-3.64727272727273\\
4.73	-3.65687609075044\\
4.74	-3.66648083623693\\
4.75	-3.67608695652174\\
4.76	-3.68569444444444\\
4.77	-3.69530329289428\\
4.78	-3.70491349480969\\
4.79	-3.71452504317789\\
4.8	-3.72413793103448\\
4.81	-3.73375215146299\\
4.82	-3.7433676975945\\
4.83	-3.7529845626072\\
4.84	-3.76260273972603\\
4.85	-3.77222222222222\\
4.86	-3.78184300341297\\
4.87	-3.79146507666099\\
4.88	-3.80108843537415\\
4.89	-3.81071307300509\\
4.9	-3.82033898305085\\
4.91	-3.82996615905245\\
4.92	-3.83959459459459\\
4.93	-3.84922428330523\\
4.94	-3.85885521885522\\
4.95	-3.86848739495798\\
4.96	-3.87812080536913\\
4.97	-3.8877554438861\\
4.98	-3.89739130434783\\
4.99	-3.90702838063439\\
5	-3.91666666666667\\
};
\addplot [color=mycolor1,solid,forget plot]
  table[row sep=crcr]{%
0	0\\
0.01	0.00386138613861386\\
0.02	0.00745098039215686\\
0.03	0.0107766990291262\\
0.04	0.0138461538461538\\
0.05	0.0166666666666667\\
0.06	0.0192452830188679\\
0.07	0.021588785046729\\
0.08	0.0237037037037037\\
0.09	0.0255963302752293\\
0.1	0.0272727272727273\\
0.11	0.0287387387387387\\
0.12	0.03\\
0.13	0.0310619469026548\\
0.14	0.0319298245614035\\
0.15	0.0326086956521739\\
0.16	0.0331034482758621\\
0.17	0.0334188034188034\\
0.18	0.0335593220338983\\
0.19	0.0335294117647059\\
0.2	0.0333333333333334\\
0.21	0.0329752066115702\\
0.22	0.0324590163934426\\
0.23	0.0317886178861788\\
0.24	0.0309677419354839\\
0.25	0.03\\
0.26	0.0288888888888889\\
0.27	0.0276377952755905\\
0.28	0.02625\\
0.29	0.0247286821705426\\
0.3	0.0230769230769231\\
0.31	0.021297709923664\\
0.32	0.0193939393939394\\
0.33	0.0173684210526315\\
0.34	0.0152238805970149\\
0.35	0.012962962962963\\
0.36	0.0105882352941177\\
0.37	0.00810218978102184\\
0.38	0.0055072463768116\\
0.39	0.00280575539568345\\
0.4	0\\
0.41	-0.00290780141843977\\
0.42	-0.00591549295774646\\
0.43	-0.00902097902097904\\
0.44	-0.0122222222222222\\
0.45	-0.0155172413793104\\
0.46	-0.0189041095890411\\
0.47	-0.0223809523809524\\
0.48	-0.025945945945946\\
0.49	-0.0295973154362416\\
0.5	-0.0333333333333334\\
0.51	-0.0371523178807947\\
0.52	-0.0410526315789474\\
0.53	-0.0450326797385621\\
0.54	-0.0490909090909091\\
0.55	-0.053225806451613\\
0.56	-0.0574358974358975\\
0.57	-0.06171974522293\\
0.58	-0.0660759493670886\\
0.59	-0.070503144654088\\
0.6	-0.0750000000000001\\
0.61	-0.0795652173913044\\
0.62	-0.0841975308641977\\
0.63	-0.0888957055214724\\
0.64	-0.0936585365853659\\
0.65	-0.0984848484848485\\
0.66	-0.103373493975904\\
0.67	-0.108323353293413\\
0.68	-0.113333333333334\\
0.69	-0.118402366863905\\
0.7	-0.123529411764706\\
0.71	-0.128713450292398\\
0.72	-0.133953488372093\\
0.73	-0.139248554913295\\
0.74	-0.144597701149425\\
0.75	-0.15\\
0.76	-0.155454545454546\\
0.77	-0.160960451977401\\
0.78	-0.166516853932584\\
0.79	-0.172122905027933\\
0.8	-0.177777777777778\\
0.81	-0.183480662983425\\
0.82	-0.189230769230769\\
0.83	-0.195027322404372\\
0.84	-0.200869565217391\\
0.85	-0.206756756756757\\
0.86	-0.212688172043011\\
0.87	-0.218663101604278\\
0.88	-0.22468085106383\\
0.89	-0.230740740740741\\
0.9	-0.236842105263158\\
0.91	-0.242984293193717\\
0.92	-0.249166666666667\\
0.93	-0.255388601036269\\
0.94	-0.261649484536082\\
0.95	-0.267948717948718\\
0.96	-0.274285714285714\\
0.97	-0.280659898477157\\
0.98	-0.287070707070707\\
0.99	-0.293517587939699\\
1	-0.3\\
1.01	-0.306517412935323\\
1.02	-0.313069306930693\\
1.03	-0.319655172413793\\
1.04	-0.326274509803922\\
1.05	-0.332926829268293\\
1.06	-0.339611650485437\\
1.07	-0.346328502415459\\
1.08	-0.353076923076923\\
1.09	-0.359856459330144\\
1.1	-0.366666666666667\\
1.11	-0.373507109004739\\
1.12	-0.380377358490566\\
1.13	-0.387276995305164\\
1.14	-0.394205607476636\\
1.15	-0.401162790697675\\
1.16	-0.408148148148148\\
1.17	-0.415161290322581\\
1.18	-0.422201834862385\\
1.19	-0.429269406392694\\
1.2	-0.436363636363637\\
1.21	-0.443484162895928\\
1.22	-0.450630630630631\\
1.23	-0.45780269058296\\
1.24	-0.465\\
1.25	-0.472222222222222\\
1.26	-0.479469026548673\\
1.27	-0.486740088105727\\
1.28	-0.494035087719298\\
1.29	-0.501353711790393\\
1.3	-0.508695652173913\\
1.31	-0.516060606060606\\
1.32	-0.523448275862069\\
1.33	-0.530858369098713\\
1.34	-0.538290598290598\\
1.35	-0.545744680851064\\
1.36	-0.553220338983051\\
1.37	-0.560717299578059\\
1.38	-0.568235294117647\\
1.39	-0.575774058577406\\
1.4	-0.583333333333333\\
1.41	-0.590912863070539\\
1.42	-0.598512396694215\\
1.43	-0.606131687242798\\
1.44	-0.613770491803279\\
1.45	-0.621428571428571\\
1.46	-0.629105691056911\\
1.47	-0.636801619433198\\
1.48	-0.644516129032258\\
1.49	-0.652248995983936\\
1.5	-0.66\\
1.51	-0.667768924302789\\
1.52	-0.675555555555556\\
1.53	-0.683359683794467\\
1.54	-0.691181102362205\\
1.55	-0.699019607843137\\
1.56	-0.706875\\
1.57	-0.714747081712062\\
1.58	-0.722635658914729\\
1.59	-0.730540540540541\\
1.6	-0.738461538461539\\
1.61	-0.74639846743295\\
1.62	-0.754351145038168\\
1.63	-0.762319391634981\\
1.64	-0.77030303030303\\
1.65	-0.778301886792453\\
1.66	-0.786315789473684\\
1.67	-0.79434456928839\\
1.68	-0.802388059701492\\
1.69	-0.810446096654275\\
1.7	-0.818518518518519\\
1.71	-0.82660516605166\\
1.72	-0.834705882352941\\
1.73	-0.842820512820513\\
1.74	-0.850948905109489\\
1.75	-0.859090909090909\\
1.76	-0.867246376811594\\
1.77	-0.875415162454874\\
1.78	-0.883597122302158\\
1.79	-0.891792114695341\\
1.8	-0.9\\
1.81	-0.908220640569395\\
1.82	-0.91645390070922\\
1.83	-0.92469964664311\\
1.84	-0.932957746478873\\
1.85	-0.941228070175439\\
1.86	-0.94951048951049\\
1.87	-0.957804878048781\\
1.88	-0.966111111111111\\
1.89	-0.974429065743945\\
1.9	-0.982758620689655\\
1.91	-0.991099656357389\\
1.92	-0.999452054794521\\
1.93	-1.0078156996587\\
1.94	-1.01619047619048\\
1.95	-1.02457627118644\\
1.96	-1.03297297297297\\
1.97	-1.04138047138047\\
1.98	-1.04979865771812\\
1.99	-1.05822742474916\\
2	-1.06666666666667\\
2.01	-1.07511627906977\\
2.02	-1.0835761589404\\
2.03	-1.09204620462046\\
2.04	-1.10052631578947\\
2.05	-1.10901639344262\\
2.06	-1.11751633986928\\
2.07	-1.12602605863192\\
2.08	-1.13454545454545\\
2.09	-1.14307443365696\\
2.1	-1.15161290322581\\
2.11	-1.16016077170418\\
2.12	-1.16871794871795\\
2.13	-1.17728434504792\\
2.14	-1.18585987261147\\
2.15	-1.19444444444444\\
2.16	-1.20303797468354\\
2.17	-1.2116403785489\\
2.18	-1.22025157232704\\
2.19	-1.22887147335423\\
2.2	-1.2375\\
2.21	-1.24613707165109\\
2.22	-1.25478260869565\\
2.23	-1.26343653250774\\
2.24	-1.2720987654321\\
2.25	-1.28076923076923\\
2.26	-1.28944785276074\\
2.27	-1.29813455657492\\
2.28	-1.30682926829268\\
2.29	-1.31553191489362\\
2.3	-1.32424242424242\\
2.31	-1.33296072507553\\
2.32	-1.34168674698795\\
2.33	-1.35042042042042\\
2.34	-1.35916167664671\\
2.35	-1.36791044776119\\
2.36	-1.37666666666667\\
2.37	-1.38543026706231\\
2.38	-1.39420118343195\\
2.39	-1.40297935103245\\
2.4	-1.41176470588235\\
2.41	-1.42055718475073\\
2.42	-1.4293567251462\\
2.43	-1.43816326530612\\
2.44	-1.44697674418605\\
2.45	-1.45579710144928\\
2.46	-1.46462427745665\\
2.47	-1.47345821325648\\
2.48	-1.48229885057471\\
2.49	-1.49114613180516\\
2.5	-1.5\\
2.51	-1.5088603988604\\
2.52	-1.51772727272727\\
2.53	-1.52660056657224\\
2.54	-1.5354802259887\\
2.55	-1.5443661971831\\
2.56	-1.55325842696629\\
2.57	-1.5621568627451\\
2.58	-1.57106145251397\\
2.59	-1.5799721448468\\
2.6	-1.58888888888889\\
2.61	-1.59781163434903\\
2.62	-1.60674033149171\\
2.63	-1.61567493112948\\
2.64	-1.62461538461538\\
2.65	-1.63356164383562\\
2.66	-1.64251366120219\\
2.67	-1.65147138964578\\
2.68	-1.6604347826087\\
2.69	-1.66940379403794\\
2.7	-1.67837837837838\\
2.71	-1.68735849056604\\
2.72	-1.69634408602151\\
2.73	-1.70533512064343\\
2.74	-1.71433155080214\\
2.75	-1.72333333333333\\
2.76	-1.73234042553191\\
2.77	-1.74135278514589\\
2.78	-1.75037037037037\\
2.79	-1.75939313984169\\
2.8	-1.76842105263158\\
2.81	-1.77745406824147\\
2.82	-1.78649214659686\\
2.83	-1.79553524804178\\
2.84	-1.80458333333333\\
2.85	-1.81363636363636\\
2.86	-1.82269430051813\\
2.87	-1.83175710594315\\
2.88	-1.84082474226804\\
2.89	-1.8498971722365\\
2.9	-1.85897435897436\\
2.91	-1.86805626598465\\
2.92	-1.87714285714286\\
2.93	-1.88623409669211\\
2.94	-1.89532994923858\\
2.95	-1.90443037974684\\
2.96	-1.91353535353535\\
2.97	-1.92264483627204\\
2.98	-1.93175879396985\\
2.99	-1.94087719298246\\
3	-1.95\\
3.01	-1.95912718204489\\
3.02	-1.96825870646766\\
3.03	-1.97739454094293\\
3.04	-1.98653465346535\\
3.05	-1.99567901234568\\
3.06	-2.0048275862069\\
3.07	-2.01398034398034\\
3.08	-2.02313725490196\\
3.09	-2.03229828850856\\
3.1	-2.04146341463415\\
3.11	-2.05063260340633\\
3.12	-2.05980582524272\\
3.13	-2.06898305084746\\
3.14	-2.07816425120773\\
3.15	-2.08734939759036\\
3.16	-2.09653846153846\\
3.17	-2.10573141486811\\
3.18	-2.11492822966507\\
3.19	-2.12412887828162\\
3.2	-2.13333333333333\\
3.21	-2.14254156769596\\
3.22	-2.15175355450237\\
3.23	-2.16096926713948\\
3.24	-2.17018867924528\\
3.25	-2.17941176470588\\
3.26	-2.18863849765258\\
3.27	-2.19786885245902\\
3.28	-2.20710280373832\\
3.29	-2.21634032634033\\
3.3	-2.22558139534884\\
3.31	-2.23482598607889\\
3.32	-2.24407407407407\\
3.33	-2.25332563510393\\
3.34	-2.26258064516129\\
3.35	-2.27183908045977\\
3.36	-2.28110091743119\\
3.37	-2.29036613272311\\
3.38	-2.29963470319635\\
3.39	-2.30890660592255\\
3.4	-2.31818181818182\\
3.41	-2.32746031746032\\
3.42	-2.33674208144796\\
3.43	-2.34602708803612\\
3.44	-2.35531531531532\\
3.45	-2.36460674157303\\
3.46	-2.37390134529148\\
3.47	-2.38319910514541\\
3.48	-2.3925\\
3.49	-2.40180400890869\\
3.5	-2.41111111111111\\
3.51	-2.42042128603104\\
3.52	-2.42973451327434\\
3.53	-2.43905077262693\\
3.54	-2.44837004405286\\
3.55	-2.45769230769231\\
3.56	-2.46701754385965\\
3.57	-2.47634573304158\\
3.58	-2.4856768558952\\
3.59	-2.49501089324619\\
3.6	-2.50434782608696\\
3.61	-2.51368763557484\\
3.62	-2.5230303030303\\
3.63	-2.53237580993521\\
3.64	-2.54172413793103\\
3.65	-2.5510752688172\\
3.66	-2.56042918454936\\
3.67	-2.56978586723769\\
3.68	-2.5791452991453\\
3.69	-2.58850746268657\\
3.7	-2.59787234042553\\
3.71	-2.60723991507431\\
3.72	-2.61661016949153\\
3.73	-2.62598308668076\\
3.74	-2.63535864978903\\
3.75	-2.64473684210526\\
3.76	-2.65411764705882\\
3.77	-2.66350104821803\\
3.78	-2.6728870292887\\
3.79	-2.68227557411273\\
3.8	-2.69166666666667\\
3.81	-2.70106029106029\\
3.82	-2.71045643153527\\
3.83	-2.71985507246377\\
3.84	-2.72925619834711\\
3.85	-2.73865979381443\\
3.86	-2.7480658436214\\
3.87	-2.75747433264887\\
3.88	-2.76688524590164\\
3.89	-2.77629856850716\\
3.9	-2.78571428571429\\
3.91	-2.79513238289206\\
3.92	-2.80455284552846\\
3.93	-2.81397565922921\\
3.94	-2.8234008097166\\
3.95	-2.83282828282828\\
3.96	-2.84225806451613\\
3.97	-2.85169014084507\\
3.98	-2.86112449799197\\
3.99	-2.87056112224449\\
4	-2.88\\
4.01	-2.88944111776447\\
4.02	-2.89888446215139\\
4.03	-2.90833001988072\\
4.04	-2.91777777777778\\
4.05	-2.92722772277228\\
4.06	-2.93667984189723\\
4.07	-2.94613412228797\\
4.08	-2.9555905511811\\
4.09	-2.96504911591356\\
4.1	-2.97450980392157\\
4.11	-2.98397260273973\\
4.12	-2.9934375\\
4.13	-3.0029044834308\\
4.14	-3.01237354085603\\
4.15	-3.02184466019418\\
4.16	-3.03131782945736\\
4.17	-3.04079303675048\\
4.18	-3.05027027027027\\
4.19	-3.05974951830443\\
4.2	-3.06923076923077\\
4.21	-3.07871401151631\\
4.22	-3.08819923371647\\
4.23	-3.09768642447419\\
4.24	-3.10717557251908\\
4.25	-3.11666666666667\\
4.26	-3.12615969581749\\
4.27	-3.13565464895636\\
4.28	-3.14515151515152\\
4.29	-3.15465028355388\\
4.3	-3.16415094339623\\
4.31	-3.17365348399247\\
4.32	-3.18315789473684\\
4.33	-3.19266416510319\\
4.34	-3.20217228464419\\
4.35	-3.21168224299065\\
4.36	-3.22119402985075\\
4.37	-3.23070763500931\\
4.38	-3.24022304832714\\
4.39	-3.24974025974026\\
4.4	-3.25925925925926\\
4.41	-3.26878003696858\\
4.42	-3.27830258302583\\
4.43	-3.28782688766114\\
4.44	-3.29735294117647\\
4.45	-3.30688073394495\\
4.46	-3.31641025641026\\
4.47	-3.32594149908592\\
4.48	-3.33547445255475\\
4.49	-3.34500910746812\\
4.5	-3.35454545454545\\
4.51	-3.3640834845735\\
4.52	-3.3736231884058\\
4.53	-3.38316455696203\\
4.54	-3.39270758122744\\
4.55	-3.40225225225225\\
4.56	-3.41179856115108\\
4.57	-3.42134649910233\\
4.58	-3.43089605734767\\
4.59	-3.44044722719141\\
4.6	-3.45\\
4.61	-3.45955436720143\\
4.62	-3.4691103202847\\
4.63	-3.47866785079929\\
4.64	-3.48822695035461\\
4.65	-3.49778761061947\\
4.66	-3.50734982332155\\
4.67	-3.51691358024691\\
4.68	-3.52647887323944\\
4.69	-3.53604569420035\\
4.7	-3.54561403508772\\
4.71	-3.55518388791594\\
4.72	-3.56475524475524\\
4.73	-3.57432809773124\\
4.74	-3.58390243902439\\
4.75	-3.59347826086957\\
4.76	-3.60305555555556\\
4.77	-3.61263431542461\\
4.78	-3.62221453287197\\
4.79	-3.63179620034542\\
4.8	-3.64137931034483\\
4.81	-3.65096385542169\\
4.82	-3.66054982817869\\
4.83	-3.6701372212693\\
4.84	-3.67972602739726\\
4.85	-3.68931623931624\\
4.86	-3.69890784982935\\
4.87	-3.70850085178876\\
4.88	-3.71809523809524\\
4.89	-3.72769100169779\\
4.9	-3.73728813559322\\
4.91	-3.74688663282572\\
4.92	-3.75648648648649\\
4.93	-3.76608768971332\\
4.94	-3.77569023569024\\
4.95	-3.78529411764706\\
4.96	-3.79489932885906\\
4.97	-3.80450586264657\\
4.98	-3.81411371237458\\
4.99	-3.82372287145242\\
5	-3.83333333333333\\
};
\addplot [color=mycolor2,solid,forget plot]
  table[row sep=crcr]{%
0	0\\
0.01	0.00485148514851485\\
0.02	0.00941176470588235\\
0.03	0.0136893203883495\\
0.04	0.0176923076923077\\
0.05	0.0214285714285714\\
0.06	0.0249056603773585\\
0.07	0.0281308411214953\\
0.08	0.0311111111111111\\
0.09	0.0338532110091743\\
0.1	0.0363636363636363\\
0.11	0.0386486486486487\\
0.12	0.0407142857142857\\
0.13	0.0425663716814159\\
0.14	0.0442105263157895\\
0.15	0.0456521739130435\\
0.16	0.0468965517241379\\
0.17	0.047948717948718\\
0.18	0.0488135593220339\\
0.19	0.0494957983193277\\
0.2	0.05\\
0.21	0.0503305785123967\\
0.22	0.0504918032786885\\
0.23	0.0504878048780488\\
0.24	0.0503225806451613\\
0.25	0.05\\
0.26	0.0495238095238095\\
0.27	0.0488976377952756\\
0.28	0.048125\\
0.29	0.0472093023255814\\
0.3	0.0461538461538461\\
0.31	0.0449618320610687\\
0.32	0.0436363636363636\\
0.33	0.0421804511278195\\
0.34	0.0405970149253731\\
0.35	0.0388888888888888\\
0.36	0.0370588235294118\\
0.37	0.0351094890510949\\
0.38	0.0330434782608696\\
0.39	0.030863309352518\\
0.4	0.0285714285714286\\
0.41	0.0261702127659574\\
0.42	0.023661971830986\\
0.43	0.021048951048951\\
0.44	0.0183333333333334\\
0.45	0.0155172413793104\\
0.46	0.0126027397260274\\
0.47	0.00959183673469388\\
0.48	0.00648648648648653\\
0.49	0.00328859060402692\\
0.5	0\\
0.51	-0.00337748344370858\\
0.52	-0.00684210526315787\\
0.53	-0.0103921568627451\\
0.54	-0.0140259740259741\\
0.55	-0.017741935483871\\
0.56	-0.0215384615384616\\
0.57	-0.0254140127388535\\
0.58	-0.029367088607595\\
0.59	-0.0333962264150942\\
0.6	-0.0375\\
0.61	-0.0416770186335403\\
0.62	-0.045925925925926\\
0.63	-0.0502453987730062\\
0.64	-0.0546341463414635\\
0.65	-0.0590909090909091\\
0.66	-0.0636144578313254\\
0.67	-0.0682035928143712\\
0.68	-0.072857142857143\\
0.69	-0.0775739644970415\\
0.7	-0.0823529411764707\\
0.71	-0.0871929824561403\\
0.72	-0.0920930232558139\\
0.73	-0.0970520231213873\\
0.74	-0.102068965517241\\
0.75	-0.107142857142857\\
0.76	-0.112272727272727\\
0.77	-0.117457627118644\\
0.78	-0.122696629213483\\
0.79	-0.127988826815642\\
0.8	-0.133333333333333\\
0.81	-0.138729281767956\\
0.82	-0.144175824175824\\
0.83	-0.149672131147541\\
0.84	-0.155217391304348\\
0.85	-0.160810810810811\\
0.86	-0.166451612903226\\
0.87	-0.172139037433155\\
0.88	-0.177872340425532\\
0.89	-0.183650793650794\\
0.9	-0.189473684210526\\
0.91	-0.195340314136126\\
0.92	-0.20125\\
0.93	-0.20720207253886\\
0.94	-0.21319587628866\\
0.95	-0.219230769230769\\
0.96	-0.22530612244898\\
0.97	-0.231421319796954\\
0.98	-0.237575757575758\\
0.99	-0.243768844221105\\
1	-0.25\\
1.01	-0.256268656716418\\
1.02	-0.262574257425743\\
1.03	-0.268916256157636\\
1.04	-0.275294117647059\\
1.05	-0.281707317073171\\
1.06	-0.288155339805825\\
1.07	-0.29463768115942\\
1.08	-0.301153846153846\\
1.09	-0.307703349282297\\
1.1	-0.314285714285714\\
1.11	-0.320900473933649\\
1.12	-0.327547169811321\\
1.13	-0.334225352112676\\
1.14	-0.340934579439252\\
1.15	-0.347674418604651\\
1.16	-0.354444444444444\\
1.17	-0.361244239631336\\
1.18	-0.368073394495413\\
1.19	-0.374931506849315\\
1.2	-0.381818181818182\\
1.21	-0.388733031674208\\
1.22	-0.395675675675676\\
1.23	-0.402645739910314\\
1.24	-0.409642857142857\\
1.25	-0.416666666666667\\
1.26	-0.423716814159292\\
1.27	-0.43079295154185\\
1.28	-0.437894736842105\\
1.29	-0.445021834061135\\
1.3	-0.452173913043478\\
1.31	-0.459350649350649\\
1.32	-0.466551724137931\\
1.33	-0.473776824034335\\
1.34	-0.481025641025641\\
1.35	-0.488297872340426\\
1.36	-0.495593220338983\\
1.37	-0.502911392405063\\
1.38	-0.510252100840336\\
1.39	-0.517615062761506\\
1.4	-0.525\\
1.41	-0.532406639004149\\
1.42	-0.539834710743802\\
1.43	-0.547283950617284\\
1.44	-0.554754098360656\\
1.45	-0.562244897959184\\
1.46	-0.569756097560976\\
1.47	-0.577287449392712\\
1.48	-0.584838709677419\\
1.49	-0.592409638554217\\
1.5	-0.6\\
1.51	-0.607609561752988\\
1.52	-0.615238095238095\\
1.53	-0.622885375494071\\
1.54	-0.630551181102362\\
1.55	-0.638235294117647\\
1.56	-0.6459375\\
1.57	-0.653657587548638\\
1.58	-0.661395348837209\\
1.59	-0.669150579150579\\
1.6	-0.676923076923077\\
1.61	-0.684712643678161\\
1.62	-0.692519083969466\\
1.63	-0.700342205323194\\
1.64	-0.708181818181818\\
1.65	-0.716037735849057\\
1.66	-0.72390977443609\\
1.67	-0.731797752808989\\
1.68	-0.739701492537313\\
1.69	-0.747620817843866\\
1.7	-0.755555555555556\\
1.71	-0.76350553505535\\
1.72	-0.771470588235294\\
1.73	-0.779450549450549\\
1.74	-0.787445255474453\\
1.75	-0.795454545454546\\
1.76	-0.803478260869565\\
1.77	-0.811516245487365\\
1.78	-0.819568345323741\\
1.79	-0.827634408602151\\
1.8	-0.835714285714286\\
1.81	-0.843807829181495\\
1.82	-0.851914893617021\\
1.83	-0.860035335689046\\
1.84	-0.868169014084507\\
1.85	-0.876315789473684\\
1.86	-0.884475524475525\\
1.87	-0.892648083623693\\
1.88	-0.900833333333333\\
1.89	-0.909031141868512\\
1.9	-0.917241379310345\\
1.91	-0.925463917525773\\
1.92	-0.933698630136986\\
1.93	-0.941945392491467\\
1.94	-0.950204081632653\\
1.95	-0.958474576271186\\
1.96	-0.966756756756757\\
1.97	-0.975050505050505\\
1.98	-0.983355704697986\\
1.99	-0.991672240802676\\
2	-1\\
2.01	-1.00833887043189\\
2.02	-1.01668874172185\\
2.03	-1.0250495049505\\
2.04	-1.03342105263158\\
2.05	-1.04180327868852\\
2.06	-1.05019607843137\\
2.07	-1.0585993485342\\
2.08	-1.06701298701299\\
2.09	-1.07543689320388\\
2.1	-1.08387096774194\\
2.11	-1.09231511254019\\
2.12	-1.10076923076923\\
2.13	-1.10923322683706\\
2.14	-1.11770700636943\\
2.15	-1.12619047619048\\
2.16	-1.1346835443038\\
2.17	-1.14318611987382\\
2.18	-1.15169811320755\\
2.19	-1.16021943573668\\
2.2	-1.16875\\
2.21	-1.17728971962617\\
2.22	-1.18583850931677\\
2.23	-1.19439628482972\\
2.24	-1.20296296296296\\
2.25	-1.21153846153846\\
2.26	-1.2201226993865\\
2.27	-1.22871559633028\\
2.28	-1.23731707317073\\
2.29	-1.24592705167173\\
2.3	-1.25454545454545\\
2.31	-1.26317220543807\\
2.32	-1.27180722891566\\
2.33	-1.28045045045045\\
2.34	-1.28910179640719\\
2.35	-1.29776119402985\\
2.36	-1.30642857142857\\
2.37	-1.31510385756677\\
2.38	-1.32378698224852\\
2.39	-1.33247787610619\\
2.4	-1.34117647058824\\
2.41	-1.34988269794721\\
2.42	-1.35859649122807\\
2.43	-1.36731778425656\\
2.44	-1.37604651162791\\
2.45	-1.38478260869565\\
2.46	-1.39352601156069\\
2.47	-1.40227665706052\\
2.48	-1.41103448275862\\
2.49	-1.4197994269341\\
2.5	-1.42857142857143\\
2.51	-1.43735042735043\\
2.52	-1.44613636363636\\
2.53	-1.45492917847025\\
2.54	-1.46372881355932\\
2.55	-1.47253521126761\\
2.56	-1.48134831460674\\
2.57	-1.49016806722689\\
2.58	-1.49899441340782\\
2.59	-1.50782729805014\\
2.6	-1.51666666666667\\
2.61	-1.52551246537396\\
2.62	-1.53436464088398\\
2.63	-1.54322314049587\\
2.64	-1.55208791208791\\
2.65	-1.56095890410959\\
2.66	-1.56983606557377\\
2.67	-1.57871934604905\\
2.68	-1.58760869565217\\
2.69	-1.59650406504065\\
2.7	-1.60540540540541\\
2.71	-1.61431266846361\\
2.72	-1.62322580645161\\
2.73	-1.63214477211796\\
2.74	-1.64106951871658\\
2.75	-1.65\\
2.76	-1.65893617021277\\
2.77	-1.66787798408488\\
2.78	-1.6768253968254\\
2.79	-1.6857783641161\\
2.8	-1.69473684210526\\
2.81	-1.70370078740157\\
2.82	-1.71267015706806\\
2.83	-1.72164490861619\\
2.84	-1.730625\\
2.85	-1.73961038961039\\
2.86	-1.74860103626943\\
2.87	-1.75759689922481\\
2.88	-1.76659793814433\\
2.89	-1.77560411311054\\
2.9	-1.78461538461538\\
2.91	-1.79363171355499\\
2.92	-1.80265306122449\\
2.93	-1.81167938931298\\
2.94	-1.82071065989848\\
2.95	-1.82974683544304\\
2.96	-1.83878787878788\\
2.97	-1.84783375314861\\
2.98	-1.85688442211055\\
2.99	-1.86593984962406\\
3	-1.875\\
3.01	-1.88406483790524\\
3.02	-1.89313432835821\\
3.03	-1.90220843672457\\
3.04	-1.91128712871287\\
3.05	-1.92037037037037\\
3.06	-1.92945812807882\\
3.07	-1.93855036855037\\
3.08	-1.94764705882353\\
3.09	-1.95674816625917\\
3.1	-1.96585365853658\\
3.11	-1.97496350364963\\
3.12	-1.98407766990291\\
3.13	-1.99319612590799\\
3.14	-2.00231884057971\\
3.15	-2.01144578313253\\
3.16	-2.02057692307692\\
3.17	-2.02971223021583\\
3.18	-2.03885167464115\\
3.19	-2.04799522673031\\
3.2	-2.05714285714286\\
3.21	-2.0662945368171\\
3.22	-2.07545023696682\\
3.23	-2.08460992907801\\
3.24	-2.09377358490566\\
3.25	-2.10294117647059\\
3.26	-2.11211267605634\\
3.27	-2.12128805620609\\
3.28	-2.13046728971963\\
3.29	-2.13965034965035\\
3.3	-2.14883720930233\\
3.31	-2.15802784222738\\
3.32	-2.16722222222222\\
3.33	-2.17642032332563\\
3.34	-2.18562211981567\\
3.35	-2.1948275862069\\
3.36	-2.20403669724771\\
3.37	-2.21324942791762\\
3.38	-2.22246575342466\\
3.39	-2.23168564920273\\
3.4	-2.24090909090909\\
3.41	-2.25013605442177\\
3.42	-2.2593665158371\\
3.43	-2.26860045146727\\
3.44	-2.27783783783784\\
3.45	-2.28707865168539\\
3.46	-2.29632286995516\\
3.47	-2.30557046979866\\
3.48	-2.31482142857143\\
3.49	-2.32407572383074\\
3.5	-2.33333333333333\\
3.51	-2.34259423503326\\
3.52	-2.35185840707965\\
3.53	-2.36112582781457\\
3.54	-2.37039647577093\\
3.55	-2.37967032967033\\
3.56	-2.38894736842105\\
3.57	-2.39822757111597\\
3.58	-2.40751091703057\\
3.59	-2.41679738562091\\
3.6	-2.42608695652174\\
3.61	-2.43537960954447\\
3.62	-2.44467532467533\\
3.63	-2.45397408207343\\
3.64	-2.46327586206896\\
3.65	-2.47258064516129\\
3.66	-2.48188841201717\\
3.67	-2.49119914346895\\
3.68	-2.50051282051282\\
3.69	-2.50982942430704\\
3.7	-2.51914893617021\\
3.71	-2.52847133757962\\
3.72	-2.53779661016949\\
3.73	-2.54712473572939\\
3.74	-2.55645569620253\\
3.75	-2.56578947368421\\
3.76	-2.57512605042017\\
3.77	-2.58446540880503\\
3.78	-2.59380753138075\\
3.79	-2.60315240083507\\
3.8	-2.6125\\
3.81	-2.62185031185031\\
3.82	-2.63120331950207\\
3.83	-2.64055900621118\\
3.84	-2.6499173553719\\
3.85	-2.65927835051546\\
3.86	-2.66864197530864\\
3.87	-2.67800821355236\\
3.88	-2.68737704918033\\
3.89	-2.69674846625767\\
3.9	-2.70612244897959\\
3.91	-2.71549898167006\\
3.92	-2.72487804878049\\
3.93	-2.73425963488844\\
3.94	-2.74364372469636\\
3.95	-2.7530303030303\\
3.96	-2.76241935483871\\
3.97	-2.77181086519115\\
3.98	-2.78120481927711\\
3.99	-2.79060120240481\\
4	-2.8\\
4.01	-2.80940119760479\\
4.02	-2.81880478087649\\
4.03	-2.82821073558648\\
4.04	-2.83761904761905\\
4.05	-2.8470297029703\\
4.06	-2.85644268774704\\
4.07	-2.86585798816568\\
4.08	-2.87527559055118\\
4.09	-2.88469548133595\\
4.1	-2.89411764705882\\
4.11	-2.90354207436399\\
4.12	-2.91296875\\
4.13	-2.92239766081871\\
4.14	-2.93182879377432\\
4.15	-2.94126213592233\\
4.16	-2.9506976744186\\
4.17	-2.96013539651838\\
4.18	-2.96957528957529\\
4.19	-2.97901734104046\\
4.2	-2.98846153846154\\
4.21	-2.99790786948177\\
4.22	-3.00735632183908\\
4.23	-3.0168068833652\\
4.24	-3.02625954198473\\
4.25	-3.03571428571429\\
4.26	-3.0451711026616\\
4.27	-3.05462998102467\\
4.28	-3.06409090909091\\
4.29	-3.07355387523629\\
4.3	-3.08301886792453\\
4.31	-3.09248587570621\\
4.32	-3.10195488721805\\
4.33	-3.11142589118199\\
4.34	-3.12089887640449\\
4.35	-3.1303738317757\\
4.36	-3.13985074626866\\
4.37	-3.14932960893855\\
4.38	-3.15881040892193\\
4.39	-3.16829313543599\\
4.4	-3.17777777777778\\
4.41	-3.18726432532348\\
4.42	-3.19675276752768\\
4.43	-3.20624309392265\\
4.44	-3.21573529411765\\
4.45	-3.22522935779816\\
4.46	-3.23472527472527\\
4.47	-3.24422303473492\\
4.48	-3.25372262773723\\
4.49	-3.26322404371585\\
4.5	-3.27272727272727\\
4.51	-3.28223230490018\\
4.52	-3.29173913043478\\
4.53	-3.30124773960217\\
4.54	-3.31075812274368\\
4.55	-3.32027027027027\\
4.56	-3.32978417266187\\
4.57	-3.33929982046679\\
4.58	-3.34881720430108\\
4.59	-3.35833631484794\\
4.6	-3.36785714285714\\
4.61	-3.37737967914439\\
4.62	-3.38690391459075\\
4.63	-3.3964298401421\\
4.64	-3.40595744680851\\
4.65	-3.41548672566372\\
4.66	-3.42501766784452\\
4.67	-3.43455026455026\\
4.68	-3.44408450704225\\
4.69	-3.45362038664323\\
4.7	-3.46315789473684\\
4.71	-3.47269702276708\\
4.72	-3.48223776223776\\
4.73	-3.49178010471204\\
4.74	-3.50132404181185\\
4.75	-3.51086956521739\\
4.76	-3.52041666666667\\
4.77	-3.52996533795494\\
4.78	-3.53951557093426\\
4.79	-3.54906735751295\\
4.8	-3.55862068965517\\
4.81	-3.56817555938038\\
4.82	-3.57773195876289\\
4.83	-3.58728987993139\\
4.84	-3.59684931506849\\
4.85	-3.60641025641026\\
4.86	-3.61597269624573\\
4.87	-3.62553662691652\\
4.88	-3.63510204081633\\
4.89	-3.64466893039049\\
4.9	-3.65423728813559\\
4.91	-3.66380710659899\\
4.92	-3.67337837837838\\
4.93	-3.68295109612142\\
4.94	-3.69252525252525\\
4.95	-3.70210084033613\\
4.96	-3.71167785234899\\
4.97	-3.72125628140703\\
4.98	-3.73083612040134\\
4.99	-3.74041736227045\\
5	-3.75\\
};
\addplot [color=mycolor3,solid,forget plot]
  table[row sep=crcr]{%
0	0\\
0.01	0.00584158415841584\\
0.02	0.0113725490196078\\
0.03	0.0166019417475728\\
0.04	0.0215384615384615\\
0.05	0.0261904761904762\\
0.06	0.0305660377358491\\
0.07	0.0346728971962617\\
0.08	0.0385185185185185\\
0.09	0.0421100917431193\\
0.1	0.0454545454545455\\
0.11	0.0485585585585585\\
0.12	0.0514285714285714\\
0.13	0.054070796460177\\
0.14	0.0564912280701754\\
0.15	0.0586956521739131\\
0.16	0.0606896551724138\\
0.17	0.0624786324786325\\
0.18	0.0640677966101695\\
0.19	0.0654621848739496\\
0.2	0.0666666666666667\\
0.21	0.0676859504132231\\
0.22	0.0685245901639344\\
0.23	0.0691869918699188\\
0.24	0.0696774193548387\\
0.25	0.07\\
0.26	0.0701587301587302\\
0.27	0.0701574803149606\\
0.28	0.07\\
0.29	0.0696899224806202\\
0.3	0.0692307692307693\\
0.31	0.0686259541984733\\
0.32	0.0678787878787879\\
0.33	0.0669924812030075\\
0.34	0.0659701492537313\\
0.35	0.0648148148148148\\
0.36	0.063529411764706\\
0.37	0.0621167883211678\\
0.38	0.0605797101449276\\
0.39	0.0589208633093525\\
0.4	0.0571428571428572\\
0.41	0.0552482269503546\\
0.42	0.0532394366197183\\
0.43	0.0511188811188811\\
0.44	0.0488888888888889\\
0.45	0.046551724137931\\
0.46	0.044109589041096\\
0.47	0.0415646258503402\\
0.48	0.038918918918919\\
0.49	0.0361744966442953\\
0.5	0.0333333333333333\\
0.51	0.0303973509933775\\
0.52	0.0273684210526316\\
0.53	0.0242483660130719\\
0.54	0.0210389610389611\\
0.55	0.0177419354838709\\
0.56	0.0143589743589745\\
0.57	0.010891719745223\\
0.58	0.0073417721518988\\
0.59	0.00371069182389949\\
0.6	0\\
0.61	-0.00378881987577628\\
0.62	-0.00765432098765439\\
0.63	-0.0115950920245398\\
0.64	-0.015609756097561\\
0.65	-0.0196969696969697\\
0.66	-0.023855421686747\\
0.67	-0.0280838323353293\\
0.68	-0.0323809523809524\\
0.69	-0.0367455621301774\\
0.7	-0.0411764705882353\\
0.71	-0.045672514619883\\
0.72	-0.0502325581395349\\
0.73	-0.0548554913294796\\
0.74	-0.0595402298850574\\
0.75	-0.0642857142857143\\
0.76	-0.069090909090909\\
0.77	-0.073954802259887\\
0.78	-0.0788764044943819\\
0.79	-0.083854748603352\\
0.8	-0.0888888888888888\\
0.81	-0.0939779005524861\\
0.82	-0.0991208791208792\\
0.83	-0.10431693989071\\
0.84	-0.109565217391304\\
0.85	-0.114864864864865\\
0.86	-0.120215053763441\\
0.87	-0.125614973262032\\
0.88	-0.131063829787234\\
0.89	-0.136560846560847\\
0.9	-0.142105263157895\\
0.91	-0.147696335078534\\
0.92	-0.153333333333333\\
0.93	-0.159015544041451\\
0.94	-0.164742268041237\\
0.95	-0.170512820512821\\
0.96	-0.176326530612245\\
0.97	-0.182182741116751\\
0.98	-0.188080808080808\\
0.99	-0.194020100502513\\
1	-0.2\\
1.01	-0.206019900497512\\
1.02	-0.212079207920792\\
1.03	-0.218177339901478\\
1.04	-0.224313725490196\\
1.05	-0.230487804878049\\
1.06	-0.236699029126214\\
1.07	-0.242946859903382\\
1.08	-0.249230769230769\\
1.09	-0.25555023923445\\
1.1	-0.261904761904762\\
1.11	-0.268293838862559\\
1.12	-0.274716981132075\\
1.13	-0.281173708920188\\
1.14	-0.287663551401869\\
1.15	-0.294186046511628\\
1.16	-0.300740740740741\\
1.17	-0.307327188940092\\
1.18	-0.31394495412844\\
1.19	-0.320593607305936\\
1.2	-0.327272727272727\\
1.21	-0.333981900452489\\
1.22	-0.340720720720721\\
1.23	-0.347488789237668\\
1.24	-0.354285714285714\\
1.25	-0.361111111111111\\
1.26	-0.367964601769911\\
1.27	-0.374845814977974\\
1.28	-0.381754385964912\\
1.29	-0.388689956331878\\
1.3	-0.395652173913043\\
1.31	-0.402640692640693\\
1.32	-0.409655172413793\\
1.33	-0.416695278969957\\
1.34	-0.423760683760684\\
1.35	-0.430851063829787\\
1.36	-0.437966101694915\\
1.37	-0.445105485232068\\
1.38	-0.452268907563025\\
1.39	-0.459456066945607\\
1.4	-0.466666666666667\\
1.41	-0.473900414937759\\
1.42	-0.481157024793388\\
1.43	-0.488436213991769\\
1.44	-0.495737704918033\\
1.45	-0.503061224489796\\
1.46	-0.510406504065041\\
1.47	-0.517773279352227\\
1.48	-0.525161290322581\\
1.49	-0.532570281124498\\
1.5	-0.54\\
1.51	-0.547450199203187\\
1.52	-0.554920634920635\\
1.53	-0.562411067193676\\
1.54	-0.56992125984252\\
1.55	-0.577450980392157\\
1.56	-0.585\\
1.57	-0.592568093385214\\
1.58	-0.60015503875969\\
1.59	-0.607760617760618\\
1.6	-0.615384615384615\\
1.61	-0.623026819923372\\
1.62	-0.630687022900763\\
1.63	-0.638365019011407\\
1.64	-0.646060606060606\\
1.65	-0.653773584905661\\
1.66	-0.661503759398496\\
1.67	-0.669250936329588\\
1.68	-0.677014925373134\\
1.69	-0.684795539033457\\
1.7	-0.692592592592593\\
1.71	-0.700405904059041\\
1.72	-0.708235294117647\\
1.73	-0.716080586080586\\
1.74	-0.723941605839416\\
1.75	-0.731818181818182\\
1.76	-0.739710144927536\\
1.77	-0.747617328519856\\
1.78	-0.755539568345324\\
1.79	-0.763476702508961\\
1.8	-0.771428571428571\\
1.81	-0.779395017793594\\
1.82	-0.787375886524823\\
1.83	-0.795371024734982\\
1.84	-0.803380281690141\\
1.85	-0.81140350877193\\
1.86	-0.81944055944056\\
1.87	-0.827491289198606\\
1.88	-0.835555555555556\\
1.89	-0.84363321799308\\
1.9	-0.851724137931035\\
1.91	-0.859828178694158\\
1.92	-0.867945205479452\\
1.93	-0.876075085324232\\
1.94	-0.88421768707483\\
1.95	-0.892372881355932\\
1.96	-0.90054054054054\\
1.97	-0.908720538720539\\
1.98	-0.916912751677852\\
1.99	-0.925117056856187\\
2	-0.933333333333333\\
2.01	-0.94156146179402\\
2.02	-0.949801324503311\\
2.03	-0.958052805280528\\
2.04	-0.966315789473684\\
2.05	-0.974590163934426\\
2.06	-0.982875816993464\\
2.07	-0.991172638436482\\
2.08	-0.999480519480519\\
2.09	-1.00779935275081\\
2.1	-1.01612903225806\\
2.11	-1.02446945337621\\
2.12	-1.03282051282051\\
2.13	-1.0411821086262\\
2.14	-1.04955414012739\\
2.15	-1.05793650793651\\
2.16	-1.06632911392405\\
2.17	-1.07473186119874\\
2.18	-1.08314465408805\\
2.19	-1.09156739811912\\
2.2	-1.1\\
2.21	-1.10844236760125\\
2.22	-1.11689440993789\\
2.23	-1.1253560371517\\
2.24	-1.13382716049383\\
2.25	-1.14230769230769\\
2.26	-1.15079754601227\\
2.27	-1.15929663608563\\
2.28	-1.16780487804878\\
2.29	-1.17632218844985\\
2.3	-1.18484848484849\\
2.31	-1.1933836858006\\
2.32	-1.20192771084337\\
2.33	-1.21048048048048\\
2.34	-1.21904191616766\\
2.35	-1.22761194029851\\
2.36	-1.23619047619048\\
2.37	-1.24477744807122\\
2.38	-1.25337278106509\\
2.39	-1.26197640117994\\
2.4	-1.27058823529412\\
2.41	-1.2792082111437\\
2.42	-1.28783625730994\\
2.43	-1.296472303207\\
2.44	-1.30511627906977\\
2.45	-1.31376811594203\\
2.46	-1.32242774566474\\
2.47	-1.33109510086455\\
2.48	-1.33977011494253\\
2.49	-1.34845272206304\\
2.5	-1.35714285714286\\
2.51	-1.36584045584046\\
2.52	-1.37454545454545\\
2.53	-1.38325779036827\\
2.54	-1.39197740112994\\
2.55	-1.40070422535211\\
2.56	-1.40943820224719\\
2.57	-1.41817927170868\\
2.58	-1.42692737430168\\
2.59	-1.43568245125348\\
2.6	-1.44444444444444\\
2.61	-1.45321329639889\\
2.62	-1.46198895027624\\
2.63	-1.47077134986226\\
2.64	-1.47956043956044\\
2.65	-1.48835616438356\\
2.66	-1.49715846994536\\
2.67	-1.50596730245232\\
2.68	-1.51478260869565\\
2.69	-1.52360433604336\\
2.7	-1.53243243243243\\
2.71	-1.54126684636119\\
2.72	-1.55010752688172\\
2.73	-1.55895442359249\\
2.74	-1.56780748663102\\
2.75	-1.57666666666667\\
2.76	-1.58553191489362\\
2.77	-1.59440318302387\\
2.78	-1.60328042328042\\
2.79	-1.6121635883905\\
2.8	-1.62105263157895\\
2.81	-1.62994750656168\\
2.82	-1.63884816753927\\
2.83	-1.6477545691906\\
2.84	-1.65666666666667\\
2.85	-1.66558441558442\\
2.86	-1.67450777202073\\
2.87	-1.68343669250646\\
2.88	-1.69237113402062\\
2.89	-1.70131105398458\\
2.9	-1.71025641025641\\
2.91	-1.71920716112532\\
2.92	-1.72816326530612\\
2.93	-1.73712468193384\\
2.94	-1.74609137055838\\
2.95	-1.75506329113924\\
2.96	-1.7640404040404\\
2.97	-1.77302267002519\\
2.98	-1.78201005025126\\
2.99	-1.79100250626566\\
3	-1.8\\
3.01	-1.80900249376559\\
3.02	-1.81800995024876\\
3.03	-1.8270223325062\\
3.04	-1.8360396039604\\
3.05	-1.84506172839506\\
3.06	-1.85408866995074\\
3.07	-1.86312039312039\\
3.08	-1.8721568627451\\
3.09	-1.88119804400978\\
3.1	-1.89024390243902\\
3.11	-1.89929440389294\\
3.12	-1.90834951456311\\
3.13	-1.91740920096852\\
3.14	-1.92647342995169\\
3.15	-1.9355421686747\\
3.16	-1.94461538461538\\
3.17	-1.95369304556355\\
3.18	-1.96277511961722\\
3.19	-1.971861575179\\
3.2	-1.98095238095238\\
3.21	-1.99004750593824\\
3.22	-1.99914691943128\\
3.23	-2.00825059101655\\
3.24	-2.01735849056604\\
3.25	-2.02647058823529\\
3.26	-2.03558685446009\\
3.27	-2.04470725995316\\
3.28	-2.05383177570093\\
3.29	-2.06296037296037\\
3.3	-2.07209302325581\\
3.31	-2.08122969837587\\
3.32	-2.09037037037037\\
3.33	-2.09951501154734\\
3.34	-2.10866359447005\\
3.35	-2.11781609195402\\
3.36	-2.12697247706422\\
3.37	-2.13613272311213\\
3.38	-2.14529680365297\\
3.39	-2.15446469248292\\
3.4	-2.16363636363636\\
3.41	-2.17281179138322\\
3.42	-2.18199095022624\\
3.43	-2.19117381489842\\
3.44	-2.20036036036036\\
3.45	-2.20955056179775\\
3.46	-2.21874439461883\\
3.47	-2.2279418344519\\
3.48	-2.23714285714286\\
3.49	-2.24634743875278\\
3.5	-2.25555555555556\\
3.51	-2.26476718403548\\
3.52	-2.27398230088496\\
3.53	-2.28320088300221\\
3.54	-2.29242290748899\\
3.55	-2.30164835164835\\
3.56	-2.31087719298246\\
3.57	-2.32010940919037\\
3.58	-2.32934497816594\\
3.59	-2.33858387799564\\
3.6	-2.34782608695652\\
3.61	-2.3570715835141\\
3.62	-2.36632034632035\\
3.63	-2.37557235421166\\
3.64	-2.3848275862069\\
3.65	-2.39408602150538\\
3.66	-2.40334763948498\\
3.67	-2.41261241970021\\
3.68	-2.42188034188034\\
3.69	-2.4311513859275\\
3.7	-2.44042553191489\\
3.71	-2.44970276008493\\
3.72	-2.45898305084746\\
3.73	-2.46826638477801\\
3.74	-2.47755274261603\\
3.75	-2.48684210526316\\
3.76	-2.49613445378151\\
3.77	-2.50542976939203\\
3.78	-2.5147280334728\\
3.79	-2.52402922755741\\
3.8	-2.53333333333333\\
3.81	-2.54264033264033\\
3.82	-2.55195020746888\\
3.83	-2.56126293995859\\
3.84	-2.57057851239669\\
3.85	-2.57989690721649\\
3.86	-2.58921810699588\\
3.87	-2.59854209445585\\
3.88	-2.60786885245902\\
3.89	-2.61719836400818\\
3.9	-2.6265306122449\\
3.91	-2.63586558044807\\
3.92	-2.64520325203252\\
3.93	-2.65454361054767\\
3.94	-2.66388663967611\\
3.95	-2.67323232323232\\
3.96	-2.68258064516129\\
3.97	-2.69193158953722\\
3.98	-2.70128514056225\\
3.99	-2.71064128256513\\
4	-2.72\\
4.01	-2.72936127744511\\
4.02	-2.73872509960159\\
4.03	-2.74809145129225\\
4.04	-2.75746031746032\\
4.05	-2.76683168316832\\
4.06	-2.77620553359684\\
4.07	-2.78558185404339\\
4.08	-2.79496062992126\\
4.09	-2.80434184675835\\
4.1	-2.81372549019608\\
4.11	-2.82311154598826\\
4.12	-2.8325\\
4.13	-2.84189083820663\\
4.14	-2.85128404669261\\
4.15	-2.86067961165049\\
4.16	-2.87007751937985\\
4.17	-2.87947775628627\\
4.18	-2.88888030888031\\
4.19	-2.89828516377649\\
4.2	-2.90769230769231\\
4.21	-2.91710172744722\\
4.22	-2.92651340996169\\
4.23	-2.93592734225621\\
4.24	-2.94534351145038\\
4.25	-2.9547619047619\\
4.26	-2.9641825095057\\
4.27	-2.97360531309298\\
4.28	-2.9830303030303\\
4.29	-2.99245746691871\\
4.3	-3.00188679245283\\
4.31	-3.01131826741996\\
4.32	-3.02075187969925\\
4.33	-3.03018761726079\\
4.34	-3.03962546816479\\
4.35	-3.04906542056075\\
4.36	-3.05850746268657\\
4.37	-3.06795158286778\\
4.38	-3.07739776951673\\
4.39	-3.08684601113173\\
4.4	-3.0962962962963\\
4.41	-3.10574861367837\\
4.42	-3.11520295202952\\
4.43	-3.12465930018416\\
4.44	-3.13411764705882\\
4.45	-3.14357798165138\\
4.46	-3.15304029304029\\
4.47	-3.16250457038391\\
4.48	-3.17197080291971\\
4.49	-3.18143897996357\\
4.5	-3.19090909090909\\
4.51	-3.20038112522686\\
4.52	-3.20985507246377\\
4.53	-3.21933092224231\\
4.54	-3.22880866425993\\
4.55	-3.23828828828829\\
4.56	-3.24776978417266\\
4.57	-3.25725314183124\\
4.58	-3.26673835125448\\
4.59	-3.27622540250447\\
4.6	-3.28571428571429\\
4.61	-3.29520499108734\\
4.62	-3.3046975088968\\
4.63	-3.3141918294849\\
4.64	-3.32368794326241\\
4.65	-3.33318584070796\\
4.66	-3.34268551236749\\
4.67	-3.35218694885362\\
4.68	-3.36169014084507\\
4.69	-3.37119507908612\\
4.7	-3.38070175438597\\
4.71	-3.39021015761821\\
4.72	-3.39972027972028\\
4.73	-3.40923211169285\\
4.74	-3.4187456445993\\
4.75	-3.42826086956522\\
4.76	-3.43777777777778\\
4.77	-3.44729636048527\\
4.78	-3.45681660899654\\
4.79	-3.46633851468048\\
4.8	-3.47586206896552\\
4.81	-3.48538726333907\\
4.82	-3.49491408934708\\
4.83	-3.50444253859348\\
4.84	-3.51397260273973\\
4.85	-3.52350427350427\\
4.86	-3.53303754266212\\
4.87	-3.54257240204429\\
4.88	-3.55210884353741\\
4.89	-3.56164685908319\\
4.9	-3.57118644067797\\
4.91	-3.58072758037225\\
4.92	-3.59027027027027\\
4.93	-3.59981450252951\\
4.94	-3.60936026936027\\
4.95	-3.61890756302521\\
4.96	-3.62845637583893\\
4.97	-3.6380067001675\\
4.98	-3.64755852842809\\
4.99	-3.65711185308848\\
5	-3.66666666666667\\
};
\addplot [color=mycolor4,solid,forget plot]
  table[row sep=crcr]{%
0	0\\
0.01	0.00683168316831683\\
0.02	0.0133333333333333\\
0.03	0.0195145631067961\\
0.04	0.0253846153846154\\
0.05	0.0309523809523809\\
0.06	0.0362264150943396\\
0.07	0.041214953271028\\
0.08	0.0459259259259259\\
0.09	0.0503669724770642\\
0.1	0.0545454545454545\\
0.11	0.0584684684684685\\
0.12	0.0621428571428571\\
0.13	0.0655752212389381\\
0.14	0.0687719298245614\\
0.15	0.0717391304347826\\
0.16	0.0744827586206897\\
0.17	0.077008547008547\\
0.18	0.0793220338983051\\
0.19	0.0814285714285715\\
0.2	0.0833333333333334\\
0.21	0.0850413223140496\\
0.22	0.0865573770491803\\
0.23	0.0878861788617886\\
0.24	0.0890322580645161\\
0.25	0.09\\
0.26	0.0907936507936508\\
0.27	0.0914173228346457\\
0.28	0.091875\\
0.29	0.0921705426356589\\
0.3	0.0923076923076923\\
0.31	0.0922900763358778\\
0.32	0.0921212121212121\\
0.33	0.0918045112781954\\
0.34	0.0913432835820895\\
0.35	0.0907407407407407\\
0.36	0.09\\
0.37	0.0891240875912408\\
0.38	0.0881159420289855\\
0.39	0.086978417266187\\
0.4	0.0857142857142857\\
0.41	0.0843262411347517\\
0.42	0.0828169014084507\\
0.43	0.0811888111888111\\
0.44	0.0794444444444445\\
0.45	0.0775862068965517\\
0.46	0.0756164383561643\\
0.47	0.0735374149659864\\
0.48	0.0713513513513514\\
0.49	0.0690604026845637\\
0.5	0.0666666666666667\\
0.51	0.0641721854304635\\
0.52	0.0615789473684211\\
0.53	0.0588888888888889\\
0.54	0.0561038961038961\\
0.55	0.0532258064516129\\
0.56	0.0502564102564103\\
0.57	0.0471974522292994\\
0.58	0.0440506329113923\\
0.59	0.0408176100628931\\
0.6	0.0375\\
0.61	0.0340993788819876\\
0.62	0.0306172839506172\\
0.63	0.0270552147239264\\
0.64	0.0234146341463414\\
0.65	0.0196969696969697\\
0.66	0.0159036144578313\\
0.67	0.0120359281437126\\
0.68	0.00809523809523804\\
0.69	0.00408284023668637\\
0.7	-1.11022302462516e-16\\
0.71	-0.0041520467836258\\
0.72	-0.00837209302325581\\
0.73	-0.0126589595375722\\
0.74	-0.0170114942528735\\
0.75	-0.0214285714285715\\
0.76	-0.025909090909091\\
0.77	-0.03045197740113\\
0.78	-0.0350561797752809\\
0.79	-0.0397206703910614\\
0.8	-0.0444444444444445\\
0.81	-0.0492265193370166\\
0.82	-0.0540659340659342\\
0.83	-0.0589617486338798\\
0.84	-0.0639130434782609\\
0.85	-0.068918918918919\\
0.86	-0.0739784946236558\\
0.87	-0.0790909090909091\\
0.88	-0.0842553191489361\\
0.89	-0.0894708994708995\\
0.9	-0.0947368421052631\\
0.91	-0.100052356020942\\
0.92	-0.105416666666667\\
0.93	-0.110829015544042\\
0.94	-0.116288659793814\\
0.95	-0.121794871794872\\
0.96	-0.12734693877551\\
0.97	-0.132944162436548\\
0.98	-0.138585858585859\\
0.99	-0.14427135678392\\
1	-0.15\\
1.01	-0.155771144278607\\
1.02	-0.161584158415842\\
1.03	-0.16743842364532\\
1.04	-0.173333333333333\\
1.05	-0.179268292682927\\
1.06	-0.185242718446602\\
1.07	-0.191256038647343\\
1.08	-0.197307692307692\\
1.09	-0.203397129186603\\
1.1	-0.20952380952381\\
1.11	-0.215687203791469\\
1.12	-0.22188679245283\\
1.13	-0.2281220657277\\
1.14	-0.234392523364486\\
1.15	-0.240697674418605\\
1.16	-0.247037037037037\\
1.17	-0.253410138248848\\
1.18	-0.259816513761468\\
1.19	-0.266255707762557\\
1.2	-0.272727272727273\\
1.21	-0.279230769230769\\
1.22	-0.285765765765766\\
1.23	-0.292331838565022\\
1.24	-0.298928571428572\\
1.25	-0.305555555555556\\
1.26	-0.312212389380531\\
1.27	-0.318898678414097\\
1.28	-0.325614035087719\\
1.29	-0.33235807860262\\
1.3	-0.339130434782609\\
1.31	-0.345930735930736\\
1.32	-0.352758620689655\\
1.33	-0.359613733905579\\
1.34	-0.366495726495727\\
1.35	-0.373404255319149\\
1.36	-0.380338983050848\\
1.37	-0.387299578059072\\
1.38	-0.394285714285714\\
1.39	-0.401297071129707\\
1.4	-0.408333333333333\\
1.41	-0.415394190871369\\
1.42	-0.422479338842975\\
1.43	-0.429588477366255\\
1.44	-0.43672131147541\\
1.45	-0.443877551020408\\
1.46	-0.451056910569106\\
1.47	-0.458259109311741\\
1.48	-0.465483870967742\\
1.49	-0.472730923694779\\
1.5	-0.48\\
1.51	-0.487290836653386\\
1.52	-0.494603174603175\\
1.53	-0.501936758893281\\
1.54	-0.509291338582677\\
1.55	-0.516666666666667\\
1.56	-0.5240625\\
1.57	-0.53147859922179\\
1.58	-0.538914728682171\\
1.59	-0.546370656370656\\
1.6	-0.553846153846154\\
1.61	-0.561340996168582\\
1.62	-0.568854961832061\\
1.63	-0.57638783269962\\
1.64	-0.583939393939394\\
1.65	-0.591509433962264\\
1.66	-0.599097744360902\\
1.67	-0.606704119850187\\
1.68	-0.614328358208955\\
1.69	-0.621970260223048\\
1.7	-0.62962962962963\\
1.71	-0.637306273062731\\
1.72	-0.645\\
1.73	-0.652710622710623\\
1.74	-0.66043795620438\\
1.75	-0.668181818181818\\
1.76	-0.675942028985507\\
1.77	-0.683718411552347\\
1.78	-0.691510791366907\\
1.79	-0.699318996415771\\
1.8	-0.707142857142857\\
1.81	-0.714982206405694\\
1.82	-0.722836879432624\\
1.83	-0.730706713780919\\
1.84	-0.738591549295775\\
1.85	-0.746491228070175\\
1.86	-0.754405594405595\\
1.87	-0.762334494773519\\
1.88	-0.770277777777778\\
1.89	-0.778235294117647\\
1.9	-0.786206896551724\\
1.91	-0.794192439862543\\
1.92	-0.802191780821918\\
1.93	-0.810204778156996\\
1.94	-0.818231292517007\\
1.95	-0.826271186440678\\
1.96	-0.834324324324324\\
1.97	-0.842390572390572\\
1.98	-0.850469798657718\\
1.99	-0.858561872909699\\
2	-0.866666666666667\\
2.01	-0.874784053156146\\
2.02	-0.882913907284768\\
2.03	-0.891056105610561\\
2.04	-0.899210526315789\\
2.05	-0.907377049180328\\
2.06	-0.915555555555555\\
2.07	-0.923745928338762\\
2.08	-0.931948051948052\\
2.09	-0.940161812297734\\
2.1	-0.948387096774194\\
2.11	-0.956623794212219\\
2.12	-0.964871794871795\\
2.13	-0.973130990415336\\
2.14	-0.98140127388535\\
2.15	-0.98968253968254\\
2.16	-0.997974683544304\\
2.17	-1.00627760252366\\
2.18	-1.01459119496855\\
2.19	-1.02291536050157\\
2.2	-1.03125\\
2.21	-1.03959501557632\\
2.22	-1.04795031055901\\
2.23	-1.05631578947368\\
2.24	-1.06469135802469\\
2.25	-1.07307692307692\\
2.26	-1.08147239263804\\
2.27	-1.08987767584098\\
2.28	-1.09829268292683\\
2.29	-1.10671732522796\\
2.3	-1.11515151515152\\
2.31	-1.12359516616314\\
2.32	-1.13204819277108\\
2.33	-1.14051051051051\\
2.34	-1.14898203592814\\
2.35	-1.15746268656716\\
2.36	-1.16595238095238\\
2.37	-1.17445103857567\\
2.38	-1.18295857988166\\
2.39	-1.19147492625369\\
2.4	-1.2\\
2.41	-1.20853372434018\\
2.42	-1.21707602339181\\
2.43	-1.22562682215743\\
2.44	-1.23418604651163\\
2.45	-1.24275362318841\\
2.46	-1.25132947976879\\
2.47	-1.25991354466859\\
2.48	-1.26850574712644\\
2.49	-1.27710601719198\\
2.5	-1.28571428571429\\
2.51	-1.29433048433048\\
2.52	-1.30295454545455\\
2.53	-1.31158640226629\\
2.54	-1.32022598870057\\
2.55	-1.32887323943662\\
2.56	-1.33752808988764\\
2.57	-1.34619047619048\\
2.58	-1.35486033519553\\
2.59	-1.36353760445682\\
2.6	-1.37222222222222\\
2.61	-1.38091412742382\\
2.62	-1.38961325966851\\
2.63	-1.39831955922865\\
2.64	-1.40703296703297\\
2.65	-1.41575342465753\\
2.66	-1.42448087431694\\
2.67	-1.43321525885559\\
2.68	-1.44195652173913\\
2.69	-1.45070460704607\\
2.7	-1.45945945945946\\
2.71	-1.46822102425876\\
2.72	-1.47698924731183\\
2.73	-1.48576407506702\\
2.74	-1.49454545454545\\
2.75	-1.50333333333333\\
2.76	-1.51212765957447\\
2.77	-1.52092838196286\\
2.78	-1.52973544973545\\
2.79	-1.53854881266491\\
2.8	-1.54736842105263\\
2.81	-1.55619422572178\\
2.82	-1.56502617801047\\
2.83	-1.57386422976501\\
2.84	-1.58270833333333\\
2.85	-1.59155844155844\\
2.86	-1.60041450777202\\
2.87	-1.60927648578811\\
2.88	-1.61814432989691\\
2.89	-1.62701799485861\\
2.9	-1.63589743589744\\
2.91	-1.64478260869565\\
2.92	-1.65367346938776\\
2.93	-1.66256997455471\\
2.94	-1.67147208121827\\
2.95	-1.68037974683544\\
2.96	-1.68929292929293\\
2.97	-1.69821158690176\\
2.98	-1.70713567839196\\
2.99	-1.71606516290727\\
3	-1.725\\
3.01	-1.73394014962594\\
3.02	-1.7428855721393\\
3.03	-1.75183622828784\\
3.04	-1.76079207920792\\
3.05	-1.76975308641975\\
3.06	-1.77871921182266\\
3.07	-1.78769041769042\\
3.08	-1.79666666666667\\
3.09	-1.80564792176039\\
3.1	-1.81463414634146\\
3.11	-1.82362530413625\\
3.12	-1.8326213592233\\
3.13	-1.84162227602906\\
3.14	-1.85062801932367\\
3.15	-1.85963855421687\\
3.16	-1.86865384615385\\
3.17	-1.87767386091127\\
3.18	-1.8866985645933\\
3.19	-1.89572792362768\\
3.2	-1.90476190476191\\
3.21	-1.91380047505938\\
3.22	-1.92284360189573\\
3.23	-1.93189125295508\\
3.24	-1.94094339622642\\
3.25	-1.95\\
3.26	-1.95906103286385\\
3.27	-1.96812646370023\\
3.28	-1.97719626168224\\
3.29	-1.9862703962704\\
3.3	-1.9953488372093\\
3.31	-2.00443155452436\\
3.32	-2.01351851851852\\
3.33	-2.02260969976905\\
3.34	-2.03170506912442\\
3.35	-2.04080459770115\\
3.36	-2.04990825688073\\
3.37	-2.05901601830664\\
3.38	-2.06812785388128\\
3.39	-2.0772437357631\\
3.4	-2.08636363636364\\
3.41	-2.09548752834467\\
3.42	-2.10461538461538\\
3.43	-2.11374717832957\\
3.44	-2.12288288288288\\
3.45	-2.13202247191011\\
3.46	-2.14116591928251\\
3.47	-2.15031319910515\\
3.48	-2.15946428571429\\
3.49	-2.16861915367483\\
3.5	-2.17777777777778\\
3.51	-2.18694013303769\\
3.52	-2.19610619469027\\
3.53	-2.20527593818985\\
3.54	-2.21444933920705\\
3.55	-2.22362637362637\\
3.56	-2.23280701754386\\
3.57	-2.24199124726477\\
3.58	-2.25117903930131\\
3.59	-2.26037037037037\\
3.6	-2.2695652173913\\
3.61	-2.27876355748373\\
3.62	-2.28796536796537\\
3.63	-2.29717062634989\\
3.64	-2.30637931034483\\
3.65	-2.31559139784946\\
3.66	-2.32480686695279\\
3.67	-2.33402569593148\\
3.68	-2.34324786324786\\
3.69	-2.35247334754797\\
3.7	-2.36170212765957\\
3.71	-2.37093418259023\\
3.72	-2.38016949152542\\
3.73	-2.38940803382664\\
3.74	-2.39864978902954\\
3.75	-2.40789473684211\\
3.76	-2.41714285714286\\
3.77	-2.42639412997904\\
3.78	-2.43564853556485\\
3.79	-2.44490605427975\\
3.8	-2.45416666666667\\
3.81	-2.46343035343035\\
3.82	-2.47269709543569\\
3.83	-2.481966873706\\
3.84	-2.49123966942149\\
3.85	-2.50051546391753\\
3.86	-2.50979423868313\\
3.87	-2.51907597535934\\
3.88	-2.52836065573771\\
3.89	-2.53764826175869\\
3.9	-2.5469387755102\\
3.91	-2.55623217922607\\
3.92	-2.56552845528455\\
3.93	-2.5748275862069\\
3.94	-2.58412955465587\\
3.95	-2.59343434343434\\
3.96	-2.60274193548387\\
3.97	-2.6120523138833\\
3.98	-2.62136546184739\\
3.99	-2.63068136272545\\
4	-2.64\\
4.01	-2.64932135728543\\
4.02	-2.65864541832669\\
4.03	-2.66797216699801\\
4.04	-2.67730158730159\\
4.05	-2.68663366336634\\
4.06	-2.69596837944664\\
4.07	-2.70530571992111\\
4.08	-2.71464566929134\\
4.09	-2.72398821218075\\
4.1	-2.73333333333333\\
4.11	-2.74268101761252\\
4.12	-2.75203125\\
4.13	-2.76138401559454\\
4.14	-2.7707392996109\\
4.15	-2.78009708737864\\
4.16	-2.78945736434109\\
4.17	-2.79882011605416\\
4.18	-2.80818532818533\\
4.19	-2.81755298651252\\
4.2	-2.82692307692308\\
4.21	-2.83629558541267\\
4.22	-2.84567049808429\\
4.23	-2.85504780114723\\
4.24	-2.86442748091603\\
4.25	-2.87380952380952\\
4.26	-2.88319391634981\\
4.27	-2.89258064516129\\
4.28	-2.9019696969697\\
4.29	-2.91136105860113\\
4.3	-2.92075471698113\\
4.31	-2.93015065913371\\
4.32	-2.93954887218045\\
4.33	-2.94894934333959\\
4.34	-2.95835205992509\\
4.35	-2.96775700934579\\
4.36	-2.97716417910448\\
4.37	-2.98657355679702\\
4.38	-2.99598513011152\\
4.39	-3.00539888682746\\
4.4	-3.01481481481482\\
4.41	-3.02423290203327\\
4.42	-3.03365313653137\\
4.43	-3.04307550644567\\
4.44	-3.0525\\
4.45	-3.06192660550459\\
4.46	-3.07135531135531\\
4.47	-3.08078610603291\\
4.48	-3.09021897810219\\
4.49	-3.09965391621129\\
4.5	-3.10909090909091\\
4.51	-3.11852994555354\\
4.52	-3.12797101449275\\
4.53	-3.13741410488246\\
4.54	-3.14685920577617\\
4.55	-3.15630630630631\\
4.56	-3.16575539568345\\
4.57	-3.17520646319569\\
4.58	-3.18465949820789\\
4.59	-3.194114490161\\
4.6	-3.20357142857143\\
4.61	-3.2130303030303\\
4.62	-3.22249110320285\\
4.63	-3.23195381882771\\
4.64	-3.24141843971631\\
4.65	-3.25088495575221\\
4.66	-3.26035335689046\\
4.67	-3.26982363315697\\
4.68	-3.27929577464789\\
4.69	-3.288769771529\\
4.7	-3.29824561403509\\
4.71	-3.30772329246935\\
4.72	-3.3172027972028\\
4.73	-3.32668411867365\\
4.74	-3.33616724738676\\
4.75	-3.34565217391304\\
4.76	-3.35513888888889\\
4.77	-3.3646273830156\\
4.78	-3.37411764705882\\
4.79	-3.38360967184801\\
4.8	-3.39310344827586\\
4.81	-3.40259896729776\\
4.82	-3.41209621993127\\
4.83	-3.42159519725557\\
4.84	-3.43109589041096\\
4.85	-3.44059829059829\\
4.86	-3.4501023890785\\
4.87	-3.45960817717206\\
4.88	-3.4691156462585\\
4.89	-3.47862478777589\\
4.9	-3.48813559322034\\
4.91	-3.49764805414552\\
4.92	-3.50716216216216\\
4.93	-3.5166779089376\\
4.94	-3.52619528619529\\
4.95	-3.53571428571429\\
4.96	-3.54523489932886\\
4.97	-3.55475711892797\\
4.98	-3.56428093645485\\
4.99	-3.57380634390651\\
5	-3.58333333333333\\
};
\addplot [color=mycolor5,solid,forget plot]
  table[row sep=crcr]{%
0	0\\
0.01	0.00782178217821782\\
0.02	0.0152941176470588\\
0.03	0.0224271844660194\\
0.04	0.0292307692307692\\
0.05	0.0357142857142857\\
0.06	0.0418867924528302\\
0.07	0.0477570093457944\\
0.08	0.0533333333333333\\
0.09	0.0586238532110092\\
0.1	0.0636363636363636\\
0.11	0.0683783783783784\\
0.12	0.0728571428571428\\
0.13	0.0770796460176991\\
0.14	0.0810526315789473\\
0.15	0.0847826086956522\\
0.16	0.0882758620689656\\
0.17	0.0915384615384615\\
0.18	0.0945762711864407\\
0.19	0.0973949579831933\\
0.2	0.1\\
0.21	0.102396694214876\\
0.22	0.104590163934426\\
0.23	0.106585365853659\\
0.24	0.108387096774194\\
0.25	0.11\\
0.26	0.111428571428571\\
0.27	0.112677165354331\\
0.28	0.11375\\
0.29	0.114651162790698\\
0.3	0.115384615384615\\
0.31	0.115954198473282\\
0.32	0.116363636363636\\
0.33	0.116616541353383\\
0.34	0.116716417910448\\
0.35	0.116666666666667\\
0.36	0.116470588235294\\
0.37	0.116131386861314\\
0.38	0.115652173913044\\
0.39	0.115035971223022\\
0.4	0.114285714285714\\
0.41	0.113404255319149\\
0.42	0.112394366197183\\
0.43	0.111258741258741\\
0.44	0.11\\
0.45	0.108620689655172\\
0.46	0.107123287671233\\
0.47	0.105510204081633\\
0.48	0.103783783783784\\
0.49	0.101946308724832\\
0.5	0.1\\
0.51	0.0979470198675497\\
0.52	0.0957894736842105\\
0.53	0.0935294117647059\\
0.54	0.0911688311688311\\
0.55	0.0887096774193549\\
0.56	0.0861538461538461\\
0.57	0.0835031847133758\\
0.58	0.0807594936708861\\
0.59	0.0779245283018869\\
0.6	0.075\\
0.61	0.0719875776397516\\
0.62	0.0688888888888889\\
0.63	0.0657055214723927\\
0.64	0.0624390243902438\\
0.65	0.0590909090909092\\
0.66	0.0556626506024096\\
0.67	0.0521556886227545\\
0.68	0.0485714285714286\\
0.69	0.0449112426035504\\
0.7	0.0411764705882353\\
0.71	0.0373684210526315\\
0.72	0.0334883720930234\\
0.73	0.0295375722543354\\
0.74	0.0255172413793104\\
0.75	0.0214285714285716\\
0.76	0.0172727272727272\\
0.77	0.0130508474576271\\
0.78	0.00876404494382022\\
0.79	0.00441340782122912\\
0.8	0\\
0.81	-0.00447513812154698\\
0.82	-0.00901098901098896\\
0.83	-0.0136065573770492\\
0.84	-0.0182608695652173\\
0.85	-0.0229729729729731\\
0.86	-0.0277419354838708\\
0.87	-0.0325668449197861\\
0.88	-0.0374468085106383\\
0.89	-0.0423809523809524\\
0.9	-0.0473684210526315\\
0.91	-0.0524083769633508\\
0.92	-0.0574999999999999\\
0.93	-0.0626424870466321\\
0.94	-0.0678350515463917\\
0.95	-0.0730769230769232\\
0.96	-0.0783673469387755\\
0.97	-0.0837055837563451\\
0.98	-0.089090909090909\\
0.99	-0.0945226130653266\\
1	-0.1\\
1.01	-0.105522388059701\\
1.02	-0.111089108910891\\
1.03	-0.116699507389163\\
1.04	-0.122352941176471\\
1.05	-0.128048780487805\\
1.06	-0.13378640776699\\
1.07	-0.139565217391304\\
1.08	-0.145384615384615\\
1.09	-0.151244019138756\\
1.1	-0.157142857142857\\
1.11	-0.163080568720379\\
1.12	-0.169056603773585\\
1.13	-0.175070422535211\\
1.14	-0.181121495327103\\
1.15	-0.187209302325582\\
1.16	-0.193333333333333\\
1.17	-0.199493087557604\\
1.18	-0.205688073394495\\
1.19	-0.211917808219178\\
1.2	-0.218181818181818\\
1.21	-0.22447963800905\\
1.22	-0.230810810810811\\
1.23	-0.237174887892377\\
1.24	-0.243571428571429\\
1.25	-0.25\\
1.26	-0.25646017699115\\
1.27	-0.26295154185022\\
1.28	-0.269473684210527\\
1.29	-0.276026200873362\\
1.3	-0.282608695652174\\
1.31	-0.289220779220779\\
1.32	-0.295862068965517\\
1.33	-0.302532188841202\\
1.34	-0.309230769230769\\
1.35	-0.315957446808511\\
1.36	-0.32271186440678\\
1.37	-0.329493670886076\\
1.38	-0.336302521008403\\
1.39	-0.343138075313808\\
1.4	-0.35\\
1.41	-0.356887966804979\\
1.42	-0.363801652892562\\
1.43	-0.370740740740741\\
1.44	-0.377704918032787\\
1.45	-0.38469387755102\\
1.46	-0.391707317073171\\
1.47	-0.398744939271255\\
1.48	-0.405806451612903\\
1.49	-0.41289156626506\\
1.5	-0.42\\
1.51	-0.427131474103586\\
1.52	-0.434285714285714\\
1.53	-0.441462450592885\\
1.54	-0.448661417322835\\
1.55	-0.455882352941176\\
1.56	-0.463125\\
1.57	-0.470389105058366\\
1.58	-0.477674418604651\\
1.59	-0.484980694980695\\
1.6	-0.492307692307692\\
1.61	-0.499655172413793\\
1.62	-0.507022900763359\\
1.63	-0.514410646387833\\
1.64	-0.521818181818182\\
1.65	-0.529245283018868\\
1.66	-0.536691729323308\\
1.67	-0.544157303370786\\
1.68	-0.551641791044776\\
1.69	-0.559144981412639\\
1.7	-0.566666666666667\\
1.71	-0.574206642066421\\
1.72	-0.581764705882353\\
1.73	-0.589340659340659\\
1.74	-0.596934306569343\\
1.75	-0.604545454545454\\
1.76	-0.612173913043478\\
1.77	-0.619819494584837\\
1.78	-0.627482014388489\\
1.79	-0.635161290322581\\
1.8	-0.642857142857143\\
1.81	-0.650569395017794\\
1.82	-0.658297872340426\\
1.83	-0.666042402826855\\
1.84	-0.673802816901408\\
1.85	-0.681578947368421\\
1.86	-0.68937062937063\\
1.87	-0.697177700348432\\
1.88	-0.705\\
1.89	-0.712837370242215\\
1.9	-0.720689655172414\\
1.91	-0.728556701030928\\
1.92	-0.736438356164383\\
1.93	-0.744334470989761\\
1.94	-0.752244897959184\\
1.95	-0.760169491525424\\
1.96	-0.768108108108108\\
1.97	-0.776060606060606\\
1.98	-0.784026845637584\\
1.99	-0.792006688963211\\
2	-0.8\\
2.01	-0.808006644518273\\
2.02	-0.816026490066225\\
2.03	-0.824059405940594\\
2.04	-0.832105263157895\\
2.05	-0.840163934426229\\
2.06	-0.848235294117647\\
2.07	-0.856319218241042\\
2.08	-0.864415584415585\\
2.09	-0.87252427184466\\
2.1	-0.880645161290323\\
2.11	-0.888778135048232\\
2.12	-0.896923076923077\\
2.13	-0.905079872204473\\
2.14	-0.913248407643312\\
2.15	-0.921428571428571\\
2.16	-0.929620253164557\\
2.17	-0.93782334384858\\
2.18	-0.946037735849057\\
2.19	-0.954263322884012\\
2.2	-0.9625\\
2.21	-0.970747663551402\\
2.22	-0.979006211180125\\
2.23	-0.987275541795666\\
2.24	-0.995555555555556\\
2.25	-1.00384615384615\\
2.26	-1.0121472392638\\
2.27	-1.02045871559633\\
2.28	-1.02878048780488\\
2.29	-1.03711246200608\\
2.3	-1.04545454545455\\
2.31	-1.05380664652568\\
2.32	-1.06216867469879\\
2.33	-1.07054054054054\\
2.34	-1.07892215568862\\
2.35	-1.08731343283582\\
2.36	-1.09571428571429\\
2.37	-1.10412462908012\\
2.38	-1.11254437869822\\
2.39	-1.12097345132743\\
2.4	-1.12941176470588\\
2.41	-1.13785923753666\\
2.42	-1.14631578947368\\
2.43	-1.15478134110787\\
2.44	-1.16325581395349\\
2.45	-1.17173913043478\\
2.46	-1.18023121387283\\
2.47	-1.18873198847262\\
2.48	-1.19724137931034\\
2.49	-1.20575931232092\\
2.5	-1.21428571428571\\
2.51	-1.22282051282051\\
2.52	-1.23136363636364\\
2.53	-1.23991501416431\\
2.54	-1.24847457627119\\
2.55	-1.25704225352113\\
2.56	-1.26561797752809\\
2.57	-1.27420168067227\\
2.58	-1.28279329608939\\
2.59	-1.29139275766017\\
2.6	-1.3\\
2.61	-1.30861495844875\\
2.62	-1.31723756906077\\
2.63	-1.32586776859504\\
2.64	-1.33450549450549\\
2.65	-1.34315068493151\\
2.66	-1.35180327868852\\
2.67	-1.36046321525886\\
2.68	-1.36913043478261\\
2.69	-1.37780487804878\\
2.7	-1.38648648648649\\
2.71	-1.39517520215633\\
2.72	-1.40387096774194\\
2.73	-1.41257372654155\\
2.74	-1.42128342245989\\
2.75	-1.43\\
2.76	-1.43872340425532\\
2.77	-1.44745358090186\\
2.78	-1.45619047619048\\
2.79	-1.46493403693931\\
2.8	-1.47368421052632\\
2.81	-1.48244094488189\\
2.82	-1.49120418848168\\
2.83	-1.49997389033943\\
2.84	-1.50875\\
2.85	-1.51753246753247\\
2.86	-1.52632124352332\\
2.87	-1.53511627906977\\
2.88	-1.5439175257732\\
2.89	-1.55272493573265\\
2.9	-1.56153846153846\\
2.91	-1.57035805626598\\
2.92	-1.57918367346939\\
2.93	-1.58801526717557\\
2.94	-1.59685279187817\\
2.95	-1.60569620253165\\
2.96	-1.61454545454545\\
2.97	-1.62340050377834\\
2.98	-1.63226130653266\\
2.99	-1.64112781954887\\
3	-1.65\\
3.01	-1.65887780548628\\
3.02	-1.66776119402985\\
3.03	-1.67665012406948\\
3.04	-1.68554455445545\\
3.05	-1.69444444444444\\
3.06	-1.70334975369458\\
3.07	-1.71226044226044\\
3.08	-1.72117647058824\\
3.09	-1.730097799511\\
3.1	-1.7390243902439\\
3.11	-1.74795620437956\\
3.12	-1.7568932038835\\
3.13	-1.76583535108959\\
3.14	-1.77478260869565\\
3.15	-1.78373493975904\\
3.16	-1.79269230769231\\
3.17	-1.80165467625899\\
3.18	-1.81062200956938\\
3.19	-1.81959427207637\\
3.2	-1.82857142857143\\
3.21	-1.83755344418052\\
3.22	-1.84654028436019\\
3.23	-1.85553191489362\\
3.24	-1.86452830188679\\
3.25	-1.87352941176471\\
3.26	-1.88253521126761\\
3.27	-1.89154566744731\\
3.28	-1.90056074766355\\
3.29	-1.90958041958042\\
3.3	-1.91860465116279\\
3.31	-1.92763341067285\\
3.32	-1.93666666666667\\
3.33	-1.94570438799076\\
3.34	-1.9547465437788\\
3.35	-1.96379310344828\\
3.36	-1.97284403669725\\
3.37	-1.98189931350114\\
3.38	-1.99095890410959\\
3.39	-2.00002277904328\\
3.4	-2.00909090909091\\
3.41	-2.01816326530612\\
3.42	-2.02723981900452\\
3.43	-2.03632054176072\\
3.44	-2.0454054054054\\
3.45	-2.05449438202247\\
3.46	-2.06358744394619\\
3.47	-2.07268456375839\\
3.48	-2.08178571428571\\
3.49	-2.09089086859688\\
3.5	-2.1\\
3.51	-2.10911308203991\\
3.52	-2.11823008849557\\
3.53	-2.12735099337748\\
3.54	-2.13647577092511\\
3.55	-2.1456043956044\\
3.56	-2.15473684210526\\
3.57	-2.16387308533917\\
3.58	-2.17301310043668\\
3.59	-2.1821568627451\\
3.6	-2.19130434782609\\
3.61	-2.20045553145336\\
3.62	-2.20961038961039\\
3.63	-2.21876889848812\\
3.64	-2.22793103448276\\
3.65	-2.23709677419355\\
3.66	-2.2462660944206\\
3.67	-2.25543897216274\\
3.68	-2.26461538461538\\
3.69	-2.27379530916844\\
3.7	-2.28297872340426\\
3.71	-2.29216560509554\\
3.72	-2.30135593220339\\
3.73	-2.31054968287526\\
3.74	-2.31974683544304\\
3.75	-2.32894736842105\\
3.76	-2.3381512605042\\
3.77	-2.34735849056604\\
3.78	-2.3565690376569\\
3.79	-2.36578288100209\\
3.8	-2.375\\
3.81	-2.38422037422037\\
3.82	-2.39344398340249\\
3.83	-2.40267080745342\\
3.84	-2.41190082644628\\
3.85	-2.42113402061856\\
3.86	-2.43037037037037\\
3.87	-2.43960985626283\\
3.88	-2.44885245901639\\
3.89	-2.4580981595092\\
3.9	-2.46734693877551\\
3.91	-2.47659877800407\\
3.92	-2.48585365853659\\
3.93	-2.49511156186613\\
3.94	-2.50437246963563\\
3.95	-2.51363636363636\\
3.96	-2.52290322580645\\
3.97	-2.53217303822938\\
3.98	-2.54144578313253\\
3.99	-2.55072144288577\\
4	-2.56\\
4.01	-2.56928143712575\\
4.02	-2.57856573705179\\
4.03	-2.58785288270378\\
4.04	-2.59714285714286\\
4.05	-2.60643564356436\\
4.06	-2.61573122529644\\
4.07	-2.62502958579882\\
4.08	-2.63433070866142\\
4.09	-2.64363457760314\\
4.1	-2.65294117647059\\
4.11	-2.66225048923679\\
4.12	-2.6715625\\
4.13	-2.68087719298246\\
4.14	-2.69019455252918\\
4.15	-2.6995145631068\\
4.16	-2.70883720930233\\
4.17	-2.71816247582205\\
4.18	-2.72749034749035\\
4.19	-2.73682080924855\\
4.2	-2.74615384615385\\
4.21	-2.75548944337812\\
4.22	-2.7648275862069\\
4.23	-2.77416826003824\\
4.24	-2.78351145038168\\
4.25	-2.79285714285714\\
4.26	-2.80220532319392\\
4.27	-2.8115559772296\\
4.28	-2.82090909090909\\
4.29	-2.83026465028355\\
4.3	-2.83962264150943\\
4.31	-2.84898305084746\\
4.32	-2.85834586466165\\
4.33	-2.86771106941839\\
4.34	-2.87707865168539\\
4.35	-2.88644859813084\\
4.36	-2.89582089552239\\
4.37	-2.90519553072626\\
4.38	-2.91457249070632\\
4.39	-2.92395176252319\\
4.4	-2.93333333333333\\
4.41	-2.94271719038817\\
4.42	-2.95210332103321\\
4.43	-2.96149171270718\\
4.44	-2.97088235294118\\
4.45	-2.9802752293578\\
4.46	-2.98967032967033\\
4.47	-2.9990676416819\\
4.48	-3.00846715328467\\
4.49	-3.01786885245902\\
4.5	-3.02727272727273\\
4.51	-3.03667876588022\\
4.52	-3.04608695652174\\
4.53	-3.0554972875226\\
4.54	-3.06490974729242\\
4.55	-3.07432432432432\\
4.56	-3.08374100719424\\
4.57	-3.09315978456014\\
4.58	-3.10258064516129\\
4.59	-3.11200357781753\\
4.6	-3.12142857142857\\
4.61	-3.13085561497326\\
4.62	-3.1402846975089\\
4.63	-3.14971580817051\\
4.64	-3.15914893617021\\
4.65	-3.16858407079646\\
4.66	-3.17802120141343\\
4.67	-3.18746031746032\\
4.68	-3.1969014084507\\
4.69	-3.20634446397188\\
4.7	-3.21578947368421\\
4.71	-3.22523642732049\\
4.72	-3.23468531468531\\
4.73	-3.24413612565445\\
4.74	-3.25358885017422\\
4.75	-3.26304347826087\\
4.76	-3.2725\\
4.77	-3.28195840554593\\
4.78	-3.29141868512111\\
4.79	-3.30088082901554\\
4.8	-3.31034482758621\\
4.81	-3.31981067125645\\
4.82	-3.32927835051546\\
4.83	-3.33874785591767\\
4.84	-3.34821917808219\\
4.85	-3.35769230769231\\
4.86	-3.36716723549488\\
4.87	-3.37664395229983\\
4.88	-3.38612244897959\\
4.89	-3.39560271646859\\
4.9	-3.40508474576271\\
4.91	-3.41456852791878\\
4.92	-3.42405405405405\\
4.93	-3.4335413153457\\
4.94	-3.4430303030303\\
4.95	-3.45252100840336\\
4.96	-3.46201342281879\\
4.97	-3.47150753768844\\
4.98	-3.48100334448161\\
4.99	-3.49050083472454\\
5	-3.5\\
};
\addplot [color=mycolor6,solid,forget plot]
  table[row sep=crcr]{%
0	0\\
0.01	0.00881188118811881\\
0.02	0.0172549019607843\\
0.03	0.0253398058252427\\
0.04	0.0330769230769231\\
0.05	0.0404761904761905\\
0.06	0.0475471698113207\\
0.07	0.0542990654205607\\
0.08	0.0607407407407407\\
0.09	0.0668807339449541\\
0.1	0.0727272727272727\\
0.11	0.0782882882882883\\
0.12	0.0835714285714285\\
0.13	0.0885840707964602\\
0.14	0.0933333333333333\\
0.15	0.0978260869565217\\
0.16	0.102068965517241\\
0.17	0.106068376068376\\
0.18	0.109830508474576\\
0.19	0.113361344537815\\
0.2	0.116666666666667\\
0.21	0.119752066115702\\
0.22	0.122622950819672\\
0.23	0.125284552845528\\
0.24	0.127741935483871\\
0.25	0.13\\
0.26	0.132063492063492\\
0.27	0.133937007874016\\
0.28	0.135625\\
0.29	0.137131782945736\\
0.3	0.138461538461538\\
0.31	0.139618320610687\\
0.32	0.140606060606061\\
0.33	0.141428571428571\\
0.34	0.142089552238806\\
0.35	0.142592592592593\\
0.36	0.142941176470588\\
0.37	0.143138686131387\\
0.38	0.143188405797101\\
0.39	0.143093525179856\\
0.4	0.142857142857143\\
0.41	0.142482269503546\\
0.42	0.141971830985916\\
0.43	0.141328671328671\\
0.44	0.140555555555556\\
0.45	0.139655172413793\\
0.46	0.138630136986301\\
0.47	0.137482993197279\\
0.48	0.136216216216216\\
0.49	0.134832214765101\\
0.5	0.133333333333333\\
0.51	0.131721854304636\\
0.52	0.13\\
0.53	0.128169934640523\\
0.54	0.126233766233766\\
0.55	0.124193548387097\\
0.56	0.122051282051282\\
0.57	0.119808917197452\\
0.58	0.11746835443038\\
0.59	0.11503144654088\\
0.6	0.1125\\
0.61	0.109875776397516\\
0.62	0.10716049382716\\
0.63	0.104355828220859\\
0.64	0.101463414634146\\
0.65	0.0984848484848485\\
0.66	0.0954216867469878\\
0.67	0.0922754491017965\\
0.68	0.0890476190476189\\
0.69	0.0857396449704142\\
0.7	0.0823529411764704\\
0.71	0.0788888888888889\\
0.72	0.0753488372093023\\
0.73	0.0717341040462428\\
0.74	0.0680459770114942\\
0.75	0.0642857142857143\\
0.76	0.0604545454545455\\
0.77	0.0565536723163841\\
0.78	0.0525842696629213\\
0.79	0.0485474860335194\\
0.8	0.0444444444444445\\
0.81	0.0402762430939226\\
0.82	0.0360439560439559\\
0.83	0.0317486338797813\\
0.84	0.027391304347826\\
0.85	0.022972972972973\\
0.86	0.0184946236559139\\
0.87	0.0139572192513367\\
0.88	0.00936170212765963\\
0.89	0.00470899470899455\\
0.9	0\\
0.91	-0.00476439790575933\\
0.92	-0.00958333333333328\\
0.93	-0.014455958549223\\
0.94	-0.0193814432989691\\
0.95	-0.0243589743589745\\
0.96	-0.0293877551020408\\
0.97	-0.0344670050761421\\
0.98	-0.0395959595959596\\
0.99	-0.0447738693467338\\
1	-0.05\\
1.01	-0.055273631840796\\
1.02	-0.0605940594059406\\
1.03	-0.0659605911330051\\
1.04	-0.071372549019608\\
1.05	-0.0768292682926829\\
1.06	-0.0823300970873787\\
1.07	-0.0878743961352658\\
1.08	-0.0934615384615385\\
1.09	-0.0990909090909091\\
1.1	-0.104761904761905\\
1.11	-0.110473933649289\\
1.12	-0.11622641509434\\
1.13	-0.122018779342723\\
1.14	-0.12785046728972\\
1.15	-0.133720930232558\\
1.16	-0.13962962962963\\
1.17	-0.145576036866359\\
1.18	-0.151559633027523\\
1.19	-0.157579908675799\\
1.2	-0.163636363636364\\
1.21	-0.16972850678733\\
1.22	-0.175855855855856\\
1.23	-0.182017937219731\\
1.24	-0.188214285714286\\
1.25	-0.194444444444444\\
1.26	-0.20070796460177\\
1.27	-0.207004405286344\\
1.28	-0.213333333333334\\
1.29	-0.219694323144105\\
1.3	-0.226086956521739\\
1.31	-0.232510822510823\\
1.32	-0.238965517241379\\
1.33	-0.245450643776824\\
1.34	-0.251965811965812\\
1.35	-0.258510638297872\\
1.36	-0.265084745762712\\
1.37	-0.27168776371308\\
1.38	-0.278319327731092\\
1.39	-0.284979079497908\\
1.4	-0.291666666666667\\
1.41	-0.298381742738589\\
1.42	-0.305123966942149\\
1.43	-0.311893004115226\\
1.44	-0.318688524590164\\
1.45	-0.325510204081633\\
1.46	-0.332357723577236\\
1.47	-0.339230769230769\\
1.48	-0.346129032258065\\
1.49	-0.353052208835341\\
1.5	-0.36\\
1.51	-0.366972111553785\\
1.52	-0.373968253968254\\
1.53	-0.38098814229249\\
1.54	-0.388031496062992\\
1.55	-0.395098039215686\\
1.56	-0.4021875\\
1.57	-0.409299610894942\\
1.58	-0.416434108527132\\
1.59	-0.423590733590734\\
1.6	-0.430769230769231\\
1.61	-0.437969348659004\\
1.62	-0.445190839694656\\
1.63	-0.452433460076046\\
1.64	-0.45969696969697\\
1.65	-0.466981132075472\\
1.66	-0.474285714285714\\
1.67	-0.481610486891386\\
1.68	-0.488955223880597\\
1.69	-0.496319702602231\\
1.7	-0.503703703703704\\
1.71	-0.511107011070111\\
1.72	-0.518529411764706\\
1.73	-0.525970695970696\\
1.74	-0.533430656934307\\
1.75	-0.540909090909091\\
1.76	-0.548405797101449\\
1.77	-0.555920577617329\\
1.78	-0.563453237410072\\
1.79	-0.571003584229391\\
1.8	-0.578571428571429\\
1.81	-0.586156583629893\\
1.82	-0.593758865248227\\
1.83	-0.601378091872792\\
1.84	-0.609014084507042\\
1.85	-0.616666666666667\\
1.86	-0.624335664335665\\
1.87	-0.632020905923345\\
1.88	-0.639722222222222\\
1.89	-0.647439446366782\\
1.9	-0.655172413793104\\
1.91	-0.662920962199313\\
1.92	-0.670684931506849\\
1.93	-0.678464163822526\\
1.94	-0.686258503401361\\
1.95	-0.69406779661017\\
1.96	-0.701891891891892\\
1.97	-0.70973063973064\\
1.98	-0.71758389261745\\
1.99	-0.725451505016723\\
2	-0.733333333333333\\
2.01	-0.741229235880399\\
2.02	-0.749139072847682\\
2.03	-0.757062706270627\\
2.04	-0.765\\
2.05	-0.772950819672131\\
2.06	-0.780915032679738\\
2.07	-0.788892508143322\\
2.08	-0.796883116883117\\
2.09	-0.804886731391586\\
2.1	-0.812903225806452\\
2.11	-0.820932475884244\\
2.12	-0.828974358974359\\
2.13	-0.83702875399361\\
2.14	-0.845095541401274\\
2.15	-0.853174603174603\\
2.16	-0.86126582278481\\
2.17	-0.869369085173502\\
2.18	-0.87748427672956\\
2.19	-0.885611285266458\\
2.2	-0.89375\\
2.21	-0.90190031152648\\
2.22	-0.910062111801243\\
2.23	-0.918235294117647\\
2.24	-0.92641975308642\\
2.25	-0.934615384615385\\
2.26	-0.942822085889571\\
2.27	-0.951039755351682\\
2.28	-0.959268292682927\\
2.29	-0.967507598784195\\
2.3	-0.975757575757576\\
2.31	-0.984018126888218\\
2.32	-0.992289156626506\\
2.33	-1.00057057057057\\
2.34	-1.0088622754491\\
2.35	-1.01716417910448\\
2.36	-1.02547619047619\\
2.37	-1.03379821958457\\
2.38	-1.04213017751479\\
2.39	-1.05047197640118\\
2.4	-1.05882352941176\\
2.41	-1.06718475073314\\
2.42	-1.07555555555556\\
2.43	-1.08393586005831\\
2.44	-1.09232558139535\\
2.45	-1.10072463768116\\
2.46	-1.10913294797688\\
2.47	-1.11755043227666\\
2.48	-1.12597701149425\\
2.49	-1.13441260744986\\
2.5	-1.14285714285714\\
2.51	-1.15131054131054\\
2.52	-1.15977272727273\\
2.53	-1.16824362606232\\
2.54	-1.17672316384181\\
2.55	-1.18521126760563\\
2.56	-1.19370786516854\\
2.57	-1.20221288515406\\
2.58	-1.21072625698324\\
2.59	-1.21924791086351\\
2.6	-1.22777777777778\\
2.61	-1.23631578947368\\
2.62	-1.24486187845304\\
2.63	-1.25341597796143\\
2.64	-1.26197802197802\\
2.65	-1.27054794520548\\
2.66	-1.27912568306011\\
2.67	-1.28771117166213\\
2.68	-1.29630434782609\\
2.69	-1.30490514905149\\
2.7	-1.31351351351351\\
2.71	-1.32212938005391\\
2.72	-1.33075268817204\\
2.73	-1.33938337801609\\
2.74	-1.34802139037433\\
2.75	-1.35666666666667\\
2.76	-1.36531914893617\\
2.77	-1.37397877984085\\
2.78	-1.3826455026455\\
2.79	-1.39131926121372\\
2.8	-1.4\\
2.81	-1.40868766404199\\
2.82	-1.41738219895288\\
2.83	-1.42608355091384\\
2.84	-1.43479166666667\\
2.85	-1.44350649350649\\
2.86	-1.45222797927461\\
2.87	-1.46095607235142\\
2.88	-1.46969072164948\\
2.89	-1.47843187660668\\
2.9	-1.48717948717949\\
2.91	-1.49593350383632\\
2.92	-1.50469387755102\\
2.93	-1.51346055979644\\
2.94	-1.52223350253807\\
2.95	-1.53101265822785\\
2.96	-1.53979797979798\\
2.97	-1.54858942065491\\
2.98	-1.55738693467337\\
2.99	-1.56619047619048\\
3	-1.575\\
3.01	-1.58381546134663\\
3.02	-1.5926368159204\\
3.03	-1.60146401985112\\
3.04	-1.61029702970297\\
3.05	-1.61913580246914\\
3.06	-1.6279802955665\\
3.07	-1.63683046683047\\
3.08	-1.6456862745098\\
3.09	-1.65454767726161\\
3.1	-1.66341463414634\\
3.11	-1.67228710462287\\
3.12	-1.68116504854369\\
3.13	-1.69004842615012\\
3.14	-1.69893719806763\\
3.15	-1.7078313253012\\
3.16	-1.71673076923077\\
3.17	-1.72563549160671\\
3.18	-1.73454545454545\\
3.19	-1.74346062052506\\
3.2	-1.75238095238095\\
3.21	-1.76130641330166\\
3.22	-1.77023696682464\\
3.23	-1.77917257683215\\
3.24	-1.78811320754717\\
3.25	-1.79705882352941\\
3.26	-1.80600938967136\\
3.27	-1.81496487119438\\
3.28	-1.82392523364486\\
3.29	-1.83289044289044\\
3.3	-1.84186046511628\\
3.31	-1.85083526682135\\
3.32	-1.85981481481482\\
3.33	-1.86879907621247\\
3.34	-1.87778801843318\\
3.35	-1.8867816091954\\
3.36	-1.89577981651376\\
3.37	-1.90478260869565\\
3.38	-1.9137899543379\\
3.39	-1.92280182232346\\
3.4	-1.93181818181818\\
3.41	-1.94083900226757\\
3.42	-1.94986425339367\\
3.43	-1.95889390519187\\
3.44	-1.96792792792793\\
3.45	-1.97696629213483\\
3.46	-1.98600896860987\\
3.47	-1.99505592841163\\
3.48	-2.00410714285714\\
3.49	-2.01316258351893\\
3.5	-2.02222222222222\\
3.51	-2.03128603104213\\
3.52	-2.04035398230088\\
3.53	-2.04942604856512\\
3.54	-2.05850220264317\\
3.55	-2.06758241758242\\
3.56	-2.07666666666667\\
3.57	-2.08575492341357\\
3.58	-2.09484716157205\\
3.59	-2.10394335511983\\
3.6	-2.11304347826087\\
3.61	-2.12214750542299\\
3.62	-2.13125541125541\\
3.63	-2.14036717062635\\
3.64	-2.14948275862069\\
3.65	-2.15860215053763\\
3.66	-2.16772532188841\\
3.67	-2.176852248394\\
3.68	-2.18598290598291\\
3.69	-2.19511727078891\\
3.7	-2.20425531914894\\
3.71	-2.21339702760085\\
3.72	-2.22254237288136\\
3.73	-2.23169133192389\\
3.74	-2.24084388185654\\
3.75	-2.25\\
3.76	-2.25915966386555\\
3.77	-2.26832285115304\\
3.78	-2.27748953974895\\
3.79	-2.28665970772443\\
3.8	-2.29583333333333\\
3.81	-2.3050103950104\\
3.82	-2.31419087136929\\
3.83	-2.32337474120083\\
3.84	-2.33256198347107\\
3.85	-2.34175257731959\\
3.86	-2.35094650205761\\
3.87	-2.36014373716632\\
3.88	-2.36934426229508\\
3.89	-2.37854805725971\\
3.9	-2.38775510204082\\
3.91	-2.39696537678208\\
3.92	-2.40617886178862\\
3.93	-2.41539553752535\\
3.94	-2.42461538461538\\
3.95	-2.43383838383838\\
3.96	-2.44306451612903\\
3.97	-2.45229376257545\\
3.98	-2.46152610441767\\
3.99	-2.47076152304609\\
4	-2.48\\
4.01	-2.48924151696607\\
4.02	-2.49848605577689\\
4.03	-2.50773359840954\\
4.04	-2.51698412698413\\
4.05	-2.52623762376238\\
4.06	-2.53549407114624\\
4.07	-2.54475345167653\\
4.08	-2.5540157480315\\
4.09	-2.56328094302554\\
4.1	-2.57254901960784\\
4.11	-2.58181996086106\\
4.12	-2.59109375\\
4.13	-2.60037037037037\\
4.14	-2.60964980544747\\
4.15	-2.61893203883495\\
4.16	-2.62821705426357\\
4.17	-2.63750483558994\\
4.18	-2.64679536679537\\
4.19	-2.65608863198459\\
4.2	-2.66538461538462\\
4.21	-2.67468330134357\\
4.22	-2.6839846743295\\
4.23	-2.69328871892925\\
4.24	-2.70259541984733\\
4.25	-2.71190476190476\\
4.26	-2.72121673003802\\
4.27	-2.73053130929791\\
4.28	-2.73984848484849\\
4.29	-2.74916824196597\\
4.3	-2.75849056603774\\
4.31	-2.76781544256121\\
4.32	-2.77714285714286\\
4.33	-2.78647279549719\\
4.34	-2.79580524344569\\
4.35	-2.80514018691589\\
4.36	-2.8144776119403\\
4.37	-2.82381750465549\\
4.38	-2.83315985130112\\
4.39	-2.84250463821892\\
4.4	-2.85185185185185\\
4.41	-2.86120147874307\\
4.42	-2.87055350553506\\
4.43	-2.87990791896869\\
4.44	-2.88926470588235\\
4.45	-2.89862385321101\\
4.46	-2.90798534798535\\
4.47	-2.9173491773309\\
4.48	-2.92671532846715\\
4.49	-2.93608378870674\\
4.5	-2.94545454545455\\
4.51	-2.9548275862069\\
4.52	-2.96420289855072\\
4.53	-2.97358047016275\\
4.54	-2.98296028880866\\
4.55	-2.99234234234234\\
4.56	-3.00172661870504\\
4.57	-3.0111131059246\\
4.58	-3.0205017921147\\
4.59	-3.02989266547406\\
4.6	-3.03928571428571\\
4.61	-3.04868092691622\\
4.62	-3.05807829181495\\
4.63	-3.06747779751332\\
4.64	-3.07687943262411\\
4.65	-3.08628318584071\\
4.66	-3.0956890459364\\
4.67	-3.10509700176367\\
4.68	-3.11450704225352\\
4.69	-3.12391915641476\\
4.7	-3.13333333333333\\
4.71	-3.14274956217163\\
4.72	-3.15216783216783\\
4.73	-3.16158813263525\\
4.74	-3.17101045296167\\
4.75	-3.1804347826087\\
4.76	-3.18986111111111\\
4.77	-3.19928942807626\\
4.78	-3.20871972318339\\
4.79	-3.21815198618307\\
4.8	-3.22758620689655\\
4.81	-3.23702237521515\\
4.82	-3.24646048109966\\
4.83	-3.25590051457976\\
4.84	-3.26534246575342\\
4.85	-3.27478632478632\\
4.86	-3.28423208191126\\
4.87	-3.2936797274276\\
4.88	-3.30312925170068\\
4.89	-3.31258064516129\\
4.9	-3.32203389830509\\
4.91	-3.33148900169205\\
4.92	-3.34094594594595\\
4.93	-3.35040472175379\\
4.94	-3.35986531986532\\
4.95	-3.36932773109244\\
4.96	-3.37879194630872\\
4.97	-3.38825795644891\\
4.98	-3.39772575250836\\
4.99	-3.40719532554257\\
5	-3.41666666666667\\
};
\addplot [color=mycolor7,solid,forget plot]
  table[row sep=crcr]{%
0	0\\
0.01	0.0098019801980198\\
0.02	0.0192156862745098\\
0.03	0.028252427184466\\
0.04	0.0369230769230769\\
0.05	0.0452380952380952\\
0.06	0.0532075471698113\\
0.07	0.0608411214953271\\
0.08	0.0681481481481481\\
0.09	0.0751376146788991\\
0.1	0.0818181818181818\\
0.11	0.0881981981981982\\
0.12	0.0942857142857143\\
0.13	0.100088495575221\\
0.14	0.105614035087719\\
0.15	0.110869565217391\\
0.16	0.115862068965517\\
0.17	0.120598290598291\\
0.18	0.125084745762712\\
0.19	0.129327731092437\\
0.2	0.133333333333333\\
0.21	0.137107438016529\\
0.22	0.140655737704918\\
0.23	0.143983739837398\\
0.24	0.147096774193548\\
0.25	0.15\\
0.26	0.152698412698413\\
0.27	0.155196850393701\\
0.28	0.1575\\
0.29	0.159612403100775\\
0.3	0.161538461538461\\
0.31	0.163282442748092\\
0.32	0.164848484848485\\
0.33	0.166240601503759\\
0.34	0.167462686567164\\
0.35	0.168518518518518\\
0.36	0.169411764705882\\
0.37	0.17014598540146\\
0.38	0.17072463768116\\
0.39	0.171151079136691\\
0.4	0.171428571428571\\
0.41	0.171560283687943\\
0.42	0.171549295774648\\
0.43	0.171398601398601\\
0.44	0.171111111111111\\
0.45	0.170689655172414\\
0.46	0.17013698630137\\
0.47	0.169455782312925\\
0.48	0.168648648648649\\
0.49	0.167718120805369\\
0.5	0.166666666666667\\
0.51	0.165496688741722\\
0.52	0.164210526315789\\
0.53	0.16281045751634\\
0.54	0.161298701298701\\
0.55	0.159677419354839\\
0.56	0.157948717948718\\
0.57	0.156114649681529\\
0.58	0.154177215189873\\
0.59	0.152138364779874\\
0.6	0.15\\
0.61	0.14776397515528\\
0.62	0.145432098765432\\
0.63	0.143006134969325\\
0.64	0.140487804878049\\
0.65	0.137878787878788\\
0.66	0.135180722891566\\
0.67	0.132395209580838\\
0.68	0.129523809523809\\
0.69	0.126568047337278\\
0.7	0.123529411764706\\
0.71	0.120409356725146\\
0.72	0.117209302325581\\
0.73	0.11393063583815\\
0.74	0.110574712643678\\
0.75	0.107142857142857\\
0.76	0.103636363636364\\
0.77	0.100056497175141\\
0.78	0.0964044943820226\\
0.79	0.09268156424581\\
0.8	0.0888888888888889\\
0.81	0.0850276243093924\\
0.82	0.0810989010989009\\
0.83	0.077103825136612\\
0.84	0.0730434782608697\\
0.85	0.0689189189189188\\
0.86	0.064731182795699\\
0.87	0.0604812834224597\\
0.88	0.0561702127659575\\
0.89	0.0517989417989417\\
0.9	0.0473684210526315\\
0.91	0.0428795811518324\\
0.92	0.0383333333333334\\
0.93	0.0337305699481865\\
0.94	0.0290721649484537\\
0.95	0.0243589743589743\\
0.96	0.0195918367346939\\
0.97	0.0147715736040609\\
0.98	0.00989898989898996\\
0.99	0.00497487437185917\\
1	0\\
1.01	-0.00502487562189047\\
1.02	-0.0100990099009901\\
1.03	-0.0152216748768474\\
1.04	-0.0203921568627452\\
1.05	-0.0256097560975608\\
1.06	-0.0308737864077671\\
1.07	-0.0361835748792272\\
1.08	-0.0415384615384615\\
1.09	-0.0469377990430622\\
1.1	-0.0523809523809524\\
1.11	-0.0578672985781992\\
1.12	-0.0633962264150945\\
1.13	-0.0689671361502346\\
1.14	-0.0745794392523365\\
1.15	-0.0802325581395351\\
1.16	-0.085925925925926\\
1.17	-0.0916589861751151\\
1.18	-0.0974311926605502\\
1.19	-0.10324200913242\\
1.2	-0.109090909090909\\
1.21	-0.114977375565611\\
1.22	-0.120900900900901\\
1.23	-0.126860986547085\\
1.24	-0.132857142857143\\
1.25	-0.138888888888889\\
1.26	-0.144955752212389\\
1.27	-0.151057268722467\\
1.28	-0.15719298245614\\
1.29	-0.163362445414847\\
1.3	-0.169565217391304\\
1.31	-0.175800865800866\\
1.32	-0.182068965517241\\
1.33	-0.188369098712446\\
1.34	-0.194700854700855\\
1.35	-0.201063829787234\\
1.36	-0.207457627118644\\
1.37	-0.213881856540085\\
1.38	-0.220336134453782\\
1.39	-0.226820083682008\\
1.4	-0.233333333333333\\
1.41	-0.239875518672199\\
1.42	-0.246446280991736\\
1.43	-0.253045267489712\\
1.44	-0.259672131147541\\
1.45	-0.266326530612245\\
1.46	-0.273008130081301\\
1.47	-0.279716599190283\\
1.48	-0.286451612903226\\
1.49	-0.293212851405623\\
1.5	-0.3\\
1.51	-0.306812749003984\\
1.52	-0.313650793650794\\
1.53	-0.320513833992095\\
1.54	-0.32740157480315\\
1.55	-0.334313725490196\\
1.56	-0.34125\\
1.57	-0.348210116731518\\
1.58	-0.355193798449613\\
1.59	-0.362200772200772\\
1.6	-0.369230769230769\\
1.61	-0.376283524904215\\
1.62	-0.383358778625954\\
1.63	-0.390456273764259\\
1.64	-0.397575757575758\\
1.65	-0.404716981132076\\
1.66	-0.41187969924812\\
1.67	-0.419063670411985\\
1.68	-0.426268656716418\\
1.69	-0.433494423791821\\
1.7	-0.440740740740741\\
1.71	-0.448007380073801\\
1.72	-0.455294117647059\\
1.73	-0.462600732600733\\
1.74	-0.46992700729927\\
1.75	-0.477272727272727\\
1.76	-0.48463768115942\\
1.77	-0.492021660649819\\
1.78	-0.499424460431655\\
1.79	-0.506845878136201\\
1.8	-0.514285714285714\\
1.81	-0.521743772241993\\
1.82	-0.529219858156029\\
1.83	-0.536713780918728\\
1.84	-0.544225352112676\\
1.85	-0.551754385964912\\
1.86	-0.559300699300699\\
1.87	-0.566864111498258\\
1.88	-0.574444444444445\\
1.89	-0.58204152249135\\
1.9	-0.589655172413793\\
1.91	-0.597285223367698\\
1.92	-0.604931506849315\\
1.93	-0.61259385665529\\
1.94	-0.620272108843537\\
1.95	-0.627966101694915\\
1.96	-0.635675675675676\\
1.97	-0.643400673400673\\
1.98	-0.651140939597315\\
1.99	-0.658896321070234\\
2	-0.666666666666667\\
2.01	-0.674451827242525\\
2.02	-0.682251655629139\\
2.03	-0.69006600660066\\
2.04	-0.697894736842105\\
2.05	-0.705737704918033\\
2.06	-0.71359477124183\\
2.07	-0.721465798045603\\
2.08	-0.729350649350649\\
2.09	-0.737249190938511\\
2.1	-0.745161290322581\\
2.11	-0.753086816720257\\
2.12	-0.761025641025641\\
2.13	-0.768977635782748\\
2.14	-0.776942675159236\\
2.15	-0.784920634920635\\
2.16	-0.792911392405063\\
2.17	-0.800914826498423\\
2.18	-0.808930817610063\\
2.19	-0.816959247648903\\
2.2	-0.825\\
2.21	-0.833052959501558\\
2.22	-0.841118012422361\\
2.23	-0.849195046439628\\
2.24	-0.857283950617284\\
2.25	-0.865384615384615\\
2.26	-0.873496932515337\\
2.27	-0.881620795107034\\
2.28	-0.889756097560976\\
2.29	-0.89790273556231\\
2.3	-0.906060606060606\\
2.31	-0.914229607250755\\
2.32	-0.922409638554217\\
2.33	-0.930600600600601\\
2.34	-0.938802395209581\\
2.35	-0.947014925373135\\
2.36	-0.955238095238095\\
2.37	-0.963471810089021\\
2.38	-0.971715976331361\\
2.39	-0.979970501474926\\
2.4	-0.988235294117647\\
2.41	-0.996510263929619\\
2.42	-1.00479532163743\\
2.43	-1.01309037900875\\
2.44	-1.02139534883721\\
2.45	-1.02971014492754\\
2.46	-1.03803468208092\\
2.47	-1.04636887608069\\
2.48	-1.05471264367816\\
2.49	-1.0630659025788\\
2.5	-1.07142857142857\\
2.51	-1.07980056980057\\
2.52	-1.08818181818182\\
2.53	-1.09657223796034\\
2.54	-1.10497175141243\\
2.55	-1.11338028169014\\
2.56	-1.12179775280899\\
2.57	-1.13022408963585\\
2.58	-1.1386592178771\\
2.59	-1.14710306406685\\
2.6	-1.15555555555556\\
2.61	-1.16401662049861\\
2.62	-1.1724861878453\\
2.63	-1.18096418732782\\
2.64	-1.18945054945055\\
2.65	-1.19794520547945\\
2.66	-1.20644808743169\\
2.67	-1.21495912806539\\
2.68	-1.22347826086957\\
2.69	-1.2320054200542\\
2.7	-1.24054054054054\\
2.71	-1.24908355795148\\
2.72	-1.25763440860215\\
2.73	-1.26619302949062\\
2.74	-1.27475935828877\\
2.75	-1.28333333333333\\
2.76	-1.29191489361702\\
2.77	-1.30050397877984\\
2.78	-1.30910052910053\\
2.79	-1.31770448548813\\
2.8	-1.32631578947368\\
2.81	-1.3349343832021\\
2.82	-1.34356020942408\\
2.83	-1.35219321148825\\
2.84	-1.36083333333333\\
2.85	-1.36948051948052\\
2.86	-1.37813471502591\\
2.87	-1.38679586563308\\
2.88	-1.39546391752577\\
2.89	-1.40413881748072\\
2.9	-1.41282051282051\\
2.91	-1.42150895140665\\
2.92	-1.43020408163265\\
2.93	-1.4389058524173\\
2.94	-1.44761421319797\\
2.95	-1.45632911392405\\
2.96	-1.4650505050505\\
2.97	-1.47377833753149\\
2.98	-1.48251256281407\\
2.99	-1.49125313283208\\
3	-1.5\\
3.01	-1.50875311720698\\
3.02	-1.51751243781095\\
3.03	-1.52627791563275\\
3.04	-1.5350495049505\\
3.05	-1.54382716049383\\
3.06	-1.55261083743842\\
3.07	-1.56140049140049\\
3.08	-1.57019607843137\\
3.09	-1.57899755501222\\
3.1	-1.58780487804878\\
3.11	-1.59661800486618\\
3.12	-1.60543689320388\\
3.13	-1.61426150121065\\
3.14	-1.62309178743961\\
3.15	-1.63192771084337\\
3.16	-1.64076923076923\\
3.17	-1.64961630695444\\
3.18	-1.65846889952153\\
3.19	-1.66732696897375\\
3.2	-1.67619047619048\\
3.21	-1.6850593824228\\
3.22	-1.6939336492891\\
3.23	-1.70281323877069\\
3.24	-1.71169811320755\\
3.25	-1.72058823529412\\
3.26	-1.72948356807512\\
3.27	-1.73838407494145\\
3.28	-1.74728971962617\\
3.29	-1.75620046620047\\
3.3	-1.76511627906977\\
3.31	-1.77403712296984\\
3.32	-1.78296296296296\\
3.33	-1.79189376443418\\
3.34	-1.80082949308756\\
3.35	-1.80977011494253\\
3.36	-1.81871559633027\\
3.37	-1.82766590389016\\
3.38	-1.83662100456621\\
3.39	-1.84558086560364\\
3.4	-1.85454545454545\\
3.41	-1.86351473922903\\
3.42	-1.87248868778281\\
3.43	-1.88146726862302\\
3.44	-1.89045045045045\\
3.45	-1.89943820224719\\
3.46	-1.90843049327354\\
3.47	-1.91742729306488\\
3.48	-1.92642857142857\\
3.49	-1.93543429844098\\
3.5	-1.94444444444444\\
3.51	-1.95345898004435\\
3.52	-1.96247787610619\\
3.53	-1.97150110375276\\
3.54	-1.98052863436123\\
3.55	-1.98956043956044\\
3.56	-1.99859649122807\\
3.57	-2.00763676148796\\
3.58	-2.01668122270742\\
3.59	-2.02572984749455\\
3.6	-2.03478260869565\\
3.61	-2.04383947939262\\
3.62	-2.05290043290043\\
3.63	-2.06196544276458\\
3.64	-2.07103448275862\\
3.65	-2.08010752688172\\
3.66	-2.08918454935622\\
3.67	-2.09826552462527\\
3.68	-2.10735042735043\\
3.69	-2.11643923240938\\
3.7	-2.12553191489362\\
3.71	-2.13462845010616\\
3.72	-2.14372881355932\\
3.73	-2.15283298097252\\
3.74	-2.16194092827004\\
3.75	-2.17105263157895\\
3.76	-2.18016806722689\\
3.77	-2.18928721174004\\
3.78	-2.198410041841\\
3.79	-2.20753653444676\\
3.8	-2.21666666666667\\
3.81	-2.22580041580042\\
3.82	-2.2349377593361\\
3.83	-2.24407867494824\\
3.84	-2.25322314049587\\
3.85	-2.26237113402062\\
3.86	-2.27152263374486\\
3.87	-2.28067761806982\\
3.88	-2.28983606557377\\
3.89	-2.29899795501022\\
3.9	-2.30816326530612\\
3.91	-2.31733197556008\\
3.92	-2.32650406504065\\
3.93	-2.33567951318458\\
3.94	-2.34485829959514\\
3.95	-2.3540404040404\\
3.96	-2.36322580645161\\
3.97	-2.37241448692153\\
3.98	-2.38160642570281\\
3.99	-2.39080160320641\\
4	-2.4\\
4.01	-2.40920159680639\\
4.02	-2.41840637450199\\
4.03	-2.42761431411531\\
4.04	-2.4368253968254\\
4.05	-2.4460396039604\\
4.06	-2.45525691699605\\
4.07	-2.46447731755424\\
4.08	-2.47370078740158\\
4.09	-2.48292730844794\\
4.1	-2.4921568627451\\
4.11	-2.50138943248532\\
4.12	-2.510625\\
4.13	-2.51986354775828\\
4.14	-2.52910505836576\\
4.15	-2.53834951456311\\
4.16	-2.54759689922481\\
4.17	-2.55684719535783\\
4.18	-2.56610038610039\\
4.19	-2.57535645472062\\
4.2	-2.58461538461538\\
4.21	-2.59387715930902\\
4.22	-2.60314176245211\\
4.23	-2.61240917782027\\
4.24	-2.62167938931298\\
4.25	-2.63095238095238\\
4.26	-2.64022813688213\\
4.27	-2.64950664136622\\
4.28	-2.65878787878788\\
4.29	-2.66807183364839\\
4.3	-2.67735849056604\\
4.31	-2.68664783427495\\
4.32	-2.69593984962406\\
4.33	-2.70523452157598\\
4.34	-2.71453183520599\\
4.35	-2.72383177570093\\
4.36	-2.73313432835821\\
4.37	-2.74243947858473\\
4.38	-2.75174721189591\\
4.39	-2.76105751391466\\
4.4	-2.77037037037037\\
4.41	-2.77968576709797\\
4.42	-2.7890036900369\\
4.43	-2.7983241252302\\
4.44	-2.80764705882353\\
4.45	-2.81697247706422\\
4.46	-2.82630036630037\\
4.47	-2.83563071297989\\
4.48	-2.84496350364964\\
4.49	-2.85429872495446\\
4.5	-2.86363636363636\\
4.51	-2.87297640653358\\
4.52	-2.88231884057971\\
4.53	-2.89166365280289\\
4.54	-2.90101083032491\\
4.55	-2.91036036036036\\
4.56	-2.91971223021583\\
4.57	-2.92906642728905\\
4.58	-2.9384229390681\\
4.59	-2.94778175313059\\
4.6	-2.95714285714286\\
4.61	-2.96650623885918\\
4.62	-2.975871886121\\
4.63	-2.98523978685613\\
4.64	-2.99460992907801\\
4.65	-3.00398230088496\\
4.66	-3.01335689045936\\
4.67	-3.02273368606702\\
4.68	-3.03211267605634\\
4.69	-3.04149384885765\\
4.7	-3.05087719298246\\
4.71	-3.06026269702277\\
4.72	-3.06965034965035\\
4.73	-3.07904013961606\\
4.74	-3.08843205574913\\
4.75	-3.09782608695652\\
4.76	-3.10722222222222\\
4.77	-3.11662045060659\\
4.78	-3.12602076124567\\
4.79	-3.1354231433506\\
4.8	-3.1448275862069\\
4.81	-3.15423407917384\\
4.82	-3.16364261168385\\
4.83	-3.17305317324185\\
4.84	-3.18246575342466\\
4.85	-3.19188034188034\\
4.86	-3.20129692832765\\
4.87	-3.21071550255537\\
4.88	-3.22013605442177\\
4.89	-3.22955857385399\\
4.9	-3.23898305084746\\
4.91	-3.24840947546531\\
4.92	-3.25783783783784\\
4.93	-3.26726812816189\\
4.94	-3.27670033670034\\
4.95	-3.28613445378151\\
4.96	-3.29557046979866\\
4.97	-3.30500837520938\\
4.98	-3.31444816053512\\
4.99	-3.3238898163606\\
5	-3.33333333333333\\
};
\addplot [color=red,only marks,mark=asterisk,mark options={solid},forget plot]
  table[row sep=crcr]{%
-5.55111512312578e-17	0\\
};
\addplot [color=red,only marks,mark=asterisk,mark options={solid},forget plot]
  table[row sep=crcr]{%
8.74787441744869e-20	0\\
};
\addplot [color=red,only marks,mark=asterisk,mark options={solid},forget plot]
  table[row sep=crcr]{%
-3.25739628826474e-22	0\\
};
\addplot [color=red,only marks,mark=asterisk,mark options={solid},forget plot]
  table[row sep=crcr]{%
-1.2412215694122e-16	0\\
};
\addplot [color=red,only marks,mark=asterisk,mark options={solid},forget plot]
  table[row sep=crcr]{%
-3.80323207149667e-19	0\\
};
\addplot [color=red,only marks,mark=asterisk,mark options={solid},forget plot]
  table[row sep=crcr]{%
-1.42196127852538e-18	0\\
};
\addplot [color=red,only marks,mark=asterisk,mark options={solid},forget plot]
  table[row sep=crcr]{%
-3.00559044891048e-17	0\\
};
\addplot [color=red,only marks,mark=asterisk,mark options={solid},forget plot]
  table[row sep=crcr]{%
1.9283552904372e-18	0\\
};
\addplot [color=red,only marks,mark=asterisk,mark options={solid},forget plot]
  table[row sep=crcr]{%
1.578393658814e-18	0\\
};
\addplot [color=red,only marks,mark=asterisk,mark options={solid},forget plot]
  table[row sep=crcr]{%
2.626844099944e-23	0\\
};
\addplot [color=red,only marks,mark=asterisk,mark options={solid},forget plot]
  table[row sep=crcr]{%
5.06656246881054e-18	0\\
};
\addplot [color=red,only marks,mark=asterisk,mark options={solid},forget plot]
  table[row sep=crcr]{%
0.1	0\\
};
\addplot [color=red,only marks,mark=asterisk,mark options={solid},forget plot]
  table[row sep=crcr]{%
0.2	0\\
};
\addplot [color=red,only marks,mark=asterisk,mark options={solid},forget plot]
  table[row sep=crcr]{%
0.3	0\\
};
\addplot [color=red,only marks,mark=asterisk,mark options={solid},forget plot]
  table[row sep=crcr]{%
0.4	0\\
};
\addplot [color=red,only marks,mark=asterisk,mark options={solid},forget plot]
  table[row sep=crcr]{%
0.5	0\\
};
\addplot [color=red,only marks,mark=asterisk,mark options={solid},forget plot]
  table[row sep=crcr]{%
0.6	0\\
};
\addplot [color=red,only marks,mark=asterisk,mark options={solid},forget plot]
  table[row sep=crcr]{%
0.7	0\\
};
\addplot [color=red,only marks,mark=asterisk,mark options={solid},forget plot]
  table[row sep=crcr]{%
0.8	0\\
};
\addplot [color=red,only marks,mark=asterisk,mark options={solid},forget plot]
  table[row sep=crcr]{%
0.9	0\\
};
\addplot [color=red,only marks,mark=asterisk,mark options={solid},forget plot]
  table[row sep=crcr]{%
1	0\\
};
\end{axis}
\end{tikzpicture}%
\end{document}
\caption{Plot of the position of the right zero and the parameter p (left). Plot of $\dot{n}$ for different p values with the zeros marked with red stars. $G=f=k=1$ is assumed, leading to $p_c = 1$.}
\label{fig:transBif}
\end{figure}
Figure~\ref{fig:transBif} indicates a trans-critical bifurcation.

\end{document}
